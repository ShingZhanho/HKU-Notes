% !TEX program = latexmk
% !TEX options = --shell-escape -synctex=1 -interaction=nonstopmode -file-line-error -pdf -cd -outdir=. ./COMP2120-Assignment4.tex

\documentclass[answers]{exam}
\usepackage[english]{babel}
\usepackage[style=apa,backend=biber]{biblatex}
\usepackage[
    format=plain,
    labelsep=newline,
    font=small,
    labelfont=bf,
    textfont=it,
    justification=justified,
    singlelinecheck=off
    ]{caption}
\usepackage{csquotes}
\usepackage{float}
\usepackage[top=2.5cm,bottom=2.5cm,left=3cm,right=3cm]{geometry}
\usepackage{graphicx}
\usepackage[hidelinks,colorlinks=true,linkcolor=blue,filecolor=blue,urlcolor=blue,citecolor=blue]{hyperref}
    \usepackage[nameinlink]{cleveref} %nameinlink ensures that the entire element is clickable in the pdf, not just the number
\usepackage{indentfirst}
\usepackage{multicol}
\usepackage[skip=1em,indent]{parskip}
\usepackage{tabularx}
\usepackage{times}
\pagestyle{foot}
\cfoot{Page \thepage\ of \numpages}
\shadedsolutions

\renewcommand{\thesubpart}{(\arabic{subpart})}
\renewcommand{\subpartlabel}{\thesubpart}

\begin{document}

\begin{center}
    \textbf
    {\Large{COMP2120 Computer Organisation} \\
    \large{24/25 Semester 2} \\
    \large{Assignment 4}}
\end{center}

This assignment is based on the CPU and simulator in Assignment 2.

\begin{figure}[H]
    \centering
    \includegraphics[width=0.6\textwidth]{figure.pdf}
    \caption{A simplified CPU}
\end{figure}

In this assignment, extra instructions are added. They are the \texttt{PUSH}, \texttt{POP}, \texttt{CALL}
and \texttt{RET} instruction. In order to implement these instructions, the CPU is modified as
follows:

\begin{enumerate}
    \item A new register (\texttt{SP}, the stack pointer) is included. \texttt{SP} provides output to S1-bus,
        and receives input from D-bus. Also, the \texttt{SP} has special hardware to increase and
        decrease its value by 4 (similar to \texttt{PC}). This is provided by the special function
        \texttt{do\_incSP()}, and \texttt{do\_decSP()}, which is in turn controlled by the flag \texttt{incSP} and
        \texttt{decSP}.
    \item A new register (\texttt{TEMP}) is included, which is directly connected to the \texttt{MAR} only,
        via a dedicated data path. Again you can move data between \texttt{MAR} and \texttt{TEMP} and
        special function \texttt{do\_MAR\_to\_TEMP()} and \texttt{do\_TEMP\_to\_MAR()} are provided, which are
        controlled by the \texttt{MAR\_to\_TEMP} and \texttt{TEMP\_to\_MAR} flag.
    \item A new flag \texttt{push\_pop} is included, which will move the \texttt{SP} to \texttt{MAR}. Otherwise, the
        CPU remains the same.
\end{enumerate}

New instructions provided include:
\begin{center}
    \begin{tabular}{cccc}
        \multicolumn{4}{l}{\texttt{PUSH Rn : SP $\leftarrow$ SP-4; mem[SP] $\leftarrow$ Rn}} \\ \hline
        \multicolumn{1}{|c|}{\texttt{00001010}} & \multicolumn{1}{c|}{\texttt{n}} & \multicolumn{1}{c|}{\texttt{00000000}} & \multicolumn{1}{c|}{\texttt{00000000}}\\ \hline
            &     &     &    \\
        \multicolumn{4}{l}{\texttt{POP Rn : Rn $\leftarrow$ mem[SP]; SP $\leftarrow$ SP+4}} \\ \hline
        \multicolumn{1}{|c|}{\texttt{00001011}} & \multicolumn{1}{c|}{\texttt{00000000}} & \multicolumn{1}{c|}{\texttt{00000000}} & \multicolumn{1}{c|}{\texttt{n}}\\ \hline
            &     &     &    \\
        \multicolumn{4}{l}{\texttt{CALL proc :}} \\ \hline
        \multicolumn{1}{|c|}{\texttt{00001100}} & \multicolumn{1}{c|}{\texttt{00000000}} & \multicolumn{1}{c|}{\texttt{11111111}} & \multicolumn{1}{c|}{\texttt{00000000}}\\ \hline
            &     &     &    \\
        \multicolumn{4}{l}{\texttt{RET :}} \\ \hline
        \multicolumn{1}{|c|}{\texttt{00001101}} & \multicolumn{1}{c|}{\texttt{00000000}} & \multicolumn{1}{c|}{\texttt{00000000}} & \multicolumn{1}{c|}{\texttt{00000000}}\\ \hline 
        \end{tabular}
\end{center}

Summary Opcode:
\begin{center}
    \begin{tabular}{|c|c|c|c|c|c|}
        \hline
        Instruction & Opcode & Instruction & Opcode & Instruction & Opcode \\ \hline
        \texttt{ADD} & \texttt{00000000} & \texttt{MOV} & \texttt{00000101} & \texttt{PUSH} & \texttt{00001010} \\ \hline
        \texttt{SUB} & \texttt{00000001} & \texttt{LD} & \texttt{00000110} & \texttt{POP} & \texttt{00001011} \\ \hline
        \texttt{NOT} & \texttt{00000010} & \texttt{ST} & \texttt{00000111} & \texttt{CALL} & \texttt{00001100} \\ \hline
        \texttt{AND} & \texttt{00000011} & \texttt{Bcc} & \texttt{00001000} & \texttt{RET} & \texttt{00001101} \\ \hline
        \texttt{OR} & \texttt{00000100} & \texttt{HLT} & \texttt{00001001} &  &  \\ \hline
    \end{tabular}
\end{center}

\subsection*{The Program}

The revised simulator program is given in sim2.py. Study the simulator code carefully.

\begin{questions}

    \question 
    
\end{questions}

\end{document}