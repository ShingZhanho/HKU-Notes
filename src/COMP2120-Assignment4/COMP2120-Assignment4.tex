% !TEX program = latexmk
% !TEX options = --shell-escape -synctex=1 -interaction=nonstopmode -file-line-error -pdf -cd -outdir=. ./COMP2120-Assignment4.tex

\documentclass[answers]{exam}
\usepackage[english]{babel}
\usepackage[style=apa,backend=biber]{biblatex}
\usepackage[
    format=plain,
    labelsep=newline,
    font=small,
    labelfont=bf,
    textfont=it,
    justification=justified,
    singlelinecheck=off
    ]{caption}
\usepackage{csquotes}
\usepackage{float}
\usepackage[top=2.5cm,bottom=2.5cm,left=3cm,right=3cm]{geometry}
\usepackage{graphicx}
\usepackage[hidelinks,colorlinks=true,linkcolor=blue,filecolor=blue,urlcolor=blue,citecolor=blue]{hyperref}
    \usepackage[nameinlink]{cleveref} %nameinlink ensures that the entire element is clickable in the pdf, not just the number
\usepackage{indentfirst}
\usepackage{multicol}
\usepackage[skip=1em,indent]{parskip}
\usepackage{tabularx}
\usepackage{times}
\pagestyle{foot}
\cfoot{Page \thepage\ of \numpages}
\shadedsolutions

\renewcommand{\thesubpart}{(\arabic{subpart})}
\renewcommand{\subpartlabel}{\thesubpart}

\begin{document}

\begin{center}
    \textbf
    {\Large{COMP2120 Computer Organisation} \\
    \large{24/25 Semester 2} \\
    \large{Assignment 4}}
\end{center}

This assignment is based on the CPU and simulator in Assignment 2.

\begin{figure}[H]
    \centering
    \includegraphics[width=0.6\textwidth]{figure.pdf}
    \caption{A simplified CPU}
\end{figure}

In this assignment, extra instructions are added. They are the \texttt{PUSH}, \texttt{POP}, \texttt{CALL}
and \texttt{RET} instruction. In order to implement these instructions, the CPU is modified as
follows:

\begin{enumerate}
    \item A new register (\texttt{SP}, the stack pointer) is included. \texttt{SP} provides output to S1-bus,
        and receives input from D-bus. Also, the \texttt{SP} has special hardware to increase and
        decrease its value by 4 (similar to \texttt{PC}). This is provided by the special function
        \texttt{do\_incSP()}, and \texttt{do\_decSP()}, which is in turn controlled by the flag \texttt{incSP} and
        \texttt{decSP}.
    \item A new register (\texttt{TEMP}) is included, which is directly connected to the \texttt{MAR} only,
        via a dedicated data path. Again you can move data between \texttt{MAR} and \texttt{TEMP} and
        special function \texttt{do\_MAR\_to\_TEMP()} and \texttt{do\_TEMP\_to\_MAR()} are provided, which are
        controlled by the \texttt{MAR\_to\_TEMP} and \texttt{TEMP\_to\_MAR} flag.
    \item A new flag \texttt{push\_pop} is included, which will move the \texttt{SP} to \texttt{MAR}. Otherwise, the
        CPU remains the same.
\end{enumerate}

New instructions provided include:
\begin{center}
    \begin{tabular}{cccc}
        \multicolumn{4}{l}{\texttt{PUSH Rn : SP $\leftarrow$ SP-4; mem[SP] $\leftarrow$ Rn}} \\ \hline
        \multicolumn{1}{|c|}{\texttt{00001010}} & \multicolumn{1}{c|}{\texttt{n}} & \multicolumn{1}{c|}{\texttt{00000000}} & \multicolumn{1}{c|}{\texttt{00000000}}\\ \hline
            &     &     &    \\
        \multicolumn{4}{l}{\texttt{POP Rn : Rn $\leftarrow$ mem[SP]; SP $\leftarrow$ SP+4}} \\ \hline
        \multicolumn{1}{|c|}{\texttt{00001011}} & \multicolumn{1}{c|}{\texttt{00000000}} & \multicolumn{1}{c|}{\texttt{00000000}} & \multicolumn{1}{c|}{\texttt{n}}\\ \hline
            &     &     &    \\
        \multicolumn{4}{l}{\texttt{CALL proc :}} \\ \hline
        \multicolumn{1}{|c|}{\texttt{00001100}} & \multicolumn{1}{c|}{\texttt{00000000}} & \multicolumn{1}{c|}{\texttt{11111111}} & \multicolumn{1}{c|}{\texttt{00000000}}\\ \hline
            &     &     &    \\
        \multicolumn{4}{l}{\texttt{RET :}} \\ \hline
        \multicolumn{1}{|c|}{\texttt{00001101}} & \multicolumn{1}{c|}{\texttt{00000000}} & \multicolumn{1}{c|}{\texttt{00000000}} & \multicolumn{1}{c|}{\texttt{00000000}}\\ \hline 
        \end{tabular}
\end{center}

Summary Opcode:
\begin{center}
    \begin{tabular}{|c|c|c|c|c|c|}
        \hline
        Instruction & Opcode & Instruction & Opcode & Instruction & Opcode \\ \hline
        \texttt{ADD} & \texttt{00000000} & \texttt{MOV} & \texttt{00000101} & \texttt{PUSH} & \texttt{00001010} \\ \hline
        \texttt{SUB} & \texttt{00000001} & \texttt{LD} & \texttt{00000110} & \texttt{POP} & \texttt{00001011} \\ \hline
        \texttt{NOT} & \texttt{00000010} & \texttt{ST} & \texttt{00000111} & \texttt{CALL} & \texttt{00001100} \\ \hline
        \texttt{AND} & \texttt{00000011} & \texttt{Bcc} & \texttt{00001000} & \texttt{RET} & \texttt{00001101} \\ \hline
        \texttt{OR} & \texttt{00000100} & \texttt{HLT} & \texttt{00001001} &  &  \\ \hline
    \end{tabular}
\end{center}

\subsection*{The Program}

The revised simulator program is given in \texttt{sim2.py}. Study the simulator code carefully.

\begin{questions}

    \question Hand assemble the following assembly code and put it in a program file. Run the
        simulator on this program. Explain what the function \texttt{SQ} does?

    \begin{verbatim}
        SUB     R4,R4,R4                    0000H:  01040404
        LD      P1,R1                       0004H:  0600ff01 00000078
        MOV     R1,R2                       000CH:  05010002
        LD      P2,R3                       0010H:  0600ff03 0000007c
    L:  MOV     R1,R10                      0018H:  0501000a
        CALL    SQ                          001CH:  0c00ff00 00000044
        ADD     R4,R11,R4                   0024H:  00040b04
        ADD     R1,R2,R1                    0028H:  00010201
        SUB     R3,R1,R5                    002CH:  01030105
        BNZ     L                           0030H:  0802ff00 00000018
        ST      R4,P                        0038H:  0704ff00 00000080
        HLT                                 0040H:  09000000

    /*  Procedure to calculate <????????>, input is R10, output is R11  */
    /*  The proc uses R12 and R13, need to save them on entry           */
    /*  and restore them on exit                                        */

    SQ: PUSH    R12                         0044H:  ....
        PUSH    R13                         0048H:  ....
        LD      P1,R13                      004CH:  ....
        SUB     R13,R13,R13                 0054H:
        MOV     R10,R12                     0058H:
    L2: ADD     R11,R10,R11                 005CH:
        SUB     R12,R13,R12                 0060H:
        BNZ     L2                          0064H:
        POP     R13                         006CH:
        POP     R12                         0070H:
        RET                                 0074H:
    P1  .WORD   1                           0078H:  00000001
    P2  .WORD   A                           007CH:  0000000a
    P   .WORD                               0080H:  00000000
    \end{verbatim}

    \begin{solution}
        \texttt{SQ} reads register \texttt{R10} as input, and calculate the square of the value of \texttt{R10}, and set the result at register \texttt{R11} as output.

        The complete assembled code is as follows:
        \begin{verbatim}
        01040404
        0600ff01
        00000078
        05010002
        0600ff03
        0000007c
        0501000a
        0c00ff00
        00000044
        00040b04
        00010201
        01030105
        0802ff00
        00000018
        0704ff00
        00000080
        09000000
        0a0c0000        ; SQ: PUSH R12
        0a0d0000        ;     PUSH R13
        0600ff0d        ; ...
        00000078        ;
        010b0b0b        ;
        050a000c        ;
        000b0a0b        ;
        010c0d0c        ;
        0802ff00        ;
        0000005c        ;
        0b00000d        ; ...
        0b00000c        ;     POP R12
        0d000000        ; RET
        00000001
        0000000a
        00000000
        \end{verbatim}
    \end{solution}

    \question Run the simulator in debug mode. Write down the data transfer/transformation
        sequences involved in the execution of the instructions \texttt{CALL} and \texttt{RET}.
        You may skip intermediate step provided by the simulator, for example the in struction fetchs step should look like:
        \begin{verbatim}
            MAR <- PC
            IR <- mem[MAR]
        \end{verbatim}
        or in English, move the value of \texttt{PC} to \texttt{MAR}. Then read memory and the result
        (\texttt{mem[MAR]}) is moved to \texttt{IR}, i.e. just write down the source and destination of the
        data movement, without the paths etc.

    \begin{solution}
        \textit{The comments in the code are optional and are not required for the submission.}
        
        For \texttt{CALL} instruction:
        \begin{verbatim}
            MAR <- PC           ; instruction fetch
            IR <- mem[MAR]
            MAR <- PC           ; address of next word
            MBR <- mem[MAR]     ; reads address of called instruction
            MAR <- MBR
            PC <- PC + 4        ; point to the instruction after return
            TEMP <- MAR         ; put called instruction address in TEMP
            SP <- SP – 4        ; decrement stack ptr for pushing
            MAR <- SP           ; get address for pushing to top of stack
            MBR <- PC           ; put return target in MBR
            mem[SP] <- MBR      ; push return target to stack
            MAR <- TEMP         ; retrieve called instruction address
            PC <- MAR           ; point PC to called instruction
        \end{verbatim}

        For \texttt{RET} instruction:
        \begin{verbatim}
            MAR <- PC           ; instruction fetch
            IR <- mem[MAR]	
            MAR <- SP           ; get address of top of stack
            SP <- SP + 4        ; pop (increment ptr)
            MBR <- mem[MAR]     ; pop top of stack to MBR (return address)
            PC <- MBR           ; point PC to where the routine was suspended
        \end{verbatim}
    \end{solution}

    \question Modify the program so that it will calculate the value of $1-2+3-4+\cdots-8+9$.
        That is,
        \begin{verbatim}
            sum = 0;
            for i=1 to 9 do sum += sq(i)
        \end{verbatim}
        Where \texttt{sq(i)} return \texttt{i} when \texttt{i} is odd, otherwise return \texttt{-i}. Note that the original
        program is already a loop from 1 to 9. Just replace the function \texttt{SQ} by
        \begin{verbatim}
            if (R10 is odd) R11 = R10;
            else R11 = 0 - R10;
        \end{verbatim}
        Since we don’t have a \texttt{NEG} instruction, to find $x$, we use $0-x$.
        To check if a number $x$ is odd, just check if the rightmost bit is 1. We can find $x$
        \texttt{AND 00000000...0001}. (i.e. 1) After \texttt{AND} operation, all bits \texttt{AND}ed with 0 will be
        0. If the rightmost bit is 0, then the result is 0. Otherwise the result is non-zero.
        Note that the address of P1, P2 and P may got changed when the length of the
        function \texttt{SQ} is changed. You may need to change the address of them in the
        program, e.g. in line 2
        \begin{verbatim}
            LD P1,R1
        \end{verbatim}
        you may need to find the new address of \texttt{P1}, and also in line 4...

        \begin{solution}
            The modified assembly code is as follows:
            \begin{verbatim}
                SUB    R4,R4,R4
                LD     P1,R1
                MOV    R1,R2
                LD     P2,R3
            L:  MOV    R1,R10
                CALL   SQ
                ADD    R4,R11,R4
                ADD    R1,R2,R1
                SUB    R3,R1,R5
                BNZ    L
                ST     R4,P
                HLT
            SQ: PUSH   R12
                PUSH   R13
                LD     P1,R13       ; set R13 = 1 for odd/even check
                SUB    R12,R12,R12  ; set R12 = 0 for negation
                AND    R10,R13,R11  ; store result at R11
                BNZ    OD           ; if not 0 -> odd -> jump to OD
                SUB    R12,R10,R11  ; input is even, negate it and put in R11
                BR     RT           ; jump to RT to prepare for return
            OD: MOV    R10,R11      ; if odd, copy R10 to R11 diretly
            RT: POP    R13
                POP    R12
                RET
            P1: .WORD  1
            P2: .WORD  A
            P:  .WORD
            \end{verbatim}
            It's assembled hexadecimal code is as follows:
            \begin{verbatim}
            01040404
            0600ff01
            00000080
            05010002
            0600ff03
            00000084
            0501000a
            0c00ff00
            00000044
            00040b04
            00010201
            01030105
            0802ff00
            00000018
            0704ff00
            00000088
            09000000
            0a0c0000
            0a0d0000
            0600ff0d
            00000080
            010c0c0c
            030a0d0b
            0802ff00
            00000070
            010c0a0b
            0800ff00
            00000074
            050a000b
            0b00000d
            0b00000c
            0d000000
            00000001
            0000000a
            00000000
            \end{verbatim}
        \end{solution}
    
\end{questions}

\end{document}