\section{Input \& Output}

\emph{I/O Module}: interface between processor and memory via system bus or central
switch; interface to peripheral devices through dedicated data links.\\
\emph{Model Requirements}: \begin{enuminline}
\item asynchronous timing
\item command decoding (\eg\texttt{SEEK})
\item data exchange
\item status reporting (\eg ready, busy, etc.)
\item address recognition
\item data buffering (speed up transactions)
\item error detection \& correction
\end{enuminline}

\subsection*{I/O Register Mapping}

\emph{Memory-mapped}: registers mapped into main memory address space; accessed as if
memory locations;\\
\emph{I/O-mapped}: mapped into separate address space; accessed via special instructions.

\subsection*{I/O Techniques}

\emph{Programmed I/O}:
Not using interrupts. CPU waits for I/O device to complete operation. CPU accesses device
via Control and Status Registers (CSR). Wastes CPU time.\\
\emph{Interrupt-driven I/O}:
($\text{Memory} \leftrightarrow \text{CPU} \leftrightarrow \text{I/O}$) CPU executes
other instructions after sending I/O command. I/O interrupts CPU when complete. CPU
sends acknowledgment (\texttt{INTA}) to I/O.
Interruptions handled \textbf{between} instruction cycles.\\
\emph{Direct Memory Access (DMA)}:
($\text{Memory} \leftrightarrow \text{I/O}$) Use Input-Output Processor (IOP). IOP
steals cycles from CPU. CPU sees elongated cycle and wait until cycle is over.
Interruptions handled \textbf{within} one instruction cycle.