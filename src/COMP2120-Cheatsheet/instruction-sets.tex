\section{Instruction Sets}

\emph{Arithmetic Operations}: treat operands as numbers; consider signs;
(\eg arithmetic shift = multiplication/division by 2, sign bit preserved).\\
\emph{Logical Operations}: treat operands as bit patterns; discard bits shifted out;
replenish new bits with 0.\\
\emph{Rotate Operations}: put bits shifted out back into the other end of the number;
are logical operations.\\

\subsection*{Instruction Operands}
Op\# \hfill Symbolic \hfill Interpretation\\
3 \hspace*{0.2\linewidth} \texttt{OP A, B, C} \hfill \texttt{A$\gets$B OP C}\\
2 \hspace*{0.2\linewidth} \texttt{OP A, B} \hfill \texttt{A$\gets$A OP B}\\
1 \hspace*{0.2\linewidth} \texttt{OP A} \hfill \texttt{AC$\gets$AC OP A}\\
0 \hspace*{0.2\linewidth} \texttt{OP} \hfill \texttt{T$\gets$(T-1) OP T}\\

\subsection*{Registers}
\emph{General Purpose Registers}: can be used for whatever reason\\
\emph{Dedicated Purpose Registers}: have a specific purpose (\eg PC, IR, SP, processor status word - PSW, flag)

\subsection*{Data Types}

\subsubsection*{Basic Data Types}
Typical lengths: 8, 16, 32, 64 bits\\
\emph{Numeric}: integer, floating point;\\
\emph{Non-numeric}: character, binary data;

\subsubsection*{MIPS Architecture}
(family of RISC, not ARM nor x86)\\
9 basic types: \begin{enuminline}
    \item (un)/signed bytes;
    \item (un)/signed half-words;
    \item (un)/signed words;
    \item double-words;
    \item single-precision floating point (32 bits);
    \item double-precision floating point (64 bits);
\end{enuminline}

\subsubsection*{ARM Architecture}
Supported lengths: \begin{enuminline}
    \item byte (8 bits);
    \item half-word (16 bits);
    \item word (32 bits);
\end{enuminline}\\
Only unsigned integers, nonnegative integers, and 2's comp integers.\\
No floating point by hardware, must be emulated.

\subsection*{Addressing Modes}
\emph{Immediate} (\texttt{OP = A}): operand is value; \textbf{Pros}: no memory reference; \textbf{Cons}: small operand magnitude\\
\emph{Direct} (\texttt{EA = A}): operand is address; \textbf{Pros}: fast \& increased magnitude; \textbf{Cons}: limited address space\\
\emph{Indirect} (\texttt{EA = (A)}): operand is address of address; \textbf{Pros}: large address space; \textbf{Cons}: multiple memory references\\
\emph{Register} (\texttt{EA = R}): operand points to register; \textbf{Pros}: fast; \textbf{Cons}: limited \# of registers (\eg 32 in MIPS)\\
\emph{Register Indirect} (\texttt{EA = (R)}): operand points to register, register has address; \textbf{Pros \& Cons}: same as indirect\\
\emph{Displacement} (\texttt{EA = A + (R)}): address is base address + offset; \textbf{Pros}: flexible; \textbf{Cons}: complex; Usage: local vars, arrays; Registers: PC, SP, base pointer register\\
\emph{Stack} (\texttt{EA = Top of Stack}): implicit; \textbf{Pros}: no memory references; \textbf{Cons}: limited applicability; Usage: \texttt{PUSH} and \texttt{POP}; Register: SP

\subsection*{Assembly Language}
\emph{Syntax}\\
\texttt{[LABEL:] OP\_NAME [OP\_1, OP\_2, ...] [\# COMMENT]}\\
\emph{Assembler Directives}\\
\texttt{.data} \hfill data segment\\
\texttt{.text} \hfill program segment\\
\texttt{.global NAME} \hfill introduce to other files\\
\texttt{.reserve EXPR} \hfill reserve space with 0\\
\texttt{.word VAL1[, ...]} \hfill write to memory\\

\subsection*{OS Support}
\emph{OS Services}: \begin{enuminline}
    \item Program creation (compilers), execution (loading, managing resources),
    \item I/O access (provide uniform interface, implementation hidden),
    \item File system management,
    \item System access (user/kernel mode),
    \item Error detection \& response (BSOD),
    \item Accounting (collect usage stats)
\end{enuminline} \\
\emph{System Calls}: special entry points to execute OS functions via kernel model
(executed by OS on behalf of user program)\\
\emph{Scheduling \& Time Sharing}: CPU time divided into time slices, each process gets a slice,
when used up, process is suspended and another process is scheduled.
OS maintains a priority queue, depends on waiting time, system load, CPU-bound etc.

\subsection*{Processor Pipelining}
Ideal throughput = $1 \text{ CPI}$.

\subsubsection*{Pipeline Hazards}
\emph{Resource}: multiple instructions need the same resource (PC, ALU, registers).
\textbf{Solution}: add more resources\\
\emph{Data}: instruction depends on result of previous instruction -- \begin{enuminline}
    \item \textbf{RAW}: occurs if read happens before write when it should be RAW;
    \item \textbf{WAR}: opposite of RAW, not occurs in pipeline but in parallel systems;
    \item \textbf{WAW}: occurs if 2nd write happens before 1st, not in pipeline but in parallel systems.
\end{enuminline}
\textbf{Solutions}: \begin{enuminline}
    \item \textbf{Stalling}: wait;
    \item \textbf{Data forwarding}: use result of previous instruction;
    \item \textbf{Rearrange instructions}: separate dependent instructions, not always possible;
\end{enuminline}\\
\emph{Control}: branch not resolved. \textbf{Unsolvable}. Mitigated by branch prediction.

\subsubsection*{Branch Prediction}
\emph{Static}: always predict taken/not taken.
\textbf{Pros}: simple, 50\% accuracy, higher in \texttt{for} loops.
\textbf{Cons}: possible high page faults.\\
\emph{Dynamic}: Use history to help predict. \begin{enuminline}
    \item \textbf{1-bit}: If wrong prediction, predict opposite; Problem: high misprediction rate at the end of loops;
    \item \textbf{2-bit}: If two consecutive wrong predictions, predict opposite.
\end{enuminline}

\subsubsection*{RISC Architecture}
\emph{Characteristics}: \begin{enuminline}
    \item Load/store architecture: only load/store instructions access memory;
    \item Fixed-length, simple, fixed-format instructions $\Rightarrow$ high clock rate, low clock cycle time;
    \item Fewer addressing modes $\Rightarrow$ low CPI;
    \item More instructions (reduction in CPI is more significant than increase in instruction count);
    \item Extensive soft/hardware pipelining;
    \item Relies on compiler optimisation.
\end{enuminline}