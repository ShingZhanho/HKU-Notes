\section{C++ Basics}

\subsection*{\texttt{<cstdlib> (stdlib.h)} and \texttt{<ctime> (time.h)}}

\begin{minted}[style=manni]{cpp}
srand(std::time(nullptr)); // seed generator
rand()                     // [0, RAND_MAX]
rand() % (b - a + 1)       // [0, b - a]
rand() % (b - a + 1) + a   // [a, b]
\end{minted}
\vspace*{-1.5\baselineskip}

\subsection*{\texttt{<cmath> (math.h)}}

\begin{minted}[style=manni]{cpp}
double sqrt(double x);
double pow(double x, double y); // x^y
double fabs(double x);          // absolute value
double ceil(double x);          // round up
double floor(double x);         // round down
\end{minted}
\vspace*{-1.25\baselineskip}

\subsection*{C Arrays}

\begin{minted}[style=manni]{cpp}
T arr[n];       // array of n elements
T arr[] = {1, 2, 3}; // array of 3 elements
T arr[3][2];    // arr of 3 arrays of 2 elements
arr[0];         // access an element of an array
/* == Take array as an argument == */
ret_type func_name(T arr[], int size);
ret_type func_name(T arr[][2], int size);
/* == Linear Search == */
int linSearch(T arr[], int size, T find) {
    for (int i = 0; i < size; ++i)
        if (arr[i] == find)
            return i;
} // avg O(n) (half), worst O(n) (all)
// add int startPos = 0; to search from startPos
/* == Selection Sort == */
void selSortAsc(T arr[], int size) {
    for (int i = 0; i < size - 1; ++i) {
        int minIndex = i;
        for (int j = i + 1; j < size; ++j) {
            if (arr[j] < arr[minIndex])
                minIndex = j;
        }
        if (minIndex == i) continue;
        T temp = arr[i];
        arr[i] = arr[minIndex];
        arr[minIndex] = temp;
    }
}
\end{minted}
\vspace*{-1.25\baselineskip}

\subsection*{\texttt{char} and \texttt{<cctype> (ctype.h)}}

\begin{minted}[style=manni]{cpp}
int isdigit(int c); // nonzero if is digit
int isalpha(int c); // nonzero if is alphabet
int isalnum(int c); // nonzero if is digit or alpha
int islower(int c); // nonzero if is lower case
int isupper(int c); // nonzero if is upper case
int tolower(int c); // ret c as lower if is upper
int toupper(int c); // ret c as upper if is lower
\end{minted}
\vspace*{-1.25\baselineskip}

\subsection*{C++ Strings and \texttt{<string>}}

\begin{minted}[style=manni]{cpp}
std::string str = "HKU ENGG1340 CompSc";
char c = str[5];                // 'N'
str[10] = '3';          // "HKU ENGG1330 CompSc"
str = str + " is fun";          // Ok
str = "Hello" + " World";       // Error
/* == Functions == */
// read line from istream into string until delim
std::getline(istream& in, string& s, char delim);
string::length(); // length of string
string::empty(); // true if empty
string::substr(int pos, int len);
// find first occurrence of str from pos, miss>npos
string::find(string str, int pos = 0);
// find last occurrence of str from pos, miss>npos
string::rfind(string str, int pos = npos);
// insert str before the character at pos, in place
string::insert(int pos, string str);
// erase len characters from pos, in place
string::erase(int pos, int len);
// replace len chars from pos with str, in place
string::replace(int pos, int len, string str);
\end{minted}
\vspace*{-1.25\baselineskip}

\subsection*{File I/O and \texttt{<fstream>}}

\begin{minted}[style=manni]{cpp}
std::ifstream fin("input.txt");
std::ofstream fout("output.txt");
fout.open("output.txt", std::ios::app); // append
fout.open(path_str.c_str());  // only support C-strs
fout.close();                 // close file
bool success = !fin.fail();   // or in.is_open();
fout << "Hi!" << std::endl;   // write to file
fin >> str;         // read one word from file
while (fin >> str);           // read until EOF
\end{minted}

\subsection*{\texttt{<sstream>}}

\begin{minted}[style=manni]{cpp}
std::istringstream iss("1 2 3");
iss >> a >> b >> c; // read from string
\end{minted}
\vspace*{-1.25\baselineskip}

\subsection*{Output Formatting by \texttt{<iostream>} and \texttt{<iomanip>}}

Default floating-point: 6 sig. fig., or scientific if too large/small.
\vspace*{-1.25\baselineskip}
\begin{minted}[style=manni]{cpp}
/* manipulators from <iostream> */
std::showpoint; // always show point
      // # of digits depends on std::setprecision
std::noshowpoint; // unset std::showpoint
std::fixed; // fixed point notation (3.14)
std::scientific; // scientific notation (3.14e+00)
std::cout.unsetf(std::ios_base::floatfield);
          // unset std::fixed and std::scientific
std::left;  // left align
std::right; // right align (default with std::setw)
/* manipulators from <iomanip> */
std::setprecision(2);
          // with std::fixed or std::scientific:
          // 3.14159 -> 3.14 or 3.14e+00
          // otherwise: 3.14159 -> 3.1
    // no padding 0 without std::showpoint if
    // not std::fixed or std::scientific
std::setw(10); // set width to 10
std::setfill('*'); // fill with * (char)

\end{minted}
\vspace*{-1.25\baselineskip}

\subsection*{Pointers and Dynamic Memory}
\begin{minted}[style=manni]{cpp}
int val = 5;
int* p = &val;      // p points to val
std::cout << *p;    // output: 5
std::cout << p;     // output: address of val
std::cout << &p;    // output: address of p
p = nullptr;        // p points to nothing
std::cout << *p;    // SIGSEGV

int* p = new int;   // allocate memory for int
delete p;           // free memory
int* p = new int[10]; // allocate memory for array
delete [] p;        // free memory

*p++;               // increments POINTER
(*p)++;             // increments VALUE

void swapByRef(int& a, int& b) {
    int temp = a; a = b; b = temp;
}
void swapByPtr(int* a, int* b) {
    int temp = *a; *a = *b; *b = temp;
}
\end{minted}
\vspace*{-1.25\baselineskip}

\subsection*{Linked List}

\begin{minted}[style=manni]{cpp}
struct Node { int data; Node* next; };
Node* head = nullptr;       // empty list
Node* n = new Node;         // create a new node
n->data = 10; n->next = nullptr; // initialize
head = n;                   // insert first node

Node* current = head;
while (current != nullptr) { // traverse the list
    doSomething(); current = current->next;
}

void insertAtFront(Node*& head, Node* n) {
    n->next = head; head = n;
}

void insertAtEnd(Node*& head, Node* n) {
    Node* current = head;
    if (current == nullptr) {
        head = n; return;
    }
    while (current->next != nullptr)
        current = current->next;
    current->next = n; n -> next = nullptr;
}

void deleteWithVal(Node*& head, int val) {
    Node* current = head, * prev = nullptr;
    while (current != nullptr) {
        if (current->data == val) {
            if (prev == nullptr)
                head = current->next;
            else
                prev->next = current->next;
            delete current;
            return;
        }
        prev = current; current = current->next;
    }
}

void reverseList(Node*& head) {
    Node *current = head, *prev = nullptr, *next;
    while (current != nullptr) {
        next = current->next;
        current->next = prev;
        prev = current;
        current = next;
    }
}

void deleteList(Node*& head) {
    Node* current = head, * next;
    while (current != nullptr) {
        next = current->next;
        delete current;
        current = next;
    }
}
\end{minted}
\vspace*{-1.25\baselineskip}

\subsection*{\texttt{struct} in C++}
\begin{minted}[style=manni]{cpp}
struct Point {
    int x, y, z; // member variables
    Point(int x, int y, int z) // constructor
        : x(x), y(y), z(z) { }
    double distance(const Point& p) {
        return sqrt(pow(x - p.x, 2) +
                    pow(y - p.y, 2) +
                    pow(z - p.z, 2));
    } // member function
    bool operator<(const Point& p) {
        return x < p.x
            || (x == p.x && y < p.y)
            || (x == p.x && y == p.y && z < p.z);
    } // operator overloading
};

Point p1(1, 2, 3), p2(4, 5, 6);
std::cout << "P1.X = " << p1.x << std::endl;
double dist = p1.distance(p2);
\end{minted}

\subsection*{STL Iterators}

Iterator templates defined in \texttt{<iterator>}, but STL containers'
iterators are defined in their own headers.

\subsubsection*{Iterator Types and Supported Operations}

\emph{Forward iterators} supports:\\
\hfill assignment, increment, dereference, equality;\\
\emph{Bidirectional iterators} supports extra:\\
\hfill decrement;\\
\emph{Random access iterators} supports extra:\\
\hfill addition, subtraction, inequality,\\
\hfill compound (it+=1), offset dereference (it[3])\\

\subsection*{STL Containers}

\subsubsection*{\texttt{<vector>}}

\begin{minted}[style=manni]{cpp}
std::vector<int> v;  // declaration
/* --- Main Member Functions --- */
v.push_back(1); // append to the end, O(1)
v.pop_back();   // remove last element, O(1)
v[i];           // access element at i, O(1)
v.size();       // size of vector, O(1)
/* --- Traverse with iterators --- */
vector<T>::iterator it; // random access iterator
for (it = v.begin(); it != v.end(); ++it) {
    std::cout << *it << ", ";
}
\end{minted}
\vspace*{-1.25\baselineskip}

\subsubsection*{\texttt{<list>}}

\begin{minted}[style=manni]{cpp}
std::list<int> l;  // declaration
/* --- Main Member Functions --- */
l.push_back(1);   // append to the end, O(1)
l.push_front(2);  // append to the front, O(1)
l.pop_back();     // remove last element, O(1)
l.pop_front();    // remove first element, O(1)
l.size();         // size of list, O(1)
l.front();        // first element, O(1)
l.back();         // last element, O(1)
/* --- Traverse with iterators --- */
list<T>::iterator it; // bidirectional iterator
// same as vector
\end{minted}
\vspace*{-1.25\baselineskip}

\subsubsection*{\texttt{<map>}}

\begin{minted}[style=manni]{cpp}
std::map<int, int> m;              // declaration
m = {{key1, val1}, {key2, val2}};  // initialise
/* --- Main Member Functions --- */
m[key];     // access value with key, O(log n)
            // create if not exist
m[key] = 1; // set value with key, O(log n)
m.count(key); // 1 if key exist, 0 otherwise
m.erase(key); // remove key, O(log n)
m.size();   // size of map, O(1)
/* --- Traverse with iterators --- */
map<K, V>::iterator it; // bidirectional iterator
for (it = m.begin(); it != m.end(); ++it) {
    std::cout << it->first << ": ";
    std::cout << it->second << ", ";
}
/* --- Use with Custom Structs S as key --- */
// S must have operator< defined
\end{minted}
\vspace*{-1.25\baselineskip}

\subsection*{STL Algorithms and \texttt{<algorithm>}}

\begin{minted}[style=manni]{cpp}
std::sort(RandomAccessIterator first,
          RandomAccessIterator last);
        // sort in range [first, last), in place
        // O(n log n) on average
std::sort(it, it + n); // sort n elements
std::sort(v.begin(), v.end()); // sort vector
    // note that the last element is not included
\end{minted}
\vspace*{-1.25\baselineskip}

\subsection*{Recursion (Binary Search)}
\begin{minted}[style=manni]{cpp}
int binSearch(int arr[], int l, int r, int x) {
    if (r >= 1) (
        int mid = l + (r - l) / 2;
        if (arr[mid] == x) return mid;
        if (arr[mid] > x)
            return binSearch(arr, l, mid - 1, x);
        return binSearch(arr, mid + 1, r, x);
    )
    return -1; // not found
}
\end{minted}
\vspace*{-1.25\baselineskip}