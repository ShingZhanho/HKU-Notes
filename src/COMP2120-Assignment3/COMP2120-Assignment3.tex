\documentclass[answers]{exam}
\usepackage[english]{babel}
\usepackage[style=apa,backend=biber]{biblatex}
\usepackage[
    format=plain,
    labelsep=newline,
    font=small,
    labelfont=bf,
    textfont=it,
    justification=justified,
    singlelinecheck=off
    ]{caption}
\usepackage{csquotes}
\usepackage{float}
\usepackage[top=2.5cm,bottom=2.5cm,left=3cm,right=3cm]{geometry}
\usepackage{graphicx}
\usepackage[hidelinks,colorlinks=true,linkcolor=blue,filecolor=blue,urlcolor=blue,citecolor=blue]{hyperref}
    \usepackage[nameinlink]{cleveref} %nameinlink ensures that the entire element is clickable in the pdf, not just the number
\usepackage{indentfirst}
\usepackage{multicol}
\usepackage[skip=1em,indent]{parskip}
\usepackage{tabularx}
\usepackage{times}
\pagestyle{foot}
\cfoot{Page \thepage\ of \numpages}
\shadedsolutions

\renewcommand{\thesubpart}{(\arabic{subpart})}
\renewcommand{\subpartlabel}{\thesubpart}

\begin{document}

\begin{center}
    \textbf
    {\Large{COMP2120 Computer Organisation} \\
    \large{24/25 Semester 2} \\
    \large{Assignment 3}}
\end{center}

\begin{questions}

    \question Consider a hypothetical machine with 1K words of cache memory. They are in a two-way set
    associative organisation, with a cache block size of 128 words, using LRU replacement algorithm.
    Suppose the cache hit time is 10 ns, the time to transfer the first word from main memory to cache
    is 60 ns, while subsequent words require 12 ns/word.
    \label{question-2way-set-cache}

    Consider the following read pattern (in blocks of 128 words, and block ID starts from 0),
    and assume each block has 48 references:

    \begin{verbatim}
        1 2 3 5 6 2 3 4 9 10 11 6 3 6 1 7 8 4 5 9 11 1 2 4 5 12 13 14 15 13 14
    \end{verbatim}

    \begin{parts}

        \part What is the cache miss penalty (i.e., time to transfer one block of data from
        main memory to cache memory)?

        \begin{solution}
            \begin{align*}
                \text{Cache miss penalty} 
                & = 60 + 12 \times 127 \\
                & = \boxed{1584 \text{ ns}}
            \end{align*}
        \end{solution}

        \part Write down the content of the cache memory (for all the blocks) at the end of
        the memory references, assuming that the cache is empty at the beginning.

        \begin{solution}
            Content of the cache memory at the end of the memory references:
            \begin{table}[H]
                \centering
                \begin{tabular}{|c|c|c|c|}
                    \hline
                    \textbf{Set 0} & \textbf{Set 1} & \textbf{Set 2} & \textbf{Set 3} \\
                    \hline
                    \texttt{4} & \texttt{13} & \texttt{14} & \texttt{11} \\
                    \hline
                    \texttt{12} & \texttt{5} & \texttt{2} & \texttt{15} \\
                    \hline
                \end{tabular}
            \end{table}
        \end{solution}

        \part Write down the number of cache misses (the first reading of a block is also
        considered a miss), and the cache hit rate.

        \begin{solution}
            Number of cache misses: $\boxed{23}$
    
            Number of blocks accessed = 31

            Number of memory access = $31 \times 48 = 1488$

            Hit rate = $1 - \dfrac{23}{1488} = \boxed{98.45\%}$
        \end{solution}

        \part Calculate the average memory access time.

        \begin{solution}
            Average memory access time = $10 + (1-0.9845) \times 1584 = \boxed{34.55 \text{ ns}}$
        \end{solution}

    \end{parts}

    \question Repeat Question \ref{question-2way-set-cache} for a direct-mapped cache organisation
    with the cache hit time being 9 ns. All other parameters remain the same.

    \begin{solution}
        \begin{parts}

            \part Since the time to transfer the first block and subsequent blocks is the same,
            the cache miss penalty is the same, $\boxed{1584 \text{ ns}}$.

            \part Content of the cache memory at the end of the memory references:
            \begin{table}[H]
                \centering
                \begin{tabular}{|c|c|c|c|c|c|c|c|}
                    \hline
                    \textbf{Line 0} & \textbf{Line 1} & \textbf{Line 2} & \textbf{Line 3} & \textbf{Line 4} & \textbf{Line 5} & \textbf{Line 6} & \textbf{Line 7} \\
                    \hline
                    \texttt{8} & \texttt{1} & \texttt{2} & \texttt{11} & \texttt{12} & \texttt{13} & \texttt{14} & \texttt{15} \\
                    \hline
                \end{tabular}
            \end{table}

            \part Number of cache misses: $\boxed{21}$
    
            Number of blocks accessed = 31
        
            Number of memory access = $31 \times 48 = 1488$
        
            Hit rate = $1 - \dfrac{21}{1488} = \boxed{98.59\%}$

            \part Average memory access time = $9 + (1-0.9859) \times 1584 = \boxed{31.35 \text{ ns}}$

        \end{parts}
    \end{solution}

    \question Consider a Hard Disk with an average seek time of 12 ms and rotation speed of
    7200 rpm, and an average number of 500 sectors per track. Assume negligible
    transfer time.

    \begin{parts}

        \part What is the average rotation latency?

        \begin{solution}
            Average rotation latency is given by:
            \begin{equation*}
                \frac{1\text{ min}}{7200\text{ rotations}} \times \frac{1}{2}\text{ rotation} \times 60\frac{\text{ seconds}}{\text{ minute}} \times 1000 \frac{\text{ ms}}{\text{ second}} = \boxed{4.17 \text{ ms}}
            \end{equation*}
        \end{solution}

        \part What is the average time to rotate for 1 sector?

        \begin{solution}
            Average time to rotate for one sector is given by:
            \begin{equation*}
                \frac{1\text{ min}}{7200\text{ rotations}} \times 60\frac{\text{ seconds}}{\text{ minute}} \times 1000 \frac{\text{ ms}}{\text{ second}} \times \frac{1\text{ rotation}}{500\text{ sectors}} = \boxed{0.0167 \text{ ms per sector}}
            \end{equation*}
        \end{solution}

        \part Consider the access of 5 sectors. Caculate the time required (ignoring tranfer
        time, but including rotation time for reading a sector) if

        \begin{subparts}

            \subpart The sectors are consecutive in the same track.

            \begin{solution}
                Time required to access 5 consecutive sectors = 
                $4.17+12+0.0167 \times 5 = \boxed{16.25 \text{ ms}}$
            \end{solution}

            \subpart The sectors are scattered in various places in the HDD.

            \begin{solution}
                Time required to access 5 scattered sectors =
                $5 \times (4.17+12 + 0.0167) = \boxed{80.93 \text{ ms}}$
            \end{solution}

        \end{subparts}

    \end{parts}
    
\end{questions}

\end{document}