\section{Introduction}

Since its discovery, the Dunhuang Grottoes have been an important site for the study of Buddhist art and
several important aspects of Chinese history, including the trade events via the Silk Road, etc.
While a large quantity of artefacts like the murals and sculptures have survived thousands of years
to be presented in this modern age, the artefacts are facing significant threats from various factors.
A study of \apaciteassub{Jiang2023-Weather-Dunhuang} has shown that the despite the lack of immediate risk
of preservation due to the drastic climate change in the past few decades, the artefacts are still
prone to deterioration due to the weathering effects of the natural environment. Therefore, the
preservation of the Dunhuang treasures is of utmost importance and a race against time
(\apaciteaspar{Yu2022-AI-Dunhuang}). Measures with high efficiency are needed to ensure the survival
of the cultural heritage.

In view of the above, the Dunhuang Research Academy (DHA) was established in 1944 and is now devoted to
applying modern technologies to the preservation of the Dunhuang Grottoes. While one takes advantage of
the efficiency and convenience of modern technologies, such as Artificial Intelligence (AI), one must
also be careful about the potential damages that may be imposed on the cultural heritage.
This essay aims to explore the current applications of the AI technology in the preservation of the 
Buddhist heritage, using the Dunhuang Caves as a case study. In the Literature Review section, some
cutting-edge AI applications in the archaeological field will be briefly explored. Then, in the
following two sections, the benefits and potential risks of these applications will be discussed.
Finally, a conclusion on the topic will be included, with a hope that the discussion would provide new
insights into the process of critically evaluating the use of AI in the preservation of cultural heritage.

\subsection{Terminology}

Since the topics ``cultural heritage'', ``preservation'' and ``AI'' are all broad, it may be helpful if
these terms are defined before the discussion begins.

\subsubsection{Buddhist Heritage and Cultural Heritage}

\subsubsection{Preservation}

\subsubsection{Artificial Intelligence}

\printbibliography
