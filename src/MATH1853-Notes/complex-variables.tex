\section{Complex Variables}

\begin{definition}[Complex Number]
    The set of \textbf{complex numbers} is defined as
    \[
        \C = \set{a+b\iu\mid a,b\in\R, \iu^2=-1}
    \]
    Further, for $z=a+b\iu\in\C$, we refer to $a$ as the \textbf{real part} of $z$, denoted $\Re(z)$,
    and $b$ as the \textbf{imaginary part} of $z$, denoted $\Im(z)$.
\end{definition}

\begin{definition}[Complex Operations]
    $\C$ is a field under the certain operations. For $z=a+b\iu, w=c+d\iu\in\C$,
    we define:
    \begin{table}[H]
        \centering
        \begin{tabular}{lll}
            \textbf{Name} & \textbf{Notation} & \textbf{Definition} \\
            \hline
            Addition & $z+w$ & $(a+c) + (b+d)\iu$ \\
            Subtraction & $z-w$ & $(a-c) + (b-d)\iu$ \\
            Multiplication & $zw$ & $(ac - bd) + (ad + bc)\iu$ \\
            Division & $\displaystyle\frac{z}{w}$, ($w\neq 0$) & $\displaystyle\frac{ac + bd}{c^2 + d^2} + \frac{bc - ad}{c^2 + d^2}\iu$ \\
            Conjugate & $\conj{z}$ & $a - b\iu$\\
            Modulus & $|z|$ & $\sqrt{a^2 + b^2}$
        \end{tabular}
        \caption{Operations on Complex Numbers}
    \end{table}
\end{definition}

\begin{theorem}[Triangle Inequality]
    For any $z,w\in\C$, we have
    \[
        |z+w| \leq |z| + |w|
    \]
    \begin{proof} The proof is as follows:
        \begin{align*}
            |z+w|^2 &= (z+w)\conj{(z+w)} \\
            &= z\conj{z} + w\conj{w} + z\conj{w} + \conj{z}w \\
            &= |z|^2 + |w|^2 + z\conj{w} + \conj{z\conj{w}} \\
            &= |z|^2 + |w|^2 + 2\Re(z\conj{w}) \qquad (\text{by }\Re(z) = \frac{z+\conj{z}}{2}) \\
            &\leq |z|^2 + |w|^2 + 2|z\conj{w}| \qquad (\sqrt{\Re(z)^2+\Im(z)^2}\geq \Re(z)) \\
            &= |z|^2 + |w|^2 + 2|z||w| \\
            &= (|z| + |w|)^2
        \end{align*}
    \end{proof}
\end{theorem}
\pagebreak

\begin{theorem}[Properties of Conjugates and Moduli]
    For any $z,w\in\C$, we have
    \begin{enumerate}
        \item \textbf{properties of conjugates:} \begin{enumerate}
            \item $\conj{(z+w)} = \conj{z} + \conj{w}$
            \item $\conj{(zw)} = \conj{z}\conj{w}$
            \item $\conj{\left(\displaystyle\frac{z}{w}\right)} = \displaystyle\frac{\conj{z}}{\conj{w}}$, ($w\neq 0$)
            \item $\conj{(\conj{z})} = z$ \qquad (\textbf{involution})
        \end{enumerate}
        \item \textbf{properties of moduli:} \begin{enumerate}
            \item $|z| \geq 0$, and $|z|=0$ if and only if $z=0$
            \item $|zw| = |z||w|$
            \item $\displaystyle\left|\frac{z}{w}\right| = \frac{|z|}{|w|}$, ($w\neq 0$)
            \item $|z|^n = |z^n|$
        \end{enumerate}
    \end{enumerate}
\end{theorem}

\begin{definition}[Polar Form of a Complex Number]
    For any $z=a+b\iu\in\C$, we can express $z$ in \textbf{polar form} as
    \[
        z = r(\cos\theta + \iu\sin\theta)
    \]
    where $r=|z|=\sqrt{a^2+b^2}$ is the \textbf{modulus} of $z$,
    and $\theta=\arg(z)=\arctan\frac{b}{a}$ is the \textbf{argument} of $z$.
\end{definition}

\begin{remark}
    In MATH1853, we limit $\theta$ to be the principal argument, i.e.,
    $\arg(z)\in[0,2\pi)\text{ or }(-\pi,\pi]$.
\end{remark}

\begin{definition}[Multiplication of Complex Numbers in Polar Form]
    Let $z_1, z_2\in\C$, where $z_1 = r_1(\cos\theta_1 + \iu\sin\theta_1)$
    and $z_2 = r_2(\cos\theta_2 + \iu\sin\theta_2)$.
    Then, the multiplication of $z_1$ and $z_2$ is given by
    \[
        z_1z_2 = r_1r_2\left[\cos(\theta_1 + \theta_2) + \iu\sin(\theta_1 + \theta_2)\right]
    \]
\end{definition}

\begin{theorem}[De Moivre's Theorem]
    If $z = r(\cos\theta + \iu\sin\theta)\in\C$, $n\in\Z$, $r>0$, then
    \[
        z^n = r^n\left[\cos(n\theta) + \iu\sin(n\theta)\right]
    \]
\end{theorem}

\begin{theorem}[Euler's Formula]
    The formula states:
    \[
        e^{\iu\theta} = \cos\theta + \iu\sin\theta
    \]
\end{theorem}

\begin{corollary}
    The trigonometric functions can be expressed in terms of exponentials as follows:
    \[
        \cos\theta = \frac{e^{\iu\theta} + e^{-\iu\theta}}{2}, \quad
        \sin\theta = \frac{e^{\iu\theta} - e^{-\iu\theta}}{2\iu}
    \]
\end{corollary}

\begin{definition}[Exponential Form of a Complex Number]
    For any $z=a+b\iu\in\C$, $z$ has the \textbf{exponential form}:
    \[
        z = re^{\iu\theta}
    \]
    where $r=|z|=\sqrt{a^2+b^2}$ and $\theta=\arg(z)=\arctan\frac{b}{a}$.
\end{definition}

\begin{theorem}[Properties of the Exponential Form]
    For any $z, w\in\C$, we have the following properties:
    \begin{table}[H]
        \centering
        \begin{tabular}{ll}
            \textbf{Property} & \textbf{Rule} \\
            \hline
            Product & $e^z e^w=e^{z+w}$ \\
            Quotient & $\displaystyle\frac{e^z}{e^w} = e^{z-w}$ \\
            Power & $(e^z)^n = e^{nz}$ \\
            De Moivre's & $(re^{\iu\theta})^n = r^n e^{\iu n\theta}$ \\
            Conjugage & $\conj{e^z} = e^{\conj{z}}$ \\
            Modulus & $|e^z| = e^{\Re(z)}$ \\
            Argument & $\arg(e^z) = \Im(z)$
        \end{tabular}
    \end{table}
\end{theorem}