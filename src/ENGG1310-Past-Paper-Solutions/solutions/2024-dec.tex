\startYear{2024 Dec}

\subsection*{Section A -- Multiple-choice Questions}
\begin{multicols}{2}

\begin{questions}

\startQwTags{2024}{Dec}{Q1}{\tgElectronics \tgElectronicsSemiconductor}
\begin{solution}
    \mcqAns{A}

    Holes are defined as the absence of electrons in the valence band, they cannot ``move'' to the conduction band.
    Therefore, D and E are incorrect.
    Electrons cannot exist in the band gap, so C is incorrect.
    Electrons have more energy in the conduction band than in the valence band, so A is correct.
\end{solution}

\startQuestion{2024}{Dec}{Q2}
\begin{solution}
    \mcqAns{C}

    In a p-type (``p'' stands for ``positive'') semiconductor, the majority carriers are holes.
    Therefore, A and B are incorrect.
    Since holes are the majority carriers, there are more holes than electrons, so C is correct and D is incorrect.
\end{solution}

\startQuestion{2024}{Dec}{Q3}
\begin{solution}
    \mcqAns{C}

    In a reverse-biased diode, no current flows through, so A and B are incorrect.
    To put the diode in reverse bias, a higher potential must be applied to the n-type side, i.e., cathode, so C is correct and D is incorrect.
\end{solution}

\startQuestion{2024}{Dec}{Q4}
\begin{solution}
    \mcqAns{E}

    In a forward-biased diode, the mathematical relationship between input voltage and output current is exponential, so only E is correct.
    (Side note: in a reverse-biased diode, the relationship remains rather constant.)
\end{solution}

\startQuestion{2024}{Dec}{Q5}
\begin{solution}
    \mcqAns{B}

    \texttt{NAND} gates (or any gates involving \texttt{NOT}) need to be constructed using active components, i.e., transistors, so only B is correct.
    The other options are all passive components.
\end{solution}

\startQuestion{2024}{Dec}{Q6}
\begin{solution}
    \mcqAns{C}
    
    Transconductance $g_m$ measures how well the input voltage is converted to output current, so C is correct.
\end{solution}

\startQuestion{2024}{Dec}{Q7}
\begin{solution}
    \mcqAns{C}

    Explanation:
    \begin{enumroman}
        \item The colour of light in LEDs is determined by the band gap of the semiconductor. Correct.
        \item An LED emits light when electrons are interjecting from the conduction band to the valence band, which only happens when forward-biased. Correct.
        \item Light is emitted when electrons falls from the conduction band to the valence band (recombination with holes), releasing energy in the form of photons. Incorrect.
        \item Same as (iii). Correct.
    \end{enumroman}
    Therefore, (i), (ii), and (iv) are correct, so C is correct.
\end{solution}

\end{questions}
\end{multicols}