\startYear{2024 May}

\subsection*{Section A -- Multiple-choice Questions}
\begin{multicols*}{2}

\startQwTags{2024}{May}{Q1}{\tgElectronics \tgElectronicsSemiconductor}
\begin{solution}
    \mcqAns{D}

    Free electrons exist in the conduction band, so D is correct.
\end{solution}

\startQwTags{2024}{May}{Q2}{\tgElectronics \tgElectronicsSemiconductor}
\begin{solution}
    \mcqAns{A}

    In an n-type semiconductor, holes are the minority carriers.
    Also, minority carriers are always thermally generated, while majority carriers are always generated by doping, so A is correct.
\end{solution}

\startQwTags{2024}{May}{Q3}{\tgElectronics \tgElectronicsDiodes}
\begin{solution}
    \mcqAns{B}

    In a forward-biased diode, current is conducted.
    Therefore, B is correct.
\end{solution}

\startQwTags{2024}{May}{Q4}{\tgElectronics \tgElectronicsBJT \tgOutSyl}
\begin{solution}
    \mcqAns{A}

    The relationship between base current and collector current is given by $I_C = \beta I_B$, where $\beta$ is the current gain of the BJT.
    Increasing $\beta$ and keeping $I_B$ constant will increase $I_C$, so A is correct.

    {\itshape
        Note:
        In this configuration, the BJT saturates when $\mathit{\beta\geq 228}$, further increasing $\mathit{\beta}$ will not increase $\mathit{I_C}$.
        However, considering the level of this course, it is assumed that the BJT will not saturate by increasing $\mathit{\beta}$.
    }
\end{solution}

\startQwTags{2024}{May}{Q5}{\tgElectronics \tgElectronicsMOSFET}
\begin{solution}
    \mcqAns{E}

    The three terminals of a metal-oxide-semiconductor field-effect transistor (MOSFET) are called gate, drain, and source, so E is correct.
    ``n, p, n'' and ``p, n, p'' are the three layers of a bipolar junction transistor (BJT), so A and B are incorrect.
    ``input, output, ground'' are related to logic gates, not MOSFETs, so C is incorrect.
    ``emitter, base, collector'' are the three terminals of a BJT, so D is incorrect.
\end{solution}

\startQwTags{2024}{May}{Q7}{\tgElectronics \tgElectronicsEM}
\begin{solution}
    \mcqAns{E}

    The electric field and the magnetic field are orthogonal (perpendicular) to each other, in phase, and travel in the same speed.
    Therefore, (i), (ii), and (iii) are all correct, so E is correct.
\end{solution}

\startQwTags{2024}{May}{Q10}{\tgElectronics \tgElectronicsSemiconductor}
\begin{solution}
    \mcqAns{C}

    Moore's law states that the number of transistors on a microchip doubles approximately every two years.
\end{solution}

\end{multicols*}