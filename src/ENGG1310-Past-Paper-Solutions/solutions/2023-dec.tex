\startYear{2023 Dec}

\subsection*{Section A -- Multiple-choice Questions}
\begin{multicols*}{2}

\startQwTags{2023}{Dec}{Q1}{\tgElectronics \tgElectronicsSemiconductor}
\begin{solution}
    \mcqAns{C}

    In a semiconductor subjected to an electric field, holes and electrons move in opposite directions.
    Electrons flows in opposite direction to the conventional current, while holes flows in the same direction as the conventional current.
    Therefore, C is correct.
\end{solution}

\startQwTags{2023}{Dec}{Q2}{\tgElectronics \tgElectronicsSemiconductor}
\begin{solution}
    \mcqAns{E}

    To form an n-type semiconductor using silicon, we dope it with an element that has more outer electrons than silicon, i.e., phosphorus, arsenic, or antimony.
    Phosphorus is ideal since its atomic size is similar to silicon, so E is correct.
\end{solution}

\startQwTags{2023}{Dec}{Q3}{\tgElectronics \tgElectronicsSemiconductor}
\begin{solution}
    \mcqAns{E}

    The bandgap energy of silicon is approximately 1.1 eV, and approximately 9 eV for silicon dioxide.
    Therefore, E is correct.
    Note that these values need to be memorised.
\end{solution}

\startQwTags{2023}{Dec}{Q4}{\tgElectronics \tgElectronicsDiodes}
\begin{solution}
    \mcqAns{C}

    A Si pn junction diode is a passive component, it cannot be used to implement \texttt{NOT} gates, so C is not true, therefore correct.
    Note that an \texttt{AND} gate can be implemented when connected in series, as shown below:
    \figHere{AND-gate-diodes.pdf}{width=0.5\textwidth}
\end{solution}

\startQwTags{2023}{Dec}{Q5}{\tgElectronics \tgElectronicsBJT}
\begin{solution}
    \mcqAns{E}

    A BJT can be considered as a voltage-controlled device, where the voltage across the base-emitter junction ($V_{BE}$) controls the current flowing from collector to emitter ($I_C$).
    Therefore, A is correct.
    The base current of a BJT is usually very small, so B is correct.
    A BJT contains two pn junctions (the base-emitter junction and the base-collector junction), so C is correct.
    The transfer characteristic of a BJT is exponential, so D is correct.
    Hence, all of the above statements are correct, so E is correct.
\end{solution}

\startQwTags{2023}{Dec}{Q6}{\tgElectronics \tgElectronicsBJT \tgElectronicsMOSFET}
\begin{solution}
    \mcqAns{C}

    Generally, when compared to MOSFETs, BJTs have higher transconductance, C is correct.
    On the other hand, MOSFETs consume less static power, have higher transistor density, and operate at higher frequencies, so A, B, and D are incorrect.
\end{solution}

\startQwTags{2023}{Dec}{Q7}{\tgElectronics \tgElectronicsSemiconductor}
\begin{solution}
    \mcqAns{C}

    The bandgap energy $E_g$ determines the conductivity of a material.
    The larger the $E_g$, the more energy is required to excite an electron from the valence band to the conduction band, so the material is less conductive.
    Hence, A and B are incorrect.
    Different $E_g$ values correspond to different wavelengths of light that can be absorbed or emitted by the material, so D is incorrect.
    The built-in potential of a diode is affected by $E_g$, so E is incorrect.
    $E_g$ does not determine the material's density, so C is correct.
\end{solution}

\end{multicols*}