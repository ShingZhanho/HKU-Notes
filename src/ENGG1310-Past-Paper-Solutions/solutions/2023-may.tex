\startYear{2023 May}

\subsection*{Section A -- Multiple-choice Questions}
\begin{multicols*}{2}

\startQwTags{2023}{May}{Q1}{\tgElectronics \tgElectronicsSemiconductor}
\begin{solution}
    \mcqAns{C}

    \begin{enumualpha}
        \item Bandgap energy in semiconductors are non-zero in order to control the flow of current, so A is incorrect.
        \item Apart from group IV elements, semiconductors are usually made by doping group IV elements with group III or group V elements, so B is incorrect.
        \item Doping will increase the number of charge carriers, therefore altering the conductivity of the material, so C is correct.
        \item Electrical conductivity depends on temperature, so D is incorrect.
    \end{enumualpha}
\end{solution}

\startQwTags{2023}{May}{Q3}{\tgElectronics \tgElectronicsSemiconductor}
\begin{solution}
    \mcqAns{B}

    When under forward bias, the n-type side of the pn junction is connected to a lower potential than the p-type side, not higher, so B is false.
\end{solution}

\startQwTags{2023}{May}{Q4}{\tgElectronics \tgElectronicsBJT}
\begin{solution}
    \mcqAns{D}

    Assuming the BJT is npn type:
    \begin{enumualpha}
        \item When $V_{BE}<0$, the base-emitter junction is reverse biased, so no current flows through, not negative.
        \item The relationship is $I_C=\beta I_B$, not $\displaystyle I_C=\frac{I_B}{\beta}$.
        \item $I_B$ is regarded as the input current, not $I_C$.
        \item $I_E=I_B+I_C$, and $I_B$ is usually very small, so one can conclude that $I_E\approx I_C$.
    \end{enumualpha}
\end{solution}

\startQwTags{2023}{May}{Q6}{\tgElectronics \tgElectronicsCMOS \tgElectronicsBJT}
\begin{solution}
    \mcqAns{E}

    \begin{enumualpha}
        \item BJTs are used for bipolar transistor integrated circuits, not CMOS. Incorrect.
        \item CMOS draws very low static power, but not zero. Incorrect.
        \item CMOS can be fabricated on both p-type and n-type Si substrates. Incorrect.
        \item When applied with a positive voltage at the gate, NMOS turn on, while PMOS turn off. Incorrect.
    \end{enumualpha}
\end{solution}

\startQwTags{2023}{May}{Q7}{\tgElectronics \tgElectronicsOptoelectronics}
\begin{solution}
    \mcqAns{B}

    Photocurrent is a result of absorption of photons, which will excite an electron from the valence band to the conduction band, so B is correct.
    All other processes are unrelated to the generation of photocurrent.
\end{solution}

\startQwTags{2023}{May}{Q8}{\tgElectronics \tgElectronicsBJT \tgElectronicsMOSFET}
\begin{solution}
    \mcqAns{E}

    MOSFETs have lower transconductance than BJTs, so E is not an advantage of MOSFETs over BJTs.
    All other options are advantages of MOSFETs over BJTs.
\end{solution}

\startQwTags{2023}{May}{Q9}{\tgElectronics \tgElectronicsAnalogueDigital}
\begin{solution}
    \mcqAns{C}

    Both analogue and digital communication systems require modulation to transmit signals over long distances, so C is NOT exclusive to digital communication.
\end{solution}

\startQwTags{2023}{May}{Q10}{\tgElectronics \tgElectronicsAM}
\begin{solution}
    \mcqAns{B}

    \begin{enumualpha}
        \item In SSB, LSB and carrier are suppressed, only USB is transmitted.
            Therefore, the bandwidth is halved instead of increased, so A is incorrect.
        \item The amount of noise is reduced on a narrower bandwith, so B is correct.
        \item The power dedicated to the carrier and the suppresed sideband is saved, so less power is used, so C is incorrect.
        \item There is less selective fading in SSB, so D is incorrect.
    \end{enumualpha}
\end{solution}

\startQwTags{2023}{May}{Q11}{\tgElectronics \tgElectronicsMultiplexing}
\begin{solution}
    \mcqAns{D}

    In WDM (wavelength-division multiplexing), the varying parameter is the wavelength of the light.
    Further, we have the relationship $\displaystyle f=\frac{c}{\lambda}$, where $c$ is a constant.
    When $\lambda$ varies, $f$ (frequency) also varies, so the varying parameter is frequency.

    {\itshape
        Note: WDM can also be regarded as a type of FDM (frequency-division multiplexing).
    }
\end{solution}

\startQwTags{2023}{May}{Q12}{\tgElectronics \tgElectronicsEM}
\begin{solution}
    \mcqAns{A}

    The electric field and the magnetic field are in phase, so the phase difference is 0.
\end{solution}

\end{multicols*}