\question Let $T:\mathbb{R}^4\to\mathbb{R}^3$ be a linear transformation defined by
\[
    T\begin{pmatrix}
        x_1 \\ x_2 \\ x_3 \\ x_4
    \end{pmatrix} = \begin{bmatrix}
        x_1 + 2x_2 + 3x_3 - x_4 \\
        3x_1 + 5x_2 + 8x_3 - 2x_4 \\
        x_1 + x_2 + 2x_3
    \end{bmatrix}
\]

\begin{parts}
    
\part Find a matrix $A$ such that $T(\bvec{x}) = A\bvec{x}$.

\begin{solution}
    By inspection, we have $\boxed{A=\displaystyle\begin{bmatrix}
        1 & 2 & 3 & -1 \\
        3 & 5 & 8 & -2 \\
        1 & 1 & 2 & 0
    \end{bmatrix}}$.
\end{solution}

\part Find a set of basis vectors spanning the nullspace of $A$. You may want to standardise
your answer by using our in-class free-column(s) approach, i.e., one-hotting each free column
respectively, to locate the nullspace vectors.

\begin{solution}
    \newenvironment{am41}{
        \begin{augmatrix}{4}{\left[}{\right]}
    }{
        \end{augmatrix}
    }

    \newenvironment{am21}{
        \begin{augmatrix}{2}{\left[}{\right]}
    }{
        \end{augmatrix}
    }

    We find $\bvec{x_N} = \displaystyle\begin{bmatrix}
        x_1 \\ x_2 \\ x_3 \\ x_4
    \end{bmatrix}$ such that $T(\bvec{x_N}) = A \bvec{x_N} = 0$.
    \begin{align*}
        &\begin{am41}
            1 & 2 & 3 & -1 & 0 \\
            3 & 5 & 8 & -2 & 0 \\
            1 & 1 & 2 & 0 & 0
        \end{am41}
        \xrightarrow[R_3-R_1]{R_2-3R_1}
        \begin{am41}
            1 & 2 & 3 & -1 & 0 \\
            0 & -1 & -1 & 1 & 0 \\
            0 & -1 & -1 & 1 & 0
        \end{am41}
        \xrightarrow[-R_2]{R_3-R_2}
        \begin{am41}
            1 & 2 & 3 & -1 & 0 \\
            0 & 1 & 1 & -1 & 0 \\
            0 & 0 & 0 & 0 & 0
        \end{am41} \\
        &\xrightarrow{R_1-2R_2}
        \begin{am41}
            1 & 0 & 1 & 1 & 0 \\
            0 & 1 & 1 & -1 & 0 \\
            0 & 0 & 0 & 0 & 0
        \end{am41}
    \end{align*}
    Solving the system, we have
    \[
        \bvec{x_N} = \begin{bmatrix}
            0 \\ 0 \\ 0 \\ 0
        \end{bmatrix} + x_3 \begin{bmatrix}
            -1 \\ -1 \\ 1 \\ 0
        \end{bmatrix} + x_4 \begin{bmatrix}
            -1 \\ 1 \\ 0 \\ 1
        \end{bmatrix} = x_3 \begin{bmatrix}
            -1 \\ -1 \\ 1 \\ 0
        \end{bmatrix} + x_4 \begin{bmatrix}
            -1 \\ 1 \\ 0 \\ 1
        \end{bmatrix}
    \]

    Then, we examine the linear independence of the set:
    \[
        \left\{\begin{bmatrix}
            -1 \\ -1 \\ 1 \\ 0
        \end{bmatrix}, \begin{bmatrix}
            -1 \\ 1 \\ 0 \\ 1
        \end{bmatrix}\right\}
    \]
    by constructing another augmented matrix:
    \[
        \begin{am21}
            -1 & -1 & 0 \\
            -1 & 1 & 0 \\
            1 & 0 & 0 \\
            0 & 1 & 0
        \end{am21}
        \xrightarrow{-R_1}
        \begin{am21}
            1 & 1 & 0 \\
            -1 & 1 & 0 \\
            1 & 0 & 0 \\
            0 & 1 & 0
        \end{am21}
        \rightarrow
        \begin{am21}
            1 & 0 & 0 \\
            0 & 1 & 0 \\
            1 & 1 & 0 \\
            -1 & 1 & 0
        \end{am21}
        \xrightarrow[R_4+R_1-R_2]{R_3-R_1-R_2}
        \begin{am21}
            1 & 0 & 0 \\
            0 & 1 & 0 \\
            0 & 0 & 0 \\
            0 & 0 & 0
        \end{am21}
    \]
    Therefore, the set is linearly independent and forms a basis of the nullspace of $A$.
    We conclude that a basis spanning the nullspace of $A$ is
    \[
        \boxed{\left\{\begin{bmatrix}
            -1 \\ -1 \\ 1 \\ 0
        \end{bmatrix}, \begin{bmatrix}
            -1 \\ 1 \\ 0 \\ 1
        \end{bmatrix}\right\}}.
    \]
\end{solution}

\end{parts}