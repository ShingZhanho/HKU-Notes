\section{Sets, Relations, \& Functions}

\begin{remark}
    The logical expressions given in this section are very important for proofs in assignments
    and exams.
\end{remark}

\subsection{Sets}

\subsubsection{Basics on Sets}

\begin{definition}[Set]
    A \textbf{set} is a collection of \textbf{unordered}, \textbf{distinct} objects,
    considered as an object in its own right.
    The objects are called the \textbf{elements} or \textbf{members} of the set.
\end{definition}

\begin{definition}[Belonging to a Set]
    An object $x$ within $A$ is said to \textbf{belong} to a set $A$, denoted
    by $x \in A$. If $x$ does not belong to $A$, we write $x \notin A$.
\end{definition}

\begin{definition}[Cardinality]
    The \textbf{cardinality} of a set $A$, denoted by $|A|$, is the number of
    elements in $A$.
\end{definition}

\begin{definition}[Empty Set]
    The \textbf{empty set}, denoted by $\mtset$, is the set with no elements.
    Its cardinality is $0$, i.e. $\card{\mtset} = 0$.
\end{definition}

\begin{remark}
    Sets can be defined in two ways:
    \begin{enumerate}
        \item \textbf{Roster form}: Explicitly list all elements of the set.
        (e.g. $A = \{1, 2, 3\}$)
        \item \textbf{Set-builder form}: Describe the properties of the elements
        of the set.\\
        (e.g. $A = \{x \mid x \text{ is a positive integer less than } 4\}
        = \{1, 2, 3\}$)
    \end{enumerate}
\end{remark}

\begin{definition}[Power Set]
    A power set of a set $A$, denoted by $\pwset{A}$, is the set of all subsets of $A$,
    i.e. all the possible combinations of elements in $A$.

    Note that $\mtset \in \pwset{A}$ and $A \in \pwset{A}$ are tautologies.
    We also have $\card{\pwset{A}} = 2^{\card{A}}$, provided that $A$ is finite.
\end{definition}

\begin{remark}
    Common sets:
    \begin{enumerate*}[label=(\arabic*)]
        \item Natural numbers: $\N = \set{0, 1, 2, \ldots}$ \textit{in the field of Computer Science,
        $0\in\N$, while in some other fields, $0\notin\N$};
        \item Integers: $\Z$, Positive integers: $\Z^+$;
        \item Rational numbers: $\Q = \set{\frac{a}{b} \mid a, b \in \Z, b \neq 0}$;
        \item Real numbers: $\R$;
        \item Complex numbers: $\C = \set{a + b\imag \mid a, b \in \R, \imag^2 = -1}$.
    \end{enumerate*}
\end{remark}

\subsubsection{Relationships of Sets}

\begin{definition}[Subset]
    A set $A$ is a \textbf{subset} of a set $B$, denoted by $A \subseteq B$,
    if every element of $A$ is also an element of $B$, or, by logical expression,
    \[
        A\subseteq B \equiv \forall x (x \in A \limp x \in B)
    \]

    Conversely, if $\exists x (x \in A \land x \notin B)$, then $A$ is not a subset of $B$,
    denoted by $A \nsubseteq B$.
\end{definition}

\begin{definition}[Proper Subset]
    A set $A$ is a \textbf{proper subset} of a set $B$, denoted by $A \subset B$,
    if $A \subseteq B$ and $A \neq B$, or, by logical expression,
    \[
        A \subset B \equiv [\forall x (x \in A \limp x \in B)] \land [\exists y (y \in B \land \neg(y \in A))]
    \]
\end{definition}

\begin{definition}[Equality of Sets]
    Two sets $A$ and $B$ are \textbf{equal}, denoted by $A = B$, if for every element $x$ in $A$,
    $x$ is also in $B$, and vice versa, or, by logical expression
    \[
        A=B \equiv \forall x (x \in A \liff x \in B)
    \]
\end{definition}

\subsubsection{Set Operations}

\begin{definition}[Intersection $\cap$]
    The \textbf{intersection} of two sets $A$ and $B$, denoted by $A \cap B$,
    is the set of elements that are in both $A$ and $B$, or, by logical expression,
    \[
        x\in A\cap B \equiv (x \in A) \land (x \in B)
    \]
\end{definition}

\begin{definition}[Union $\cup$]
    The \textbf{union} of two sets $A$ and $B$, denoted by $A \cup B$,
    is the set of elements that are in either $A$ or $B$ (or in both), or, by logical expression,
    \[
        x\in A\cup B \equiv (x \in A) \lor (x \in B)
    \]
\end{definition}

\begin{definition}[Difference $-$]
    The \textbf{difference} of two sets $A$ and $B$, denoted by $A - B$,
    is the set of elements that are in $A$ but not in $B$, or, by logical expression,
    \[
        x \in A - B \equiv (x \in A) \land (x \notin B)
    \]
\end{definition}

\begin{definition}[Complement $\scomp{S}$]
    The \textbf{complement} of a set $A$ in the universal set $U$, denoted by $\scomp{A}$,
    is the set of elements that are not in $A$, or, by logical expression,
    \[
        x \in \scomp{A} \equiv x \notin A
    \]
    
    Note that we have $\scomp{A} = U - A$.
\end{definition}

\begin{definition}[Cartesian Product $\times$]
    The \textbf{Cartesian product} of two sets $A$ and $B$, denoted by $A \times B$,
    is the set of all \textbf{ordered pairs} $(a, b)$ where $a \in A$ and $b \in B$.
\end{definition}

\begin{remark}
    Note that $A\times B\neq B\times A$ for non-empty sets $A$ and $B$.
    Generally, we have $A\times \mtset = \mtset\times A = \mtset$.
\end{remark}