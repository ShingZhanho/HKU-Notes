\section{Sets, Relations, \& Functions}

\begin{remark}
    The logical expressions given in this section are very important for proofs in assignments
    and exams.
\end{remark}

\subsection{Sets}

\subsubsection{Basics on Sets}

\begin{definition}[Set]
    A \textbf{set} is a collection of \textbf{unordered}, \textbf{distinct} objects,
    considered as an object in its own right.
    The objects are called the \textbf{elements} or \textbf{members} of the set.
\end{definition}

\begin{definition}[Belonging to a Set]
    An object $x$ within $A$ is said to \textbf{belong} to a set $A$, denoted
    by $x \in A$. If $x$ does not belong to $A$, we write $x \notin A$.
\end{definition}

\begin{definition}[Cardinality]
    The \textbf{cardinality} of a set $A$, denoted by $|A|$, is the number of
    elements in $A$.
\end{definition}

\begin{definition}[Empty Set]
    The \textbf{empty set}, denoted by $\mtset$, is the set with no elements.
    Its cardinality is $0$, i.e. $\card{\mtset} = 0$.
\end{definition}

\begin{remark}
    Sets can be defined in two ways:
    \begin{enumerate}
        \item \textbf{Roster form}: Explicitly list all elements of the set.
        (e.g. $A = \{1, 2, 3\}$)
        \item \textbf{Set-builder form}: Describe the properties of the elements
        of the set.\\
        (e.g. $A = \{x \mid x \text{ is a positive integer less than } 4\}
        = \{1, 2, 3\}$)
    \end{enumerate}
\end{remark}

\begin{definition}[Power Set]
    A power set of a set $A$, denoted by $\pwset{A}$, is the set of all subsets of $A$,
    i.e. all the possible combinations of elements in $A$.

    Note that $\mtset \in \pwset{A}$ and $A \in \pwset{A}$ are tautologies.
    We also have $\card{\pwset{A}} = 2^{\card{A}}$, provided that $A$ is finite.
\end{definition}

\begin{remark}
    Common sets:
    \begin{enumerate*}[label=(\arabic*)]
        \item Natural numbers: $\N = \set{0, 1, 2, \ldots}$ \textit{in the field of Computer Science,
        $0\in\N$, while in some other fields, $0\notin\N$};
        \item Integers: $\Z$, Positive integers: $\Z^+$;
        \item Rational numbers: $\Q = \set{\frac{a}{b} \mid a, b \in \Z, b \neq 0}$;
        \item Real numbers: $\R$;
        \item Complex numbers: $\C = \set{a + b\imag \mid a, b \in \R, \imag^2 = -1}$.
    \end{enumerate*}
\end{remark}

\subsubsection{Relationships of Sets}

\begin{definition}[Subset]
    A set $A$ is a \textbf{subset} of a set $B$, denoted by $A \subseteq B$,
    if every element of $A$ is also an element of $B$, or, by logical expression,
    \[
        A\subseteq B \equiv \forall x (x \in A \limp x \in B)
    \]

    Conversely, if $\exists x (x \in A \land x \notin B)$, then $A$ is not a subset of $B$,
    denoted by $A \nsubseteq B$.
\end{definition}

\begin{definition}[Proper Subset]
    A set $A$ is a \textbf{proper subset} of a set $B$, denoted by $A \subset B$,
    if $A \subseteq B$ and $A \neq B$, or, by logical expression,
    \[
        A \subset B \equiv [\forall x (x \in A \limp x \in B)] \land [\exists y (y \in B \land \neg(y \in A))]
    \]
\end{definition}

\begin{definition}[Equality of Sets]
    Two sets $A$ and $B$ are \textbf{equal}, denoted by $A = B$, if for every element $x$ in $A$,
    $x$ is also in $B$, and vice versa, or, by logical expression
    \[
        A=B \equiv \forall x (x \in A \liff x \in B)
    \]
\end{definition}

\subsubsection{Set Operations}

\begin{definition}[Intersection $\cap$]
    The \textbf{intersection} of two sets $A$ and $B$, denoted by $A \cap B$,
    is the set of elements that are in both $A$ and $B$, or, by logical expression,
    \[
        x\in A\cap B \equiv (x \in A) \land (x \in B)
    \]
\end{definition}

\begin{definition}[Union $\cup$]
    The \textbf{union} of two sets $A$ and $B$, denoted by $A \cup B$,
    is the set of elements that are in either $A$ or $B$ (or in both), or, by logical expression,
    \[
        x\in A\cup B \equiv (x \in A) \lor (x \in B)
    \]
\end{definition}

\begin{definition}[Difference $-$]
    The \textbf{difference} of two sets $A$ and $B$, denoted by $A - B$,
    is the set of elements that are in $A$ but not in $B$, or, by logical expression,
    \[
        x \in A - B \equiv (x \in A) \land (x \notin B)
    \]
\end{definition}

\begin{definition}[Complement $\scomp{S}$]
    The \textbf{complement} of a set $A$ in the universal set $U$, denoted by $\scomp{A}$,
    is the set of elements that are not in $A$, or, by logical expression,
    \[
        x \in \scomp{A} \equiv x \notin A
    \]
    
    Note that we have $\scomp{A} = U - A$.
\end{definition}

\begin{definition}[Cartesian Product $\times$]
    The \textbf{Cartesian product} of two sets $A$ and $B$, denoted by $A \times B$,
    is the set of all \textbf{ordered pairs} $(a, b)$ where $a \in A$ and $b \in B$.
\end{definition}

\begin{remark}
    Note that $A\times B\neq B\times A$ for non-empty sets $A$ and $B$.
    Generally, we have $A\times \mtset = \mtset\times A = \mtset$.
\end{remark}

\subsubsection{Set Theory \& Logic}

Sets are in one-to-one correspondence with predicates.

\begin{theorem}[Sets and Predicates]
    Let $U$ be the set of all possible values of $x$, then this universal set is equivalent
    to the universe of discourse for predicate logics.

    For every set $A \subseteq U$, we can define a predicate that depends on $A$:
    \[
        P_A(x) : (x \in A)
    \]

    Similarly, for every predicate $P(x)$, we can define a truth set $A$:
    \[
        A := \set{x \in U \mid P(x)}
    \]
\end{theorem}

\subsection{Relations}

\subsubsection{Definition \& Properties of Relations}

\begin{definition}[Relation]
    A \textbf{relation} from a set $A$ to a set $B$ is a subset of $A \times B$, denoted by
    $R \subseteq A\times B$.

    When element $a$ is in relation with element $b$ by $R$, we write $aRb$, i.e.,
    \[
        a\rel b \equiv (a, b) \in R
    \]
    Otherwise, we write $a \notrel b$.
\end{definition}

There are several special types of relations when we discuss relations on a set $A$ itself,
i.e. $R \subseteq A \times A$.

\begin{definition}[Reflexive Relation]
    A \textbf{reflexive relation} is defined to be:
    \[
        \forall x\in A (x\rel x)
    \]
\end{definition}
\begin{example}
    Let $A$ be the set of all people and define $x\rel y: \text{``}x$ has a biological father $y$''.
    Since everyone must be born from a biological father (regardless of whether they know who
    he is, or whether the father is still alive), we have $\forall x\in A (x\rel x)$.
    Thus, this relation is reflexive.
\end{example}

\begin{definition}[Symmetric Relation]
    A \textbf{symmetric relation} is defined to be:
    \[
        \forall x, y \in A (x\rel y \limp y\rel x)
    \]    
\end{definition}

\begin{example}
    Continue to let $A$ be the set of all people and redefine $x\rel y:$ ``$x$ is married to $y$''.
    Since if $x$ is married to $y$, then $y$ must be married to $x$, we have
    $\forall x, y \in A (x\rel y \limp y\rel x)$.
    Thus, this relation is symmetric.
    \begin{remark}
        It does not matter if $\exists x_0, y_0\in A$ who are single and not married to anyone,
        since when $(x_0, y_0) \notin R$, the implication $(x_0\rel y_0) \limp (y_0\rel x_0)$ is 
        still true.
    \end{remark}
\end{example}

\begin{definition}[Transitive Relation]
    A \textbf{transitive relation} is defined to be:
    \[
        \forall x, y, z \in A [(x\rel y) \land (y\rel z) \limp (x\rel z)]
    \]
\end{definition}

\begin{example}
    Let $A$ be the set of all people in Hong Kong, and define $x\rel y:$ ``$x$ studies in the
    same university as $y$''.
    Since if $x$ studies in the same university as $y$, and $y$ studies in the same university
    as $z$, then $x$ must study in the same university as $z$, we have
    $\forall x, y, z \in A [(x\rel y) \land (y\rel z) \limp (x\rel z)]$.
    Thus, this relation is transitive.

    \begin{remark}
        It does not matter if $\exists x,y_0,z\in A$, where $y_0$ studies in a different
        university from $x$ and $z$, since when $(x\rel y_0) \land (y_0\rel z)$ is false,
        the implication $[(x\rel y_0) \land (y_0\rel z)] \limp (x\rel z)$ is still true.
    \end{remark}
\end{example}

\begin{definition}[Equivalence Relation]
    An \textbf{equivalence relation} is a relation that is reflexive, symmetric, and transitive.
\end{definition}

\subsubsection{Equivalence Classes \& Partitions}

\begin{lemma}[Equivalence of Elements]
    Let $R$ be an equivalence relation on a set $A$. Two elements $x, y \in A$ are said to be
    \textbf{equivalent} if $x\rel y$.
\end{lemma}

\begin{definition}[Equivalence Class]
    The \textbf{equivalence class} of an element $x \in A$ is the set of all elements in $A$
    that are equivalent to $x$, denoted by $[x]$, or, more formally,
    \[
        [x] := \set{y \in A \mid x\rel y}
    \]

    Every element of $[x]$ is said to be a \textbf{representative} of the equivalence class $[x]$.
\end{definition}

\begin{example}
    Let $A = \Z^+$, and define $x\rel y:$ ``$x-y$ is even''.
    Observe that any integer subtracted by itself is $0$, which is even,
    so the relation is reflexive.
    Also, when $x-y$ is even, then $y-x = -(x-y)$ is also even, so the relation is symmetric.
    Finally, when $x-y$ and $y-z$ are both even, then $x-z = (x-y) + (y-z)$ is also even,
    so the relation is transitive.
    Thus, this relation is an equivalence relation.

    Easily, we have $[1] = \set{1, 3, 5, 7, \ldots}$ and $[2] = \set{2, 4, 6, 8, \ldots}$.
\end{example}

\begin{lemma}[Disjoint Sets]
    Two sets $A_1$ and $A_2$ are said to be \textbf{disjoint} if $A_1 \cap A_2 = \mtset$,
    i.e. they have no elements in common.
\end{lemma}

\begin{theorem}[Distinct Equivalence Classes are Disjoint]
    If $[x]\cup[y]\neq\mtset$, then $[x] = [y]$.
\end{theorem}

\begin{definition}[Partition of a Set]
    A list of subsets $A_1, A_2, \ldots, A_k \subseteq A$ forms a \textbf{partition} of $A$
    if the following conditions are satisfied:
    \begin{enumerate}
        \item $\displaystyle\bigcup_{i=1}^k A_i = A$
        \item $\forall i \neq j : A_i \cap A_j = \mtset $
    \end{enumerate}
\end{definition}

\begin{example}
    Let $A$ be the set of all people, and define $x\rel y:$ ``$x$ and $y$ are born in the same
    month''.
    We skip the verification that this is an equivalence relation.
    
    The equivalence classes are: $[\text{People born in Jan}], [\text{People born in Feb}], \ldots,
    [\text{People born in Dec}]$.

    Observe that the union of all these equivalence classes must be $A$, since everyone must be born
    in some month, and any two equivalence classes are disjoint, since no one can be born in two different months.
    Thus, these equivalence classes form a partition of $A$.
\end{example}

\subsection{Functions}

\subsubsection{Definition \& Terminology}

\begin{definition}[Function]
    A \textbf{function} is a special type of relation from a set $A$ to a set $B$
    with the property that for every $a \in A$, there is exactly one $b \in B$
    such that $a$ is related to $b$.
    
    Explicitly, a relation $R$ from $A$ to $B$ is a function if it satisfies:
    \begin{enumerate}
        \item $\forall a\in A, \exists b\in B : a\rel b$
        \item $\forall a\in A, \forall b_1, b_2 \in B : (a\rel b_1) \land (a\rel b_2) \limp (b_1 = b_2)$
    \end{enumerate}

    \textbf{Notation}: For a relation $R$ that is a function, we write $y=f(x)$.

    \textbf{Complete Notation of a Function}: To completely define a function, we write it in
    the form:
    \[
        f: A\to B \quad f(x) = (\text{the rule to get } y \text{ from } x)
    \]
\end{definition}

\begin{definition}[Domain, Codomain, Preimage \& Image]
    For a function $y=f(x)$, or $R: A \to B$, we have:
    \begin{enumerate}
        \item The set $A$ is called the \textbf{domain} of $f$;
        \item The elements $x \in A$ are called the \textbf{preimages};
        \item The set $B$ is called the \textbf{codomain} of $f$;
        \item The elements $y \in B$ are called the \textbf{images};
    \end{enumerate}
\end{definition}

\begin{definition}[Range]
    The \textbf{range} of a function $f:A\to B$ is the set of all elements in $B$ that
    are images of elements in $A$, i.e.
    \[
        f(A) := \set{y\in B \mid \exists x\in A (y=f(x))}
    \]

    Note that we have the property $f(A) \subseteq B$, i.e. the range of $f$ is not necessarily
    the same as the codomain of $f$.
\end{definition}

\subsubsection{Types of Functions}

\begin{definition}[Injective Function]
    A function $f:A\to B$ is said to be \textbf{injective} (one-to-one) if
    \[
        \forall x_1, x_2 \in A [(x_1  \neq x_2) \limp (f(x_1) \neq f(x_2))]
    \]
\end{definition}

\begin{example}
    $f:\R\to\R\quad f(x) = x^2$ is not injective, since $f(1) = f(-1) = 1$,
    while $f:\R^+\to\R\quad f(x) = x^2$ is injective.
\end{example}

\begin{definition}[Surjective Function]
    A function $f:A\to B$ is said to be \textbf{surjective} (onto) if
    \[
        \forall y \in B, \exists x \in A : f(x) = y
    \]

    Consequently, we have $f(A) = B$ for any surjective $f:A\to B$.
\end{definition}

\begin{definition}[Bijective Function]
    A function $f:A\to B$ is said to be \textbf{bijective} if it is both injective and surjective.
    In other words, for every $y \in B$, there exists a unique $x \in A$ such that $f(x) = y$.
\end{definition}

\subsubsection{Operations on Functions}

\begin{definition}[Composition of Functions]
    Given two functions $f:A\to B$ and $g:B\to C$, we can define $g\circ f:A\to C$ by
    \[
        (g\circ f)(x) := g(f(x))
    \]
\end{definition}

\begin{definition}[Inverse of a Function]
    Let $f:A\to B$ be a bijective function. Then there exists a function $g:B\to A$ such that
    \[
        \forall x \in A \quad (g\circ f)(x) = x
    \]
    and
    \[
        \forall y \in B \quad (f\circ g)(y) = y
    \]
    This function $g$ is called the \textbf{inverse} of $f$, denoted by $f^{-1}$.
\end{definition}

\subsubsection{Real-valued Functions}

\begin{definition}[Real-valued Function]
    A function $f:A\to B$ is said to be a \textbf{real-valued function} if $B \subseteq \R$.
\end{definition}

\begin{theorem}
    Let $A,B\subseteq\R$, we say $f:A\to B$ to be:
    \begin{enumerate}
        \item \textbf{strictly increasing} if $\forall x_1, x_2 \in A [(x_1 < x_2) \limp (f(x_1) < f(x_2))]$;
        \item \textbf{strictly decreasing} if $\forall x_1, x_2 \in A [(x_1 < x_2) \limp (f(x_1) > f(x_2))]$;
        \item \textbf{non-increasing} if $\forall x_1, x_2 \in A [(x_1 < x_2) \limp (f(x_1) \geq f(x_2))]$;
        \item \textbf{non-decreasing} if $\forall x_1, x_2 \in A [(x_1 < x_2) \limp (f(x_1) \leq f(x_2))]$.
    \end{enumerate}
\end{theorem}

\begin{remark}
    Note that ``not non-decreasing'' does not imply ``decreasing''.
    For example, $f:\R\to\R\quad f(x) = x^2$ is not non-decreasing, but it is not decreasing either.
\end{remark}