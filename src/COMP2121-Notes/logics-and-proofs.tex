\section{Logic \& Proofs}

\subsection{Propositional Logic}

\begin{definition}[Proposition]
    A statement that can be \textbf{unambiguously} determined to be either true of false.
\end{definition}

\begin{definition}[Logical Operators]
    The commonly used logical operators are:
    \begin{itemize}
        \item \textbf{Negation}: $\lNot P$
        \item \textbf{Conjunction} (AND): $P \lAnd Q$
        \item \textbf{Disjunction} (OR): $P \lOr Q$
        \item \textbf{Exclusive OR} (XOR): $P \lXor Q$
        \item \textbf{Implication/Conditional} (if ..., then ...): $P \lImp Q$
        \item \textbf{Biconditional} (if and only if): $P \lIff Q$
    \end{itemize}
    where $P$ and $Q$ are propositions.
\end{definition}

\begin{remark}
    The truth tables for the conditional and biconditional operators are as follows:
    \begin{table}[H]
        \centering
        \begin{tabular}{cccc}
            $P$ & $Q$ & $\boldsymbol{P \lImp Q}$ & $\boldsymbol{P \lIff Q}$ \\ \hline
            F   & F   & T                     & T                     \\
            F   & T   & T                     & F                     \\
            T   & F   & F                     & F                     \\
            T   & T   & T                     & T                     \\
        \end{tabular}
    \end{table}

    Note that the ``implication'' operator is distinct from the ``implication'' used in natural
    language. $P \lImp Q$ does not contain cause-and-effect information.
\end{remark}

\begin{definition}[Sufficiency and Necessity]
    When $P \lImp Q$ is true, we say that:
    \begin{itemize}
        \item $P$ is a \textbf{sufficient} condition for $Q$.
        \item $Q$ is a \textbf{necessary} condition for $P$.
    \end{itemize}
\end{definition}

The truth value of a proposition $P$ is denoted by $\lTV{P}$. For composite propositions,
we often need to simplify the expression.

\subsubsection{Boolean Algebra}

\begin{definition}[Algebraic Rules for Boolean Algebra]
    Consider two propositions $P$ and $Q$, and denote true by $1$ and false by $0$, we have:
    \begin{multicols}{2}
        \begin{enumerate}
            \item $\lTV{\lNot P} = \lTV{P} \lXor 1$
            \item $\lTV{P \lAnd Q} = \lTV{P}\lTV{Q}$
            \item $\lTV{P \lXor Q} = \lTV{P} \lXor \lTV{Q}$
            \item $\lTV{P \lOr Q} = \lTV{P} \lXor \lTV{Q} \lXor \lTV{P}\lTV{Q}$
            \item $\lTV{P \lIff Q} = \lTV{P} \lXor \lTV{Q} \lXor 1$
            \item $\lTV{P \lImp Q} = \lTV{P}\lTV{Q} \lXor \lTV{P} \lXor 1$
        \end{enumerate}
    \end{multicols}
\end{definition}

\begin{example}
    \textbf{Question}: Compute the truth values of $(P \lImp Q) \lAnd (Q \lImp P)$ as a function of
    $\lTV{P}$ and $\lTV{Q}$.

    \textbf{Solution}: Denote $x=\lTV{P}$ and $y=\lTV{Q}$, we have:
    \begin{align*}
        &\phantom{\mathrel{=}} \lTV{(P \lImp Q) \lAnd (Q \lImp P)} \\
        &= (xy \lXor x \lXor 1) \lAnd (yx \lXor y \lXor 1) \qquad\text{(Rule \#6)} \\
        &= (xy \lXor x \lXor 1)(xy \lXor y \lXor 1) \qquad\text{(Rule \#2)} \\
        &= x^2y^2 \lXor xy^2 \lXor xy \lXor x^2y \lXor xy \lXor x \lXor xy \lXor y \lXor 1 \\
        &= xy \lXor xy \lXor xy \lXor xy \lXor xy \lXor x \lXor xy \lXor y \lXor 1 \qquad\text{(}\forall b\in\{0, 1\}:(b^2 = b)\text{)}\\
        &= \boxed{x \lXor y \lXor 1}
    \end{align*}
    Note that this expression is equivalent to $P \lIff Q$ (by Rule \#5).
\end{example}

\subsubsection{Logical Equivalence}

\begin{definition}[Tautology and Contradiction]
    A \textbf{tautology} is a proposition that is always true, denoted by $T$.
    A \textbf{contradiction} is a proposition that is always false, denoted by $F$.
\end{definition}

\begin{definition}[Logical Equivalence]
    Two propositions $P$ and $Q$ are said to be logically equivalent if $P \lIff Q$ is a tautology,
    denoted as $P \equiv Q$ or $P \Leftrightarrow Q$.
\end{definition}

\begin{theorem}[Important Laws of Logical Equivalence]
    These are some important laws for simplifying composite logics:
    \begin{enumerate}
        \item \textbf{Double Negation Law}: $\neg(\neg P) \equiv Q$
        \item \textbf{Identity Laws}: $P \lAnd T \equiv P$ and $P \lOr F \equiv P$
        \item \textbf{Domination Laws}: $P \lOr T \equiv T$ and $P \lAnd F \equiv F$
        \item \textbf{Idempotent Laws}: $P \lAnd P \equiv P$ and $P \lOr P \equiv P$
        \item \textbf{Negation Laws}: $P \lAnd \neg P \equiv F$ and $P \lOr \neg P \equiv T$
        \item \textbf{Biconditional Law}: $(P \lIff Q) \equiv (P \lImp Q) \lAnd (Q \lImp P)$
        \item \textbf{Implication Law}: $(P \lImp Q) \equiv (\neg P \lOr Q)$
        \item \textbf{Contraposition Law}: $(P \lImp Q) \equiv (\neg Q \lImp \neg P)$
        \item \textbf{De Morgan's Laws}:
        \begin{enumerate}
            \item $\neg(P \lAnd Q \lAnd R \lAnd \cdots) \equiv \neg P \lOr \neg Q \lOr \neg R \lOr \cdots$
            \item $\neg(P \lOr Q \lOr R \lOr \cdots) \equiv \neg P \lAnd \neg Q \lAnd \neg R \lAnd \cdots$
        \end{enumerate}
        \item \textbf{Distributivity Laws}:
        \begin{enumerate}
            \item $P \lOr (Q \lAnd R) \equiv (P \lOr Q) \lAnd (P \lOr R)$
            \item $P \lAnd (Q \lOr R) \equiv (P \lAnd Q) \lOr (P \lAnd R)$
        \end{enumerate}
        \item \textbf{Absorption Laws}:
        \begin{enumerate}
            \item $P \lOr (P \lAnd Q) \equiv P$
            \item $P \lAnd (P \lOr Q) \equiv P$
        \end{enumerate}
    \end{enumerate}
\end{theorem}

\begin{remark}
    Other important logical equivalences involving Biconditional and Exclusive Or
    \begin{multicols}{2}
        \begin{enumerate}
            \item \textbf{Biconditional}:
            \begin{enumerate}
                \item $P \liff Q \equiv \neg P \liff \neg Q$
                \item $P \liff Q \equiv (P \land Q) \lor (\neg P \land \neg Q)$
                \item $\neg (P \liff Q) \equiv P \liff \neg Q$
            \end{enumerate}
            \columnbreak
            \item \textbf{Exclusive Or}:
            \begin{enumerate}
                \item $P \oplus Q \equiv (P \lor Q) \land \neg (P \land Q)$
                \item $P \liff Q \equiv \neg P \oplus Q \equiv P \oplus \neg Q$
                \item $P \lor Q \equiv (P \land Q) \lxor (P \oplus Q)$
                \item $P \lxor \boldsymbol{T} \equiv \neg P$ and $P \lxor \boldsymbol{F} \equiv P$
            \end{enumerate}
        \end{enumerate}
    \end{multicols}
\end{remark}

\subsection{Predicate Logic}

\begin{definition}[Universe of Discourse]
    For a variable $x$, the set of values under consideration is called the \textbf{Universe of Discourse},
    or the \textbf{Domain}.
\end{definition}

\begin{definition}[Predicate]
    A predicate $P(x)$ is a statement that depends on a variable $x$ so that $P(x)$ is a proposition
    for every $x$ in the universe of discourse.
\end{definition}

\begin{definition}[The Universal Quantifier and The Existential Quantifier]
    The notation $\boldsymbol{\forall x P(x)}$
    denotes ``$P(x)$ holds \textbf{for every $\boldsymbol{x}$} in the universe of discourse''.
    Consider its logical equivalence:--
    $$
    \forall x P(x) \equiv P(x_0) \lAnd P(x_1) \lAnd P(x_2) \lAnd \cdots \lAnd P(x_n)
    $$
    where $\{x_0, x_1, x_2, \cdots, x_n\}$ is the universe of discourse.

    Similarly, the notation $\boldsymbol{\exists x P(x)}$ denotes ``$P(x)$ holds for \textbf{at least one $\boldsymbol{x}$}
    in the universe of discourse'', and has the logical equivalence of:--
    $$
    \exists x P(x) \equiv P(x_0) \lOr P(x_1) \lOr P(x_2) \lOr \cdots \lOr P(x_n)
    $$.
\end{definition}

\begin{theorem}[Examples and Counterexamples]
    To guarantee that the proposition $\forall x P(x)$ is false, it is enough to find one
    \textbf{counterexample} $x_0$ such that $P(x_0)$ is false.
    To guarantee that the proposition $\exists x P(x)$ is true, it is enough to find one
    \textbf{example} $x_0$ such that $P(x_0)$ is true.
\end{theorem}

\begin{theorem}[Negations of $\forall$ and $\exists$ Predicates]
    Predicates involving $\forall$ and $\exists$ can be negated as:
    \begin{multicols}{2}
        \begin{enumerate}
            \item $\neg(\forall xP(x)) \equiv \exists x \neg P(x)$
            \item $\neg(\exists xP(x)) \equiv \forall x \neg P(x)$
        \end{enumerate}
    \end{multicols}
\end{theorem}

\begin{remark}
    Quantifiers cannot be exchanged:
    $$
    \forall y [\exists x : P(x,y)] \nequiv \exists x [\forall y : P(x, y)]
    $$
    Let $P(x,y)$ be ``$x$ opens $y$'', $x\in\{\text{all keys in the world}\}$, and
    $y\in\{\text{all doors in the world}\}$. The left hand side means ``for every door, there exists
    at least one key that opens the door'' while the right hand side means ``there exists
    at least one key that opens every door.''
\end{remark}

\begin{remark}
    Quantifiers are only distributive with respect to certain operators:
    \begin{align*}
        \forall x [P(x) \lAnd Q(x)] &\equiv [\forall x P(x)] \lAnd [\forall y Q(y)] \\
        \exists x [P(x) \lOr Q(x)] &\equiv [\exists x P(x)] \lOr [\exists y Q(y)]
    \end{align*}
    Other combinations do not work.
\end{remark}

\subsection{Proofs}

\begin{definition}[Valid Argument]
    An argument is said to be valid if it is impossible for the conclusion to be $\lfalse$
    when all its premises are $\ltrue$.
\end{definition}

\begin{definition}[Logical Implication]
    $P$ is said to be logically implying $Q$ if $P \limp Q$ is a tautology, denoted
    as $P \limply Q$.
\end{definition}

\begin{definition}[Open Problems]
    When the proof of a statement remains unknown, the statement is called an \textbf{open problem}.
    Once a proof is found, the problem no longer remains open.
\end{definition}

\begin{theorem}[Rules of Inference in Propositional Logic] \quad\par
    \begin{center}
        \begin{tabular}{ll}
            \textbf{Name} & \textbf{Logical Implication} \\
            \hline
            \textit{Modus ponens} & $((P \limp Q) \land P) \limply Q$ \\
            \textit{Modus tollens} & $((P \limp Q) \land \neg Q) \limply \neg P$ \\
            Hypothetical syllogism & $((P \limp Q) \land (Q \limp R)) \limply (P \limp R)$ \\
            Disjunctive syllogism & $((P \lor Q) \land \neg P) \limply Q$ \\
            Addition & $P \limply (P \lor Q)$ \\
            Simplification & $(P \land Q) \limply P$ \\
            Resolution & $((P \lor Q) \land (\neg P \lor R) \limply (Q \lor R))$
        \end{tabular}
    \end{center}
\end{theorem}

\begin{theorem}[Rules of Inference in Predicate Logic] \quad\par
    \begin{center}
        \begin{tabular}{ll}
            \textbf{Name} & \textbf{Logical Implication} \\
            \hline
            Universal instantiation & $(\forall x P(x)) \limply P(c)$ for any arbitrary $c$ \\
            Universal generalization & $P(c) \limply (\forall x P(x))$ for any arbitrary $c$ \\
            Existential instantiation & $(\exists x P(x)) \limply P(c)$ for some $c$ \\
            Existential generalization & $P(c) \limply (\exists x P(x))$ for some $c$ \\
        \end{tabular}
    \end{center}
\end{theorem}

\begin{theorem}[Combining Rules of Inference] \quad\par
    \begin{center}
        \begin{tabular}{ll}
            \textbf{Name} & \textbf{Logical Implication} \\
            \hline
            Universal \textit{modus ponens} & $((\forall x (P(x) \limp Q(x))) \land P(a)) \limply Q(a)$ for a particular $a$ \\
            Universal \textit{modus tollens} & $((\forall x (P(x) \limp Q(x))) \land \neg Q(a)) \limply \neg P(a)$ for a particular $a$
        \end{tabular}
    \end{center}
\end{theorem}

Several methods for proving statements are introduced below.

\subsubsection{Direct Proof}

A \textbf{direct proof} of a statement $P \limp Q$ starts with the assumption that $P$
is true, and uses a sequence of logical implications to arrive at the conclusion that $Q$
is true.

\begin{example} \quad\par
    \textbf{Thesis}: If $n$ is an odd integer, then $n^2$ is an odd integer.

    \begin{proof}
        Assume $n$ to be an odd integer, then, by definition, for some integer $k$,
        $n=2k+1$.

        Then, we have $n^2 = (2k+1)^2 = 2(2k^2 + 1) + 1$.
        Since $k$ is an integer, $2k^2 + 1$ is an integer.

        By definition of an odd number, $n^2=2(2k^2 + 1) + 1$ is an odd integer.
    \end{proof}
\end{example}

\subsubsection{Proof by Contraposition}

Recall the Contraposition Law: $P \limp Q \equiv \neg Q \limp \neg P$. To prove $P \limp Q$
by contraposition, we first assume that the conclusion, namely $Q$, is false, and then
show that the premise $P$ must also be false.

\begin{example} \quad\par
    \textbf{Thesis}: For every integer $n$, $n$ is even if $n^2$ is even.

    \begin{proof}
        Denote $P:$ ``$n^2$ is even'' and $Q:$ ``$n$ is even''. The thesis is equivalent
        to $P \limp Q$. To prove by contraposition, we prove $\neg Q \limp \neg P$, i.e.,
        ``If $n$ is odd, then $n^2$ is odd''.

        Note that this is exactly the same as the previous example, which has already
        been proven true.

        Therefore, when $Q$ is false (i.e., $n$ is odd), $P$ must also be false (i.e.,
        $n^2$ is odd), and this shows that $P \limply Q$.
    \end{proof}
\end{example}

\subsubsection{Proof by Contradiction}

Suppose we would like to prove $P$ is true, and we can find a contradiction $\lfalse$,
such that $\neg P \limply \lfalse$. Because $\neg P \limp \lfalse$ is true, but the
consequence is $\lfalse$, for the conditional to be true, the premise $\neg P$ must
be false, i.e., $P$ must be true.

\begin{example} \quad\par
    \textbf{Thesis}: $\exists x \in \R (x^2 + 1 = 0)$ is false.

    \begin{proof}
        We start by assuming the negation of the thesis, namely
        $\exists x \in \R (x^2 + 1 = 0)$ is true.

        Consider the fact that $\forall x_0 \in \R (x_0^2 \ge 0)$, which also implies
        $x_0^2 + 1 \ge 1$.

        Combined with the assumption, we have $x_0^2 + 1 = 0 \ge 1$, which is a contradition.

        Therefore, the negation of the thesis is false, and the thesis must be true.
    \end{proof}
\end{example}

\subsubsection{Proof by Cases}

Consider the logical equivalence:
$$
(P_1 \lor P_2 \lor \cdots \lor P_n) \limp Q \equiv (P_1 \limp Q) \land (P_2 \limp Q) \land \cdots \land (P_n \limp Q)
$$
To prove $P \limp Q$, where $P \equiv P_1 \lor P_2 \lor \cdots \lor P_n$, we can prove
$P_i \limp Q$ for each $i=1, 2, \ldots, n$.

\subsubsection{Equivalence Proof}

To prove $P \equiv Q$, we can prove both the sufficiency $P \limp Q$ and the necessity
$Q \limp P$.