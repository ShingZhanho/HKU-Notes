\section{Logic \& Proofs}

\begin{definition}[Proposition]
    A statement that can be \textbf{unambiguously} determined to be either true of false.
\end{definition}

\begin{definition}[Logical Operators]
    The commonly used logical operators are:
    \begin{itemize}
        \item \textbf{Negation}: $\lNot P$
        \item \textbf{Conjunction} (AND): $P \lAnd Q$
        \item \textbf{Disjunction} (OR): $P \lOr Q$
        \item \textbf{Exclusive OR} (XOR): $P \lXor Q$
        \item \textbf{Implication/Conditional} (if ..., then ...): $P \lImp Q$
        \item \textbf{Biconditional} (if and only if): $P \lIff Q$
    \end{itemize}
    where $P$ and $Q$ are propositions.
\end{definition}

\begin{remark}
    The truth tables for the conditional and biconditional operators are as follows:
    \begin{table}[H]
        \centering
        \begin{tabular}{cccc}
            $P$ & $Q$ & $\boldsymbol{P \lImp Q}$ & $\boldsymbol{P \lIff Q}$ \\ \hline
            F   & F   & T                     & T                     \\
            F   & T   & T                     & F                     \\
            T   & F   & F                     & F                     \\
            T   & T   & T                     & T                     \\
        \end{tabular}
    \end{table}

    Note that the ``implication'' operator is distinct from the ``implication'' used in natural
    language. $P \lImp Q$ does not contain cause-and-effect information.
\end{remark}

\begin{definition}[Sufficiency and Necessity]
    When $P \lImp Q$ is true, we say that:
    \begin{itemize}
        \item $P$ is a \textbf{sufficient} condition for $Q$.
        \item $Q$ is a \textbf{necessary} condition for $P$.
    \end{itemize}
\end{definition}

\subsection{Computing Truth Values}

The truth value of a proposition $P$ is denoted by $\lTV{P}$. For composite propositions,
we often need to simplify the expression.

\subsubsection{Boolean Algebra}

\begin{definition}[Algebraic Rules for Boolean Algebra]
    Consider two propositions $P$ and $Q$, and denote true by $1$ and false by $0$, we have:
    \begin{multicols}{2}
        \begin{enumerate}
            \item $\lTV{\lNot P} = \lTV{P} \lXor 1$
            \item $\lTV{P \lAnd Q} = \lTV{P}\lTV{Q}$
            \item $\lTV{P \lXor Q} = \lTV{P} \lXor \lTV{Q}$
            \item $\lTV{P \lOr Q} = \lTV{P} \lXor \lTV{Q} \lXor \lTV{P}\lTV{Q}$
            \item $\lTV{P \lIff Q} = \lTV{P} \lXor \lTV{Q} \lXor 1$
            \item $\lTV{P \lImp Q} = \lTV{P}\lTV{Q} \lXor \lTV{P} \lXor 1$
        \end{enumerate}
    \end{multicols}
\end{definition}

\begin{example}
    \textbf{Question}: Compute the truth values of $(P \lImp Q) \lAnd (Q \lImp P)$ as a function of
    $\lTV{P}$ and $\lTV{Q}$.

    \textbf{Solution}: Denote $x=\lTV{P}$ and $y=\lTV{Q}$, we have:
    \begin{align*}
        &\phantom{\mathrel{=}} \lTV{(P \lImp Q) \lAnd (Q \lImp P)} \\
        &= (xy \lXor x \lXor 1) \lAnd (yx \lXor y \lXor 1) \qquad\text{(Rule \#6)} \\
        &= (xy \lXor x \lXor 1)(xy \lXor y \lXor 1) \qquad\text{(Rule \#2)} \\
        &= x^2y^2 \lXor xy^2 \lXor xy \lXor x^2y \lXor xy \lXor x \lXor xy \lXor y \lXor 1 \\
        &= xy \lXor xy \lXor xy \lXor xy \lXor xy \lXor x \lXor xy \lXor y \lXor 1 \\
        &= \boxed{x \lXor y \lXor 1} \qquad\text{(}\forall b\in\{0, 1\}:(b^2 = b)\text{)}
    \end{align*}
    Note that this expression is equivalent to $P \lIff Q$ (by Rule \#5).
\end{example}

\subsubsection{Logical Equivalence}

\begin{definition}[Tautology and Contradiction]
    A \textbf{tautology} is a proposition that is always true, denoted by $T$.
    A \textbf{contradiction} is a proposition that is always false, denoted by $F$.
\end{definition}

\begin{definition}[Logical Equivalence]
    Two propositions $P$ and $Q$ are said to be logically equivalent if $P \lIff Q$ is a tautology,
    denoted as $P \equiv Q$ or $P \Leftrightarrow Q$.
\end{definition}

\begin{theorem}[Important Laws of Logical Equivalence]
    These are some important laws for simplifying composite logics:
    \begin{enumerate}
        \item \textbf{Double Negation Law}: $\neg(\neg P) \equiv Q$
        \item \textbf{Identity Laws}: $P \lAnd T \equiv P$ and $P \lOr F \equiv P$
        \item \textbf{Domination Laws}: $P \lOr T \equiv T$ and $P \lAnd F \equiv F$
        \item \textbf{Idempotent Laws}: $P \lAnd P \equiv P$ and $P \lOr P \equiv P$
        \item \textbf{Negation Laws}: $P \lAnd \neg P \equiv F$ and $P \lOr \neg P \equiv T$
        \item \textbf{Biconditional Law}: $(P \lIff Q) \equiv (P \lImp Q) \lAnd (Q \lImp P)$
        \item \textbf{Implication Law}: $(P \lImp Q) \equiv (\neg P \lOr Q)$
        \item \textbf{Contraposition Law}: $(P \lImp Q) \equiv (\neg Q \lImp \neg P)$
        \item \textbf{De Morgan's Laws}:
        \begin{enumerate}
            \item $\neg(P \lAnd Q \lAnd R \lAnd \cdots) \equiv \neg P \lOr \neg Q \lOr \neg R \lOr \cdots$
            \item $\neg(P \lOr Q \lOr R \lOr \cdots) \equiv \neg P \lAnd \neg Q \lAnd \neg R \lAnd \cdots$
        \end{enumerate}
        \item \textbf{Distributivity Laws}:
        \begin{enumerate}
            \item $P \lOr (Q \lAnd R) \equiv (P \lOr Q) \lAnd (P \lOr R)$
            \item $P \lAnd (Q \lOr R) \equiv (P \lAnd Q) \lOr (P \lAnd R)$
        \end{enumerate}
        \item \textbf{Absorption Laws}:
        \begin{enumerate}
            \item $P \lOr (P \lAnd Q) \equiv P$
            \item $P \lAnd (P \lOr Q) \equiv P$
        \end{enumerate}
    \end{enumerate}
\end{theorem}