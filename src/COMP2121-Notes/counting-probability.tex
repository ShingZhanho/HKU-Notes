\section{Counting \& Probability}

\subsection{Fundamentals of Counting}

\begin{theorem}[Product Rule] \label{thm:product-rule}
    If a procedure can be broken down into $k$ tasks, where the first task can be done in $n_1$
    ways, and for each way of doing the first task the second task can be done in $n_2$ ways,
    and so on, then the entire procedure can be done in $n_1 n_2 \cdots n_k$ ways.
\end{theorem}

\begin{theorem}[Product Rule for Finite Sets]
    For some finite sets $A_1, A_2, \ldots, A_k$, the number of \textbf{ordered list}
    $(a_1, a_2, \ldots, a_k)$ where $a_i \in A_i$ for $i=1,2,\ldots,k$ is
    \[
        \card{A_1} \cdot \card{A_2} \cdots \card{A_k} = \prod_{i=1}^k \card{A_i}
    \]
\end{theorem}

\begin{corollary}
    For some finite sets $A_1, A_2, \ldots, A_k$, if all of them are identical, say, $A$,
    then a \textbf{sequence of length $\bm{n}$ with entries from $\bm{A}$} is an ordered list
    $(a_1, a_2, \ldots, a_k)$ where $a_i \in A$ for $i=1,2,\ldots,k$. The number of such
    $n$-sequences is
    \[
        \card{A}^n
    \]
\end{corollary}

With the \nameref{thm:product-rule}, we can solve problems like:

\begin{example}[Counting Functions]
    Find the number of functions $f:A\to B$, provided that $A$ and $B$ are finite sets.

    \textbf{Solution:}
    A function $f:A\to B$ can be constructed by assigning each element in $A$ to an element in $B$.
    This can be broken down into $\card{A}$ tasks, where the $i$-th task is to assign $f(a_i)\in B$
    for some $a_i \in A$. The $i$-th task can be done in $\card{B}$ ways.

    By the \nameref{thm:product-rule}, the total number of ways to construct such function is
    \[
        \boxed{\card{B}^{\card{A}}}
    \]
\end{example}

\begin{example}[Counting Injective Functions]
    Find the number of injective functions $f:A\to B$, provided that $A$ and $B$ are finite sets.

    \textbf{Solution:}
    First, note that if $\card{A} > \card{B}$, then there is no injective function from $A$ to $B$.
    Now, suppose $\card{A} \leq \card{B}$. An injective function $f:A\to B$ can be constructed by
    several steps. First, we choose an element $a_1\in A$. There are $\card{B}$ ways to choose
    a $f(a_1)\in B$. Next, we choose another element $a_2\in A$. Since $f$ is injective,
    there are $\card{B}-1$ ways to choose $f(a_2)\in B$. Continuing this way, we can see that
    the $i$-th task can be done in $\card{B}-i+1$ ways.

    Therefore, the number of injective functions from $A$ to $B$ is
    \[
        \boxed{\card{B}(\card{B}-1)(\card{B}-2)\cdots(\card{B}-\card{A}+1) = \frac{\card{B}!}{(\card{B}-\card{A})!}}
    \]
    provided that $\card{A} \leq \card{B}$.
\end{example}

\begin{theorem}[Inclusion-Exclusion Principle]
    For some sets $A_1, A_2, \ldots, A_n$,
    \[
        \card{\bigcup_{i=1}^n A_i}
        = \sum_{i} \card{A_i}
        - \sum_{i<j} \card{A_i \cap A_j}
        + \sum_{i<j<k} \card{A_i \cap A_j \cap A_k}
        - \cdots
        + (-1)^{n+1} \card{A_1 \cap A_2 \cap \cdots \cap A_n}
    \]
\end{theorem}

\subsection{Permutations and Combinations}

\begin{definition}[Permutation]
    A \textbf{permutation} of a set $A$ is an ordered arrangement of all the elements of $A$.
    The number of permutations of a set with $n$ elements is $n!$.
\end{definition}

\begin{definition}[$r$-Permutation]
    An \textbf{$\bm{r}$-permutation} of a set $A$ is an ordered arrangement of $r$ elements of $A$.
    The number of $r$-permutations of a set with $n$ elements is denoted by
    \[
        P(n,r) = \frac{n!}{(n-r)!},\quad r\in[0,n]
    \]
\end{definition}

\begin{definition}[$r$-Combination]
    An \textbf{$\bm{r}$-combination} of a set $A$ is an unordered selection of $r$ elements of $A$.
    Thus, an $r$-combination is simply a subset of $A$ with $r$ elements. The number of
    $r$-combinations of a set with $n$ elements is denoted by
    \[
        C(n,r) = \binom{n}{r} = \frac{n!}{r!(n-r)!},\quad r\in[0,n]
    \]
    The notation $\binom{n}{r}$ is called a \textbf{binomial coefficient}.
\end{definition}

Below are some useful properties of binomial coefficients:

\begin{corollary}[Symmetry of Binomial Coefficients]
    Let $n$ and $r$ be nonnegative integers with $r \leq n$. Then
    \[
        \binom{n}{r} = \binom{n}{n-r}
    \]
\end{corollary}

\begin{corollary}[Sum of Binomial Coefficients]
    Let $n$ be a nonnegative integer. Then
    \[
        \sum_{k=0}^n \binom{n}{k} = 2^n
    \]
\end{corollary}

\begin{corollary}[Recursive Formula of Binomial Coefficients]
    Let $n>m\geq 1$ be integers. Then
    \[
        \binom{n}{m} = \binom{n-1}{m} + \binom{n-1}{m-1}
    \]
\end{corollary}

\begin{theorem}[Vandermonde's Identity]
    Let $m,n$ and $r$ be nonnegative integers with $r$ not exceeding either $m$ or $n$. Then
    \[
        \sum_{k=0}^r \binom{m}{k} \binom{n}{r-k} = \binom{m+n}{r}
    \]
\end{theorem}

\begin{definition}[Equivalence Upon Reshuffling]
    Two permutations are said to be \textbf{equivalent upon reshuffling} if they correspond
    to the same combination. For example, the permutations $(a,b,c)$, $(b,c,a)$ and $(c,a,b)$
    are equivalent upon reshuffling, since they correspond to the same combination $\{a,b,c\}$.
    The number of $k$-permutations that are equivalent upon reshuffling is $k!$.
\end{definition}