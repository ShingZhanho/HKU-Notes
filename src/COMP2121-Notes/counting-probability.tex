\section{Counting \& Probability}

\subsection{Fundamentals of Counting}

\begin{theorem}[Product Rule] \label{thm:product-rule}
    If a procedure can be broken down into $k$ tasks, where the first task can be done in $n_1$
    ways, and for each way of doing the first task the second task can be done in $n_2$ ways,
    and so on, then the entire procedure can be done in $n_1 n_2 \cdots n_k$ ways.
\end{theorem}

With the \nameref{thm:product-rule}, we can solve problems like:

\begin{example}[Counting Elements of a Cartesian Product of Sets]
    Given sets $A_1$, $A_2$, \ldots, $A_k$, find $\card{\displaystyle\prod_{i=1}^k A_i}$.
    
    \textbf{Solution:}
    The Cartesian product contains elements of the form $(a_1, a_2, \ldots, a_k)$
    where $a_i \in A_i$. Constructing such product can be broken down into $k$ tasks,
    namely choosing $a_1, a_2, \ldots, a_k$. The $i$-th task can be done in $\card{A_i}$ ways.

    By the \nameref{thm:product-rule}, the total number of ways to construct such product is
    \[
        \boxed{\card{A_1}\cdot\card{A_2}\cdots\card{A_k} = \prod_{i=1}^k \card{A_i}}
    \]
\end{example}

\begin{example}[Counting Functions]
    Find the number of functions $f:A\to B$, provided that $A$ and $B$ are finite sets.

    \textbf{Solution:}
    A function $f:A\to B$ can be constructed by assigning each element in $A$ to an element in $B$.
    This can be broken down into $\card{A}$ tasks, where the $i$-th task is to assign $f(a_i)\in B$
    for some $a_i \in A$. The $i$-th task can be done in $\card{B}$ ways.

    By the \nameref{thm:product-rule}, the total number of ways to construct such function is
    \[
        \boxed{\card{B}^{\card{A}}}
    \]
\end{example}

\begin{example}[Counting Injective Functions]
    Find the number of injective functions $f:A\to B$, provided that $A$ and $B$ are finite sets.

    \textbf{Solution:}
    First, note that if $\card{A} > \card{B}$, then there is no injective function from $A$ to $B$.
    Now, suppose $\card{A} \leq \card{B}$. An injective function $f:A\to B$ can be constructed by
    several steps. First, we choose an element $a_1\in A$. There are $\card{B}$ ways to choose
    a $f(a_1)\in B$. Next, we choose another element $a_2\in A$. Since $f$ is injective,
    there are $\card{B}-1$ ways to choose $f(a_2)\in B$. Continuing this way, we can see that
    the $i$-th task can be done in $\card{B}-i+1$ ways.

    Therefore, the number of injective functions from $A$ to $B$ is
    \[
        \boxed{\card{B}(\card{B}-1)(\card{B}-2)\cdots(\card{B}-\card{A}+1) = \frac{\card{B}!}{(\card{B}-\card{A})!}}
    \]
    provided that $\card{A} \leq \card{B}$.
\end{example}

\begin{theorem}[Inclusion-Exclusion Principle]
    For some sets $A_1, A_2, \ldots, A_n$,
    \[
        \card{\bigcup_{i=1}^n A_i}
        = \sum_{i} \card{A_i}
        - \sum_{i<j} \card{A_i \cap A_j}
        + \sum_{i<j<k} \card{A_i \cap A_j \cap A_k}
        - \cdots
        + (-1)^{n+1} \card{A_1 \cap A_2 \cap \cdots \cap A_n}
    \]
\end{theorem}

