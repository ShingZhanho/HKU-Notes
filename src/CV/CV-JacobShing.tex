%%%%%%%%%%%%%%%%%
% This is an sample CV template created using altacv.cls
% (v1.7.4, 30 Jul 2025) written by LianTze Lim (liantze@gmail.com), based on the
% CV created by BusinessInsider at http://www.businessinsider.my/a-sample-resume-for-marissa-mayer-2016-7/?r=US&IR=T
%
%% It may be distributed and/or modified under the
%% conditions of the LaTeX Project Public License, either version 1.3
%% of this license or (at your option) any later version.
%% The latest version of this license is in
%%    http://www.latex-project.org/lppl.txt
%% and version 1.3 or later is part of all distributions of LaTeX
%% version 2003/12/01 or later.
%%%%%%%%%%%%%%%%

%% Use the "normalphoto" option if you want a normal photo instead of cropped to a circle
\documentclass[10pt,a4paper,withhypeper,normalphoto]{altacv}
%% AltaCV uses the fontawesome5 and simpleicons packages.
%% See http://texdoc.net/pkg/fontawesome5 and http://texdoc.net/pkg/simpleicons for full list of symbols.

% Change the page layout if you need to
\geometry{left=1.25cm,right=1.25cm,top=1.5cm,bottom=1.5cm,columnsep=1.2cm}

% The paracol package lets you typeset columns of text in parallel
\usepackage{paracol}


% Change the font if you want to, depending on whether
% you're using pdflatex or xelatex/lualatex
% WHEN COMPILING WITH XELATEX PLEASE USE
% xelatex -shell-escape -output-driver="xdvipdfmx -z 0" mmayer.tex
\iftutex
  % If using xelatex or lualatex:
  \setmainfont{Lato}
\else
  % If using pdflatex:
  \usepackage[default]{lato}
\fi

% Change the colours if you want to
\definecolor{VividPurple}{HTML}{285399}
\definecolor{SlateGrey}{HTML}{2E2E2E}
\definecolor{LightGrey}{HTML}{666666}
% \colorlet{name}{black}
% \colorlet{tagline}{PastelRed}
\colorlet{heading}{VividPurple}
\colorlet{headingrule}{VividPurple}
% \colorlet{subheading}{PastelRed}
\colorlet{accent}{VividPurple}
\colorlet{emphasis}{SlateGrey}
\colorlet{body}{LightGrey}

% Change some fonts, if necessary
% \renewcommand{\namefont}{\Huge\rmfamily\bfseries}
% \renewcommand{\personalinfofont}{\footnotesize}
% \renewcommand{\cvsectionfont}{\LARGE\rmfamily\bfseries}
% \renewcommand{\cvsubsectionfont}{\large\bfseries}

% Change the bullets for itemize and rating marker
% for \cvskill if you want to
\renewcommand{\cvItemMarker}{{\small\textbullet}}
\renewcommand{\cvRatingMarker}{\faCircle}
% ...and the markers for the date/location for \cvevent
% \renewcommand{\cvDateMarker}{\faCalendar*[regular]}
% \renewcommand{\cvLocationMarker}{\faMapMarker*}


% If your CV/résumé is in a language other than English,
% then you probably want to change these so that when you
% copy-paste from the PDF or run pdftotext, the location
% and date marker icons for \cvevent will paste as correct
% translations. For example Spanish:
% \renewcommand{\locationname}{Ubicación}
% \renewcommand{\datename}{Fecha}

% Custom commands on top of the template
\newcommand{\cvicontag}[2]{\cvtag{\raisebox{-.4ex}{\color{accent}#1} \color{body}#2}}

% make \cvachievment compatible with \simpleicon
\newcommand{\cvachievementAlt}[3]{%
  \begin{tabularx}{\linewidth}{@{}p{2em} @{\hspace{1ex}} >{\raggedright\arraybackslash}X@{}}
  \multirow{2}{*}{\,\,\Large\color{accent}#1} & \bfseries\textcolor{emphasis}{#2}\\
  & #3
  \end{tabularx}%
  \smallskip
}

\begin{document}
\name{ZHAN HO JACOB SHING}
\tagline{HKU Beng(CompSc) Student}

\photoR{2.5cm}{portrait}
\personalinfo{%
  \email{jacobszh@connect.hku.hk}
  \phone{+852 5482 9448}
  \location{Hong Kong}\\
  \linkedin{jacobszh}
  \github{ShingZhanho} 

  %% You can add your own arbitrary detail with
  %% \printinfo{symbol}{detail}[optional hyperlink prefix]
  % \printinfo{\faPaw}{Hey ho!}
  %% Or you can declare your own field with
  %% \NewInfoFiled{fieldname}{symbol}[optional hyperlink prefix] and use it:
  % \NewInfoField{gitlab}{\faGitlab}[https://gitlab.com/]
  % \gitlab{your_id}
	%%
  %% For services and platforms like Mastodon where there isn't a
  %% straightforward relation between the user ID/nickname and the hyperlink,
  %% you can use \printinfo directly e.g.
  % \printinfo{\faMastodon}{@username@instace}[https://instance.url/@username]
  %% But if you absolutely want to create new dedicated info fields for
  %% such platforms, then use \NewInfoField* with a star:
  % \NewInfoField*{mastodon}{\faMastodon}
  %% then you can use \mastodon, with TWO arguments where the 2nd argument is
  %% the full hyperlink.
  % \mastodon{@username@instance}{https://instance.url/@username}
}

\makecvheader

%% Depending on your tastes, you may want to make fonts of itemize environments slightly smaller
\AtBeginEnvironment{itemize}{\small}

%% Set the left/right column width ratio to 6:4.
\columnratio{0.6}

% Start a 2-column paracol. Both the left and right columns will automatically
% break across pages if things get too long.
\begin{paracol}{2}

\cvsection{Experience}

\cvevent{Student Teaching Associate}{The University of Hong Kong}{Sept 2025 -- Ongoing}{Hong Kong}
\begin{itemize}
  \item Led weekly tutorials for 30+ students for the course COMP1117 (Computer Programming) to help students develop skills in the Python programming language
\end{itemize}

\divider

\cvevent{Programme Associate}{The University of Hong Kong}{Jun 2025 -- Aug 2025}{Hong Kong}
\begin{itemize}
\item Arranged and organised logistics for the HKU Summer Institute 2025 by creating information systems with the use of AI technologies
\item Facilitated communication between different departments of the university to organise student activities
\end{itemize}

\divider

\cvevent{Document Handler}{Hong Kong Examinations and Assessment Authority}{Apr 2024 -- May 2024}{Hong Kong}

\begin{itemize}
\item Handled classfied documents for the authority and facilitated logistics of documents
\item Facilitated the grading process of the 2024 HKDSE examinations by using information systems
\end{itemize}

\cvsection{Awards}

\cvachievement
  {\faTrophy}
  {Y S and Christabel Lung Undergraduate Scholarship for Engineering Students}
  {2024/25 Academic Year}


\cvsection{Projects}

\cvachievement{\faPython}{SpeedSlide}
{
https://github.com/ShingZhanho/ENGG1330-Project-24Fall
\begin{itemize}
  \item Course project, overall grade and all components A+
  \item A text-based game implementing the Sliding Puzzle game
  \item Implemented in pure Python, with a TUI (text user interface) library built from scratch,
        leveraging OOP concepts
\end{itemize}
}

\divider

\cvachievementAlt{\simpleicon{cplusplus}}{SHOOT!}{
https://github.com/ShingZhanho/ENGG1340-Project-25Spring
\begin{itemize}
  \item Course project, overall grade 19.3/20
  \item A text-based 2D arena shooter game implemented in C++ with the FTXUI library
  \item Event driven architecture, further reinforced OOP concepts, multi-threading,
        memory management
\end{itemize}
}

% use ONLY \newpage if you want to force a page break for
% ONLY the currentc column
% \newpage

% %% Switch to the right column. This will now automatically move to the second
% %% page if the content is too long.
\switchcolumn


\cvsection{Education}

\cvevent{B.Eng.\ in Computer Science}{The University of Hong Kong}{Sept 2024 -- Ongoing}{}
\begin{itemize}
  \item CGPA: 3.87/4.30
\end{itemize}

\divider

\cvevent{M.B., B.S.\ {\footnotesize (Incomplete \& Voluntarily Withdrawn)}}{The University of Hong Kong}{Sept 2023 -- Mar 2024}{}
\begin{itemize}
  \item Pass in all assessment components (GPA system not applicable to the medicine programme)
  \item Withdrawn and transffered to pursue his true passion in Computer Science
\end{itemize}

\divider

\cvevent{HKDSE}{WEO Chang Pui Chung Memorial School}{Sept 2017 -- Jun 2023}{}
\begin{itemize}
  \item ICT: Level 5**
  \item Chinese, English, Mathematics (Compulsory),
  Chemistry: Level 5*
  \item Mathematics (Extended Module II): Level 5
  \item Liberal Studies: Level 4
\end{itemize}

\cvsection{Strengths}

% Don't overuse these \cvtag boxes — they're just eye-candies and not essential. If something doesn't fit on a single line, it probably works better as part of an itemized list (probably inlined itemized list), or just as a comma-separated list of strengths.

\textbf{\color{accent}Personalities}
\cvtag{Hard-working}
\cvtag{Motivated}
\cvtag{Quick Learner}

\divider

\textbf{\color{accent}Technical Skills}
\cvicontag{\simpleicon{dotnet}}{C\#}
\cvicontag{\simpleicon{python}}{Python}
\cvtag{\color{accent}\LaTeX\hspace*{.5ex}\color{body}LaTeX}
\cvicontag{\kern-.8ex\raisebox{.4ex}{\faJava}\hspace*{-.5ex}}{Java}
\cvicontag{\kern-.8ex\raisebox{.4ex}{\faGit*}\hspace*{-.5ex}}{Git \& GitHub}
\cvicontag{\kern-.8ex\raisebox{.3ex}{\faSwift}\hspace*{-.5ex}}{Swift \& SwiftUI}
\cvicontag{\simpleicon{github}}{GitHub Actions}
\cvicontag{\simpleicon{cplusplus}}{C++}

\cvsection{Languages}

\textcolor{emphasis}{\textbf{Cantonese \& Mandarin}}\hfill Native

\divider

\textcolor{emphasis}{\textbf{English}}\hfill Professional

\divider

\textcolor{emphasis}{\textbf{French}}\hfill Elementary

\begin{itemize}
  \item Currently minoring in French, A+ in both FREN1001 and FREN1002
  \item DELF A2 (\textit{to be taken in Nov 2025})
\end{itemize}

\end{paracol}

\end{document}
