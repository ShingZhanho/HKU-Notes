\question \textbf{Questions from Quiz 1}

\begin{parts}

    \part Bo makes 3 axes or 2 cups daily. What is her opportunity cost for a cup?

    \begin{oneparchoices}
        \choice 1.5
        \correctchoice 1.5 axes
        \choice $\frac{2}{3}$
        \choice $\frac{2}{3}$ axes
        \choice $\frac{2}{3}$ cups
    \end{oneparchoices}

    \begin{solution}
        The answer is 1.5 axes. She could either spend a day making 3 axes or 2 cups. In one-half of a day, she could either make $\frac{3}{2}$ axes or 1 cup.
    \end{solution}

    \part How was the invention of fire and cooking most advantageous to the evolution of hunter-
gatherer societies?

    \begin{choices}
        \choice Made self-sufficiency possible for hunter-gatherer societies.
        \choice Encouraged both men and women to participate in hunting equally.
        \choice Allowed hunter-gatherer societies to hunt and gather at night.
        \correctchoice Encouraged specialisation by gender and allowed the sharing of food.
        \choice Allowed hunter-gatherer societies to migrate north to colder regions.
    \end{choices}

    \begin{solution}
        Cooking predisposed humans to the exchange of different kinds of foods.
        As economist Haim Ofek stated, ``fire itself is hard to start but easy to share; likewise, cooked food is hard to make but easy to share.''
        Within hunter-gatherer societies, fire and cooking gave rise to the first division of labour and speialisation, where women gathered and cooked, and men hunted.
    \end{solution}

    \part Small, isolated populations tend to

    \begin{choices}
        \choice concentrate on mining.
        \choice suffer fewer wars, allowing more resources to be spent on technological development.
        \correctchoice lose abilities.
        \choice diversify their abilities.
        \choice focus on a few skills or products, thus becoming expert beyond larger populations.
    \end{choices}

    \begin{solution}
        In Chapter 2, Ridley claimed about the Tasmanians, ``Isolation - self-sufficiency - caused the shrivelling of their technology.''
    \end{solution}

    \part In recent decades, economic equality has

    \begin{choices}
        \choice decreased globally and increased within the most populous countries.
        \choice not changed in a consistent direction over time.
        \choice decreased globally and decreased within the most populous countries.
        \correctchoice increased globally and decreased within the most populous countries.
        \choice increased globally and increased within the most populous countries.
    \end{choices}

    \begin{solution}
        Ridley said that ``while inequality has increased within some countries, globally it has been falling.
        The recent enrichment of China and India has increased inequality within those countries by making
        the income of the rich grow faster than that of the poor - an income gap is an inevitable
        consequence of an expanding economy. Yet the global effect of the growth of China and India has
        been to reduce the difference between rich and poor worldwide.''
    \end{solution}

    \part What was likely the first specialization of labor?

    \begin{choices}
        \choice Children sewed clothes with needle and thread, and adults grew fibers for thread.
        \choice Particular adults cut stones to construct buildings.
        \choice Children made clay objects, and adults laid bricks.
        \correctchoice Children and women gathered, and men hunted.
        \choice Individuals became copper smelters.
    \end{choices}

    \begin{solution}
        Matt Ridley said in Chapter 2, ``The first and deepest division of labour is the sexual one.
        It is an iron rule documented in virtually all foraging people that `men hunt, women and children gather'.''
    \end{solution}

    \part What is the collective brain?

    \begin{choices}
        \choice The human mind's toolkit of abilities allowing an individual to perform a wide array of tasks
        \choice A Marxist concept of communal intellectual work to perform a specific commercial or industrial job, analogous to collective farms
        \choice The human mind's tendency to acquire knowledge, allowing the development of civilization
        \choice The combination of multiple computers to allow applications such as cloud computing or artificial intelligence
        \correctchoice The set of many people who contribute bits of knowledge that can make goods and services
    \end{choices}

    \begin{solution}
        In Chapter 1 of The Rational Optimist, Ridley said, ``This is what I mean by the collective brain.
        As Friedrich Hayek first clearly saw, knowledge `never exists in concentrated or integrated form but
        solely as the dispersed bits of incomplete and frequently contradictory knowledge which all the
        separate individuals possess'.''
    \end{solution}

    \part Matt Ridley argues that gains in prosperity might best be seen by

    \begin{choices}
        \choice gains in self-sufficiency of the average person,
        \choice increases in pollution.
        \correctchoice a reduction in the amount of work it takes to acquire a given item.
        \choice a reduction in the difference between what the richest people have as compared to the poorest.
        \choice increases in the cost of housing.
    \end{choices}

    \begin{solution}
        Matt Ridley said, ``This is what prosperity is: the increase in the amount of goods or services you can earn with the same amount of work.''
    \end{solution}

    \part Once hominids learned how to make Acheulean hand axes,

    \begin{choices}
        \choice other stone tools appeared in rapid succession.
        \choice the bronze age quickly followed.
        \choice stone tools disappeared from the archeological record.
        \correctchoice tool development stagnated for a long time.
        \choice Tyrannian foot spears were soon invented.
    \end{choices}

    \begin{solution}
        Ridley says at the start of Chapter 2 that hominids made the Acheulean biface almost
        unchanged for about a million years, and then, after a slight change, another 600,000 years
        with almost no change. Thus, the answer is that tool development stagnated for a long time.
    \end{solution}

    \part Which does Matt Ridley claim is unique to humans?

    \begin{oneparchoices}
        \choice Using tools
        \correctchoice Barter
        \choice Reciprocity
        \choice Cannibalism
        \choice Violence by groups
    \end{oneparchoices}

    \begin{solution}
        Ridley said that ``There is strikingly little use of barter in any other animal species." Later he
        said that "this appetite for barter, had somehow appeared in our African ancestors some
        time before 100,000 years ago.''
    \end{solution}

    \part To explain the theme of “The Rational Optimist”, Matt Ridley makes the analogy that sex is to biological evolution as

    \begin{choices}
        \choice a hand axe is to a computer mouse.
        \choice optimism is to pessimism.
        \choice language is to big brains.
        \correctchoice exchange is to cultural development.
        \choice Neanderthals are to modern humans.
        Feedback
    \end{choices}

    \begin{solution}
        In the Prologue to ``The Rational Optimist'', Matt Ridley says, ``Exchange is to cultural evolution as sex is to biological evolution.''
    \end{solution}

    \part Compared to the distant past, each person is, on average worldwide,

    \begin{choices}
        \correctchoice less often a victim of violence.
        \choice less prosperous.
        \choice less happy.
        \choice about as prosperous.
        \choice more self-sufficient.
    \end{choices}

    \begin{solution}
        ``Less often a victim of violence.;; Ridley states that ``violence was a chronic and ever-present
        threat...The warfare death rate of 0.5 per cent of the population per year that was typical of
        many hunter-gatherer societies would equate to two billion people dying during the
        twentieth century (instead of 100 million).''
    \end{solution}

    \part Compared to now, Matt Ridley thinks the environment in a century, like the condition of humanity itself, will be

    \begin{choices}
        \choice a hellish nightmare that will make Venus look like paradise
        \choice the same
        \choice rather worse
        \choice non-existent because we will destroy Earth
        \correctchoice far better
    \end{choices}

    \begin{solution}
        The Prologue of The Rational Optimist says, ``It will not be
        easy, but it is perfectly possible, indeed probable, that in the year 2110, a century after this book is
        published, humanity will be much, much better off than it is today, and so will the ecology of the planet
        it inhabits.''
    \end{solution}

    \part According to Richard Dawkins, what is the main role of a living individual?

    \begin{choices}
        \correctchoice To act as a disposable survival machine for carrying genes
        \choice To be an essentially immortal messenger of coded information
        \choice To contribute to the stability of the environment of other individuals and species
        \choice To cooperate with other individuals
        \choice To compete with other individuals
    \end{choices}

    \begin{solution}
        In the video of Dawkins in Lecture 1, he said,
        ``An individual organism is a throwaway
        survival machine for the self-replicating, coded information which it contains.''
    \end{solution}

    \part According to Sapiens, the Cognitive Revolution was characterized by emergence of all of the following abilities, EXCEPT

    \begin{choices}
        \choice Being readily able to gossip
        \choice Innovative collaboration among a small group of individuals
        \correctchoice Being able to warn other individuals about imminent dangers
        \choice Telling of myths, legends, and religions
        \choice Sharing of information about one's social relationships
    \end{choices}

    \begin{solution}
        The Cognitive Revolution, which happened between 70,000-30,000 years ago,
        enabled new ways of thinking and communication among Homo sapiens, distinguishing the human
        language and those of other hominids. The sharing of information about one's current physical
        surroundings, such as warning one another in the presence of a predator, was key for survival and
        social cooperation, but this was a shared trait among other animals and archaic humans.
    \end{solution}

    \part Matt Ridley thinks that the best way to get happier is:

    \begin{oneparchoices}
        \correctchoice Political and social liberty
        \choice Social Security
        \choice Wealth
        \choice Scientific Innovations
        \choice Longevity
    \end{oneparchoices}

    \begin{solution}
        Ridley argues that wealth is not the best route to happiness. ``Social and political liberation is far more effective.''
    \end{solution}

    \part Which one of the following is not an example of an imagined reality?

    \begin{oneparchoices}
        \choice The Japanese Yen
        \choice Human rights
        \choice The United States
        \choice Google
        \correctchoice Climate change
    \end{oneparchoices}

    \begin{solution}
        Yuval N. Harari states that the entities that bring us together, such as the notion of
        gods, money, nations, corporations, and laws of justice, are all a product of our common imagination.
    \end{solution}

    \part Which individual can make a computer mouse?
    
    \begin{choices}
        \choice An assembly line worker
        \choice A designer
        \correctchoice Nobody
        \choice Any of these: engineer, designer, or assembly line worker
        \choice An engineer
    \end{choices}

    \begin{solution}
        The answer is nobody. In the TED Talk where he introduced the ideas in the book,
        Ridley said that nobody knows how to make a mouse. A CEO knows how to run a company and an
        engineer can design a mouse, but no one person can do it alone. It requires many people, maybe
        millions, to contribute to making a mouse.
    \end{solution}

    \part According to Ridley, which of the following has been the most effective factor causing people to become happier over time?

    \begin{choices}
        \choice We are getting more interdependent.
        \choice Poverty has reduced drastically in the past few decades.
        \choice Our average life expectancy increased over time.
        \choice Combined efforts by the government keep housing prices high.
        \correctchoice We have more free choices in life.
    \end{choices}

    \begin{solution}
        Ridley stated,
        ``It is the increase in free choice since 1981 that has been responsible for
        the increase in happiness... (Ridley, 2011, under sub-theme Happiness of Chapter 1)''
    \end{solution}

    \part According to the Flynn effect, which of the following generations would have had the highest average score in an IQ test if all of them took it when they were 20?

    \begin{choices}
        \correctchoice Generation Z (those born from 1997 onwards)
        \choice Generation X (those born from 1965 to 1980)
        \choice Boomer (those born from 1946 to 1964)
        \choice Generation Y (those born from 1981 to 1996)
        \choice We can't determine from this information, as an individual's IQ varies based on multiple biological and environmental factors.
    \end{choices}

    \begin{solution}
        The Flynn effect, as explained in The Rational Optimist, describes the general increase
        of IQ among populations throughout the 20th century (i.e., ``... steady, progressive and ubiquitous improvement in the average IQ scores...''
        (Ridley, 2011, p. 19))
    \end{solution}

    \part Which of the following does Matt Ridley think contributed most to the development of human civilization?

    \begin{oneparchoices}
        \choice Imitation
        \choice Language
        \choice Standing upright
        \choice Evolution of a large brain
        \correctchoice Specialization
    \end{oneparchoices}

    \begin{solution}
        Matt Ridley said, ``The cumulative accretion of knowledge by specialists that allows us each to consume more and more different
        things by each producing fewer and fewer is, I submit, the central story of humanity...This is history's greatest theme: the
        metastasis of exchange, specialisation and the invention it has called forth, the `creation' of time." Large brains or language were
        likely not the key because Neanderthals may have talked and had brains at least as large as homo sapiens, yet Neanderthal
        civilization didn't advance because they didn't specialise.
    \end{solution}

    \part According to Richard Dawkins, what is the main role of a living individual?

    \begin{choices}
        \choice To cooperate with other individuals
        \choice To be an essentially immortal messenger of coded information
        \choice To contribute to the stability of the environment of other individuals and species
        \correctchoice To act as a disposable survival machine for carrying genes
        \choice To compete with other individuals
    \end{choices}

    \begin{solution}
        In the video of Dawkins in Lecture 1, he said, ``An individual organism is a throwaway survival machine for the self-replicating, coded information which it contains.''
    \end{solution}

    \part Matt Ridley argues that Homo sapiens (which Ridley calls \textit{Homo dynamicus}) advanced technologically

    \begin{choices}
        \choice when they lived in relative isolation, not threatened by people they did not know.
        \correctchoice when they had more opportunities to interact with other people outside their living groups.
        \choice as a result of a specific mutation in the FOXP2 gene.
        \choice due to changing climate, putting a premium on adaptability.
        \choice when they started eating a diet rich in both fruit and vegetables, developing larger brains as a result.
    \end{choices}

    \begin{solution}
        Ridley refuted theories that a change in climate or in a gene caused technological advancement. Instead he claimed that the cause
        was that homo sapiens ``started, for the very first time, to exchange things between unrelated, unmarried individuals; to share,
        swap, barter and trade.''
    \end{solution}

    \part What is true of comparative advantage?

    \begin{choices}
        \correctchoice If I take longer than you to make an object, I might still benefit by producing and trading it with you.
        \choice If I don't want something, I have a comparative advantage in producing it.
        \choice Self-sufficiency shows some people have a comparative advantage in everything they do.
        \choice The great apes make use of comparative advantage when engaging in reciprocal grooming.
        \choice The excavations at Boxgrove Quarry illustrate that it is not always profitable to have a comparative advantage.
    \end{choices}

    \begin{solution}
        This may seem paradoxical, but as long as there is a difference in opportunity cost between two people, they may benefit by
        trading. In the fish-banana example in Lecture 2, Ann is more efficient at both farming and fishing than Bob, yet Bob still gains by
        trading. As I said in the lecture, ``Even though Ann is better at both fishing and banana harvesting, she can gain by trading with Bob,
        and so could Bob.'' He, in effect, gains from trade by sharing her efficiency.
        
        The great apes answer is wrong because they do not necessarily have a comparative advantage; they may be equally good in
        grooming. Reciprocation (like for like) is not barter.
        
        The Boxgrove Quarry answer is wrong. All I mentioned about Boxgrove Quarry was that hominids there used tools and buried their
        dead: nothing about comparative advantage or trade.
    \end{solution}

    \part How was the invention of fire and cooking most advantageous to the evolution of hunter-gatherer societies?

    \begin{choices}
        \choice Facilitated the development of culture and human civilization with new dishes and cuisines.
        \choice Allowed hunter-gatherer societies to hunt more effectively.
        \correctchoice Encouraged specialisation by gender and allowed the sharing of food.
        \choice Encouraged both men and women to participate in hunting equally.
        \choice Allowed hunter-gatherer societies to migrate north to colder regions.
    \end{choices}

    \begin{solution}
        Cooking predisposed humans to the exchange of different kinds of foods. As economist Haim Ofek stated, ``fire itself is hard to
        start but easy to share; likewise, cooked food is hard to make but easy to share.'' Within hunter-gatherer societies, fire and cooking
        gave rise to the first division of labour and specialisation, where women gathered and cooked, and men hunted.
    \end{solution}

    \part Why did civilization regress in Tasmania after about 10,000 years ago?

    \begin{choices}
        \choice Hunting drove their main food source to extinction
        \choice Genetic mutations that degraded health
        \choice Repeated attacks from outsiders
        \correctchoice Self-sufficiency
        \choice Depletion of resources
    \end{choices}

    \begin{solution}
        In Chapter 2 of The Rational Optimist, Ridley wrote, ``Isolation - self-sufficiency - caused the shrivelling of their technology.''
        Rising sea levels caused by the end of the ice age cut-off Tasmania from the Australian mainland. This isolation led to self-sufficiency for the hunter-gatherer tribes,
        which experienced stagnation and technological regression because they lacked the numbers to sustain their existing technology.
    \end{solution}

    \part What was the most likely cause of human expansion and success in the last 200,000 years?

    \begin{choices}
        \choice The emergence of reciprocity, which only exists in modern humans
        \choice Invention of the stone biface axe
        \correctchoice Exchange among unrelated, unmarried people, allowing specialisation
        \choice A mutation or mutations allowing speech
        \choice Climate change
    \end{choices}

    \begin{solution}
        Ridley asserted ``that the answer lies not in climate, nor genetics, nor in archaeology, nor even entirely in `culture',
        but in economics. Human beings had started to do something to and with each other that in effect began to build a collective intelligence.
        They had started, for the very first time, to exchange things between unrelated, unmarried individuals''
    \end{solution}

\end{parts}