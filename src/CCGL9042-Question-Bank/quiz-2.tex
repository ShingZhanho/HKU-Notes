\question \textbf{Questions from Quiz 2}

\begin{parts}

    \part What is a way to avoid some tragedies of the commons?

    \begin{choices}
        \choice Nepotism
        \choice Establishing a stronger welfare state
        \correctchoice Privatization
        \choice Specialisation
        \choice An enforced agreement among all parties that the resource should remain freely accessible to all of them.
    \end{choices}

    \begin{solution}
        Privatization could avoid some tragedies of the commons by limiting usage of a resource to a sustainable level.
    \end{solution}

    \part What did both Google and J.P. Morgan say was most important to them?

    \begin{oneparchoices}
        \correctchoice Trust
        \choice Freedom
        \choice Innovation
        \choice Intelligence
        \choice A zero-sum game
    \end{oneparchoices}

    \begin{solution}
        In Chapter 3, Ridley says, ``Google's code of conduct echoes Morgan: `Trust is the foundation upon which our success and prosperity rest…' ''.
    \end{solution}

    \part Which of the following statements is most accurate?

    \begin{choices}
        \choice Comparative advantages result in unilaterally beneficial trade.
        \correctchoice Self-interest can create public good through trade.
        \choice According to Michael Shermer, poverty is the barrier to charity.
        \choice Altruism is important for trade.
        \choice Conditions for TIT-FOR-TAT strategy success in the iterated prisoners' dilemma are recognizing cooperation, remembering interaction, retaliating people, and rewarding cheating.
    \end{choices}

    \begin{solution}
        The correct answer is that self-interest creates public good through trade. If everyone is acting in his
        own interest, the collective result tends to be mutually beneficial. See lecture slide, ``Doing good by
        being selfish?''
        
        Comparative advantages do not generally result in unilaterally beneficial trade. They lead to
        mutually beneficial trade. See lecture slide, ``The roots of prosperity.''
        
        According to Michael Shermer, poverty is not the barrier to charity. The book says, ``As Michael
        Shermer comments, `Poverty is not a barrier to charity, but welfare is.' ''
        
        The four Res are repeating interaction, remembering people, recognizing cheating and rewarding
        cooperation. See lecture slide, ``Conditions for cooperation.''
        
        Altruism is not important for trade. Altruism is good for people but not necessary for trade. See
        lecture slide, ``Altruism.''
    \end{solution}

    \part Which of the following best facilitate trade?

    \begin{choices}
        \choice Markets, a medium of exchange, tariffs, and trust
        \choice A medium of exchange, self-sufficiency, specialisation, and trust
        \correctchoice A medium of exchange, justice, middlemen, and trust
        \choice Altruism, markets, specialisation, mobility, and trust
        \choice Altruism, a medium of exchange, self-sufficiency, and trust
    \end{choices}

    \begin{solution}
        A medium of exchange, justice, middlemen, and trust help trade take place. Altruism, self-
        sufficiency, or tariffs don't facilitate trade.
    \end{solution}

    \part Which one of the following statements is most accurate according to Hamilton's rule?

    \begin{choices}
        \correctchoice Hamilton's rule says that I would let my grandfather die to save three of my aunts.
        \choice Hamilton's rule explains the phenomenon of selflessly donating living organs for strangers.
        \choice Hamilton's rule states that I should always challenge one who threatens my reputation, even if the challenge risks my life.
        \choice Hamilton's rule explains why we should always think of our own individual interests and be selfish.
        \choice Under Hamilton's rule, I should sacrifice my own life in exchange for anyone from my cousin's family.
    \end{choices}

    \begin{solution}
        Hamilton's rule states that an action benefits one's gene if the cost of the action to one's
        fitness is less than the benefit to other individuals, multiplied by their relatedness to the
        actor (``Hamilton's rule'' slide). Since my relatedness to my grandfather is $\frac{1}{4}$, which is less than my
        relatedness to three aunts ($3\times\frac{1}{8}$), Hamilton's rule says that I would save 3 aunts instead of my
        grandfather.
    \end{solution}

    \part Which is a major implication one could infer from the Elephant Chart on global income inequality from 1988 to 2008?

    \begin{choices}
        \choice The poor have gotten poorer because the ``trunk'' of the Elephant Chart tilts upwards.
        \correctchoice Income inequality, from a global perspective, fell as incomes in the developing world rose, reducing the gap with incomes in developed countries.
        \choice There has been major growth in real incomes in all countries around the world.
        \choice Crises of income inequality have arisen in well-developed countries due to the stagnation of real wages and the hollowing-out of the richest 1\% in these countries.
        \choice All Asian countries, especially Japan, have had drastic income growth as a result of the phenomenon of ``Resurgent Asia'', whilst European countries and the United States struggled to perform well economically.
    \end{choices}

    \begin{solution}
        Rising incomes among most of the poor people of the world have reduced the gap between rich and
        poor, thus income inequality, from a global perspective, fell as incomes in the developing world rose,
        reducing the gap with incomes in developed countries. The “Resurgent Asia" answer is wrong
        because it cannot be directly inferred from the graph, and because income did not grow in all Asian
        countries. Lower- and middle-class Japanese did not do well (at 1:26 of the Video).
    \end{solution}

    \part TIT-FOR-TAT success in society doesn't need

    \begin{choices}
        \correctchoice returning profits.
        \choice rewarding cooperation.
        \choice remembering people.
        \choice repeating interaction.
        \choice recognizing cheating.
    \end{choices}

    \begin{solution}
        TIT-FOR-TAT success in society needs 4 Re: repeating interaction, remembering people, recognizing cheating, and rewarding cooperation.
    \end{solution}

    \part Which of these will least likely raise oxytocin levels?

    \begin{oneparchoices}
        \choice Prayer
        \choice Massage
        \correctchoice Anxiety
        \choice Group dancing
        \choice Chatting with friends
    \end{oneparchoices}

    \begin{solution}
        Oxytocin suppresses the activity of the amygdala, the organ that expresses fear. Thus anxiety is unlikely to raise oxytocin levels.
    \end{solution}

    \part What does Matt Ridley feel about markets?

    \begin{choices}
        \correctchoice A system of individual self-interest that drives the creation of trust
        \choice The tendency to become a top-down system monopolised by large corporations that deliver enormous cost savings to consumers
        \choice The tendency for competition to promote altruistic behaviour
        \choice The erosion of generosity because selfishness is the fundamental principle for the creation of markets
        \choice A necessary evil that allows people to become wealthy despite many drawbacks such as loss of jobs
    \end{choices}

    \begin{solution}
        A system of individual self-interest that drives the creation of trust, as Ridley discussed in the
        section ``If trust makes markets work, can markets generate trust?''.
    \end{solution}

    \part Which is correct about oxytocin?

    \begin{choices}
        \choice Oxytocin is a hormone that exists only in humans.
        \correctchoice Oxytocin is a hormone that raises trust or warm feelings.
        \choice The trust game shows that oxytocin has an effect on reciprocity.
        \choice Oxytocin does not affect fear.
        \choice Oxytocin is a hormone that makes you feel angry.
    \end{choices}

    \begin{solution}
        According to Chapter 3 of ``The Rational Optimist'', in the section entitled ``The trust juice'',
        the trust game experiment proves that oxytocin improves trusting, even more than risk-taking.
        
        Oxytocin does not make you feel angry because the book explains that oxytocin is a
        ``chemical that evolution uses to make mammals feel good about each other''. Oxytocin is
        common to all mammals, not only humans. The trust game experiment also shows that
        oxytocin doesn't affect the trustee's decision on back transfer, thus explaining why oxytocin
        does not have an effect on reciprocity. Oxytocin can suppress ``the activity of the amygdala,
        the organ that expresses fear.''
    \end{solution}

    \part Which of these is LEAST likely to help facilitate trade?

    \begin{choices}
        \choice Both retaliation and forgiveness
        \choice Reciprocal altruism
        \correctchoice Anonymous altruism
        \choice Self-interest
        \choice Ownership
    \end{choices}

    \begin{solution}
        A lecture slide entitled “Altruism” said that helping even without reciprocation is not important for trade.
    \end{solution}

    \part Which is the most accurate statement regarding Harari's arguments on the imagined order?

    \begin{choices}
        \choice Our interests and pleasures exist independently of the imagined order.
        \choice The imagined order is the prime command from a supreme being.
        \correctchoice The natural order, unlike the imagined order, does not require believers.
        \choice Sapiens live in only one collective imagined order.
        \choice There is actually no order since entropy leads everything to chaos.
    \end{choices}

    \begin{solution}
        The natural order does not require believers since it does not cease to exist even if
        people stop believing in it. It is objective and is independent from human consciousness and beliefs.

        The imagined order is inter-subjective, but cannot be altered by a single individual.

        Sapiens live within layers of many imagined orders. Examples are the law, money, the notion of gods, and nations.
        
        Our interests and pleasures are shaped and programmed by the imagined order. They are not
        independent desires.
        
        Harari argues that ``there is no way out of the imagined order''.
        Running awaytowards freedom only leads to a ``bigger prison'', aka a larger imagined reality.
    \end{solution}

    \part What is a conclusion from the ultimatum, dictator, and blinded dictator games?

    \begin{choices}
        \choice The data are strongly consistent with the theory of purely altruistic behaviour by natural selection.
        \choice All people are nice and trustworthy.
        \choice People are inherently mistrustful and will always treat each other unfairly unless they are closely related.
        \choice People's decision-making seems to be driven by curiosity.
        \correctchoice People in interconnected societies have a systemic bias toward cooperation and fairness.
    \end{choices}

    \begin{solution}
        People in interconnected societies have a systemic bias toward cooperation and
        fairness. Ridley commented ``that the more people are immersed in the collective brain of the modern
        commercial world, the more generous they are.''
        
        The 3 games show that people may act nice. That
        behavior can be based on fear, which in turn can be driven by image-consciousness.
    \end{solution}

    \part Which of these is the most successful strategy for the iterated prisoner's dilemma?

    \begin{choices}
        \choice Cooperate after opponent defects, and defect after opponent cooperates
        \choice Defect at random times
        \choice Alternate between cooperating and defecting
        \choice Express altruism, be nice, and forgive
        \correctchoice Be nice, retaliate, and forgive
    \end{choices}

    \begin{solution}
        The best strategy is generally tit-for-tat, which involves being nice, retaliating, and forgiving (``Evolutionary simulation'' slide).
    \end{solution}

    \part Which of the following best describes the broadest sense of what Harari means by `Building Pyramids'?

    \begin{choices}
        \choice The emergence of cultural systems during the Agricultural Revolution
        \choice The illustration of inequality among homo sapiens
        \choice The erection of huge stone monuments by Egyptian, Mayan, and other civilizations
        \correctchoice The creation of structures, societal hierarchies, and orders
        \choice The hierarchy of human needs and desires
    \end{choices}

    \begin{solution}
        In Sapiens Chapter 6, Harari explores human progress during the Agricultural
        Revolution. He highlights the creation of imagined orders and social systems (ie. social pyramids)
        during this period, which also comes along with the building of physical structures, such as lawns and
        houses. The term `Building Pyramids' illustrates this.
    \end{solution}

    \part Which of the following factors would be most likely to deter long-term cooperation?

    \begin{choices}
        \choice Altruism
        \correctchoice Anonymity
        \choice An ability to remember
        \choice Positive experience of commerce and trade
        \choice Release of more oxytocin
    \end{choices}

    \begin{solution}
        The correct answer is anonymity because it blocks the ability to remember trade partners for building long term cooperation and
        future reciprocation. Anonymity would eliminate the risk to one's reputation that would arise from betraying one's trading partner,
        and the blinded and unblinded dictator games demonstrate that reputation is an important factor in trade.
        The other options would not deter cooperation. An increase in blood oxytocin level promotes trust (Chapter 3). Commerce and
        trade are already a kind of cooperation. With repeating interaction, trust can be built and promote further cooperation (slide:
        ``Conditions for cooperation''). An ability to remember encourages cooperation because it encourages trust after positive
        interactions. Altruism would increase cooperation.
    \end{solution}

    \part Which is NOT an example of the tragedy of the commons?

    \begin{choices}
        \choice Logging of old-growth forests that leads to the loss of livelihood for indigenous populations depending on it for survival.
        \choice Collective ownership of grassland by a village where there is no incentive to stop overgrazing by livestock.
        \choice Littering in public spaces such as parks and recreation areas.
        \correctchoice An agreement for reducing the burning of fossil fuels and carbon dioxide emissions.
        \choice Overcrowding in popular tourist locations like Venice or Machu Picchu.
    \end{choices}

    \begin{solution}
        An agreement for reducing the burning of fossil fuels and carbon dioxide emissions would avoid the tragedy of the commons.
    \end{solution}

    \part According to Matt Ridley, the unexpected increase in productivity that the US saw in the 1990's was due particularly to

    \begin{choices}
        \correctchoice improvements in efficiency first introduced by Walmart.
        \choice huge expansion in the use of steam engines to reduce the cost of transportation.
        \choice economic changes in Europe leading to the introduction of the Euro currency.
        \choice the mass adoption of smartphones.
        \choice reductions in tariffs, thus increasing imports of cheaper products from China.
    \end{choices}

    \begin{solution}
        Ridley states that, for productivity, ``the 1990s surge in the United States was caused by (drum roll of excitement) logistical changes
        in business (groan of disappointment), especially in the retail business and especially in just one firm-Wal-Mart.''
    \end{solution}

    \part Which of the following is/are incorrect about oxytocin?

    \begin{enumerate}
        \item Oxytocin is a hormone that exists only in humans.
        \item Oxytocin is a hormone that raises trust or warm feelings.
        \item The trust game shows that oxytocin has an effect on reciprocity.
        \item Oxytocin does not affect fear.
    \end{enumerate}


    \begin{choices}
        \choice 2, 3, 4
        \choice 2
        \choice 1, 2
        \correctchoice 1, 3, 4
        \choice 3
    \end{choices}

    \begin{solution}
        Oxytocin is a hormone that specifically increases trusting. According to Chapter 3 of ``The Rational Optimist'', in the section entitled `The trust juice'',
        the trust game experiment proves that oxytocin improves trusting, even more than risk-taking.
        Oxytocin does not make you feel angry because the book explains that oxytocin is a ``chemical that evolution uses to make mammals feel good about each other''.
        Oxytocin is common to all mammals, not only humans. The trust game experiment also shows that oxytocin doesn't affect the trustee's decision on back transfer,
        thus explaining why oxytocin does not have an effect on reciprocity. Oxytocin can suppress ``the activity of the amygdala, the organ that expresses fear.''
    \end{solution}

    \part What is the largest group of people whose incomes rose substantially in the 1990s and 2000s, according to analysis of the elephant graph?

    \begin{choices}
        \choice The middle class in Switzerland
        \choice Africa
        \choice The richest 1\%
        \choice The poorest 1\% of people
        \correctchoice Resurgent Asia
    \end{choices}

    \begin{solution}
        In the video on the elephant graph, Branko Milanovic and Paul Solman discussed that the middle third or so of the global population had a big rise in income,
        and that this constituted Resurgent Asia, such as middle-class Indians, Chinese, and South Koreans.
        The middle class of developed nations is the wrong answer because these were the largest group whose income did NOT rise substantially.
        In the elephant graph, the big dip for the richest 30\% of the world’s income distribution (except the top 1\%)
        represents the hollowing out of the middle class in developed countries, such as France, the UK, and the US. Income did not rise for the poorest 5\% of people.
    \end{solution}

    \part Natural resources, economic liberty, teaching program, strong government, and infrastructure might all predict prosperity of a country.
    Which would Matt Ridley likely consider most important?

    \begin{oneparchoices}
        \choice Natural resources
        \choice Infrastructure
        \correctchoice Economic liberty
        \choice Teaching program
        \choice Strong government
    \end{oneparchoices}

    \begin{solution}
        Ridley said, ``a country's economic freedom predicts its prosperity better than its mineral wealth, education system or infrastructure do.''
        He feels fragmented government could actually better support economic development than strong government.
    \end{solution}

    \part For the tit-for-tat strategy to help people work together, all of these are needed, EXCEPT:

    \begin{choices}
        \choice People know when other people are being unfair.
        \choice People can identify each other.
        \correctchoice People have different skills.
        \choice People often deal with each other.
        \choice People favor working together.
    \end{choices}

    \begin{solution}
        Tit-for-tat needs the 4 Re's: repeating interaction, remembering people, recognizing cheating, and rewarding cooperation.
        Thus tit-for-tat needs people to know when other people are being unfair, to often deal with each other, to favor working together, 
        and to know each other. But it doesn't need people to have different skills.
    \end{solution}

    \part TIT-FOR-TAT success in society does NOT need

    \begin{choices}
        \correctchoice retaining kindness.
        \choice remembering people.
        \choice recurring contact.
        \choice recognizing cheating.
        \choice rewarding cooperation.
    \end{choices}

    \begin{solution}
        TIT-FOR-TAT success in society needs 4 Re: repeating interaction, remembering people, recognizing cheating, and rewarding cooperation.
        The first is equivalent to recurring contact. Therefore, the answer is retaining kindness.
    \end{solution}

    \part Which of these will least likely raise oxytocin levels?

    \begin{choices}
        \choice Chatting with friends
        \correctchoice Public speaking
        \choice Group dancing
        \choice Massage
        \choice Prayer
    \end{choices}

    \begin{solution}
        Oxytocin suppresses the activity of the amygdala, the organ that expresses fear. Thus public speaking is unlikely to raise oxytocin levels.
    \end{solution}

\end{parts}