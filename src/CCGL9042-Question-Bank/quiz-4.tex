\question \textbf{Questions from Quiz 4}

\begin{parts}
    
    \part What financial invention was least likely used before the last century?

    \begin{choices}
        \correctchoice cryptocurrency
        \choice mortgages
        \choice mutual funds
        \choice loan forgiveness measures by government
        \choice compound interest
    \end{choices}

    \begin{solution}
        The slide titled ``AN ANCIENT WALL STREET'' said that 4 of these existed in ancient Ur. But
        cryptocurrency was not mentioned. (It was developed in the last few years. An example is Bitcoin.)
    \end{solution}

    \part Which have the most urbanized population and will see the most growth of cities over the
        next several years: developed or developing regions?

    \begin{choices}
        \choice Developing regions are the most urban, and developed region cities will grow the most.
        \correctchoice Developed regions are the most urban, and developing region cities will grow the most.
        \choice Developing regions are the most urban, and developing region cities will grow the most.
        \choice Developed regions are the most urban, and both developed and developing region cities will grow roughly equally.
        \choice Developed regions are the most urban, and developed region cities will grow the most.
    \end{choices}

    \begin{solution}
        The slide titled ``URBANIZATION BY COUNTRY'' shows that the most urbanized regions are
        developed areas such as North America, Western Europe, Australia, New Zealand, Japan, South
        Korea, and Saudi Arabia (though much of South America is also urbanized despite being less
        developed), while the least urbanized countries are in Africa and much of South Asia, which are
        among the least developed regions. The slide titled ``Distribution of Cities'' shows that most of the
        new large cities between 2014 and 2030 will be in South America, Africa, and South Asia, which are
        developing regions.
    \end{solution}

    \part High tariffs tended to

    \begin{choices}
        \choice become more common in recent decades.
        \choice keep growth the same.
        \choice protect consumers.
        \choice speed growth.
        \correctchoice slow growth.
    \end{choices}

    \begin{solution}
        High tariffs tend to slow growth. Chapter 5 mentions this in several places, for example: ``Farm
        subsidies and import tariffs on cotton, sugar, rice and other products cost Africa \$500 billion a year
        in lost export opportunities - or twelve times the entire aid budget to the continent.''
    \end{solution}

    \part In general, is city living better for the environment than is suburban living?

    \begin{choices}
        \choice No, because constructing cities requires deforestation and reclamation, which harm the environment.
        \correctchoice Yes, because cities tend to use less energy per person due to mass transit, smaller homes and shorter errands.
        \choice Yes, because cities have greater population density, which promotes the development of renewable energy to protect the environment.
        \choice City living and suburban living have the same impact on the environment.
        \choice No, because the high population density in a city means greater production of waste products, like carbon dioxide and rubbish, as well as greater consumption of energy.
    \end{choices}

    \begin{solution}
        Slides: ``15\% rule'', ``Are cities green?'', ``Cities are green''.
    \end{solution}

    \part Historically, what was most important in keeping city populations stable or increasing?

    \begin{choices}
        \correctchoice High immigration from rural areas
        \choice Death rate less than in rural areas
        \choice Low emigration to rural areas
        \choice Birth rate higher than in rural areas
        \choice Both a higher birth rate and a lower death rate than in rural areas
    \end{choices}

    \begin{solution}
        In The Rational Optimist: ``Since people have generally done more dying than procreating when in
        cities, big cities have always depended on rural immigrants to sustain their numbers.''
    \end{solution}

    \part According to Matt Ridley's "The Rational Optimist", which is the sequence of the urban
        revolution?

    \begin{choices}
        \choice intensification of trade $\to$ emperor $\to$ agricultural surplus
        \choice emperor $\to$ intensification of trade $\to$ agricultural surplus
        \choice emperor $\to$ agricultural surplus $\to$ intensification of trade
        \choice agricultural surplus $\to$ emperor $\to$ intensification of trade
        \correctchoice intensification of trade $\to$ agricultural surplus $\to$ emperor
    \end{choices}

    \begin{solution}
        Matt Ridley suggests that the intensification of trade triggers agricultural surpluses (as farmers are
        motivated to turn their produce into valuable goods via trade), emperors then emerge as results of
        temptation to capture the gains of the society's income: ``To argue, therefore, that emperors or
        agricultural surpluses made the urban revolution is to get it backwards. Intensification of trade came
        first. Agricultural surpluses were summoned forth by trade, which offered farmers a way of turning
        their produce into valuable goods from elsewhere. Emperors, with their ziggurats and pyramids,
        were often made possible by trade. Throughout history, empires start as trade areas before they
        become the playthings of military plunderers from within or without.''
    \end{solution}

    \part What started in Uruk?

    \begin{choices}
        \choice Alcoholic drinks
        \choice Cities over a million people
        \choice Perfection of camels for moving cargo
        \choice Trading beyond family members
        \correctchoice Writing as a long-lasting record using symbols
    \end{choices}

    \begin{solution}
        The slide ``URUK'' says that writing originated in Uruk. 
        Alcoholic drinks were made for ~9 ka
        in China, predating Uruk (slide: ``CITY PROBLEMS: POLLUTION''). Uruk may have been the
        first city $>10,000$ people, not 1 million. Trading beyond family members happened in the
        Ubaid period before Uruk (Rational Optimist: ``Ubaid Mesopotamia, by exporting grain and
        cloth, drew its neighbours into exporting timber and later metal.''), and likely much earlier.
        Camels were perfected for cargo carrying much later (Rational Optimist: ``The camel had
        been domesticated several thousand years earlier, but it was in the early centuries AD that it
        was at last made into a reliable beast of burden.'')
    \end{solution}

    \part Why were Phoenicians successful?

    \begin{choices}
        \choice They were effective warriors.
        \choice Their land was rich in minerals.
        \choice They had strong central government.
        \correctchoice They had excellent shipbuilding ability.
        \choice They had religious motivation.
    \end{choices}

    \begin{solution}
        Chapter 5 says that Phoenicians built the best ships and used them to trade around the
        Mediterranean Sea.
    \end{solution}

    \part According to the 15\% rule, which of the following is true?

    \begin{choices}
        \choice When the population of a city doubles, GDP and crime rise roughly 100\%.
        \choice When the population of a city doubles, GDP and crime rise roughly 85\%.
        \correctchoice When the population of a city doubles, GDP and crime rise roughly 115\%.
        \choice When the population of a city doubles, GDP and infrastructure rise roughly 100\%.
        \choice When the population of a city doubles, GDP and infrastructure rise roughly 85\%.
    \end{choices}

    \begin{solution}
        The correct answer is that when the population of a city doubles, GDP and crime rise
        roughly 115\% (slide: 15\% rule). Some things which are good, like GDP, and others which are
        bad, like crime, both increase 15\% faster than the population when it comes to city living.
    \end{solution}

    \part According to the 15\% rule discussed in Lecture 5, which of the following is most likely correct?

    \begin{choices}
        \choice Doubling population leads to a roughly 85\% increase in the GDP and wages of a city.
        \correctchoice Doubling population leads to a roughly 115\% increase in the traffic congestion in a city.
        \choice Doubling population leads to a roughly 85\% increase in crime in a city.
        \choice Doubling population leads to a roughly 115\% increase in the infrastructure of a city.
        \choice Growth rate for crime tends to be inversely proportional to population growth rate.
    \end{choices}

    \begin{solution}
        The correct answer is that traffic grows about 15\% faster than the population (see slide ``15\% RULE'').
    \end{solution}

    \part What is/was the largest migration?
    
    \begin{choices}
        \choice The Partition of India
        \choice Refugees fleeing Syria
        \correctchoice The Urbanization of China
        \choice Mexicans moving to the US
        \choice The Great Atlantic Migration
    \end{choices}

    \begin{solution}
        The slide titled ``CHINA'S MIGRATION TO CITIES'' has a table listing 200-500 million people in the Chinese Urbanization and says,
        ``Largest migration ever''.
    \end{solution}

    \part Which of the following statements about the rise of towns and cities is most likely correct?

    \begin{choices}
        \correctchoice The earliest towns developed as meeting points for buyers and sellers.
        \choice The rise of empires resulted in the beginning of the urban revolution.
        \choice The rise of Norte Chico civilization supports the idea that cereal stores and warfare made cities possible.
        \choice Trade developed because trade agents were sent abroad to acquire things on behalf of a city.
        \choice One of the earliest cities, Uruk, at its peak contained 1\% of the world's population of 50 million people.
    \end{choices}

    \begin{solution}
        Ridley said in the beginning of Chapter 4,
        ``Cities exist for trade. They are places where
        people come to divide their labour, to specialise and exchange.''
        In the section titled ``Cotton and Fish'', he asked,
        ``So what was driving people together into these South American towns? The answer, in a
        word, is trade.'' The slide titled ``Main points of Chapter 5'' says,
        ``Trade creates cities''.
        Norte Chico
        towns had no evidence of large-scale hoarding of grain or construction related to warfare.
        Intensification of trade came before emperors. Ridley claimed in the section,
        ``The flag follows trade'',
        ``To argue, therefore, that emperors or agricultural surpluses made the urban revolution is to get it
        backwards. Intensification of trade came first.''
        Cities arose independently in at least some places
        because the Americas, which had no contact with Asia, developed cities. Trade did not develop due to
        empires acquiring things; it emerged from the interactions of individuals. People responded to the
        incentive of profits.
    \end{solution}

    \part In general, is suburban living better for the environment than is city living?

    \begin{choices}
        \choice No, because cities have greater population density, which promotes the development
            of renewable energy to protect the environment.
        \choice Yes, because the high population density in a city means greater production of waste
            products, like carbon dioxide and rubbish, as well as greater consumption of energy.
        \choice Yes, because constructing cities requires deforestation and reclamation, which harm
            the environment.
        \choice City living and suburban living have the same impact on the environment.
        \correctchoice No, because cities tend to use less energy per person for transport and in
            buildings.
    \end{choices}


    \begin{solution}
        Slides: ``15\% rule'', ``Are cities green?'', ``Cities are green''.
    \end{solution}

    \part What does Matt Ridley claim is a key reason that the economy of Europe grew more than
        many other parts of the world during the last millennium?

    \begin{choices}
        \choice Alphabetic language
        \choice Silver and gold from the Americas
        \choice The spread of agriculture
        \choice The spiritual direction provided by the Catholic church
        \correctchoice The absence of a unifying government
    \end{choices}

    \begin{solution}
        During the last millenium, lack of a unifying government differentiated Europe from
        other leading regions, such as China, thus aiding trade and exchange.
    \end{solution}

    \part Why were the people in and around Tyre successful?

    \begin{choices}
        \choice They had strong central government.
        \choice They built large cities.
        \choice They developed a sophisticated written language.
        \correctchoice They had excellent shipbuilding ability.
        \choice They had rich silver mines.
    \end{choices}

    \begin{solution}
        Chapter 5 says that Phoenicians (containing the city of Tyre) built the best ships and
        used them to trade around the Mediterranean Sea.
    \end{solution}

    \part Money has the power to bring people together, but also to destroy humanity. Harari thinks
        the main reason is because

    \begin{choices}
        \choice People are inherently bad.
        \choice As people and organizations accumulate greater wealth, they can fund bigger and more dangerous weapons.
        \correctchoice People trust money more than humans.
        \choice People enjoy their economic freedom.
        \choice People believe money is more important than knowledge and innovation.
    \end{choices}

    \begin{solution}
        In Sapiens Chapter 10, Harari wrote: ``When everything is convertible, and when trust
        depends on anonymous coins and cowry shells, it corrodes local traditions, intimate relations and human value...''.
        He suggests that between transactions, trust is not invested in humans, but in the money itself.
    \end{solution}

    \part The Phoenicians were a Mediterranean civilisation that originated from Lebanon and became extremely prosperous in the Ancient
        World. What was one of the main reasons for their success?

    \begin{choices}
        \choice They were resource-rich and relied on trading crustaceans and incense with Romans.
        \choice They built good quality ships that allowed them to raid other nations and cities and take their wealth.
        \correctchoice They traded with many markets spread across the Mediterranean region.
        \choice They founded Uruk, the first known city, and benefited from taxing merchants.
        \choice They had a strong central government and emperor who guided commerce efficiently.
    \end{choices}

    \begin{solution}
        The Phoenicians were among the greatest traders of the ancient world and owed much of their prosperity to trade. They were
        skilled shipbuilders and created a strong navy that allowed them to trade across the Mediterranean. The Phoenicians were not
        ruled by any emperors, did not have a strong military (rarely fighting wars), and were not a resource-rich nation. Uruk was founded
        by the Sumerians in present-day Iraq, not by the Phoenicians.
    \end{solution}

    \part According to David Hume, political fragmentation is often the friend, not the enemy, of economic advance. What does he mean by that?

    \begin{choices}
        \choice Democratic city states are the only places where economic progress can occur.
        \choice Establishing colonies all over the world can boost economic growth.
        \choice Political fragmentation lays the foundation for political unification.
        \correctchoice Political fragmentation inhibits leaders from imposing excessive regulation.
        \choice Trade made possible the cross-fertilisation of ideas that led to great discoveries.
    \end{choices}

    \begin{solution}
        The correct answer is that political fragmentation prevents overregulation of citizens by leaders (the section titled ``The Moloch
        state'').
    \end{solution}

    \part Which is most accurate about sweatshops?

    \begin{choices}
        \correctchoice Workers choose jobs with them due to the relatively high pay.
        \choice Their main attraction is good working conditions.
        \choice They force poor workers to move from farms.
        \choice Their spread during the Industrial Revolution provided affordable candy to the majority of the population: an example of
        the virtuous cycle of luxury spending that boosted the economy for everyone.
        \choice People work there mainly because they want to live in cities.
    \end{choices}

    \begin{solution}
        The pay of sweatshops is generally higher than for farming or other jobs (``SWEATSHOPS'' slide).
    \end{solution}

    \part According to Ridley, economic advancement after 5000 years ago was the result of

    \begin{choices}
        \choice the rise of Christianity, which facilitated trust in fellow humans and therefore trade.
        \choice strong central governments, facilitating trade across vast areas, like Greece under the leadership of Philip of Macedon.
        \choice the development of agriculture.
        \choice the invention of slavery, mobilizing large numbers of people to a common good.
        \correctchoice trade among people living in relatively independent city-states.
    \end{choices}

    \begin{solution}
        The answer is trade among people living in relatively independent city-states. The answer is not ``strong central governments,
        facilitating trade across vast areas, like Greece under the leadership of Philip of Macedon'' because Ridley says that ``as soon as
        Greece was unified into an empire by a thug - Philip of Macedon in 338 BC - it lost its edge.'' Christianity developed much later.
        Ridley doesn't mention slavery as producing economic advancement.
    \end{solution}

    \part Cities raise efficiency in several ways. Which of the following increases SLOWER than the population increases?

    \begin{oneparchoices}
        \choice Patents
        \choice Job income
        \correctchoice The length of sewage pipes
        \choice GDP
        \choice Infectious disease
    \end{oneparchoices}

    \begin{solution}
        In the lecture slide titled ``15\% RULE'', I said that GDP, wages, patents, and crime increase 15\% faster than population,
        but doubling the population only raises infrastructure 85\%. I listed sewage pipes as an example. Infectious disease is not mentioned. 
    \end{solution}

    \part History's earliest known form of money, which was measured in silas, was which of the following?

    \begin{oneparchoices}
        \correctchoice Barley
        \choice Dutch guilders
        \choice Dinosaur bones
        \choice Cigarettes
        \choice Lydian coins
    \end{oneparchoices}

    \begin{solution}
        Sapiens Chapter 10: ``History's first known money is the Sumerian barley money.
        It appeared in Sumer around 3000 BC, at the same time and place in which writing appeared.''
    \end{solution}

    \part The American economist Edward Glaeser proclaimed that ``Thoreau was wrong. Living in the country is not the right way to care for the
        Earth. The best thing that we can do for the planet is build more skyscrapers.'' What did he mean by this?

    \begin{choices}
        \choice Urbanisation has caused exponential population growth, but skyscrapers can act as a carbon storage system and trap carbon dioxide emissions within cities.
        \choice City dwellers consume significantly less food per capita than country dwellers and therefore cause less damage to natural ecosystems.
        \choice Despite the "nostalgia for mud" felt by many Westerners, city living is cleaner and more comfortable.
        \choice The downdraught effect from skyscrapers causes harmful toxins in air pollution to dissipate and has been empirically proven to be effective against climate change.
        \correctchoice Per capita, urban residents have a smaller environmental footprint than do rural residents.
    \end{choices}

    \begin{solution}
        The world's cities contain half of the world's population, but they occupy less than 3 percent of the world's land area and reduce the
        possibility of potential degradation to natural ecosystems. Living in cities also uses less energy, and city-dwellers are shown to usually
        have a lower carbon footprint per capita than country dwellers. While city living may be more comfortable than rural life, this was also not his point.
        The lecture and reading did not discuss skyscrapers as means to prevent air pollution or absorb carbon dioxide.
    \end{solution}

\end{parts}