\question \textbf{Questions from Quiz 3}

\begin{parts}

    \part Which of the following statements is most accurate?

    \begin{choices}
        \choice Norman Borlaug was a chemist who helped develop the Haber-Borlaug process to make synthetic fertilizer.
        \choice Farming started when humankind started to bake bread.
        \choice It requires millennia to selectively breed the largest and the smallest dogs.
        \correctchoice Farming extends specialization to other species.
        \choice The first genetic modification of a food crop was performed several decades ago.
    \end{choices}

    \begin{solution}
        The correct answer is that farming extends specialization to other species (slide: ``Farming begins:
        summary'').
        
        The first genetic modification of a food crop occurred with domestication thousands of
        years ago (slide: ``History of genetically modified food'').
        
        Humankind began to bake bread at least 23
        ka ago, while farming became a common practice around 10 ka ago. The gap is due to the gathering
        of wild grains and wheat (slides: ``(Not-so-fresh) Bread'' and ``Farming begins'').
        
        The Haber-Bosch process turns nitrogen and methane (a natural gas) into ammonia to make fertilizer (slides:
        ``Haber-Bosch process'').
        
        It is a quick process to selectively breed the largest and the smallest dogs,
        within a century (slide: ``Dogs'').
        
        Farming extends specialization to other species, to provide foods
        and material for us (slide: ``Farming begins: summary'').
    \end{solution}

    \part According to Matt Ridley, what caused farming to start around 11500 years ago?

    \begin{choices}
        \choice People wanted to change their diet.
        \choice The invention of copper smelting
        \choice There were no longer any big animals to hunt.
        \correctchoice The world became much warmer, wetter and more predictable.
        \choice People realized vegetables provide more nutrients to the human body.
    \end{choices}

    \begin{solution}
        Ridley, in ``The Rational Optimist'' stated that ``around 11,500 years ago the temperature
        of the Greenland ice cap shot up by ten degrees (centigrade) in half a century; throughout the world
        conditions became dramatically warmer, wetter and more predictable. In the Levant intensification
        of cereal use could resume, the Natufians could return to settled homes and soon something
        prompted some body to start deliberately saving seed to plant.''
    \end{solution}

    \part Which statement about GMOs is most accurate?

    \begin{choices}
        \correctchoice Growing GMO crops lowers the production cost of farmers.
        \choice The DNA change in GMOs is injecting an unknown gene with unknown function into the genome ofan organism.
        \choice Growing GMO crops will increase the use of herbicide and pesticide.
        \choice Since the introduction of GMO crops to India, the suicide rate of farmers has increased substantially.
        \choice Superweeds arise mainly due to GMO crops.
    \end{choices}

    \begin{solution}
        According to the lecture slide titled ``Main benefits of GMO crops'', the main benefits of GMO crops
        include lower cost. The other choices are incorrect because the DNA change of GMOs is known.
        herbicides induce superweeds regardless of the use of GM crops, the suicide rate of Indian farmers
        did not change when GM crops began to be grown, and GM crops can actually reduce the use of
        herbicide and pesticide.
    \end{solution}

    \part What were the causes of “The Wheat Problem”?

    \begin{choices}
        \choice The growing population and the lack of rainwater
        \choice The lack of sunlight and the lack of suitable acres to plough
        \choice The growing population and the lack of sunlight
        \choice The lack of sunlight and the lack of rainwater
        \correctchoice The growing population and the lack of suitable acres to plough
    \end{choices}

    \begin{solution}
        The Rational Optimist quotes the British chemist Sir William Crookes as stating that ``all civilisations
        stand in deadly peril of not having enough to eat'' due to the growing population and the lack of
        acres to plough in the Americas. The Caucasian race ``will be squeezed out of existence by races to
        whom wheaten bread is not the staff of life''. This problem could not be solved unless a scientific
        process can fix nitrogen from the air.
    \end{solution}

    \part While shopping together, a family member asks you whether it's better to buy organic fruit
    because it doesn't have chemicals. Which of the following would be factual replies?

    \begin{enumerate}
        \item Though organic food doesn't have chemicals, it is not necessarily safer.
        \item The levels of synthetic pesticides on fruit are generally much less than would be harmful.
        \item Plants make chemicals that can be toxic.
    \end{enumerate}

    \begin{oneparchoices}
        \choice None of the above
        \correctchoice 2 and 3
        \choice 1
        \choice 2
        \choice 1 and 3
    \end{oneparchoices}

    \begin{solution}
        The lecture slide titled ``Food is chemicals'' says ``Everything, including food, is chemicals.'' Therefore,
        organic food does have chemicals, and (1) is wrong. The slide titled ``Are pesticides safe?'' shows that
        even though apples commonly have pesticide residues, apples have much less than would be
        harmful (with mean exposure hundreds to thousands of times less than the reference dose), thus (2)
        is correct. The slide titled ``Is food safe?'' says that tobacco makes nicotine, a pesticide, and the ``Food
        is chemicals'' slide says that many food plants make cyanide, thus (3) is correct. Bruce Ames also
        supported this in his video.
    \end{solution}

    \part What was a key factor which helped in proving Malthus's prediction wrong?

    \begin{choices}
        \correctchoice The Haber-Bosch process for producing ammonia fertilizer
        \choice Adoption of organic farming around the world, leading to the availability of nutritious food
        \choice Stabilization of the world population in the 20th century, allowing more people to eat
        \choice The movement to grow food locally, raising efficiency
        \choice India and Pakistan gaining independence, allowing them to produce more wheat
    \end{choices}

    \begin{solution}
        In the book, Ridley lists three factors as the chief reasons: ``Borlaug's genes, sexually recombined
        with Haber's ammonium and Rudolf Diesel's internal combustion engine, have rearranged sufficient
        atoms ... to ensure that Malthus was wrong for at least another half-century''. Organic Farming didn't
        contribute to higher yields, and in fact can even take up more land and do the opposite. It was not
        independence but rather Borlaug's intervention in India and Pakistan which increased their wheat
        yield. Growing food locally tends to decrease, not increase, yields, as Ridley explained: ``Should the
        world decide...that countries should largely grow and eat their own food..., then of course a very
        much higher acreage will be needed''. Ridley doesn't list the stabilizing of populations as one of the
        factors; in fact, population grew dramatically.
    \end{solution}

    \part Which of the following statements is most correct about artificial selection in domestication?

    \begin{choices}
        \choice It requires genetic engineering.
        \choice It needs millenia to significantly change traits.
        \correctchoice It attempts to enhance some favorable quality.
        \choice It selects traits that increase the likelihood of survival and hence reproduction.
        \choice It is beneficial to humans, but not to the domesticated organism.
    \end{choices}

    \begin{solution}
        Its goal is to accentuate a desirable trait by selective breeding, according to a slide titled ``Domestication''.
    \end{solution}

    \part Scientists bred new traits into crops as part of the Green Revolution. Which of the following
        traits were important in growing more food: resistance to plant diseases, taller plants to
        support more grain, or more protein for better nutrition?

    \begin{choices}
        \choice Disease resistance and taller plants
        \choice Taller plants and more protein
        \choice Taller plants, disease resistance, and more protein
        \choice None of these
        \correctchoice More protein and disease resistance
    \end{choices}

    \begin{solution}
        The new crop breeds were shorter, not taller. The slides said that Norman Borlaug bred high
        yield wheat strains that were short so they won't collapse, and that were multiply disease-
        resistant. In The Rational Optimist, Ridley said, ``fertiliser caused the crop to grow tall and
        thick, whereupon it fell over, or `lodged'. Vogel began crossing Norin 10 with other wheats to
        make new short-strawed varieties.'' Later, he said, “thanks to Borlaug's persistence, the
        dwarf wheats prevailed... Borlaug's wheat - and dwarf rice varieties that followed - ushered
        in the Green Revolution.'' The 10 minute video you previewed before the lecture also clearly
        explained the importance of shorter plants, that ``some strains of wheat were modified to
        have a high protein level to compensate for a lack of protein'' in the diet of many people,
        and that ``agronomists were also able to create strains that were resistant to lethal plant
        diseases.''
    \end{solution}

    \part How may farming have created a positive feedback loop to encourage more farming?

    \begin{choices}
        \correctchoice By supporting a larger population, which gathered more wild plants, decreasing their availability and forcing people to grow more plants to obtain enough food
        \choice By accumulating capital to fund copper smelting, supplying copper-based synthetic pesticides and herbicides
        \choice By reducing hunting, leading to more manure from wild animals, supplying more fertiliser for farming
        \choice By allowing the domestication of dogs, which supplied labor to grow plants on farms
        \choice By domesticating plants, increasing the demand for fertiliser from fish and other animals obtained by hunting
    \end{choices}

    \begin{solution}
        The lectures stated the best answer: ``By supporting a larger population, which gathered
        more wild plants, decreasing their availability and forcing people to grow more plants to
        obtain enough food''. Increasing the demand for animals from hunting doesn't necessarily
        encourage farming. Farming may have increased rather than decreased hunting. Farming
        may have led to accumulation of capital to fund copper smelting, but the products were
        things like axes, not synthetic chemicals. Dogs were probably domesticated before
        widespread farming.
    \end{solution}

    \part Why did Matt Ridley feel that ``the invention of metal smelting was an almost inevitable consequence of the invention of agriculture''?

    \begin{choices}
        \choice Back in those days, people had to compete with farmers, so they invented metal smelting to earn money.
        \correctchoice Farming created capital, a high density of people, and markets, thus making feasible the high investment required for smelting.
        \choice Slash and burn agriculture required the burning of forests, which occasionally melted surface-exposed copper deposits, triggering the idea of copper smelting.
        \choice People found that domesticating wheat was not enough for survival, so they invented metal smelting to produce more sharp, powerful weapons to hunt animals.
        \choice Copper tools were important for planting seeds and harvesting crops.
    \end{choices}

    \begin{solution}
        The correct answer is that farming created capital, a high density of people, and markets,
        thus making feasible the high investment required for smelting (see Chapter 4 and the slide
        titled ``Capital'').
    \end{solution}

    \part ``The Rational Optimist'' contains a quote by Professor Robert Paarlberg, who said,
        ``Europeans are imposing the richest of tastes on the poorest of people.'' What did he mean
        by this?

    \begin{choices}
        \choice The lucrative European market for luxury food induces farmers in low-income areas to
        grow these export crops instead of cheap local foods.
        \choice The invention by Haber and Bosch of an efficient way to make nitrogen fertilizer allowed
        the growth of an abundance of cheap food, contributing to the spread of overeating and
        thus medical problems such as diabetes and atherosclerosis.
        \choice By advertising fast food and junk food, companies are burdening the food budgets of poor
        people.
        \choice By promoting the purchase of GM seeds, which are expensive, European companies are
        impoverishing farmers in developing countries.
        \correctchoice Opposition to the use of GM foods decreases agricultural efficiency and thus raises food
        prices, which many people in developed regions can easily afford but which many people in
        poor regions cannot.
    \end{choices}

    \begin{solution}
        The answer is that opposition to the use of GM foods decreases agricultural efficiency and
        thus raises food prices, which many people in developed regions can easily afford but which
        many people in poor regions cannot. ``The Rational Optimist'' quotes Kenyan scientist
        Florence Wambugu: ``You people in the developed world are certainly free to debate the
        merits of genetically modified foods, but can we eat first?''
    \end{solution}

    \part Which statement about chemicals and food safety is most accurate?

    \begin{choices}
        \choice Food containing synthetic chemicals over certain levels is unsafe to consume.
        \choice It is generally good to try to minimize the amount of chemicals in food.
        \choice Food is unsafe to consume if the total level of all chemicals is above a certain level.
        \choice Food containing natural chemicals over certain levels is unsafe to consume.
        \correctchoice Food containing particular chemicals over certain levels is unsafe to consume.
    \end{choices}

    \begin{solution}
        Referring to the slide titled ``Food is chemicals'', knowing which chemical was used and
        how much of it was used are the determinants of whether the food is safe to consume, not whether
        the chemicals are natural or synthetic. For example, to many people, cyanide might seem to be toxic,
        however, it is actually a natural chemical in some food (e.g., almond and potato). However, we would
        not die after eating those foods because the amount we normally consume gives us much less than a
        toxic amount of cyanide. The slide titled ``Are pesticides safe?'' shows that many apples have some
        synthetic chemicals (e.g., Carbaryl), but the amounts are far below dangerous levels.
    \end{solution}

    \part Why did Matt Ridley feel that farming led to the use of copper?

    \begin{choices}
        \choice Back in those days, hunter-gatherers had to compete with farmers, so they invented metal smelting to earn money.
        \choice People found that domesticating wheat was not enough for survival, so they invented metal smelting to produce more sharp, powerful weapons to hunt animals.
        \choice Copper tools were important for planting seeds and harvesting crops.
        \correctchoice Producing copper needed intensive investment, which farming enabled by accumulating wealth and people, and creating markets.
        \choice Slash and burn agriculture required the burning of forests, which occasionally melted surface-exposed copper deposits, triggering the idea of copper smelting.
    \end{choices}

    \begin{solution}
        The correct answer is that farming created capital, a high density of people, and
        markets, thus making feasible the high investment required for smelting (see Chapter 4 and the slide
        titled ``Capital'').
    \end{solution}

    \part Which one of the following statements is TRUE regarding lactose intolerance?
    
    \begin{choices}
        \choice All adults in East and South-East Asia possess functional lactase enzyme.
        \choice Converting milk to cheese allowed lactose intolerant individuals to consume the product without
            diarrhea because the microbes in cheese produce chemicals that
            numb the nerves in the gut to prevent spasms. For this reason, cheese was an early anesthetic.
        \choice Clay strainers were historically used to extract the liquids from the solids in milk, and the liquids were used to make the lactase enzyme.
        \correctchoice Mutations for adult production of lactase enzyme likely spread because they raised the chances of survival by providing nutrition in low-food and low-light environments.
        \choice Drinking more milk allowed individuals to mutate themselves and produce the lactase enzyme to counteract the negative side-effects of milk.
    \end{choices}

    \begin{solution}
        It is true that the mutations for adult expression of lactase enzyme most likely spread in
        the population because they boosted an individual's chances of survival in populations facing famine
        or in dark climates where Vitamin D could not be reliably and naturally synthesized through sunlight.
        
        The other responses are incorrect because:
        
        A low proportion of East and South-East Asian adult
        populations possess the lactase enzyme gene
        
        Consumption of cheese without the negative
        side-effects of diarrhea is possible because cheese has little or no lactose
        
        Individuals do not mutate
        themselves in response to the environment. Random mutations gave rise to versions of the lactase
        enzyme gene, and it evolved through the process of natural selection
        
        Strainers separated solids
        from the liquids in the milk, and the solids were used to make cheese. The liquids were not used to
        make lactase enzymes
    \end{solution}

    \part Which is most likely TRUE about agriculture?
    
    \begin{choices}
        \choice Its introduction made people much more peaceful and self-sufficient.
        \choice It started independently and became common on all inhabited continents within a
            relatively short period of a few thousand years. Australian farms stretched from coast
            to coast.
        \correctchoice It became feasible after the Ice Age ended and the climate stabilized to a
            temperature suitable for farming.
        \choice It spread within a few thousand years once homo sapiens learned to control fire.
        \choice It was invented in the Fertile Crescent (in the Middle East), and its knowledge was
            carried and implemented throughout the world within a few months.
    \end{choices}
        
    \begin{solution}
        In ``The Rational Optimist,'' Ridley says,
        ``The Fertile Crescent was probably the place
        where agriculture first took hold, and from there the habit gradually spread south to Egypt, west into
        Asia Minor and east to India, but farming was quickly invented in at least six other places in a short
        time''. Therefore, knowledge of farming was not carried throughout the world; rather farming
        developed independently in multiple locations. Farming did become common within a period of
        several thousand years in the Americas, Africa, and Asia, but not in Australia because Australians
        ``knew, or found out the hard way, that farming does not work in a highly volatile climate.''
        Farming did not make people peaceful.
        ``Wherever archaeologists look, they find evidence that early farmers
        fought each other incessantly and with deadly effect.''
        Ridley says that ``around 11,500 years ago the
        temperature of the Greenland ice cap shot up by ten degrees (centigrade) in half a century;
        throughout the world conditions became dramatically warmer, wetter and more predictable. In the
        Levant, intensification of cereal use could resume''. Thus, stabilization of the climate and a
        temperature suitable for farming made agriculture feasible.
    \end{solution}

    \part Which of the following statements about the development of agriculture is most correct?

    \begin{choices}
        \choice Agriculture moved people up the trophic pyramid.
        \correctchoice Agriculture was a slow, cumulative transition of human diet intensification.
        \choice Early farmers were self-sufficient and did not rely on trade.
        \choice Agriculture led to the invention of ovens for baking bread.
        \choice Once people started farming, they did not go back to hunting and gathering.
    \end{choices}

    \begin{solution}
        Agriculture could not develop overnight. Early humans were hunter-gatherers, and their diet had been dependent on meat.
        Shifting a meaty diet towards a carbohydrate diet took time. Ridley said, ``It was the culmination of a long, slow intensification of
        human diet that took tens of thousands of years.''
        
        The other answers were wrong because early trading was frequent in places where farming existed, people regressed from
        farming to hunting and gathering during a cold snap, agriculture moved people down the trophic pyramid, and ovens were used to
        bake bread from wild grain long before farming.
    \end{solution}

    \part Which of the following statements about Ötzi is most accurate?

    \begin{choices}
        \choice He died approximately 2500 years ago.
        \choice The weapons found with his body indicate that his main job was a weapons maker.
        \choice His body was discovered in the mountains of Greece.
        \correctchoice He provided evidence that trading and specialization occurred in the Metal Age.
        \choice The plant and animal material found with his body indicate that his main job was a farmer.
    \end{choices}

    \begin{solution}
        He provided evidence that trading and specialization occurred in the Metal Age. The Rational Optimist says, ``Copper smelting
        was a practice that makes no sense for an individual trying to meet his own needs, or even for a self-sufficient tribe...''. It also said,
        ``And yet there were things that Oetzi no doubt made for himself: ...directly consume between one-third and two-thirds of what
        they produce, and exchange the rest for other goods''. Thus, trading might have started in the metal age already, which allowed
        Ötzi to own such a number of tools. The other answers are wrong because the tools found on his body do not provide any
        evidence about his occupation, and the slides said he died about 5300 years ago in Italy.
    \end{solution}

    \part Which one of the following statements about the mummified iceman Oetzi is TRUE?

    \begin{choices}
        \choice He demonstrated that a single human possessed the knowledge to make many different tools.
        \choice None of these
        \choice He was a prehistoric hunter-gather with a collective brain that lived and died 100,000 years ago.
        \choice He was a self-sufficient nomad who roamed the Serengeti and hunted Neanderthals.
        \correctchoice He took advantage of the specialised labour of other people and animals.
    \end{choices}

    \begin{solution}
        Oetzi was the mummified `iceman' found in the Alps. He lived about 5,300 years ago in an Alpine valley, 2,000 years after
        agriculture reached southern Europe. Oetzi was carrying a diverse inventory of tools, and it would have been unfathomable for
        him to have possessed the knowledge or time to collect the raw materials and craft the items himself. This demonstrated that
        Oetzi consumed the specialised labour of other people and animals.
    \end{solution}

    \part Which does Ridley feel would pressure wilderness and threaten rainforests?

    \begin{choices}
        \choice limiting our carbon footprint
        \correctchoice growing food locally
        \choice a vegetarian diet
        \choice eating fish instead of beef
        \choice farming on the Sun
    \end{choices}

    \begin{solution}
        In Chapter 4, Ridley says: ``Should the world decide…that countries should largely grow and eat their own food ...,
        then of course a very much higher acreage will be needed. My country happens to be as useless at growing bananas and cotton
        as Jamaica is at growing wheat and wool. ... good night rainforests. But as long as some sanity prevails, then yes,
        my grandchildren can both eat well and visit larger and wilder nature reserves than I can. It is a vision I am happy to strive for.
        Intensive yields are the way to get there.''
    \end{solution} 

    \part What are the main benefits of GM Crops
    
    \begin{choices}
        \choice Better-tasting crops, ability to use natural pesticides, lower toxicity to humans and animals.
        \choice Decreased resistance to herbicides and pesticides, cheaper for farmers and consumers, and advanced nutritional delivery for human consumption.
        \choice More nutritious, less herbicide and pesticide use, decreased resistance to herbicides and pesticides.
        \correctchoice Cheaper for farmers and consumers, more nutritious, less herbicide and pesticide use.
        \choice Higher efficiency, better-tasting crops, and advanced nutritional delivery for human consumption.
    \end{choices}

    \begin{solution}
        GM crops can decrease the cost to farmers and consumers, and reduce the use of herbicides and pesticides by including
        genes in the crop that confers resistance to diseases, insects/pests and herbicides. Examples include Bt corn and cotton.
        Furthermore, GM crops can also improve nutritional value by containing additional nutrients that are lacking from diets of
        many people in developing countries. The most famous example of this is Golden Rice, which was modified to prevent vitamin A
        deficiency by containing enhanced levels of beta-carotene.

        GM crops may increase, not decrease, risk of resistance to herbicides and pesticides.
        Improving taste has not yet been a demonstrable benefit of GMO crops.
    \end{solution}

\end{parts}