\question \textbf{Questions from Quiz 6}

\begin{parts}

    \part Which of the following is the best method to estimate and compare the cost of living in different countries?

    \begin{choices}
        \choice Vast majority income
        \choice GDP expenditure method
        \choice GDP per capita
        \choice Gini coefficient
        \correctchoice Purchasing power parity
    \end{choices}

    \begin{solution}
        Purchasing power parity can adjust GDP of different countries to make their respective cost of living
        comparable. See slides ``Disadvantages of GDP growth rate" or "GDP correction''.
    \end{solution}

    \part Which of the Millenium Development Goals was met?

    \begin{choices}
        \choice Have the proportion of people without safe water \& sanitation decrease.
        \choice Allow all children to complete primary school.
        \choice Raise foreign development aid by rich nations to 0.7\% of their economies.
        \correctchoice Reduce the proportion of people who earn less than about HK\$300/month by half.
        \choice Decrease the child mortality rate to one-third.
    \end{choices}

    \begin{solution}
        The proportion of people living in extreme poverty was halved. Only a few countries—mainly around
        the North Sea—met the goal of raising foreign development aid to 0.7\% of their economies.
    \end{solution}

    \part Which of the following is a correct calculation regarding GDP? All values are within a period of one quarter year.

    \begin{choices}
        \choice In a country, workers gain \$10 billion as salary, companies gain \$25 billion as profit, and investors
        gain \$5 billion. Companies pay \$2 billion as profit tax to the government, which gains \$2 billion as
        tax revenue. The GDP of this country is \$44 billion.
        \choice Ken purchases \$10 million of raw materials from Company A. He produces jackets worth \$25 million.
        However, he finds that the jackets have quality problems, thus he destroys them. Ken's contribution
        to GDP from these events is \$15 million.
        \choice In a country, expenditure is \$200 billion for private consumption, \$200 billion for government
        consumption, and \$150 billion for net investment. Imports are \$50 billion. Exports are \$50 billion.
        GDP is \$750 billion.
        \correctchoice Ben has a toy factory which imports \$600 million of materials from Country A and purchases \$200
        million of local materials. He exports the toys for \$2000 million to Country B and sells the rest of the
        toys in the local market for \$1000 million. Ben's factory contributes \$2200 million to GDP.
        \choice In a country, Company A imported \$1 million of materials to produce clothes. Company B bought
        the clothes for \$2 million and resold them for \$2.5 million profit. The contribution to the GDP of this
        country is \$5.5 million.
    \end{choices}

    \begin{solution}
        The toy factory answer is correct. Using the production method, the sales of products and services
        minus the purchases are $\$ 2\text{ billion} + \$ 1\text{ billion} - \$ 600\text{ million} - \$ 200\text{ million} = \$ 2.2\text{ billion}$.
        The correct answer is: Ben has a toy factory which imports \$600 million of materials from Country A
        and purchases \$200 million of local materials. He exports the toys for \$2000 million to Country B
        and sells the rest of the toys in the local market for \$1000 million. Ben's factory contributes \$2200
        million to GDP.
    \end{solution}

    \part Which one of the following statements is TRUE regarding extreme poverty?

    \begin{choices}
        \choice In the 20th century, extreme poverty only existed in Sub-Saharan Africa and the Asian region.
        \choice In the last 20 years, the ratio of people not living in extreme poverty versus people living in extreme
        poverty has remained stable.
        \choice In 2013, Sub-Saharan Africa had less \% of its population living in extreme poverty in comparison to
        the world average.
        \choice In 2018, extreme poverty is defined as living below USD 49 per day.
        \correctchoice In the last 30 years, East Asia and the Pacific region has experienced the greatest decline in extreme
        poverty as a percentage of its population.
    \end{choices}

    \begin{solution}
        East Asia and the Pacific region have experienced the greatest decline in extreme poverty as a share
        of the population compared to any other region in the world. In 2013, Sub-Saharan Africa had a
        greater \% of its population living in extreme poverty ($\thicksim$40\%) in comparison to the world average
        ($\thicksim$10\%). The ratio of the world's people not living in extreme poverty versus the world's people living
        in extreme poverty, in fact, skyrocketed in the last century. Extreme poverty exists in all regions of
        the world, not just Sub-Saharan Africa and Asia. The definition of extreme poverty is living below
        USD1.90 per day in 2018 (as measured in 2011 international prices).
    \end{solution}

    \part DALYs show

    \begin{choices}
        \choice alternative ways to measure economic growth, besides GDP.
        \correctchoice which diseases affect people the most.
        \choice a summary of conditions every 24 hours.
        \choice a composite measure of national conditions.
        \choice Trevor Noah.
    \end{choices}

    \begin{solution}
        See Slide ``Health Adjusted Life Years'', and spend a few minutes playing
        with \url{https://vizhub.healthdata.org/gbd-compare/}, which may be useful for your poster if you are
        health minister.
    \end{solution}

    \part All of the following are problems with using GDP as a measure of the quality of life of a country's citizens, EXCEPT

    \begin{choices}
        \choice GDP doesn't necessarily reflect the working conditions or amount of effort required to produce the income.
        \choice Due to income inequality, average income of a country gives limited insight into the quality of life of the poorest citizens.
        \correctchoice GDP doesn't include government spending on large infrastructure projects that may or may not be of much benefit to the population.
        \choice GDP only measures economic success and ignores important factors like happiness.
        \choice Work is excluded from GDP if it is volunteering.
    \end{choices}

    \begin{solution}
        GDP does include government spending on infrastructure.
    \end{solution}

    \part Which key dimension is common to the calculations of all of the following: Human
        Development Index, Human Capital Index, and Social Progress Index?

    \begin{oneparchoices}
        \choice Child Mortality
        \correctchoice Education
        \choice Happiness
        \choice CO2 Emissions
        \choice Economic Freedom
    \end{oneparchoices}

    \begin{solution}
        Education is a common aspect measured by all three indices. Human Development Index
        (HDI) measures life expectancy (which is linked to health), education, and per capita income.
        Human Capital Index, developed by the World Bank, measures child mortality, schooling,
        and health. Social Progress Index contains 54 indicators, including education.
    \end{solution}

    \part A Human Capital Index value of 0.29 for a country means that

    \begin{choices}
        \choice the country is losing 29\% of its potential income.
        \choice the country ranks lower than 29\% of countries in its composite score of health and education.
        \choice the country ranks better than 29\% of countries in its composite score of health and education.
        \correctchoice the country is losing 71\% of its potential income.
        \choice GDP per capita of the country is 0.29 times its happiness index.
    \end{choices}

    \begin{solution}
        On slide ``Human Capital Index'', the video on the Human Capital Index explained that the
        value is the proportion of potential that is being used.
    \end{solution}

    \part What is the likeliest reason that world economic growth slowed in the last several decades?

    \begin{choices}
        \choice More and more countries are shrinking their economies to protect the environment.
        \correctchoice Countries that developed earliest now drag down growth.
        \choice The global financial crisis around 2008 and COVID-19 more recently
        \choice The fraction of not poor compared to poor people rose greatly, but the poor contribute much less to gross world product.
        \choice Economic growth has returned to about the rate before the Industrial Revolution.
    \end{choices}

    \begin{solution}
        Slide ``Gross World Product growth rate'' says: ``First countries to industrialize now grow
        slowly but are big and thus dilute growth rate of world?''
    \end{solution}

    \part Suppose the Gini coefficient of Hong Kong increased from 0.43 to 0.53 from 2000 to 2010.
        What can we most confidently conclude happened during that period in Hong Kong?

    \begin{choices}
        \choice Average income increased.
        \correctchoice The income gap widened.
        \choice Income increased for high-income households but decreased for low-income households.
        \choice Government reformed the tax system.
        \choice Average standard of living rose.
    \end{choices}

    \begin{solution}
        The correct answer is that the income gap widened (1st slide titled ``Gini coefficient''). We
        don't know whether income increased for high-income households but decreased for low-income
        households; that's because both rich and poor income might have decreased during
        the period, or both might have increased during the period.
    \end{solution}

    \part Disability Adjusted Life Years (DALYs) are

    \begin{choices}
        \choice a measure of the quality of life, not life expectancy.
        \choice mainly related to chronic diseases (e.g., heart disease or cancer) in the poorest countries and diseases related to children (e.g., diarrhea) in the richest countries.
        \choice a measure of life expectancy, not quality of life.
        \choice one component in the calculation of the Human Development Index.
        \correctchoice mainly related to chronic diseases (e.g., heart disease or cancer) in the richest countries and diseases related to children (e.g., diarrhea) in the poorest countries.
    \end{choices}

    \part Neglecting other data that are not provided, in which of these countries will a person receiving the median national income have the highest quality of life?

    \begin{choices}
        \correctchoice Airsealand: high GDP/person, low GDP growth rate, Gini coefficient = 0.2, life expectancy = 70 years
        \choice Yaheaia: low GDP/person, high GDP growth rate, Gini coefficient = 0.2, life expectancy = 70 years
        \choice Stanistan: high GDP/person, high GDP growth rate, Gini coefficient = 0.7, life expectancy = 70 years
        \choice Dictatorial Aristocrats' Kingdom of Splunge: low GDP/person, high GDP growth rate, Gini coefficient = 0.2, life expectancy = 50 years
        \choice West Easter Island: high GDP/person, high GDP growth rate, Gini coefficient = 0.1, life expectancy = 40 years
    \end{choices}

    \begin{solution}
        High GDP/person and life expectancy generally produce a better quality of life than low
        GDP/person or life expectancy. For a person with median income, income would be higher in a
        country with more equal distribution of income, or a low Gini coefficient.
    \end{solution}

    \part Which of the following ways would lead to the most happiness, according to Buddhism?

    \begin{choices}
        \choice Ignoring external conditions and focusing on retaining and extending pleasant inner feelings
        \choice Forming emotional bonds with those we value the most
        \choice Meditating at least thrice a day in order to experience nirvana and enlightenment on Earth
        \correctchoice Detaching oneself from emotional cravings and accepting feelings, whether positive or negative
        \choice Finding a purpose to make one's life more meaningful
    \end{choices}

    \begin{solution}
        Chapter 19 of Sapiens says that according to Buddhist teachings, happiness is
        \begin{enumerate*}[label=(\arabic*)]
            \item independent of external conditions,
            \item also independent of our inner feelings, and that
            \item we should let go of the pursuits of both external achievements and inner feelings (emotional cravings).
        \end{enumerate*}
        The purpose of letting go is not to ignore all feelings, but to help us accept them and live in the present
        moment.
    \end{solution}

    \part What is most notable about the Social Progress Index (SPI) and GDP per capita (GPC) of Costa Rica, New Zealand and Kuwait (as of several years ago)?

    \begin{choices}
        \correctchoice Costa Rica had a particularly high SPI, given its GPC. New Zealand had about the highest SPI. Kuwait had a particularly low SPI, given its GPC.
        \choice Costa Rica had an SPI of 100\%. New Zealand had about the highest GPC. Kuwait had a particularly low SPI, given its GPC.
        \choice Costa Rica had an SPI of 0\%. New Zealand had about the highest SPI. Kuwait had a particularly high SPI, given its GPC.
        \choice Costa Rica had a particularly low SPI, given its GPC. New Zealand had a particularly high SPI, given its GPC. Kuwait had about the highest GPC.
        \choice Costa Rica had about the highest SPI. New Zealand had about the highest GPC. Kuwait had a particularly low SPI, given its GPC.
    \end{choices}

    \begin{solution}
        In the TED talk introducing the SPI, Costa Rica was a superstar, with an unusually high
        SPI for its GPC, New Zealand had the highest SPI, and Kuwait had a very high GPC but not a high
        SPI.
    \end{solution}

    \part Which of the following is NOT one of Harari's viewpoints on human progress and happiness?

    \begin{choices}
        \correctchoice Modern humans will never be happier than their ancestors because the world is too corrupt.
        \choice We cannot know for sure if modern humans are happier than their ancestors since we don't know much about the science of happiness.
        \choice The pursuit of constant happiness is impractical since biology does not allow for humans to always feel happy.
        \choice The acceleration of human progress has caused suffering of some other species.
        \choice Human progress is undeniable and is shown through evidence such as medical advancements and lower child mortality.
    \end{choices}

    \begin{solution}
        In Sapiens Chapter 19, Harari posted several arguments against the notion that
        humanity has been improving through past centuries. He believes that humans are not necessarily
        happier now than then, since there are many factors to consider (the subjectivity of happiness, the
        wellbeing of other species, the biochemistry of happiness, and many others). Despite this, Harari also
        acknowledges that it is undeniable that humans have progressed in certain aspects of life; this is
        shown through the decrease in child mortality and other medical advances. Ultimately, Harari believes
        that it is too early to conclude that
        \begin{enumerate*}[label=(\arabic*)]
            \item humans are happier and better than our ancestors and
            \item vice versa, due to our knowledge gap on happiness.
        \end{enumerate*}
    \end{solution}

    \part The World Happiness Report identified several key factors as important in happiness. Which of the following are key factors to happiness?

    \begin{choices}
        \choice Consumerism, low unemployment, entertainment, and press freedom
        \correctchoice Freedom from corruption, support network, income, and lifespan
        \choice Health care availability, income, cultural achievements, and education
        \choice Generosity, income, the right-to-vote, cultural achievements, and lifespan
        \choice Income, freedom from corruption, press freedom, cultural achievements
    \end{choices}

    \begin{solution}
        The factors identified as important to happiness which are included in the list are:
        Having someone to count on (support network), income (in real GDP per capita), freedom from
        corruption, mental health, generosity, and lifespan. Other factors identified which are not in the list are:
        Mental health and perceived freedom to make life choices.
    \end{solution}

    \part According to Sapiens, studies on wealth and subjective wellbeing found that:

    \begin{choices}
        \choice Higher-income individuals tend to be happier than lower-income individuals, but only for incomes in the top 0.01\%.
        \choice Progressive taxes always lead to a greater subjective wellbeing for low-income individuals.
        \choice As income rises, happiness steadily decreases.
        \correctchoice Income rise is positively associated with subjective wellbeing.
        \choice Money has no effect on subjective wellbeing.
    \end{choices}

    \begin{solution}
        Studies have found that happiness is positively correlated with wealth, especially for
        lower-income individuals. After it passes a certain threshold, findings suggest that the rise in
        happiness levels will not be as significant. However, there is no evidence that happiness levels will fall
        after this point (Sapiens Chapter 19: ``If you're a top executive earning \$250,000 a year and you win
        \$1 million in the lottery ... your surge is likely to last only a few weeks. According to the empirical
        findings, it's almost certainly not going to make a big difference to the way you feel over the long
        run.'').
    \end{solution}

    \part If Luxembourg and Angola have a monthly GDP per capita of US\$3,200 and US\$3,400,
        respectively. can you conclude that most people in Angola have a better quality of life than
        most people in Luxembourg? Why or why not?

    \begin{choices}
        \correctchoice No, because Angola has an element of the economy that benefits only a small portion of the population.
        \choice Yes, because most people in Angola earn \$200 more than most people in Luxembourg.
        \choice No, because the inflation rate of Luxembourg is much lower than that of Angola.
        \choice Yes, because all countries have the same income distribution.
        \choice No, because European countries, such as Luxembourg, have among the world's highest income inequality.
    \end{choices}

    \begin{solution}
        Angola has high inequality (slide ``Inequality Adjusted HDI'') and abundant resources
        (slide ``Disadvantages of GDP growth rate''). It is quite possible that the benefits of the high
        GDP/person are focused on a small group of people in Angola, unlike Luxembourg. Thus, no,
        GDP/person does not necessarily indicate quality of life. As for other options, we did not discuss
        either country concealing a relatively large portion of their wages and production.
    \end{solution}

    \part Which one of the following Millennium Development Goals was met?

    \begin{choices}
        \choice Halve the spread of HIV/AIDS by 2015.
        \choice Halve the proportion of the population without access to the internet.
        \choice Between 1990 and 2015, reduce by two-thirds the under-five mortality rate.
        \choice Raise foreign development aid by rich nations to 0.7\% of their economies.
        \correctchoice Resolve the debt problems of developing countries.
    \end{choices}

    \begin{solution}
        The only Millennium Development Goal met on the list given was the resolution of debt problems of developing countries. The
        world fell short of the following goals: 
        \begin{enumerate*}[label=(\roman*)]
            \item Raise foreign development aid by rich nations to 0.7\% of their economies; and
            \item Between 1990 and 2015, reduce by two-thirds the under-five mortality rate.
        \end{enumerate*}
        Halving the proportion of the population without access to safe
        drinking water and sanitation was a goal that was met, but the goals did not mention improving internet connection. The goal for
        HIV/AIDS was to begin to reverse the spread, not to halve it.
    \end{solution}

    \part In the last two decades, the world Gini coefficient

    \begin{oneparchoices}
        \choice stayed the same.
        \choice could not be estimated.
        \choice rose.
        \choice fluctuated up and down dramatically.
        \correctchoice fell.
    \end{oneparchoices}

    \begin{solution}
        Slide ``Gini coefficient'' showing the graph reveals that the Gini coefficient has fallen for the last 2 decades.
    \end{solution}

    \part What is an advantage to using GDP growth rate?

    \begin{choices}
        \choice GDP growth rate accounts for all economic transactions and activities within a country.
        \choice Unbeneficial production does not raise GDP.
        \choice GDP growth rate has some positive correlation with the quality of life in a country.
        \correctchoice GDP growth rate is easy to measure and compare.
        \choice Accounts for increases in the quality of products and services over time.
    \end{choices}

    \begin{solution}
        GDP growth rate is easy to measure and compare. However, GDP growth rate does not accurately measure the well being of a
        country as it ignores activities such as the volunteer or underground economy, unbeneficial production raises GDP, and GDP does
        not take the increase in quality of products and services into account.
    \end{solution}

    \part Listed below are several desirable goals for a country. If a country could raise any of these from a low rank to a high rank among
        countries, which one would MOST improve the average general life satisfaction of its people, according to findings from the World
        Happiness Report?

    \begin{choices}
        \correctchoice Low level of psychological disorders
        \choice Fair treatment in dealings among individuals and organizations, such as business and government
        \choice Percentage of children receiving free education
        \choice High level of education
        \choice Low level of air pollution
    \end{choices}

    \begin{solution}
        The correct answer is low level of psychological disorders because mental health is the biggest single factor in happiness according
        to the World Happiness Report (3rd slide titled ``World Happiness Report'').
    \end{solution}

    \part Harari believes that ... will lead to continuous happiness.

    \begin{choices}
        \choice Being immortal
        \correctchoice None of the above
        \choice Being free from illnesses
        \choice Having a lot of money
        \choice Getting married to the love of your life
    \end{choices}

    \begin{solution}
        None of the options above is correct, because Harari believes that
        \begin{enumerate*}[label=(\arabic*)]
            \item happiness depends on subjective expectations,
            \item happiness mainly depends on happy hormones (serotonin, dopamine, oxytocin), and
            \item we cannot be constantly happy.
        \end{enumerate*}
    \end{solution}

    \part What is a disadvantage of using GDP growth rate to determine the development level of a country?

    \begin{choices}
        \choice GDP growth rate ignores quality changes of products, thus increasingly overestimating the standard of living.
        \correctchoice GDP growth rate cannot reflect the income distribution of a country.
        \choice GDP growth rate includes black market transactions, and this may overestimate the real production of the country.
        \choice The effect of inflation on GDP growth rate cannot be adjusted.
        \choice Because imports are added to the GDP, the GDP growth rate may overestimate the standard of living.
    \end{choices}

    \begin{solution}
        The correct answer is that the GDP growth rate cannot reflect the income distribution of a country. The GDP growth rate lumps
        together everyone in a country.
    \end{solution}

    \part According to the graph in the lecture, approximately what is the current ratio of the number of people in extreme poverty to the
        number of people not in extreme poverty?

    \begin{oneparchoices}
        \choice 1000
        \choice 10
        \choice 0.001
        \correctchoice 0.1
        \choice 1
    \end{oneparchoices}

    \begin{solution}
        Slide ``Poverty: global'' shows a little over half a billion people are extremely poor, and around 7 billion are not,
        giving a ratio around one tenth. This question was somewhat challenging because it required that, when you were reviewing the slide,
        you were curious enough to ask yourself, ``I wonder how many people are in extreme poverty?'' or to read the text on the slide,
        ``World: ratio of not poor to poor has rocketed'', and wonder what the ratio is now, and try to estimate it from the graph.
    \end{solution}

    \part Within a region, the number of people in extreme poverty divided by the population there is P. Where was P the largest in 1990?

    \begin{oneparchoices}
        \choice Latin America \& Caribbean
        \choice Central Asia \& Europe
        \choice N. America
        \choice The world
        \correctchoice E. Asia \& Pacific
    \end{oneparchoices}

    \begin{solution}
        Slide ``Poverty: regional'' shows that about 60\% of people lived below the extreme poverty line in East Asia \& Pacific then, much higher than in other regions.
    \end{solution}

\end{parts}