\question \textbf{Questions from Quiz 9}

\begin{parts}

    \part Which of the following are major reasons that non-renewable resources aren't depleted?

    \begin{enumerate}
        \item Increasing cost induces users to buy less.
        \item Since known reserves cannot increase, producers ration them to meet anticipated future needs.
        \item New reserves are discovered.
        \item Decreasing cost induces users to buy less.
    \end{enumerate}

    \begin{oneparchoices}
        \correctchoice 1 and 3
        \choice 2 and 3
        \choice 1 and 2
        \choice 2 and 4
        \choice 3 and 4
    \end{oneparchoices}

    \begin{solution}
        Slide ``Resources: why don't we use up?'' says we don't use up finite resources because we
        continually discover more, and if price rises, demand falls.
    \end{solution}

    \part According to Matt Ridley, which of the following is most accurate regarding the pessimistic
        predictions about the future of humanity?

    \begin{choices}
        \choice People should never listen to such pessimists as these apocaholics profits from the natural pessimism of human nature.
        \choice Apocaholics, on average, are right in their predictions and thus are showered with awards and honors and are rarely challenged.
        \choice Pessimists are accurate about their predictions as they incorporate the strategy of extrapolation in their methodology.
        \choice People are often pessimistic about their own future life whereas they are optimistic about the future of society.
        \correctchoice The pessimistic predictions are often accepted and the optimistic ones are criticized.
    \end{choices}

    \begin{solution}
        Even though apocaholics profit from the pessimistic tendency of people, Ridley never really says that
        people should never listen to them. Therefore, option ``People should never listen to such pessimists
        as these apocaholics profits from the natural pessimism of human nature.'' is wrong.
    \end{solution}

    \part Based on past decades, global grain production will likely \fillin[][2cm] this century.

    \begin{oneparchoices}
        \choice rise and then fall
        \choice fall and then rise
        \choice stay about the same
        \choice fall
        \correctchoice rise
    \end{oneparchoices}

    \begin{solution}
        Slide ``Food supply: rice'' shows steadily rising rice production.
    \end{solution}

    \part The pessimists' error is

    \begin{choices}
        \choice ignoring tipping points.
        \choice obfuscation.
        \choice seeing a future with totally new technologies.
        \choice rejecting current trends by developing new technologies.
        \correctchoice extrapolationism.
    \end{choices}

    \begin{solution}
        The introduction of Chapter 9 in The Rational Optimist says, ``The pessimists' mistake is
        extrapolationism: assuming that the future is just a bigger version of the past. As Herb Stein once
        said, `If something cannot go on forever, then it will not.' ''
    \end{solution}
    
    \part In 1962, more American children were dying of cancer than any other disease because

    \begin{choices}
        \choice chemicals in the environment as a result of modern manufacturing processes were at an all-time high in the US.
        \choice of radiation from nuclear bomb testing.
        \choice of secondhand exposure to cigarette smoke.
        \choice of the increasing use of DDT to kill mosquitos in the US.
        \correctchoice rates of other childhood diseases declined in the 20th century.
    \end{choices}

    \begin{solution}
        Other childhood diseases decreased, leaving cancer as the biggest cause of death in
        children. See Cancer section of Chapter 9 of The Rational Optimist.
    \end{solution}

    \part Coffee and cabbage contain many carcinogens

    \begin{choices}
        \choice at dangerous levels from synthetic pesticides.
        \choice produced by biofuel farming.
        \choice at dangerous levels from nuclear bomb tests.
        \choice made by the plants, and should be avoided.
        \correctchoice whose doses are safe.
    \end{choices}

    \begin{solution}
        Cancer section of Chapter 9 of The Rational Optimist: ``As Bruce Ames famously
        demonstrated in the late 1990s, cabbage has forty-nine natural pesticides in it, more than
        half of which are carcinogens. In drinking a single cup of coffee you encounter far more
        carcinogenic chemicals than in a year's exposure to pesticide residues in food. This does not
        mean that coffee is dangerous, or contaminated: the carcinogens are nearly all natural
        chemicals found in the coffee plant and the dose is too low to cause disease, as it is in the
        pesticide residue. Ames says, `We've put a hundred nails in the coffin of the cancer story and
        it keeps coming back out.' ''.
    \end{solution}

    \part Which of the following statement is most accurate?

    \begin{choices}
        \choice Natural pesticides are less dangerous than synthetic pesticides.
        \choice The reasons for the pessimistic view of the world are mainly pessimism bias on oneself, extrapolation, romanticism, illusion of control, and negative news bias.
        \correctchoice One reason why extrapolation may be wrong is that technology may change.
        \choice Pessimism is better than optimism because it helps to generate better policies to prevent complacency and disasters.
        \choice Extreme environmentalists advocate that a sustainable solution to protect the environment is to steadily expand the economy.
    \end{choices}

    \begin{solution}
        The other answers are wrong because extreme environmentalists advocate that a
        sustainable solution to protect environment is to shrink the economy, many unregulated
        natural pesticides far more dangerous than synthetic pesticides, and the reasons for the
        pessimistic view of the world are mainly pessimism bias on the world, extrapolation,
        romanticism, illusion of control, and negative news bias. Also, pessimism may prevent
        complacency and disasters but it may also lead to bad policies (``Summary'' slide). We cannot
        easily say whether pessimism or optimism is better.
    \end{solution}

    \part What is NOT an example of eugenics?

    \begin{choices}
        \choice The forced sterilisation of people with negative genetic abnormalities.
        \correctchoice Parenting classes to help all parents raise better children.
        \choice Subsidies rewarding more intelligent people when they have more children.
        \choice Killing targeted groups to keep them from having children.
        \choice Prohibition against breeding by criminals and feeble-minded persons.
    \end{choices}

    \begin{solution}
        Eugenics is the manipulation of breeding of people to increase characteristics regarded as
        desirable. Parenting classes to help all parents raise better children is not an example of
        eugenics because it aims to improve the family environment and parenting behaviours, and
        not to encourage selective breeding to engineer children with ``better'' genes.
    \end{solution}

    \part Governments have generally become \fillin[][2cm] democratic since 1900, and \fillin[][2cm] democratic since 1980.

    \begin{oneparchoices}
        \choice less; more
        \choice no more and no less; no more and no less
        \correctchoice more; more
        \choice less; less
        \choice more; less
    \end{oneparchoices}

    \begin{solution}
        The correct answer is that governments have generally become more democratic since
        1900, while becoming more democratic since 1980 (graph on ``Democracy'' slide).
    \end{solution}

    \part DDT

    \begin{choices}
        \choice killed millions of people.
        \choice reduced the Earth's protective ozone layer.
        \choice greatly reduced polio cases until its use was banned.
        \choice was harmless to birds.
        \correctchoice saved millions of human lives.
    \end{choices}

    \begin{solution}
        The Cancer section of Chapter 9 of The Rational Optimist says,
        ``DDT's miraculous ability to halt epidemics of malaria and typhus, saving
        perhaps 500 million lives in the 1950s and 1960s (according to the US National Academy of
        Sciences), far outweighed any negative effect it had on human health.''
    \end{solution}
    
    \part We may prefer negative news

    \begin{choices}
        \choice due to confirmation bias.
        \correctchoice to alert us to risks and learn, aiding our survival.
        \choice since bad things always happen (at least somewhere).
        \choice due to our tendency to extrapolate in faulty ways.
        \choice for a sense of control.
    \end{choices}

    \begin{solution}
        Slide ``Negative news bias'' says that bad news is more newsworthy because people
        pay more attention to it, and it's more important to our survival.
    \end{solution}

    \part According to the lecture, the human tendency to focus on problems may

    \begin{choices}
        \choice explain why we tend to analyze and reject negative forecasts.
        \choice explain why we tend to be optimistic about the world.
        \correctchoice help us identify obstacles so that we might overcome them.
        \choice explain why the actual proportion of deaths from suicide is much higher than the share of attention that suicide gets in the media.
        \choice explain why we tend to be pessimistic about ourselves.
    \end{choices}

    \begin{solution}
        The summary slide says, ``Such pessimism may prevent complacency and disasters...''
    \end{solution}

    \part Which is most likely to happen with renewable or non-renewable resources in the long run as people get richer?

    \begin{choices}
        \choice Rising food consumption will, over the long run, almost totally deplete the supply of wheat, a renewable resource, by the end of this century.
        \choice Global supplies of gold, a non-renewable resource, will run out by the end of this century.
        \correctchoice Rising human interference in animal habitats will cause multiple species of animals, a renewable resource, to become extinct during this century.
        \choice Rising construction will raise the long-term, real price of copper, a non-renewable resource, to at least ten times its current level by the end this century.
        \choice Rising energy use will lead us to run out of one or more types of fossil fuel, a non-renewable resource, by 2050.
    \end{choices}

    \begin{solution}
        The slides showed that over the long term non-renewable resources (like copper, iron,
        and coal) do not become totally depleted and don't rise in price, and that food (like rice) production
        tends to rise over the long term. But I showed several animals that have become extinct. Thus, it's
        most likely that some animals will become extinct.
    \end{solution}

    \part According to Tali Sharot, the happiest people have

    \begin{choices}
        \choice abilities below the average of the population.
        \choice low income.
        \correctchoice high expectations.
        \choice no expectations.
        \choice realistic expectations but act pessimistically.
    \end{choices}

    \begin{solution}
        See TED talk (Explanation: by Tali Sharot).
    \end{solution}

    \part According to the lecture, history suggests that real costs of non-renewable resources in this century will likely

    \begin{choices}
        \choice rise steadily and slowly.
        \correctchoice fluctuate in the short run but not change very much in the long run.
        \choice rise exponentially.
        \choice fall steadily and slowly.
        \choice fall exponentially.
    \end{choices}

    \begin{solution}
        Slides ``Resources: Iron'', ``Resources: Copper'', and ``Resources: Coal'' show that
        prices of non-renewable resources didn't change dramatically in the long run.
    \end{solution}

    \part Which of the following are major reasons that non-renewable resources aren't depleted?
    
    \begin{enumerate}
        \item Increasing cost induces users to buy less.
        \item Since known reserves cannot increase, producers ration them to meet anticipated future needs.
        \item New reserves are discovered.
        \item Decreasing cost induces users to buy less.
    \end{enumerate}

    \begin{oneparchoices}
        \correctchoice 1 and 3
        \choice 1, 2 and 3
        \choice 1 and 2
        \choice 2 and 3
        \choice 3 and 4
    \end{oneparchoices}

    \begin{solution}
        Slide ``Resources: why don't we use up?'' says we don't use up finite resources because we continually discover more, and if price
        rises, demand falls.
    \end{solution}

    \part We may prefer bad news

    \begin{choices}
        \choice for a sense of control.
        \choice due to confirmation bias.
        \choice since bad things always happen (at least somewhere).
        \choice due to extrapolation.
        \correctchoice to learn and alert us to risks, helping us survive.
    \end{choices}

    \begin{solution}
        Slide ``Negative news bias'' says that bad news is more newsworthy because people pay more attention to it, and it's more
        important to our survival.
    \end{solution}

    \part According to the lecture, history suggests that, in the long run, real costs of non-renewable resources in this century will likely

    \begin{choices}
        \choice fall exponentially.
        \correctchoice not change very much.
        \choice fall steadily and slowly.
        \choice rise steadily and slowly.
        \choice rise exponentially.
    \end{choices}

    \begin{solution}
        Slides ``Resources: Iron'', ``Resources: Copper'', and ``Resources: Coal'' show that prices of non-renewable resources didn't change
        dramatically in the long run.
    \end{solution}

    \part The flush toilet is used as an example in Pinker's video to illustrate that

    \begin{choices}
        \choice most of us acknowledge that the flush toilet is important in our everyday life.
        \choice the flush toilet reduces the transmission of cholera.
        \correctchoice people often neglect that what is basic in their life could affect their well-being.
        \choice we should return to traditional ways of doing things.
        \choice the flush toilet is accessible for almost 95\% of the population in the world.
    \end{choices}

    \begin{solution}
        The example of the flush toilet is used by Pinker to explain how people who overlooked the basics in their lives could affect their
        well-being.
    \end{solution}

    \part In the foreseeable future, which is more likely to occur?

    \begin{choices}
        \choice The development of renewable energy will be arrested
        \correctchoice Multiple animal species will near extinction.
        \choice Rare earth elements, which are used in electric car batteries and wind turbines, will be used up.
        \choice Multiple non-renewable resources will run out.
        \choice Pesticides will greatly increase cancer rates.
    \end{choices}

    \begin{solution}
        See slide ``Resources''. Somewhat ironically, non-renewable
        resources tend not to run out (they may get more scarce, but that raises their price, driving more extraction and less
        consumption). By contrast, renewable resources (living species) may become extinct. Some animal species become extinct every
        year.
    \end{solution}

    \part According to Matt Ridley, which of the following is most accurate regarding the pessimistic predictions about the future of humanity?

    \begin{choices}
        \choice Pessimists are accurate about their predictions as they incorporate the strategy of extrapolation in their methodology.
        \correctchoice The pessimistic predictions are often accepted and the optimistic ones are criticized.
        \choice Apocaholics, on average, are right in their predictions and thus are showered with awards and honors and are rarely challenged.
        \choice People should never listen to such pessimists as these apocaholics profits from the natural pessimism of human nature.
        \choice People are often pessimistic about their own future life whereas they are optimistic about the future of society.
    \end{choices}

    \begin{solution}
        Even though apocaholics profit from the pessimistic tendency of people, Ridley never really says that people should never listen to
        them. Therefore, option ``People should never listen to such pessimists as these apocaholics profits from the natural pessimism of
        human nature.'' is wrong.
    \end{solution}

    \part Which of these diseases killed the most people since 1900?

    \begin{oneparchoices}
        \choice vCJD
        \correctchoice Flu
        \choice SARS
        \choice COVID-19
        \choice Ebola
    \end{oneparchoices}

    \begin{solution}
        H1N1 flu possibly caused 50 million deaths in 1918 (``Plague'' section of The Rational Optimist, Chapter 9).
        From recent news, COVID-19 killed many people, but far fewer than H1N1 flu. All the other answer choices contributed to much
        lower death tolls, according to The Rational Optimist.
    \end{solution}

    \part Which of these diseases killed the most people since 1990?

    \begin{oneparchoices}
        \choice Ebola
        \choice Smallpox
        \CorrectChoice COVID-19
        \choice H5N1 flu
        \choice vCJD
    \end{oneparchoices}

    \begin{solution}
        The correct answer is COVID-19. H1N1 flu possibly caused 50 million deaths in 1918
        (``Plague'' section of The Rational Optimist, Chapter 9). From recent news, COVID-19 killed many people,
        but far fewer than H1N1 flu. All the other answer choices contributed to much lower death tolls, according to The Rational Optimist.
        However, since the question asks about the period \textbf{since 1990}, the answer is COVID-19.
    \end{solution}

    \part Which one of the following infectious diseases has affected the fewest people in the last 30 years?

    \begin{oneparchoices}
        \choice Mad-cow disease
        \correctchoice Smallpox
        \choice Bird flu (H5N1)
        \choice AIDS
        \choice Ebola
    \end{oneparchoices}

    \begin{solution}
        ``it is now more than forty years since smallpox was exterminated'' (The Rational Optimist).
    \end{solution}

    \part According to the lecture, the human tendency to focus on problems may

    \begin{choices}
        \choice explain why we tend to be pessimistic about ourselves.
        \choice explain why we tend to be optimistic about the world.
        \choice explain why we tend to analyze and reject negative forecasts.
        \choice explain why the actual proportion of deaths from suicide is much higher than the share of attention that suicide gets in the media.
        \correctchoice help us identify obstacles so that we might overcome them.
    \end{choices}

    \begin{solution}
        The summary slide says, ``Such pessimism may prevent complacency and disasters...''
    \end{solution}

    \part Which one of the following statements is TRUE regarding the current state of the world over the long term?

    \begin{choices}
        \choice Forests are declining within North America due to acid rain.
        \correctchoice More than 10\% of people globally believe the world will end in this century.
        \choice In the face of falling supply, real grain prices rose by at least 10 times in approximately the last 100 years.
        \choice Rising demands of population growth will soon outweigh the world's food production.
        \choice Democracies are dying and autocratic governments are on the rise.
    \end{choices}

    \begin{solution}
        Over the last century, rice production has increased while grain prices have fallen. Democracy is on the rise and autocratic governments are becoming fewer.
        Studies show there is no decline in forests in the United States and Canada due to acid rain.
        Fears of starvation due to a narrowing margin between food production and population growth are simply untrue.
        Food production has grown, while fertility rates have dropped due to the demographic transition.
        Large percentages, but less than 50\% of people think the world will end in their lifetime.
    \end{solution}

    \part Based on long term trends, several decades from now global grain production will likely \fillin[][2cm].

    \begin{choices}
        \correctchoice continue rising
        \choice remain at about current levels
        \choice continue falling
        \choice no longer keep up with rising demand
        \choice come to a turning point at which it stops rising and begins to fall
    \end{choices}

    \begin{solution}
        Slide ``Food supply: rice'' shows steadily rising rice production.
    \end{solution}

    \part To give us perspective, Steven Pinker feels that journalists should

    \begin{choices}
        \correctchoice give quantitative and historical background over the last decade or so.
        \choice ask a citizen for an opinion and an expert to suggest how to feel.
        \choice interview a balanced number of people on both sides of an issue.
        \choice give only current facts.
        \choice express their opinions.
    \end{choices}

    \begin{solution}
        On slide ``Pessimism bias'' at 8:42 in the video, he says: ``I think our intellectual and journalistic culture has to become more evidence based,
        data oriented, quantitative. The current practice of journalism—report a story, ask a person on the street their comments on the story,
        and then a columnist or pundit tells people how to emote with regard to the story—that's not a way to give people an accurate understanding of the world.
        Stories, events of the day, should be put into historical context. By historical context I don't mean the Roman Empire.
        I mean like the last 10 years, the last 20 years. And they should be put in quantitative perspective.''
    \end{solution}

    \part Which resource is LEAST likely to ever be completely depleted?

    \begin{oneparchoices}
        \choice Tuna fish
        \choice Polar bears
        \choice Avocados
        \choice Rhinoceroses
        \correctchoice Rare-earth metals
    \end{oneparchoices}

    \begin{solution}
        The answer is rare-earth metals. No non-renewable resource has ever been fully depleted, while some living species have become extinct. See slide ``Resources''.
    \end{solution}

\end{parts}