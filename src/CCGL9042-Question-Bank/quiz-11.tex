\question \textbf{Questions from Quiz 11}

\begin{parts}
    
    \part  In the last 4 decades, the number of people born each day has

    \begin{choices}
        \choice roughly doubled and then roughly halved.
        \correctchoice stayed about the same.
        \choice fluctuated between about 130 million and 145 million.
        \choice fallen roughly in half.
        \choice approximately doubled.
    \end{choices}

    \begin{solution}
        A Demographics in 2050 slide shows the number of annual births not varying much,
        but remaining within a narrow range from about 130 million to 145 million per year for the last 40
        years. That's per year, not per day. Therefore, the answer that is most correct is that the number has
        stayed about the same.
    \end{solution}

    \part What may most likely happen to the world population in 2050?

    \begin{choices}
        \choice Population will roughly double in size due to the demographic transition.
        \choice Longevity will remain at the current level of about 80 years.
        \correctchoice The ratio of the number of elderly to the number of children will grow much bigger.
        \choice People will no longer experience illnesses and major disease outbreaks.
        \choice It will increase gradually until 2035 and remain steady after.
    \end{choices}

    \begin{solution}
        It is likely that elderly people will make up a higher percentage of the population in the
        future. The other statements are possible, but unlikely.
    \end{solution}

    \part Which of the following is incorrect concerning dematerialization?

    \begin{choices}
        \choice Dematerialization will predominate in 30 years' time.
        \correctchoice Dematerialization occurs mainly in developing countries.
        \choice Transitioning from using CD to Spotify for music is an example of dematerialization.
        \choice Dematerialization is facilitated by innovation and technological advancement.
        \choice There will be an adverse trend of dematerialization in poor countries.
    \end{choices}

    \begin{solution}
        Dematerialization is defined as the provision of a product or a service with the
        reduction of material used. The process of dematerialization is facilitated by technological
        advancement and is the major trend in developed countries. Poor countries, as mentioned in the
        lecture, would need to use more materials, especially when they are getting through industrialization
        to improve their economic condition (see slide Income and Environment).
    \end{solution}

\end{parts}