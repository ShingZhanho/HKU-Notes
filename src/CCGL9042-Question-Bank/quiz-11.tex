\question \textbf{Questions from Quiz 11}

\textbf{\color{red}IMPORTANT NOTICE:}
In the previous years, quizzes 10 and 11 were held as one single quiz.
The author has did his best to separate the questions from previous years' question bank
into quiz 10 and quiz 11. However, there is still a chance that some questions from quiz 10
will appear in quiz 11, and vice versa.
You are advised to revise both sections when preparing for these two quizzes.


\begin{parts}
    
    \part  In the last 4 decades, the number of people born each day has

    \begin{choices}
        \choice roughly doubled and then roughly halved.
        \correctchoice stayed about the same.
        \choice fluctuated between about 130 million and 145 million.
        \choice fallen roughly in half.
        \choice approximately doubled.
    \end{choices}

    \begin{solution}
        A Demographics in 2050 slide shows the number of annual births not varying much,
        but remaining within a narrow range from about 130 million to 145 million per year for the last 40
        years. That's per year, not per day. Therefore, the answer that is most correct is that the number has
        stayed about the same.
    \end{solution}

    \part What may most likely happen to the world population in 2050?

    \begin{choices}
        \choice Population will roughly double in size due to the demographic transition.
        \choice Longevity will remain at the current level of about 80 years.
        \correctchoice The ratio of the number of elderly to the number of children will grow much bigger.
        \choice People will no longer experience illnesses and major disease outbreaks.
        \choice It will increase gradually until 2035 and remain steady after.
    \end{choices}

    \begin{solution}
        It is likely that elderly people will make up a higher percentage of the population in the
        future. The other statements are possible, but unlikely.
    \end{solution}

    \part Which of the following is incorrect concerning dematerialization?

    \begin{choices}
        \choice Dematerialization will predominate in 30 years' time.
        \correctchoice Dematerialization occurs mainly in developing countries.
        \choice Transitioning from using CD to Spotify for music is an example of dematerialization.
        \choice Dematerialization is facilitated by innovation and technological advancement.
        \choice There will be an adverse trend of dematerialization in poor countries.
    \end{choices}

    \begin{solution}
        Dematerialization is defined as the provision of a product or a service with the
        reduction of material used. The process of dematerialization is facilitated by technological
        advancement and is the major trend in developed countries. Poor countries, as mentioned in the
        lecture, would need to use more materials, especially when they are getting through industrialization
        to improve their economic condition (see slide Income and Environment).
    \end{solution}

    \part Though the economies of the poorest countries now almost all grow slower than 7\%/year,
        why may the Sustainable Development Goal of economic growth $>$7\%/year in the poorest
        countries eventually be achieved?

    \begin{choices}
        \choice The poorest countries may become more civilized and take more initiatives to protect
        the environment, offering more opportunities to women, youth and people with
        disabilities.
        \correctchoice Rising wages in other countries will drive employers to hire from the shrinking
        pool of low-cost workers, driving up income in the poorest countries.
        \choice The poorest countries may develop world-leading technology.
        \choice Growth fluctuates over time.
        \choice People with talent may migrate to the poorest countries.
    \end{choices}

    \begin{solution}
        The slides showing GDP/capita yearly growth for poor countries says,
        ``But in future, cheap labor scarcer, boosting growth in the last poor countries.''
    \end{solution}

    \part Based on Sapiens, what does Harari think would replace the laws of natural selection in the future?

    \begin{choices}
        \choice Evolution
        \choice Memes
        \correctchoice Conscious design of life
        \choice The laws of physics
        \choice Decisions by extraterrestrial life
    \end{choices}

    \begin{solution}
        Harari thinks that the laws of natural selection will be replaced by the laws of intelligent
        design: in other words, conscious design of life.
    \end{solution}

    \part What major category of death and disability is likely to happen more frequently per person by 2050?

    \begin{choices}
        \choice Cancer (including lung and stomach cancer)
        \correctchoice No major category
        \choice Accidental injuries
        \choice Diarrhea
        \choice Stroke
    \end{choices}

    \begin{solution}
        The Health in 2050 slide says, ``Almost every disease will fall.'' The light colors on the
        graph show that all major categories have decreased over the last few decades (HIV grew, but it's
        falling now, as we learned earlier in the course.)
    \end{solution}

    \part Arthur C. Clarke stated 3 laws. What is their meaning?

    \begin{choices}
        \choice Supernatural phenomena are sometimes real.
        \choice We will increasingly explore space.
        \choice Ignore old scientists.
        \correctchoice The future is often too amazing for us to predict.
        \choice It's tough to make predictions, except about the future.
    \end{choices}

    \begin{solution}
        Clarke's first law states in part that if a respected scientist predicts that something is
        impossible, he is very probably wrong. In other words, many things that seem beyond achieving now
        will eventually be achieved because technology will make advances too great for us to foresee.
    \end{solution}

    \part What does Clay Shirky think are the two main things that have led to today's cognitive surplus?

    \begin{choices}
        \choice Free time and high IQ
        \choice TV and education
        \choice Internet memes and the media
        \choice Internet memes and TV
        \correctchoice Free time and the media
    \end{choices}

    \begin{solution}
        Clay Shirky stated that there are 2 parts that make a cognitive surplus:
        \begin{enumerate*}[label=(\arabic*)]
            \item The cumulative free time and talent in the developed world, and
            \item the media landscape that allows us to work cooperatively and cumulatively together.
        \end{enumerate*}
    \end{solution}

    \part Which is not an example of biological engineering, as mentioned by Harari?

    \begin{choices}
        \choice The potato with a gene of an Arctic fish in its genetic makeup
        \choice The genetically modified E. coli that is used to produce biofuel
        \correctchoice The voles who are in monogamous relationships
        \choice The Vacanti mouse with an ear grown on its back
        \choice The green fluorescent rabbit Alba
    \end{choices}

    \begin{solution}
        The species of voles that he mentioned are naturally monogamous. All other examples are biologically engineered. (see
        the introduction and Of Mice and Men of Sapiens)
    \end{solution}

    \part Which is/are the most cost-effective way(s) to improve happiness in developed and poor countries respectively?

    \begin{choices}
        \choice Improve poverty in both developed and poor countries.
        \choice Treat mental illness in both developed and poor countries.
        \correctchoice Treat mental illness in developed countries and cut poverty in poor countries.
        \choice Facilitate economic development in both developed and poor countries.
        \choice Treat physical illness in developed countries and cut poverty in poor countries.
    \end{choices}

    \begin{solution}
        Treating mental illness (e.g., anxiety and depression) is the most cost-effective way to improve happiness for developed countries
        while cutting poverty would be the most crucial for poor countries (see slide Happiness).
    \end{solution}

    \part By 2050, the people not in extreme poverty will likely

    \begin{choices}
        \choice rise in absolute number but fall as a proportion of the population.
        \correctchoice rise in absolute number and rise as a proportion of the population.
        \choice remain about the same in absolute number.
        \choice fall in absolute number and fall as a proportion of the population.
        \choice fall in absolute number but rise as a proportion of the population.
    \end{choices}

    \begin{solution}
        The slides showing ``World population living in extreme poverty'' says ``Far fewer people in extreme poverty.'' Since the overall
        population will rise by 2050, the proportion in extreme poverty will also fall. Thus, the number and proportion of people not in
        extreme poverty will both rise.
    \end{solution}

    \part What is the main implication of Harari's "Frankenstein Prophecy"?

    \begin{choices}
        \choice Scientists have a long way to go until they can successfully create a new life form.
        \choice Technological advancements will vastly improve human lives.
        \choice Interfering with nature's creations will likely lead to suffering and catastrophe.
        \correctchoice Future scientific engineering may lead to the extinction of our own species.
        \choice Aliens from outer space may colonize our civilization.
    \end{choices}

    \begin{solution}
        Harari thinks that in the future, scientists may be able to engineer life forms superior to us, which may eventually lead to the
        extinction of our own species. He wrote: ``The pace of technological development will soon lead to the replacement of Homo
        sapiens by completely different beings with not only different physiques, but also very different cognitive and emotional worlds.''
    \end{solution}

    \part Ridley feels that in the future

    \begin{choices}
        \choice The world will be more top-down, as corporations and governments become larger and stronger.
        \choice Predators and parasites will quickly disappear.
        \choice Humans will live on the Sun.
        \correctchoice The flame of innovation will decline occasionally in some places.
        \choice Most wilderness will disappear.
    \end{choices}

    \begin{solution}
        In Chapter 11 of ``The Rational Optimist'', Ridley says, ``However reactionary and cautious Europe and the Islamic world
        and perhaps even America become, China will surely now keep the torch of catallaxy alight, and India, and maybe Brazil,
        not to mention a host of smaller free cities and states.'' Thus, innovation will fluctuate and will decline from time to
        time in some places. He says that predators and parasites will still cause problems and that human nature (instincts) won't
        change for better or worse, but most likely things will go well, including the expansion of wilderness.
    \end{solution}

    \part By 2050, the size of the middle class will most likely be closest to

    \begin{oneparchoices}
        \choice 2 billion
        \choice 0.5 billion
        \choice 14 billion
        \correctchoice 7 billion
        \choice 1 billion
    \end{oneparchoices}

    \begin{solution}
        The slides showing the graph of the middle class says:
        ``Huge growth of middle class.'' and 
        ``From ~4 billion now to ~7 billion.''
        7 billion is near the entire human population now.
    \end{solution}

    \part In the next two decades, compared to the four before 2020, the global number of people dying each month is forecast to

    \begin{choices}
        \choice fall faster.
        \correctchoice rise faster.
        \choice rise from about 600 million to about 900 million.
        \choice remain about the same.
        \choice fall at about the same rate.
    \end{choices}

    \begin{solution}
        A Demographics in 2050 slide shows global deaths. They are forecast to rise from about 60 million to 90 million annually,
        not monthly. That is a much faster rise than in the four decades before the COVID-19 spike in deaths. Thus, the answer is ``rise faster''.
    \end{solution}

\end{parts}