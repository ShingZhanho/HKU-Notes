\question \textbf{Questions from Quiz 5}

\begin{parts}
    
    \part How did the British escape the Malthusian trap in the 1800s and avoid the fate of Japan and other countries?

    \begin{choices}
        \choice For centuries the relatively rich in Britain had fewer and fewer children, reducing the demand for goods and keeping the population below the carrying capacity.
        \correctchoice So-called “Ghost acres” providing food from other countries as well as emigration into America released the Malthusian pressure on British society.
        \choice As the industrial revolution started, an increase in rice production decreased the mortality rate.
        \choice The government set up family planning committees across the country to educate woman about the Malthusian trap.
        \choice By returning to self-sufficiency and only providing for their own family, British people created a stable society with enough supply.
    \end{choices}

    \begin{solution}
        Based on the section ``British exceptionalism'' in The Rational Optimist. So-called ``Ghost acres''
        providing food from other countries as well as emigration into America released the Malthusian
        pressure on British society. The other answers are wrong since ``the relatively rich had more children
        than the relatively poor'', self-sufficiency is characteristic for a poor economy and not an escape
        from the Malthusian trap, during much of the industrial revolution fertility was high in Britain, and
        there simply was no planning committee created by the government.
    \end{solution}

    \part Which of the following factors may have helped the United Kingdom (UK) to escape the Malthusian Trap during the 1800s?

    \begin{choices}
        \choice Longer working hours, agricultural and industrial innovation, and increased immigration
        \choice Import of food, imported labour from colonies, and the discovery of oil
        \choice A rising number of workers to supply labor to factories, import of food, and government policies enforcing contraception
        \correctchoice Agricultural and industrial innovations, import of food, and government policies conducive to business
        \choice Innovations in agriculture and industry that improved efficiency of workers in producing food, thus increasing self-sufficiency
    \end{choices}

    \begin{solution}
        The following factors may have helped the United Kingdom to escape the Malthusian Trap during
        the 1800s:
        \begin{itemize}
            \item Decreased price of food due to food imports that resulted in a decreased percentage of incomepeople expended on food.
            \item Agricultural and industrial innovations made essentials (such as food, clothing) cheaper.
            \item Government policies were conducive to businesses (fewer monopolies compared to France or China).
            \item Britain's control over the production of iron and coal helped in the widespread application of steam power.
            \item Emigration (not immigration) of excess population to North America and its colonies.
        \end{itemize}
        However, there is no evidence to suggest an increase in work hours or working efficiency (measured
        in food produced per hour).
    \end{solution}

    \part Which of the following statements about the Malthusian trap is correct?

    \begin{choices}
        \choice Invention and innovation permanently improve humanity's standard of living.
        \choice Population always exceeds food production.
        \choice Death rate remains constant.
        \correctchoice Population changes due to changes in resources.
        \choice Carrying capacity remains constant.
    \end{choices}

    \begin{solution}
        Referring to the lecture slides ``Carrying capacity'' and ``Malthusian trap'', when there are more
        resources available, the population increases. the rate of population growth exceeds the rate of
        food production, and famine occurs. Then, the population falls.
    \end{solution}

    \part According to Ridley, what was the general trend of European economies from the 1300s through the 1400s?

    \begin{choices}
        \choice Wages fell because of disasters such as poor harvests and the Black Death.
        \choice Hands-off policies by central governments, affected by classical economist Adam Smith, leading to anarchism - a precursor to the Enlightenment movement.
        \choice Consolidation of power by the central government, gradually transitioning European economies into socialism.
        \choice Constant warfare and expansion, leading to the relief of overpopulation through emigration to the new colonies in the Americas.
        \correctchoice A shift from labour-intensive economies toward capital-intensive economies.
    \end{choices}

    \begin{solution}
        In The Rational Optimist, Ridley argues that 14th century Europe ``was on a trajectory towards a
        labour-intensive revolution.'' However, by the 15th century, Europe had transitioned to a ``labour-
        saving industrial trajectory instead,'' and forming a capital-intensive economy. This trend is observed
        to have continued throughout the 17th century with ``famine, plague, and war'' reducing the
        European population.
    \end{solution}

    \part Which reason best explains why per capita income in England was higher in the 1400s than any later time until the Industrial Revolution?

    \begin{choices}
        \choice greater economic freedom
        \correctchoice insufficient labour supply
        \choice discovered new natural resources
        \choice developed advanced technology in agriculture
        \choice the introduction of electricity
    \end{choices}

    \begin{solution}
        The Rational Optimist: ``the plague may have been one of the sparks that lit the Renaissance,
        because the shortage of labour shifted income from rents to wages as landlords struggled to find
        both tenants and employees. With rising wages, some of the surviving peasantry could once more
        just afford the oriental luxuries and fine cloth that Lombard and Hanseatic merchants supplied. ...
        Per capita income in England was probably higher in1450 than it would be again before 1820.''
    \end{solution}

    \part Which of the following is most accurate regarding the Malthusian Trap?

    \begin{choices}
        \choice Research shows that working longer hours helped the UK escape the Malthusian Trap.
        \choice Countries like India, Bangladesh, China and Peru adopted a purely voluntary approach to promote contraception.
        \correctchoice Cheap food was one factor that helped the UK escape the Malthusian Trap during the Industrial Revolution.
        \choice The Malthusian Trap is a feedback loop in which income increases, causing population to rise to a new carrying capacity, inevitably leading income to rise further.
        \choice Malthus concluded that the power in the earth to produce subsistence for men is indefinitely greater than the power of population.
    \end{choices}

    \begin{solution}
        The correct answer is cheap food. Slides titled ``More resources?'' showed that Britain imported
        wheat from the US from the mid-1800s, leading wheat to get cheaper. Since people on average
        spent most of their income on food, cheaper food allowed more people to survive and even to buy
        luxuries.
    \end{solution}

    \part Which of the following factors may have contributed most to the demographic transition
        seen in the United Kingdom (UK) during the Industrial Revolution, when there was a
        decrease in the fertility rates of women (measured by children/woman) as income increased?

    \begin{choices}
        \choice The increased cost of raising children, decreased child survival, advances in contraception
        \choice Pollution, vaccination, World-War I
        \correctchoice Female literacy, urbanisation, vaccination
        \choice Vaccination, female literacy, women's right-to-vote
        \choice Urbanisation, pollution, women's right-to-vote
    \end{choices}

    \begin{solution}
        The correct answer is female literacy, urbanisation, and vaccination. Medical advances such
        as vaccination allowed child mortality to decrease, and, as a result, more children survived.
        This decreased the need to have more children to ensure the desired number survived.
        Urbanisation made the cost of raising children more expensive and decreased the economic
        benefit of children compared to agricultural societies. Female literacy allowed women to
        work in jobs, offering an alternative to being a housewife. Women in the UK did not get the
        right to vote until 1918, and World War I didn't happen until 1914. and neither was
        discussed in the chapter or lecture. Pollution may have decreased child survival, which
        would have had the opposite effect.
    \end{solution}

    \part Which of the following about population growth is correct?

    \begin{choices}
        \choice A decreasing rate of fertility is beneficial to nearly all industries in the world.
        \choice The percentage growth rate in world population is now close to the historical maximum.
        \correctchoice Hong Kong has one of the highest rates of contraceptive use and lowest rates of fertility in the world.
        \choice The fertility rate was low before women could receive education.
        \choice The high population growth rate of England in the preceding centuries made it favourable to start the Industrial Revolution.
    \end{choices}

    \begin{solution}
        From a graph of contraceptive use on one of the lecture slides titled ``Birth control'', Hong
        Kong is the point with the highest contraceptive prevalence.
    \end{solution}

    \part Which statement BEST summarises Matt Ridley's principal argument regarding what has led
        to a decrease in the human population growth rate?

    \begin{choices}
        \correctchoice An increase in the division of labour and trade led to a decrease in population growth.
        \choice As people increasingly learn about the population explosion and its effects on the
        environment and the economy, they increasingly decide to forego having traditional, large
        families.
        \choice Climate change has led to a stabilisation in the growth of the world food supply in calories
        per head, and the limited food has restricted human population growth.
        \choice Coercive contraception programs have decreased fertility in many, though not all,
        countries with the fastest growing populations.
        \choice In the most rapidly growing countries, high child mortality tends to discourage parents
        from having more children.
    \end{choices}

    \begin{solution}
        Ridley's main argument is that an increase in the division of labour and trade leads to a
        decrease or halt in population growth. ``It is somewhat distasteful to the intelligentsia to
        accept that consumption and commerce could be the friend of population control...The
        more interdependent and well-off we all become, the more the population will stabilise well
        within the resources of the planet.''
        The growth of the world food supply in calories per head has dramatically increased not
        stabilized. There is no evidence to suggest that carcinogens found in GMO crops have led to
        a decrease in male and female fertility rates.
    \end{solution}

    \part What is the most important cause of the demographic transition?

    \begin{oneparchoices}
        \choice rising income
        \choice better education
        \correctchoice not known
        \choice female emancipation
        \choice urbanization
    \end{oneparchoices}

    \begin{solution}
        ``The Rational Optimist'' says about the demographic transition: ``Deliciously, nobody really
        knows how to explain this mysteriously predictable phenomenon... the best that can be said
        for sure about the demographic transition is that countries lower their birth rates as they
        grow healthier, wealthier, better educated, more urbanised and more emancipated.'' Thus,
        we don't know which, if any, of those causes is most important.
    \end{solution}

    \part What was a major reason that the Industrial Revolution happened in England first and not China?

    \begin{choices}
        \choice China lacked a culture of invention.
        \choice Before the Industrial Revolution, China had been economically inferior to Europe.
        \choice England's mild, wet climate yielded an ample supply of food.
        \choice The culture in Europe was more advanced.
        \correctchoice Before the Industrial Revolution, English wages were among the highest in the world, while China had cheap labor.
    \end{choices}

    \begin{solution}
        According to the video on slide ``Which one?'', wages in England were among the highest in
        the world, encouraging automation. In China, wages were lower.
    \end{solution}

    \part How did America most contribute to England escaping the Malthusian Trap?

    \begin{choices}
        \choice America was the hub of innovation and new technologies which helped the Britishers.
        \correctchoice America admitted migrants from Britain, which helped reduce the rapid rise in the British population.
        \choice America had a huge source of coal, which England imported in exchange for wheat.
        \choice America had no major role to play in England coming out of the Malthusian Trap.
        \choice Americans provided demand for British wool textiles, and thus England had an inflow of money.
    \end{choices}

    \begin{solution}
        The correct answer is that people emigrated to the USA, and food was exported from
        America to England. Both of these factors helped in removing the pressure on population.
        Thus, England did not fall into the Malthusian trap. Ridley says, ``Two things, says the
        historian Kenneth Pomeranz, were vital to Europe's achievement: coal and America... Not
        only did the Americas ship back their produce; they also allowed a safety valve for
        emigration to relieve the Malthusian pressure of the population boom induced by
        industrialization.''
        
        Britain did earn from its wool textile exports, including to America, but cotton textiles were
        the major export (``It was cotton textiles that drove the early Industrial Revolution.''--Crash
        Course video on slide ``Which one?'').
    \end{solution}

    \part What helped Britain escape Malthus' trap?

    \begin{choices}
        \choice big homeland:empire ratio
        \choice low wages
        \choice working longer hours
        \choice importing people to Britain
        \correctchoice natural resources
    \end{choices}

    \begin{solution}
        Slide ``Which one?'' listed the factors. Coal and iron are one of the possible factors.
        The other choices here are the opposite of those listed or are not listed.
    \end{solution}

    \part If improved health care and nutrition save children in the poorest regions from dying, what
        change is most likely to happen there within a generation?

    \begin{choices}
        \choice As people gain confidence that their children will survive, they will have more children.
        \choice Population decline
        \correctchoice People will have fewer children.
        \choice A demographic transition, leading to a Malthusian trap
        \choice More children will survive, creating overpopulation, resulting in starvation.
    \end{choices}

    \begin{solution}
        A graph on lecture slide ``Modern demographic transition'' shows a strong correlation
        between infant mortality and fertility, and the video on the same slide explains that confidence in
        survival of children leads parents to have fewer. A common misconception, as described in The
        Rational Optimist, is that more children surviving leads to overpopulation: ``Jeffrey Sachs recounts that
        on `countless occasions' after a lecture a member of the audience has `whispered' to him `if we save
        all those children, won't they simply starve as adults?' Answer: no. If we save children from dying,
        people will have smaller families.'' / Within just one generation, the population is unlikely to decline.
        Even if fertility falls below the replacement level, population continues to rise for a long time as the
        generation born during the high fertility period grows and has their own children.
    \end{solution}

    \part How did high wages and cheap energy most likely help trigger the Industrial Revolution in Britain?

    \begin{choices}
        \choice It led the government to tax energy sources and legislate lower wages, supplying abundant, cheap labor for factories.
        \choice It led to reduced mortality, increasing the population to supply abundant labor for factories.
        \choice It led British textile production to move to the largest British colony, India
            where wages were much lower and agricultural efficiency supplied abundant, cheap cotton.
            The profits flowed to Britain, leading people there to buy many more manufactured goods from other countries.
        \correctchoice It led companies to create ways to replace workers with coal power or monetary investment.
        \choice It led consumers to buy large cars, stimulating the auto industry.
    \end{choices}

    \begin{solution}
        In the Crash Course video on lecture slide ``Which one?'', John Green cited historian
        Robert Allen as saying that high wages and cheap energy led directly to the Industrial Revolution by
        motivating firms to develop ways to substitute money and energy for labor. Although Indian cotton and
        labor were cheap, British textile production mechanized instead of moving to India.
    \end{solution}

    \part Of the following government policies to deal with a fast growing population, which would Ridley likely prefer?

    \begin{choices}
        \choice Governments should move urban workers to farms to raise food production until escape from the Malthusian trap is achieved.
        \correctchoice Governments should do nothing on this issue.
        \choice Governments should delegate reproduction to the leader, like with social insects.
        \choice Governments should slow economic growth to decrease population growth to a sustainable level.
        \choice Governments should require abortions for parents having $>3$ children.
    \end{choices}

    \begin{solution}
        The answer is nothing. The coercive policies are wrong. Ridley said,
        ``Yet the tragedy is that this top-down coercion was not only counter-productive; it was unnecessary. Birth rates were
        already falling rapidly in the 1970s all across the continent of Asia quite voluntarily. They fell just as far
        and just as fast without coercion.'' He also concludes by quoting Ron Bailey's saying -
        ``There is no need to impose coercive population control measures; economic freedom actually generates a benign
        invisible hand of population control.'' / Self-sufficiency is wrong. Ridley says,
        ``Human beings are a species that stops its own population expansions once the division of labour reaches the point at
        which individuals are all trading goods and services with each other, rather than trying to be
        self-sufficient.''
        Since he thinks that population expansions can be stopped by specialization and
        trading, people should not be self-sufficient. / Also, he says that ``Probably by far the best policy for
        reducing population is to encourage female education.''
        So, governments should encourage females to receive education.
    \end{solution}

    \part Why did English wages fall roughly in half in the 1200s?

    \begin{oneparchoices}
        \choice self-sufficiency
        \choice falling wool production
        \choice plague
        \correctchoice rising population
        \choice decreasing rents
    \end{oneparchoices}

    \begin{solution}
        ``The Rational Optimist'' says: ``In the thirteenth century the population of England
        seems to have doubled... the population boom overtook the economy's productivity. Rents inflated
        and wages deflated: the rich were bidding up land prices while the poor were bidding down wages.''
    \end{solution}

    \part Which is the most important cause of the demographic transition?

    \begin{choices}
        \choice Urbanization is the major reason why the birth rate decreases. This is because raising a child in cities is much more expensive than in rural area.
        \choice Female emancipation creates alternative productive uses of time and effort—jobs instead of childrearing—thus decreasing birth rates.
        \choice The increase in wealth allows people to afford more luxuries—diversions from having many children.
        \choice Successful government policy on birth control is the key to lowering birth rates.
        \correctchoice There is no consensus on a single factor being the major cause of the demographic transition.
    \end{choices}

    \begin{solution}
        In the section titled ``An unexplained phenomenon'', the book mentions that,
        ``Deliciously, nobody really knows how to explain this mysteriously predictable phenomenon.''
        Later, Ridley says, ``In other words, the best that''
        can be said for sure about the demographic transition is that countries lower their birth rates as they
        grow wealthier, more urbanized and more educated and emancipated.''
    \end{solution}

    \part What steps occur in the demographic transition?

    \begin{choices}
        \choice No observable pattern
        \correctchoice Mortality falls, then population rises, then fecundity decreases.
        \choice Fecundity decreases, then population falls, then mortality falls.
        \choice Mortality increases, then population falls, then fecundity decreases.
        \choice Population rises, then mortality falls, then fecundity decreases.
    \end{choices}

    \begin{solution}
        The correct answer is that mortality falls, then population rises, then fecundity decreases,
        as discussed in the video on the slide titled ``Modern demographic transition'' and in the book:
        ``The pattern is always the same: mortality falls first, causing a population boom, then a few decades later,
        fecundity falls quite suddenly and quite rapidly.''
    \end{solution}

    \part Just after economic expansions ended in crashes (before the Industrial Revolution), populations increased their

    \begin{choices}
        \choice literacy.
        \choice luxury spending.
        \choice specialization.
        \choice comfort.
        \correctchoice self-sufficiency.
    \end{choices}

    \begin{solution}
        ``The Rational Optimist'' says, ``Until 1800 this was how every economic boom ended: with a partial return to self-sufficiency''.
    \end{solution}

    \part Which of the following is an expected consequence of a demographic transition leading to lower population growth?

    \begin{choices}
        \choice A resurgence of childhood diseases
        \choice More business for toy stores
        \correctchoice An increase in abandoned homes
        \choice Increased competition for jobs
        \choice Food shortages
    \end{choices}

    \begin{solution}
        An expected consequence of lower population growth is an increase in abandoned homes.
        Other consequences include more demand for elderly-related businesses (e.g., geriatricians, elderly care homes, and diaper manufacturers).
        The shrinking youth population should decrease, not increase, competition for graduate jobs.
    \end{solution}

\end{parts}