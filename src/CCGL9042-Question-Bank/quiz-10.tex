\question \textbf{Questions from Quiz 10}

\begin{parts}

    \part In The Rational Optimist, Matt Ridley argued that climate change has been doing more good
        than harm. Which of the following were one of his stated reasons for his argument?

    \begin{choices}
        \choice Warmer weather can make unbearably cold winters more bearable for everybody.
        \choice Climate change has been increasing global temperatures, increasing the stretch of tropic
        regions which support huge biodiversities. The speciation rate may outweigh the extinctions
        partly caused by climate change.
        \choice There are many concrete pieces of evidence that species have been wiped out due to
        climate change, and the resulting simplification of ecosystems is better for the environment.
        \choice The melting of ice caps means more free water enters our global water cycle and thus
        places experiencing overly dry climates may experience more pleasant weather at nobody's
        expense.
        \correctchoice The increased level of carbon dioxide boosts crop growth, as well as the growth of wild
        vegetation, so climate change is not necessarily bad.
    \end{choices}

    \begin{solution}
        ``Not only will
        the warmth improve yields from cold lands and the rainfall improve yields from some dry
        lands, but the increased carbon dioxide will itself enhance yields, especially in dry areas.''
        (The Rational Optimist)
    \end{solution}

    \part Which of these statements about climate change is most accurate?

    \begin{choices}
        \correctchoice The degree of climate changes and the expense and impact of methods to prevent them
        are difficult to measure, thus it is hard to tell whether mitigation is a good investment.
        \choice The Copenhagen Consensus assessed climate change action as higher priority than health
        action.
        \choice Sea level will continue to rise up to a decade or so after CO2 emissions stop.
        \choice Domesticated crops such as wheat were bred at the pre-industrial CO2 level of around
        270 ppm, thus their yields will fall as CO2 level rises above that level.
        \choice Because electricity has come to play such a large role in the economy, and since it relies
        heavily on coal, the electricity sector is responsible for the majority of greenhouse gas
        emissions.
    \end{choices}

    \begin{solution}
        Whether
        mitigation is cost-effective is complicated as warming, effects, and cost-effectiveness of
        solutions are hard to measure. See slide ``Is mitigation cost-effective?''
    \end{solution}

    \part Why does the increase of carbon dioxide in the air lead to global warming?

    \begin{choices}
        \choice Carbon dioxide provide plants a material for photosynthesis, an exothermic reaction.
        \choice Carbon dioxide in air absorbs ultraviolet light from the Sun, heating the carbon dioxide,
        which then bumps into other air molecules, transferring the heat to them.
        \choice Carbon dioxide in air refracts ultraviolet light from the Sun, causing more heat to transfer
        to the land.
        \correctchoice Carbon dioxide absorbs infrared light coming from the Earth and re-emits the light, which
        heats other molecules.
        \choice Carbon dioxide in air dissolves in water, where it forms small bubbles which inhibit
        formation of ice. Water is darker than ice, so it absorbs more sunlight and thus gets warmer.
    \end{choices}

    \begin{solution}
        See slide ``Greenhouse effect''.
    \end{solution}

    \part What communist ideal does Matt Ridley predict will be at least partly realized in the near future?

    \begin{choices}
        \correctchoice We'll get what we need and give what we can.
        \choice Profit will no longer drive commerce.
        \choice Religion will disappear.
        \choice Economic class differences will vanish.
        \choice Countries will merge into a world government, which he suggests be named Utoptimistan.
    \end{choices}

    \begin{solution}
        Chapter 11 of The Rational Optimist: ``People are willing to share their photographs on Flickr,
        their thoughts on Twitter, their friends on Facebook, their knowledge on Wikipedia, their
        software patches on Linux, their donations on GlobalGiving, their community news on
        Craigslist, their pedigrees on Ancestry.com, their genomes on 23andMe, even their medical
        records on PatientsLikeMe. Thanks to the internet, each is giving according to his ability to
        each according to his needs, to a degree that never happened in Marxism.''
    \end{solution}

    \part In The Rational Optimist, which of these factors did research find explained most of the
        variation in economic growth among different countries throughout the world?

    \begin{choices}
        \correctchoice secure rights to property
        \choice fossil fuel reserves
        \choice the number of hours per day that people work
        \choice diversity of large animal species
        \choice a history of strong colonial rule
    \end{choices}

    \begin{solution}
        In the ``Bound to fail?'' section of Chapter 10 of The Rational Optimist: ``When Daron
        Acemoglu and his colleagues compared property rights with economic growth throughout
        the world, they found that the first explained an astonishing three quarters of the variation
        in the second''.
    \end{solution}

    \part Which of the following best enhanced economic growth of African countries?

    \begin{choices}
        \correctchoice good property rights
        \choice strong colonial rule
        \choice rich mineral wealth
        \choice government planning of the economy
        \choice AIDS from other countries
    \end{choices}

    \begin{solution}
        The answer is good property rights. In the ``Bound to fail?'' section of Chapter 10 of The
        Rational Optimist: ``When Daron Acemoglu and his colleagues compared property rights
        with economic growth throughout the world, they found that the first explained an
        astonishing three quarters of the variation in the second''.
    \end{solution}

    \part If the Hong Kong government gives the Global Fund HK\$150 from each of its 7.5 million people, about how many lives would it save?		

    \begin{oneparchoices}
        \correctchoice 100,000
        \choice 10,000
        \choice 100
        \choice 10
        \choice 1,000
    \end{oneparchoices}

    \begin{solution}
        HK\$150 $\times$ 7.5 million is about HK\$1100 million. The			
        Global Fund saves lives for about HK\$11,000 each (``Current problems'' slide). HK\$1100 million $\div$ HK\$11,000 = 100,000.
    \end{solution}

    \part How may African governments follow the path of East Asian economic development, according to The Rational Optimist?

    \begin{choices}
        \correctchoice Setting up free-trade areas
        \choice Converting to a pictographic language
        \choice Providing free education
        \choice Funding industrial research grants to promote innovation
        \choice Growing rice
    \end{choices}

    \begin{solution}
        In the section ``The world is your oyster'' in Chapter 10 of The Rational Optimist: ``In
        1978 China was about as poor and despotic as Africa is now. It changed because it deliberately
        allowed free-trading zones to develop in emulation of Hong Kong. So, says the economist Paul
        Romer, why not repeat the formula? Use Western aid to create a new `charter city' in Africa on
        uninhabited land, free to trade with the rest of the world, and allow it to draw in people from the
        surrounding nations.''
    \end{solution}

    \part Which of the following statements is most accurate?

    \begin{choices}
        \choice The amount of fossil fuel used is not consistent with CO2 added to the atmosphere.
        \correctchoice To motivate mitigation, governments can implement CO2 dividends.
        \choice An advantage of the global priorities method to solve problems is that it requires little estimation of effects and costs to mitigate the problems.
        \choice Over half of greenhouse gas emissions are from electricity generation.
        \choice The rise of sea level will stop when CO2 emissions stop.
    \end{choices}

    \begin{solution}
        See slide ``How can we motivate mitigation?''
    \end{solution}

    \part From the lecture, which world sea level rise is in the range forecast by 2100?

    \begin{oneparchoices}
        \choice 1 dekameter
        \correctchoice 50 cm
        \choice 250 cm
        \choice 10 cm
        \choice 2 cm
    \end{oneparchoices}

    \begin{solution}
        Slide ``Sea level'' says that a 3-10 dm rise is predicted in the 21st century. 50 cm, or 5 dm, is in that range.
        The amount of sea level rise is very important to know because a large rise could
        affect billions of people and trillions of dollars of infrastructure.
    \end{solution}

    \part Which statement is most accurate?

    \begin{choices}
        \choice Earth has warmed 2.5-3.5 $^\circ\text{C}$ since 1850.
        \choice Electricity generation causes most of the greenhouse gas emissions.
        \choice The sea may rise 5 m by 2100.
        \choice Sea level rise will stop when CO2 emissions stop.
        \correctchoice Carbon dividends can motivate mitigation.
    \end{choices}

    \begin{solution}
        (``How can we motivate mitigation?'' slide). The sea may rise 3-10 dm by 2100, not 5 m (``Sea level'' slide). The rise of
        sea level will not stop when CO2 emissions stop; the rise will continue for centuries after CO2
        emissions stop (``Sea level'' slide). There is no single sector responsible for most greenhouse gas
        emission; electricity generation accounts for 25\% of greenhouse gas emission (``Sources'' slides).
        Earth has warmed only about 1 $^\circ\text{C}$ since 1850; I estimate that Earth will warm 2.5-3.5 $^\circ\text{C}$
        by 2100 (``How much has Earth warmed already?'' and ``How much will Earth warm?'' slides).
    \end{solution}

    \part How would redistributing some money from rich people to poor people affect overall
        happiness or satisfaction in a population?

    \begin{choices}
        \choice There would be an overall increase in happiness because, as income rises,
        additional money has less and less effect, until it decreases happiness.
        \choice There would not be much overall effect because the decrease in happiness of the
        rich would roughly balance the increase in happiness of the poor.
        \choice There would be an overall fall in happiness due to the 100x multiplier effect, in which
        rich people, through stimulating economic activity, can create up to approximately
        100x as much happiness as do poor people.
        \choice There would be an overall decrease in happiness because the life satisfaction of rich
        people is much higher than for poor people.
        \correctchoice There would be an overall increase in happiness because each unit of money
        affects poor people far more than rich people.
    \end{choices}

    \begin{solution}
        As shown in the lecture, ``Each dollar raises life satisfaction $\sim$100x more for someone
        in extreme poverty than for someone earning HK\$33,000/month.'' There would be an overall increase
        in happiness because each unit of money affects poor people far more than rich people.
    \end{solution}

    \part Why did the Copenhagen Consensus prioritise health problems over the problem of climate change?

    \begin{choices}
        \choice Because there is almost no solution to the problem of climate change
        \correctchoice Because solving health problems is much more efficient than solving the problem of climate change
        \choice Because people have more difficulty understanding long-term problems, like climate change, than short-term problems, like health
        \choice Because health problems affect the environment more than does climate change
        \choice Because environmentalists are exaggerating the negative impact of global warming
    \end{choices}

    \begin{solution}
        The Global Priorities video in the lecture presents climate change as a
        real and major problem that has solutions. But health problems are also big and have solutions, and
        these can be accomplished much more cheaply, thus helping many more people with the same
        amount of money.
    \end{solution}
    
    \part If changes in fossil fuel use continue the trends of the last 7 decades for the next 7 decades,  what level will atmospheric CO, likely be in 2100?

    \begin{oneparchoices}
        \choice 300-400 ppm
        \correctchoice $>$600 ppm
        \choice $<$300 ppm
        \choice 500-600 ppm
        \choice 400-500 ppm
    \end{oneparchoices}

    \begin{solution}
        Slide ``What will CO2 level be?'' shows that a freeze in the rate of CO2 emission would
        lead to a level of about 770 ppm in 2100. Since CO2 emission has been rising for the last 7 decades,
        a continuation of that trend would lead to an atmospheric level even higher than 770 ppm.
    \end{solution}

    \part Ridley feels that the biggest cause of Botswana's economic success was

    \begin{choices}
        \choice a higher education level among citizens than in other African countries.
        \correctchoice good institutions and property rights.
        \choice abundant resources: cattle and diamonds.
        \choice efficient, one-party government.
        \choice an abundant labor force.
    \end{choices}

    \begin{solution}
        ``But its biggest advantage is one that the rest of Africa could easily
        have shared: good institutions. In particular, Botswana turns out to have secure, enforceable property rights that are fairly widely
        distributed and fairly well respected.'' (Chapter 10, section ``Bound to fail?'')
    \end{solution}

    \part Which of the following statements about climate change is TRUE?

    \begin{choices}
        \choice Replacing fossil fuels with renewable energy is the optimal strategy to slow down global warming because electricity
        generation contributes the majority of our greenhouse gas emissions.
        \choice The three largest sources of greenhouse gas are electricity generation, manufacturing and transportation.
        \choice In recent years, the contribution of natural gas to carbon dioxide emission has exceeded that of coal.
        \correctchoice Compared with health action, the Copenhagen Consensus generally puts a lower priority on climate change action.
        \choice If we could stop increasing the rate of fossil fuel burning, and just keep it at the current rate, carbon dioxide in the air
        would quickly return to the pre-industrial level.
    \end{choices}

    \begin{solution}
        The Copenhagen Consensus generally was that climate change action is important, but that other action, such as
        giving Vitamin A supplements to prevent blindness, or preventing HIV transmission, helps people more for lower cost.
    \end{solution}

    \part What does Ridley suggest is the best way to raise Africa's life quality in the long run?

    \begin{choices}
        \choice Reduce population growth by education.
        \choice Build government-operated factories to boost industrialization.
        \choice Replicate the success of Europe's early growth by raising trade barriers to divide the largest countries into 10,000 mini-
        nations to enhance competition and improve the conditions for entrepreneurs.
        \correctchoice Standardize informal laws and make the legal system more accessible.
        \choice Provide social welfare benefits to help equalize income.
    \end{choices}

    \begin{solution}
        In the section titled ``The world is
        your oyster'', he said, ``It is this kind of institutional reform that will in the end do far more for African living standards than dams,
        factories, aid or population control.''
    \end{solution}

    \part Since money is limited, Bjorn Lomborg suggested in his TED talk that \fillin[][2cm] should prioritize \fillin[][2cm] to improve the world.

    \begin{choices}
        \choice scientists; solutions to problems
        \choice scientists; problems
        \choice economists; solutions to problems
        \correctchoice economists; problems
        \choice scientists; research
    \end{choices}

    \begin{solution}
        Scientists provide solutions to the problems within their fields of
        interest but are not trained to compare the costs and benefits of solutions in different areas. That is what economists do.
        We should not prioritize problems because many are big but insoluble, like aging. Instead we should prioritize solutions to
        problems.
    \end{solution}

    \part According to Bjorn Lomborg's talk about prioritizing, which project came first in the priority list drawn up by the Copenhagen Consensus in 2004?

    \begin{choices}
        \choice Deal with malaria
        \choice Manage climate change
        \correctchoice Deal with HIV/AIDS 
        \choice Eliminate barriers to trade
        \choice Solve global obesity
    \end{choices}

    \begin{solution}
        According to the Global priorities video, dealing with HIV/AIDS was the first priority.
    \end{solution}

    \part According to the video of political scientist Bjorn Lomborg, which one of the following was the lowest priority problem we should tackle, in terms of cost-effectiveness?

    \begin{oneparchoices}
        \choice Barriers to Free Trade
        \choice HIV/ AIDS
        \correctchoice Climate Change
        \choice Malaria
        \choice Malnutrition
    \end{oneparchoices}

    \begin{solution}
        Lomborg stated that the resources required to address the problem of climate change could be more cost-effectively
        diverted to programs for HIV/AID treatment and prevention, micronutrients for malnutrition, promoting free trade, 
        and for treatment and prevention of malaria. He did not deny the problem of climate change but suggested that resources
        should be prioritised to tackle problems in the most cost-effective way to yield the greatest benefit to humankind.
    \end{solution}

    \part Which one of the following is NOT an effect of human-induced greenhouse gas emission?

    \begin{choices}
        \correctchoice Increase in sea alkalinity
        \choice Changes in global weather patterns
        \choice Opening of some new shipping routes
        \choice Increased growth of some plants
        \choice Increased species extinctions
    \end{choices}

    \begin{solution}
        Increasing carbon dioxide in the air dissolves in the sea, decreasing sea alkalinity, not increasing it.
    \end{solution}

    \part Which one of the following is NOT an effect of human-induced greenhouse gas emission?

    \begin{choices}
        \choice Increased species extinctions
        \choice Increased growth of some plants
        \correctchoice Decrease in sea acidity
        \choice Changes in global weather patterns
        \choice Opening of some new shipping routes
    \end{choices}

    \begin{solution}
        Increasing carbon dioxide in the air dissolves in the sea, increasing sea acidity, not decreasing it.
    \end{solution}

    \part In The Rational Optimist, which of these factors did research find explained most of the differences in level of growth in economies of countries throughout the world?

    \begin{choices}
        \correctchoice trust in ownership
        \choice a history of strong colonial rule
        \choice number of patents
        \choice natural resources
        \choice the number of hours per day that people work
    \end{choices}

    \begin{solution}
        Trust in ownership, or property rights. In the ``Bound to fail?'' section of Chapter 10 of The Rational Optimist:
        ``When Daron Acemoglu and his colleagues compared property rights with economic growth throughout the world,
        they found that the first explained an astonishing three quarters of the variation in the second''.
    \end{solution}

\end{parts}