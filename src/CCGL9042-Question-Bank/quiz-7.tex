\question \textbf{Questions from Quiz 7}

\begin{parts}
    
    \part What did Matt Ridley mean by ``the secret of the industrial revolution was shifting from current solar power to stored solar power?''

    \begin{choices}
        \choice The invention of batteries allowed the storage of energy generated from solar panels.
        \choice The abolition of slavery was due to the widespread adoption of machinery.
        \choice The smog from industries decreased sunlight and caused severe environmental damage to forests.
        \correctchoice Shifting energy sources from hay, timber, water, and wind to fossil fuels supplied energy to sustain the Industrial Revolution.
        \choice The Agricultural Revolution provided the biomass needed to power the Industrial Revolution and therefore escape the Malthusian trap.
    \end{choices}

    \begin{solution}
        Ridley was referring to the transition from the use of timber, hay, water, and wind (as current solar
        power) to the use of fossil fuels, especially coal (as stored solar power), which was key to sustaining
        the momentum of the industrial revolution.
    \end{solution}

    \part Battery capacity is the amount of energy a battery can store. A news article reported that
        Tesla's Gigafactory would soon ``get to 35 GWh of battery cell production'', but the article
        did not say in what time interval that would be measured. How could the article better
        express the production rate of the battery factory?

    \begin{choices}
        \choice Battery capacity per length, using units of energy (not power)
        \choice Progress of the assembly line, using units of speed (not energy)
        \correctchoice Battery capacity per time period, using units of power (not energy)
        \choice Progress of the assembly line, using units of speed (not power)
        \choice Using units of million kWh (not GWh)
    \end{choices}

    \begin{solution}
        Bigger batteries store more energy. A good way to express the production rate of a battery factory
        would be the energy-holding ability of the batteries it manufactures in a certain time, which you'd
        measure in units of energy/time, or power. The error in the article was that a GWh is a unit of
        energy, not power (because a Watt is a unit of power, and power multiplied by time is energy). That's
        like saying that a soda factory has been upgraded to fill bottles with 20 liters of Coca Cola. Without
        saying how long it takes to fill those bottles, it doesn't make much sense. 20 liters/second would
        make sense.
    \end{solution}

    \part What is the biggest source of energy?

    \begin{oneparchoices}
        \correctchoice Oil
        \choice Nuclear power
        \choice Coal
        \choice Natural gas
        \choice Hydroelectricity
    \end{oneparchoices}

    \begin{solution}
        The correct answer is oil (``Shares of primary energy use'' slide). Coal is the biggest source of
        electricity only, not of all energy (``Electricity'' slide).
    \end{solution}

    \part According to Swanson's Law, how will the price of electricity and the share of electricity
        prodced by fossil fuels likely change in the next several decades?

    \begin{choices}
        \choice The price of electricity will likely rise, and the share of electricity produced by fossil fuels will likely fall.
        \correctchoice The price of electricity will likely fall, and the share of electricity produced by fossil fuels will likely fall.
        \choice The price of electricity will likely rise, and the share of electricity produced by fossil fuels will likely rise.
        \choice The price of electricity will likely stay about the same, and the share of electricity produced by fossil fuels will likely stay about the same.
        \choice The price of electricity will likely fall, and the share of electricity produced by fossil fuels will likely rise.
    \end{choices}

    \begin{solution}
        Swanson's law states a consistent relationship between the cost of solar panels and the deployment
        of solar panels. The likely consequence of this trend in the future is a fall in the cost of electricity
        produced by solar panels, resulting in an expansion in the use of solar energy, causing a fall in the
        overall price of electricity, and decreased reliance on fossil fuels as a share of electricity produced.
    \end{solution}

    \part Which is true about non-renewable energy sources?

    \begin{choices}
        \correctchoice The U.S. government once predicted that U.S. oil would run out by 1924.
        \choice Since the UK passed the period of peak coal in the 1800s, many coal mines were shut because worldwide coal reserves are running out.
        \choice We could use up all of the proven reserves of oil in the world within three decades.
        \choice Civilizations throughout history eventually decayed because non-renewable energy sources were depleted.
        \choice If not for fossil fuels, for half of the world's population to maintain the world's average lifestyle, the other half would need to serve them as slaves.
    \end{choices}

    \begin{solution}
        Governments and other experts have predicted repeatedly over the last century that oil would run
        out soon. See slide ``Finite fossil fuels finishing?''
    \end{solution}

    \part What does Wright's Law state?

    \begin{choices}
        \choice The cost to produce each unit of a product tends to be directly proportional to the amount spent on research to improve the product.
        \choice The cost to manufacture a mass-produced item falls with time.
        \choice By increasing their wages, workers will devote more effort, thus reducing the cost of production by roughly the same as the amount spent on increased wages.
        \choice Through trade and exchange, an invention, such as airplanes, multiplies its benefit to the rest of the economy by a value that is roughly constant for each invention.
        \correctchoice The cost to produce a product falls with the cumulative amount of the product that is made.
    \end{choices}

    \begin{solution}
        See slide ``Solar: getting cheaper''.
    \end{solution}

    \part News articles often use units for electricity that are wrong or not meaningful. For example,
        an article said that 200 million megawatt hours (MWh) of electricity is wasted, but didn't
        state the time period--let's assume one year--or express the number in terms that are
        familiar to people, such as the number of typical power plants that generate that much
        energy in a year. Estimate how many traditional power plants generate that much energy in
        a year. Consider the capacity factor. You shouldn't need a calculator because the numbers
        are simple. To easily estimate in your head, round the number of days in a year to 400 and
        the number of hours in a day to 25 $= \frac{1}{4}$ of 100.

    \begin{oneparchoices}
        \correctchoice About 40-100
        \choice About 4000-10,000
        \choice About 200,000
        \choice About 40,000-100,000
        \choice About 20
    \end{oneparchoices}

    \begin{solution}
        How many hours in a year? $\thicksim$400 days/yr $\times$ $\thicksim\frac{1}{4}\times$ 100 h/day $= 100 \times 100$ h/yr = 10k h/yr
        $\frac{200\text{ million MWh/yr}}{10\text{k h/yr}}$ = 20 thousand MW = 20 GW
        A typical power plant is 1 GW, and capacity factor is around $\frac{1}{5}$ to $\frac{1}{2}$ (Capacity Factor
        slide), so a typical power plant produces a long-term average of around $\frac{1}{5}$ to $\frac{1}{2}$ GW. Thus,
        you'd need around 40-100 typical power plants.
    \end{solution}

    \part What is the Jevons paradox?

    \begin{choices}
        \choice Rising total energy use lowers energy efficiency.
        \choice Rising energy efficiency lowers total energy use.
        \correctchoice Rising energy efficiency raises total energy use.
        \choice New energy reserves more than replace those being used.
        \choice Rising total energy use raises energy efficiency.
    \end{choices}

    \begin{solution}
        Rising energy efficiency raises total energy use. Ridley quotes Jevons: ``new modes of
        economy will lead to an increase of consumption''.
    \end{solution}

    \part Were energy sources mostly renewable or non-renewable? What about now?

    \begin{choices}
        \choice 300 years ago and now, they are mostly renewable.
        \choice 300 years ago and now, they are mostly non-renewable.
        \choice 300 years ago they were mostly renewable, but now they are a roughly even mix of renewable and non-renewable.
        \correctchoice 300 years ago they were mostly renewable, but now they are mostly non-renewable.
        \choice 300 years ago they were mostly non-renewable, but now they are mostly renewable.
    \end{choices}

    \begin{solution}
        Slide 3 and 7 showed that we used renewable energy in the past and mostly fossil fuels now.
        The Rational Optimist discusses a huge rise in coal use in Britain from around 1750 through
        the 1800s, and not much coal use in China in the 1700s.
    \end{solution}

    \part Which of the following is the correct description of the history of power generation?

    \begin{choices}
        \choice Several centuries ago, we used mostly renewable power (PAWWW power). Now we use
        mostly non-renewable power, and in the future non-renewables (e.g., oil) will likely continue
        to dominate.
        \correctchoice Several centuries ago, we used mostly renewable power (PAWWW power). Now we use
        mostly non-renewable power, and in the future renewables (e.g., solar and wind) will likely
        dominate.
        \choice Humans have always been consuming mostly non-renewable power and will continue to
        consume mostly non-renewables in the future. Renewable power will likely have a relatively
        small stake in the market.
        \choice Several centuries ago, we used mostly non-renewable power. Now we continue to use
        mostly non-renewable power, and in the future other renewables (e.g., solar and wind) will
        likely dominate.
        \choice Several centuries ago, we used mostly non-renewable power. Now we use mostly
        renewable power (PAWWW power), and in the future other renewables (e.g., solar and
        wind) will likely dominate.
    \end{choices}

    \begin{solution}
        In the past, we used renewable power (PAWWW power). Now we use mostly non-renewable
        power, and in the future renewables (e.g., solar and wind) will likely dominate (see slide ``A
        short history of energy'').
    \end{solution}

    \part In his TEDx talk, what did Professor David MacKay consider as the three major categories of
        energy sources to replace fossil fuels in the UK?

    \begin{choices}
        \choice Onshore wind, offshore wind, and solar, because these take the least area.
        \correctchoice He did not advocate for or against particular sources, but he considered renewables in the
        country, renewables outside the country, and nuclear.
        \choice Solar, wind, and biomass. Nuclear uses the least area, but he didn't recommend it because
        it has popularity problems.
        \choice Concentrating solar, photovoltaic solar, and wind, because these take the least area.
        \choice Nuclear, solar, and wind. He doesn't recommend biomass because it takes too much area,
        as illustrated by his story about biofuels grown on the road verge.
    \end{choices}

    \begin{solution}
        In his talk ``A reality check on renewables'', MacKay listed `6 levers to pull', 3 of which
        discussed the alternative energy sources for replacing fossil fuels. He didn't specify or
        exclude particular sources, but he considered the possibility of increasing renewables inside
        the country, increasing renewables outside the country (for example in places like Australia,
        Canada, Russia, Libya, Kazakhstan), and nuclear power.
    \end{solution}

    \part Which of the following statements is TRUE?
    
    \begin{choices}
        \choice Compared to forests and streams, coal, as non-renewable energy, suffered more
        from diminishing returns and rising prices.
        \choice Daniel Defoe in 1728 stated that a rich demand from a few people is more important
        than a low demand from each of many people.
        \choice Nylon was the fiber on which the British textile boom was built because of the UK's
        warm, wet climate for growing nylon.
        \choice According to the Jevons paradox, energy consumption falls after the energy
        efficiency rises.
        \correctchoice Like Britain, China also had a vibrant textile industry by 1700, but the industry
        didn't become mechanized because it was far from the coalfields.
    \end{choices}

    \part Which is TRUE of current world energy use?
    
    \begin{choices}
        \choice Nuclear power constitutes about 15\% of world energy use.
        \choice Oil contributes a bigger share of world energy use than it did a decade ago.
        \correctchoice Natural gas makes up about a quarter of world energy use.
        \choice Hydroelectricity contributes about 15\% of world energy use.
        \choice Fossil fuels make up about 95\% of world energy use.
    \end{choices}

    \begin{solution}
        Slide ``Shares of primary energy use'' shows that natural gas is about a quarter.
    \end{solution}

    \part World oil reserves are $\thicksim$1.7 trillion barrels. The world uses-37 billion barrels of oil per year.
        Based on past experience with fossil fuels, when is it most likely that the world will run out of oil?

    \begin{choices}
        \choice  In about 2-4 decades, partly because use is slowly growing
        \choice  Within 2 decades, partly because developing countries are rapidly increasing their use of oil
        \choice  In about 6-10 decades, partly because renewable energy sources are being used more and more
        \correctchoice  Not within a century, partly because new reserves will be found
        \choice  In about 4-6 decades, partly because use is almost constant
    \end{choices}

    \begin{solution}
        Slides ``Finite fossil fuels finishing?'' and ``Why don't we run out?'' showed that reserves tend to keep
        increasing, despite use also increasing.
    \end{solution}

    \part According to Ridley, who are the modern slaves?

    \begin{choices}
        \choice Money in all its forms
        \correctchoice Machines and the power that runs them
        \choice Animals who raise our food and provide our entertainment and companionship
        \choice Medical treatments that improve and prolong our lives
        \choice Memes, or facts that serve the collective brain
    \end{choices}

    \begin{solution}
        The answer is machines and the power that runs them. Referring to the last paragraph
        of Chapter 7 of The Rational Optimist, Ridley claims, ``...we will need the watts from somewhere. They
        are our slaves. Thomas Edison deserves the last word: `I am ashamed at the number of things around
        my house and shops that are done by animals  human beings, I mean - and ought to be done by a
        motor without any sense of fatigue or pain. Hereafter a motor must do all the chores.''
    \end{solution}

    \part How does Ridley refute the claim that the Industrial Revolution drove down most living
        standards and contributed to poverty, inequality, child labour and pollution?

    \begin{choices}
        \choice Children working in factories were not only labourers, but they received a valuable,
        hands-on education in the leading edge technologies.
        \choice He admits that living standards did worsen during the Industrial Revolution, but
        argues that that was because workers then were lazy, and they deserved it.
        \choice The Industrial Revolution improved agricultural efficiency, reducing the need for farm
        land. This freed up land for housing and recreation, thus improving living conditions.
        \correctchoice Poverty, inequality, child labour and pollution had often been even worse
        before the Industrial Revolution.
        \choice With technological advance, smog decreased in industrial cities, so in fact, the
        environment was improving.
    \end{choices}

    \begin{solution}
        In Chapter 7 of The Rational Optimist, Ridley says,
        
        ``Is it really necessary to point out
        that poverty, inequality, child labour, disease and pollution existed before there were factories? In the
        case of poverty, the rural pauper of 1700 was markedly worse off than the urban pauper of 1850 and
        there were many more of him. In Gregory King's survey of the British population in 1688, 1.2 million
        labourers lived on just £4 a year and 1.3 million `cottagers' - peasants - on just £2 a year. That is to
        say, half the entire nation lived in abject poverty; without charity they would starve. During the
        industrial revolution, there was plenty of poverty but not nearly as much as this nor nearly as severe.
        Even farm labourers' income rose during the industrial revolution. 
        
        ``As for inequality, in terms of both
        physical stature and number of surviving children, the gap narrowed between the richest and the
        poorest during industrialisation. That could not have happened if economic inequality increased.
        
        ``As for child labour, a patent for a hand-driven linen-spinning machine from 1678, long before powered
        mills, happily boasts that `a child three or four years of age may do as much as a child of seven or
        eight years old.'
        
        ``As for disease, deaths from infectious disease fell steadily throughout the period. As
        for pollution, smog undoubtedly increased in industrial cities, but the sewage-filled streets of Samuel
        Pepys's London were more noisome than anything in Elizabeth Gaskell's Manchester of the 1850s.''
    \end{solution}

    \part Which statement is most accurate?

    \begin{choices}
        \choice Biofuels produce much more energy than the energy (other than solar) it takes to produce them.
        \choice Swanson's law refers to the observation that, as more solar panels are made, their cost per watt rises.
        \choice Energy is power divided by time.
        \correctchoice Early energy was completely renewable, using trees, hydropower, wind, domesticated animals, and human labor.
        \choice The long-term real price of oil has risen as reserves have decreased.
    \end{choices}

    \begin{solution}
        The correct answer is that early energy was completely renewable (see slide ``A short history of energy''). Power is energy divided by
        time. The long-term real price of oil has not consistently changed (slide titled ``The long-term price of oil did not change much''), and
        reserves have grown (slide titled ``Finite fossil fuels finishing?''). Swanson's law refers to the observation that, as solar panel
        production increases, the cost of each watt made by each new panel decreases (slides titled ``Solar''). The energy to grow biofuels
        roughly equals energy from them (``Biofuels'' slide).
    \end{solution}

    \part How do energy intensity (joules per dollar of GDP) and total energy use change as the economy of a region develops?

    \begin{choices}
        \choice First, energy intensity and total energy use fall. Then, energy intensity and total energy use fall.
        \correctchoice First, energy intensity and total energy use rise. Then, energy intensity falls, but total energy use doesn't fall.
        \choice First, energy intensity and total energy use rise. Then, energy intensity rises, but total energy use falls.
        \choice First, energy intensity falls, but total energy use rises. Then, energy intensity and total energy use fall.
        \choice First, energy intensity and total energy use fall. Then, energy intensity and total energy use rise.
    \end{choices}

    \begin{solution}
        As Matt Ridley explained in ``The Rational Optimist,'' industrialization first raises energy intensity and energy use. As energy
        efficiency rises, energy intensity falls, but people find new uses of energy, thus keeping total energy use from falling: Jevon's
        paradox.
    \end{solution}

    \part What was a major point of David MacKay's video about getting off of fossil fuels?

    \begin{choices}
        \choice Nuclear is not the most cost-effective energy source.
        \choice Renewables are not good choices for supplying energy.
        \choice Because fossil fuels supply 90\% of energy, it is impossible to replace them.
        \choice Trying to conserve energy cannot help much.
        \correctchoice Land area is important to consider when choosing energy sources.
    \end{choices}

    \begin{solution}
        The correct answer is ``Land area is important to consider when choosing energy sources.'' He discussed a graph of population per
        land area and power per person to compare different energy sources based on power per area. He said he likes renewables. He
        didn't discuss cost of nuclear energy. Even though fossil fuels supply 90\% of energy, he said, ``We need big action because we get
        90\% of our energy from fossil fuels, and so you need to push hard on most, if not all, of these levers.'', implying the alternatives to
        fossil fuels are feasible, if they are used heavily and in concert. He showed how he saved a large fraction of his energy use, implying
        that he thinks conserving energy can help significantly.
    \end{solution}

    \part Based on Harari's views, in today's world,

    \begin{choices}
        \choice Rich people are more likely to suffer a shorter lifespan than poor people.
        \choice The rich are deprived while the poor splurge on luxuries.
        \choice Poverty is no longer a problem anywhere in the world due to the abundance of affordable food and resources.
        \correctchoice Middle class people are more likely to spend beyond their means than are wealthy people.
        \choice We have successfully solved wealth inequality between the rich and the poor.
    \end{choices}

    \begin{solution}
        Sapiens Chapter 17: ``Today, the tables have turned. The rich take great care managing their assets and investments while the less
        well-heeled go into debt buying cars and televisions they don't really need.''
    \end{solution}

    \part In the last decade or so, electricity use has been

    \begin{choices}
        \correctchoice rising only in developing countries.
        \choice falling only in developing countries.
        \choice falling similarly in developed and developing countries.
        \choice rising only in developed countries.
        \choice rising similarly in developed and developing countries.
    \end{choices}

    \begin{solution}
        Slide ``Electricity'' said that electricity, from 2006 to 2016, grew 3\%/yr in world, 8\%/yr in China, 0\%/yr in US, 7\%/yr in India, 0\%/yr in
        Hong Kong. ~ no growth in developed countries. Since electricity use hasn't been changing in developed countries, but the world is
        using more electricity, all the growth must be from developing countries.
    \end{solution}

    \part What did Ridley say was a shocking irony?

    \begin{choices}
        \choice Originally, we used renewable energy.
        \choice Past civilizations fell due to depletion of non-renewable energy.
        \choice Fossil fuel cost keeps rising, but the economy keeps growing.
        \choice Non-renewable resources are infinite.
        \correctchoice Energy from non-renewable sources made economic growth sustainable.
    \end{choices}

    \begin{solution}
        In the Introduction to the chapter, Ridley said: ``This leads to a shocking irony. I am about to argue that economic growth only
        became sustainable when it began to rely on non-renewable, non-green, non-clean power.''
    \end{solution}

    \part What is the biggest source of electricity, other than fossil fuels?

    \begin{oneparchoices}
        \correctchoice Hydro
        \choice Geothermal
        \choice Wind
        \choice Nuclear
        \choice Solar
    \end{oneparchoices}

    \begin{solution}
        Slide ``Electricity'' shows it as hydroelectricity.
    \end{solution}

    \part Fossil fuel

    \begin{choices}
        \correctchoice saves land that would otherwise be used to produce renewable energy.
        \choice will likely run out by 2050.
        \choice is increasingly used now to replace renewable power sources, whose cost has been rising over the long term.
        \choice production stopped rising early in the Industrial Revolution in England.
        \choice helped China lead the world in iron production in the 1700s.
    \end{choices}

    \begin{solution}
        Fossil fuel is underground and uses little land to extract, while solar and wind power are collected over a wide area of the surface.
        We will not run out of fossil fuel because we will switch to other sources of energy before using up all fossil fuels.
    \end{solution}

    \part Based on Sapiens, why did Britain initially turn to coal as their energy source, prior to the Industrial Revolution?

    \begin{choices}
        \choice British inventors were inspired by the invention of gunpowder in China, which successfully converted heat energy into movement.
        \choice The government owned most coal mines, giving it a financial incentive to promote coal over other energy sources.
        \correctchoice The population increase had used up much of Britain's supply of wood, forcing the use of an alternative energy source.
        \choice Government tax incentives
        \choice The scarcity of coal in the world would help 17th century Britain achieve greater power and dominance over rivaling countries.
    \end{choices}

    \begin{solution}
        Sapiens Chapter 17: ``As the British population swelled, forests were cut down to fuel the growing economy and make way for houses and fields.
        Britain suffered from an increasing shortage of firewood. It began burning coal as a substitute.''
    \end{solution}

    \part Based on The Rational Optimist, which of the following is the major reason biofuels are promoted nowadays in the U.S.?

    \begin{choices}
        \choice Biofuels are cheaper than fossil fuels.
        \choice Biofuels are needed to fill the gap in supply due to a shortage of fossil fuels.
        \correctchoice Large corporations persuaded the politicians and provided political funding.
        \choice Growing biofuels is environmentally friendly as it consumes less water and fertilizer than does extracting fossil fuels.
        \choice Burning biofuels will generate a pleasant aroma that is much preferred to coal smoke.
    \end{choices}

    \part \textit{Unknown Question (This question could not be displayed due to a system error of Moodle.)}

    \begin{choices}
        \choice The cost of transporting coal overland was minimal.
        \choice Britain had no other renewable energy sources available.
        \correctchoice Coal was effectively limitless in supply and easily accessible.
        \choice It was significantly cheaper than water power.
        \choice Coal was the primary source of fuel for iron production from the start.
    \end{choices}

    \begin{solution}
        Coal became a crucial energy source for Britain's industrial revolution not because it was initially cheaper
        than alternatives like water power, but because it was effectively limitless in supply and easily accessible.
        This abundance and accessibility allowed for sustained industrial growth without the diminishing returns that
        water power and timber experienced. Additionally, Britain's coal fields were close to the surface and navigable
        waterways, facilitating cheap transportation, which made it even more crucial as an energy source.
    \end{solution}

\end{parts}