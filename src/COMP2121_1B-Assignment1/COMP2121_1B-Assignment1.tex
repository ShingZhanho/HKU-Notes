\documentclass[answers]{exam}
\usepackage[x11names]{xcolor}
\usepackage[left=1.5cm,right=1.5cm,top=1.5cm,bottom=2cm]{geometry}
\usepackage{newtxtext,newtxmath}
\usepackage{amsmath}
\usepackage{bm}
\usepackage{float}
\usepackage{multicol,multirow}
\usepackage{tabularx}
\usepackage[inline,shortlabels]{enumitem}
\usepackage{cancel}
\usepackage{venndiagram}
% for long multiplications
\makeatletter
    \providecommand\text\mbox
    \newenvironment{arithmetic}[1][]{\begin{tabular}[#1]{Al}}{\end{tabular}}
    \newcolumntype{A}{>{\bgroup\def~{\phantom{0}}$\@testOptor}r<{\@gobble\\$\egroup}}
    \def\@testOptor\ignorespaces#1#2\\{%
    \ifx#1\times
        \@OperatorRow{#1}{#2}\@tempa%
    \else\ifx#1+
        \@OperatorRow+{#2}\@tempa%
    \else\ifx#1\discretionary% detects the soft hyphen, \-
        \@ShortSubtractRow{#2}\@tempa%
    \else\ifx#1-
        \@OperatorRow-{#2}\@tempa%
    \else
        \@NormalRow{#1#2}\@tempa%
    \fi\fi\fi\fi
    \@tempa}
    \def\@OperatorRow#1#2#3{%
    \@IfEndRow#2\@gobble\\{%
        \def#3{\underline{{}#1 #2}\\}%
    }{%
        \def#3{\underline{{}#1 #2{}}}%
    }}

\def\@NormalRow#1#2{%
    \@IfEndRow#1\@gobble\\{%
        \def#2{#1\\}%
    }{%
        \def#2{#1{}}%
    }}

\def\@IfEndRow#1\@gobble#2\\#3#4{%
    \ifx#2\@gobble
        #4%
    \else
        #3%
    \fi}

\makeatother

\pagestyle{foot}
\cfoot{Page \thepage\ of \numpages}
\bracketedpoints
\renewcommand{\solutiontitle}{\noindent\textbf{Answers:}\par\noindent}

% Logic symbols
\newcommand{\lxor}{\oplus}
\newcommand{\limp}{\rightarrow}
\newcommand{\liff}{\leftrightarrow}
\newcommand{\ltrue}{\bm{T}}
\newcommand{\lfalse}{\bm{F}}

% Set symbols
\newcommand{\set}[1]{\left\{#1\right\}} % set braces
\newcommand{\card}[1]{\left|#1\right|} % cardinality of a set
\newcommand{\pwset}[1]{\mathcal{P}(#1)} % power set
\newcommand{\mtset}{\varnothing} % empty set
\renewcommand{\emptyset}{\varnothing} % empty set

\renewcommand{\thesubpart}{(\arabic{subpart})}
\renewcommand{\subpartlabel}{\thesubpart}

% commands for question 4b
\newcommand{\tnot}[1]{\texttt{NOT}\left(#1\right)}
\newcommand{\topr}[3]{\texttt{#1}\left(#2, #3\right)}

% psol (proof-solution) environment
\newenvironment{psol}{
    \renewcommand{\solutiontitle}{\noindent\textbf{Proof:}\par\noindent}
    \begin{solution}
}{
    \par\hfill\textbf{Q.E.D.}
    \end{solution}
    \renewcommand{\solutiontitle}{\noindent\textbf{Answers:}\par\noindent}
}

\newcommand{\ds}{\displaystyle}

\allowdisplaybreaks

\newcommand{\A}{\mathbb{A}}
\newcommand{\B}{\mathbb{B}}
\newcommand{\C}{\mathbb{C}}
\newcommand{\D}{\mathbb{D}}
\newcommand{\E}{\mathbb{E}}
\newcommand{\F}{\mathbb{F}}
\newcommand{\K}{\mathbb{K}}
\newcommand{\N}{\mathbb{N}}
\newcommand{\Q}{\mathbb{Q}}
\newcommand{\R}{\mathbb{R}}
\newcommand{\T}{\mathbb{T}}
\newcommand{\X}{\mathbb{X}}
\newcommand{\Y}{\mathbb{Y}}
\newcommand{\Z}{\mathbb{Z}}

\begin{document}

\begin{center}
    \textbf
    {\Large{COMP2121 Discrete Mathematics} \\
    \large{25/26 Semester 1} \\
    \large{Assignment 1 (Proposed Solutions)}}\\
    SHING, Zhan Ho Jacob \qquad 3036228892
\end{center}

\textit{
    To distinguish, tautologies are represented as a bold T ($\ltrue$),
    and contradictions a bold F ($\lfalse$).
}

\begin{questions}
    
    \question Let $C$ be the statement ``Lady Furina has clues'', $M$ be the statement ``Lady Furina has a motive'', $S$ be
    the statement ``Lady Furina solves the case'' and $T$ be the statement ``The truth is revealed''.

    \begin{parts}

        \part Translate $(C \land \neg M) \limp \neg S$ into English.
        \begin{solution}
            The statement translates to ``If Lady Furina has clues and does not have motive,
            then she does not solve the case.''
        \end{solution}

        \part Rewrite the following sentence using logical operators: ``The truth is revealed only if Lady Furina
        solves the case and has clues.''
        \begin{solution}
            The required logical expression is $T \limp (S\land C)$.
        \end{solution}

        \part Determine whether the argument
        \begin{quote}
            ``If Lady Furina solves the case, then she has clues. The truth is not revealed unless she
            has clues. Therefore, if she solves the case, the truth is revealed.''
        \end{quote}
        is valid. Justify your answer. 
        \begin{solution}
            If we rewrite the argument in logical expressions, we have:
            \begin{enumerate}
                \item $S \limp C$ \label{q1c:stm1}
                \item $\neg C \limp \neg T$ \label{q1c:stm2}
                \item Therefore, from propositions \ref{q1c:stm1} and \ref{q1c:stm2}, $S \limp T$ \label{q1c:stm3}
            \end{enumerate}
            For the argument to be valid, we require the conclusion (Proposition \ref{q1c:stm3})
            is never false while all the premises (Propositions \ref{q1c:stm1} and \ref{q1c:stm2}) are true.
            It can be verified by trying to find a counterexample, i.e., combinations of $S$, $C$, and $T$
            such that the conclusion is $\lfalse$ while all its premises are $\ltrue$.

            For the conclusion to be $\lfalse$, the only possible case is when $S$ is $\ltrue$ and
            $T$ is $\lfalse$.

            For Proposition \ref{q1c:stm1} to be $\ltrue$ when $S$ is $\ltrue$, $C$ can only be
            $\ltrue$.
            
            When $C$ is $\ltrue$ and $T$ is $\lfalse$, Proposition \ref{q1c:stm2} is $\ltrue$.

            Therefore, when $S$ is $\ltrue$, $C$ is $\ltrue$, and $T$ is $\lfalse$, all premises
            are $\ltrue$ but the conclusion is $\lfalse$, i.e., the argument is \textbf{invalid}.
        \end{solution}

    \end{parts}

    \question In a town live three types of people: knights, who always tell the truth, knaves, who always lie, and spies,
    who sometimes tell the truth and sometimes lie. You meet three individuals on the road and you know
    one of them is a knight, one is a knave and one is a spy. However, the three persons are foreigners. They
    can understand English, but they can only say in their own language, namely ``Ja'' and ``Da''. You know
    they mean either ``yes'' or ``no'', but you don’t know which means ``yes'' and which means ``no''.

    You ask A ``If I asked you if B is the spy, would you say Ja?'' A answers ``Ja''.

    Then you ask C “If I asked you if you were the knave, would you say Ja?” C answers ``Ja''.

    Finally, you ask C again ``If I asked you if A is the spy, would you say Ja?'' C answers ``Da''.

    According to the above information, can you determine who is the knight, who is the knave and who is
    the spy?
    \begin{solution}
        \textit{The originally proposed solution was incorrect and is hence deleted.
        For copyright reasons, official solutions cannot be provided here.}
    \end{solution}

    \question You are an experienced engineer who wants to design a new computer. A computer is composed of many
    logical gates, each implements a logical operator. Therefore, your task is to implement every logical
    operator to realize universal computation. You ask your assistant to buy some chips that integrates
    the basic logical gates inside, so that you can combine them to realize every possible logical operator.
    Unfortunately, your assistant only bought you chips for \texttt{NAND} gate, which implements $\neg(A\land B)$ for
    two arbitrary inputs $A$ and $B$. This troubles you a lot, because you’ll have to implement other logical
    operators by yourself.

    \begin{parts}
        
        \part Since you want to realize universal computation, first you have to decide how many different logical
        operators you have to implement. For example, a 1-to-1 logical operator takes a 1-bit binary number
        $A \in\set{0, 1}$ as input and 1-bit binary number $B \in\set{0, 1}$ as output. There are four 1-to-1 logical
        operators in total. Two examples are \texttt{IDENTITY} $(B= A)$ and \texttt{NOT} $(B= \neg A)$.
        
        A 2-to-1 logical operator takes two 1-bit binary numbers $A, B \in\set{0, 1}$as input and outputs $C \in
        \set{0, 1}$. Determine the total number of possible 2-to-1 logical operators. 
        \begin{solution}
            Note that for two inputs $A$ and $B$, there are four possible combinations,
            i.e., $(A, B) \in \{(0,0), (0, 1), (1, 0), (1, 1)\}$. A logical operator
            is a function $f$ that maps the combinations of inputs to an output, i.e.,
            $f: \{0, 1\}^2 \to \{0, 1\}$. Consider the truth table of $f$:
            \begin{center}
                \begin{tabular}{ccc}
                    $\bm{A}$ & $\bm{B}$ & $\bm{C = f(A, B)}$ \\
                    \hline
                    0 & 0 & $f(0, 0)$ \\
                    0 & 1 & $f(0, 1)$ \\
                    1 & 0 & $f(1, 0)$ \\
                    1 & 1 & $f(1, 1)$ 
                \end{tabular}
            \end{center}
            Two logical operators $f$ and $g$ are said to be the same if and only if
            for all combinations of inputs, $f(A, B) = g(A, B)$. Therefore, the problem
            reduces to counting the number of different ways to fill the truth table,
            which is given by $2^4 = 16$.

            Therefore, there are \fbox{\textbf{16}} possible 2-to-1 logical operators.
        \end{solution}

        \part Even though there are many different logical operators, you found that you can implement all of
        them using only \texttt{NAND} operators. Please implement \texttt{NOT}, \texttt{OR}, \texttt{AND}, \texttt{XOR}, \texttt{IMPLIES} operators
        using \texttt{NAND} only. For simplicity, you may write, e.g., $\neg(A\land B)$ as $\texttt{NAND}(A, B)$.
        
        [Hint: here’s an example of using \texttt{NOR} to implement \texttt{NOT}: $\texttt{NOT}(A)\equiv \texttt{NOR}(A, A)$, since $(A\lor A) \equiv A$.]
        \begin{solution}
            \begin{itemize}
                \item \textbf{\texttt{NOT} operator}
                
                Note that $\texttt{NAND}(A, A) \equiv \neg (A \land A) \equiv \neg A \lor \neg A \equiv \neg A$.
                
                Therefore, the \texttt{NOT} operator is implemented as $\boxed{\texttt{NOT}(A) = \texttt{NAND}(A, A)}$.

                \item \textbf{\texttt{OR} operator}
                
                Note that $\neg(\neg A \land \neg B) \equiv A \lor B$,
                which means $\texttt{OR}(A, B)=\texttt{NAND}(\texttt{NOT}(A), \texttt{NOT}(B))$.
                
                Further expand the $\texttt{NOT}$ operators as implemented before, we have:
                $\boxed{\texttt{OR}(A, B)=\texttt{NAND}\left[\texttt{NAND}(A, A), \texttt{NAND}(B,B)\right]}$.

                \item \textbf{\texttt{AND} operator}
                
                Note that if we apply \texttt{NAND} on the result of $\texttt{NAND}(A, B)$, we have:
                $$\neg\left[\neg\left(A\land B\right)\land\neg\left(A\land B\right)\right]
                \equiv \left(A\land B\right)\lor\left(A \land B\right)
                \equiv A \land \left(B \lor B\right)
                \equiv A \land B$$

                Therefore, the \texttt{AND} operator is implemented as
                \fbox{$\topr{AND}{A}{B}=\topr{NAND}{\topr{NAND}{A}{B}}{\topr{NAND}{A}{B}}$}.

                \item \textbf{\texttt{XOR} operator}
                
                Recall that $A\lxor B
                \equiv (A\lor B)\land\neg(A\land B)$. From this expression, we continue to derive:
                \begin{align*}
                    (A\lor B)\land\neg(A\land B)
                    &\equiv (A\land\neg(A\land B))\lor(B\land\neg(A\land B)) \qquad\text{(Distributive Law)}\\
                    &\equiv \neg[\neg(A\land\neg(A\land B))\land\neg(B\land\neg(A\land B))] \qquad\text{(De Morgan's Law)}\\
                    &\equiv \topr{NAND}{
                        \topr{NAND}{A}{\topr{NAND}{A}{B}}
                    }{
                        \topr{NAND}{B}{\topr{NAND}{A}{B}}
                    }
                \end{align*}

                Therefore, \texttt{XOR} is implemented as \fbox{$\topr{XOR}{A}{B}
                =\topr{NAND}{
                        \topr{NAND}{A}{\topr{NAND}{A}{B}}
                    }{
                        \topr{NAND}{B}{\topr{NAND}{A}{B}}
                    }$}.

                \item \textbf{\texttt{IMPLIES} operator}
                
                By logical equivalence, we have $A \limp B \equiv \neg A \lor B$.

                Expand the expression, we have:
                \begin{align*}
                    \neg A\lor B
                    &\equiv \topr{OR}{\tnot{A}}{B} \\
                    &\equiv \topr{NAND}{A}{\tnot{B}} \\
                    &\equiv \topr{NAND}{A}{\topr{NAND}{B}{B}}
                \end{align*}

                Therefore, \texttt{IMPLIES} is as \fbox{$
                \topr{IMPLIES}{A}{B} = 
                \topr{NAND}{A}{\topr{NAND}{B}{B}}
                $}.
            \end{itemize}
        \end{solution}
    \end{parts}

    \question Quantifiers and Predicates

    \begin{parts}
        
        \part Rewrite expression
        $\neg(\exists y\neg(\forall x(P(x)\land Q(y)))\limp\exists z R(z))$ so that all the negation signs immediately precedes predicates.
        \begin{solution}
            \begin{align*}
                \neg[\exists y\neg\{\forall x[P(x)\land Q(y)]\}\limp\exists zR(z)]
                &\equiv \neg[\exists y{\exists x[\neg P(x)\lor\neg Q(y)]}\limp\exists zR(z)]\\
                &\equiv \neg[\neg\{\exists x\exists y[\neg P(x)\lor\neg Q(y)]\}\lor\exists zR(z)]\\
                &\equiv \neg[\forall x\forall y[P(x)\land Q(y)]\lor\exists zR(z)]\\
                &\equiv \neg[(\forall xP(x)\land\forall yQ(y))\lor\exists zR(z)]\\
                &\equiv \neg(\forall xP(x)\land\forall yQ(y))\land\forall z\neg R(z)\\
                &\equiv \boxed{[\exists x\neg P(x)\lor\exists y\neg Q(y)]\land\forall z\neg R(z)}
            \end{align*}
        \end{solution}

        \part Consider the two predicates $P(x)$ and $Q(x)$ with the same universe of discourse.

        Prove that $\forall xP(x)\land\exists xQ(x) \equiv \forall x \exists y (P(x)\land Q(y))$.
        \begin{psol}
            Rename the variable on the left-hand side, we have:
            $$\forall xP(x)\land\exists yQ(y) \equiv \forall x\exists y(P(x)\land Q(y))$$
            To prove equivalence, we prove both the sufficiency and necessity.
            \begin{enumerate}
                \item \textbf{Sufficiency:} $\forall xP(x)\land\exists yQ(y) \Rightarrow \forall x\exists y(P(x)\land Q(y))$
                
                Assume that the L.H.S. is true, then for all $x$, $P(x)$ must hold.
                Also, there must exist at least one $y$, say $y_0$, such that $Q(y_0)$ holds.
                Therefore, for all $x$, we can always find a $y$, which is $y_0$, such that
                $P(x)\land Q(y)$ holds. We have displayed that the R.H.S. must be true if
                the L.H.S. is true.

                \item \textbf{Necessity:} $\forall xP(x)\land\exists yQ(y) \Leftarrow \forall x\exists y(P(x)\land Q(y))$
                
                Assume that the R.H.S. (i.e., $\forall x\exists y(P(x)\land Q(y))$) is true,
                then for all $x$, we can always find a $y$, say $y_0$, such that $P(x)\land Q(y_0)$
                holds. Therefore, for all $x$, $P(x)$ must hold, and there must exist at least one $y$,
                which is $y_0$, such that $Q(y_0)$ holds. We have displayed that the L.H.S. must be true if
                the R.H.S. is true.
            \end{enumerate}
            Since both the sufficiency and necessity have been proved, we conclude that
            $\forall xP(x)\land\exists xQ(x) \liff \forall x \exists y (P(x)\land Q(y))$
            is a tautology, i.e., the two statements are logically equivalent.
        \end{psol}

        \part Justify whether $\exists x \forall y P(x, y) \limp \forall y \exists x P(x, y)$
        is always a tautology for any predicate $P(x, y)$.
        \begin{solution}
            We propose that the given proposition is not always a tautology, and we attempt
            to find an example, such that the proposition is a contradiction.

            $\exists x \forall y P(x,y) \limp \forall y \exists x P(x,y)$ is a contradiction
            iff $\exists x \forall y P(x,y)$ is $\ltrue$ and $\forall y \exists x P(x,y)$ is
            $\lfalse$.

            If the antecedent is true, then, for at least one $x$, say $x_0$, $P(x_0, y)$
            holds for all $y$.

            If the consequent is false, then, for at least one $y$, say $y_0$, there does not
            exist any $x$ such that $P(x, y_0)$ holds.

            However, if $P(x_0, y)$ holds for all $y$, then $P(x_0, y_0)$ must also hold.
            Therefore, such $y_0$ does not exist, and the consequent can never be false if
            the antecedent is true.

            Therefore, we cannot find a counterexample such that the antecedent is true
            while the consequent is false. Hence, the given proposition is always a tautology.
        \end{solution}

    \end{parts}

    \question Proofs

    \begin{parts}
        
        \part Consider the following proof that uses logical equivalence/implication:
        \begin{align}
            \forall x(\neg P(x)\limp Q(x))\land\forall y\neg Q(y)
            &\equiv \forall x(P(x)\lor Q(x))\land\forall y\neg Q(y) \\
            &\equiv \forall x[(P(x)\land Q(x))\lor\neg Q(x)] \\
            &\equiv \forall x[P(x)\land\neg Q(x)] \\
            &\Rightarrow \forall xP(x) \\
            &\Rightarrow \exists xP(x).
        \end{align}
        Determine whether the proof is corrct.
        \begin{solution}
            Step 1: Valid. This is implication law ($P \limp Q \equiv \neg P \lor Q$).

            Step 2: Valid. Renaming the variable $x$ to $y$ does not change the meaning of
            the statement. This is also the distribution law of $\forall$ over $\land$ --
            $\forall x P(x) \land \forall y Q(y) \equiv \forall x (P(x) \land Q(x))$.

            Step 3: Valid. This step combines application of several laws:
            \begin{align*}
                &\equiv \forall x[(P(x)\land\neg Q(x)) \lor (Q(x)\land\neg Q(x))] \qquad\text{(Distribution law)}\\
                &\equiv \forall x[(P(x)\land\neg Q(x)) \lor \lfalse] \qquad\text{(Contradiction law)}\\
                &\equiv \forall x[P(x)\land\neg Q(x)] \qquad\text{(Identity law)}
            \end{align*}

            Step 4: Valid. This is the simplification (i.e., $P\land Q \limp P$).

            Step 5: Valid. First, since $\forall x P(x)$ is true, then $P(c)$ must be true
            for any $c$ in the universe of dicourse. Then, since there exists at least one
            $x$, $c$ in this case, such that $P(x)$ is true, then $\exists x P(x)$ is true.

            Therefore, the proof is sound and valid.
        \end{solution}

        \part Proof by induction that
        \[
            \sum_{k=1}^n \cos(kx) = \frac{1}{2}\left(\frac{\sin\left[\left(n+\frac{1}{2}\right)x\right]}{\sin\left(\frac{1}{2}x\right)}-1\right)
        \]
        holds for all integers $n\geq 1$.
        \begin{psol}
            Denote $P(n) : \ds{\sum_{k=1}^n \cos(kx) = \frac{1}{2}\left(\frac{\sin\left[\left(n+\frac{1}{2}\right)x\right]}{\sin\left(\frac{1}{2}x\right)}-1\right)}$.

            \textbf{Base Case:} $n=1$.
            \begin{equation*}
                \text{L.H.S.} = \sum_{k=1}^1 \cos(kx) = \cos(x)
            \end{equation*}
            \begin{align*}
                \text{R.H.S.} &= \frac{1}{2}\left(\frac{\sin\frac{3}{2}x}{\sin\frac{1}{2}x}-1\right)\\
                &= \frac{1}{2} \left(\frac{\sin\frac{3}{2}x-\sin\frac{1}{2}x}{\sin\frac{1}{2}x}\right)\\
                &= \frac{1}{2} \left(\frac{2\cos x\sin\frac{1}{2}x}{\sin\frac{1}{2}x}\right)\\
                &= \cos x \\
                &= \text{L.H.S.}
            \end{align*}
            Therefore, $P(1)$ holds.

            \textbf{Inductive Step:} Assume that $P(n)$ holds for $n\geq 1$. We show that $P(n+1)$ also holds.
            \begin{align*}
                \text{L.H.S.} &= \sum_{k=1}^{n+1} \cos(kx) \\
                &= \sum_{k=1}^n \cos(kx) + \cos\left((n+1)x\right) \\
                &= P(n) + \cos\left((n+1)x\right) \\
                &= \frac{1}{2}\left(\frac{\sin\left[\left(n+\frac{1}{2}\right)x\right]}{\sin\frac{1}{2}x}-1\right) + \cos\left((n+1)x\right) \\
                &= \frac{1}{2}\left(\frac{\sin\left[\left(n+\frac{1}{2}\right)x\right]-\sin\frac{1}{2}x}{\sin\frac{1}{2}x}\right) + \cos\left((n+1)x\right) \\
                &= \frac{1}{2}\left(\frac{2\cos\left[\frac{n+1}{2}x\right]\sin\left[\frac{n}{2}x\right]}{\sin\frac{1}{2}x}\right) + \cos\left((n+1)x\right) \\
                &= \frac{\cos\left[\frac{n+1}{2}x\right]\sin\left[\frac{n}{2}x\right] + \cos\left((n+1)x\right)\sin\frac{1}{2}x}{\sin\frac{1}{2}x} \\
                &= \frac{\sin\left[\frac{2n+1}{2}x\right]-\sin\frac{1}{2}x+\sin\left[\frac{2n+3}{2}x\right]-\sin\left[\frac{2n+1}{2}x\right]}{2\sin\frac{1}{2}x} \\
                &= \frac{1}{2}\left(\frac{\sin\left[\frac{2n+3}{2}x\right]}{\sin\frac{1}{2}x}-1\right) \\
                &= \frac{1}{2}\left(\frac{\sin\left[\left(\left(n+1\right)+\frac{1}{2}\right)x\right]}{\sin\frac{1}{2}x}-1\right) \\
                &= \text{R.H.S.}
            \end{align*}
            Therefore, $P(n) \Rightarrow P(n+1)$.

            By the principle of mathematical induction, $P(n)$ holds for all $n\geq 1$.
        \end{psol}

        \newpage

        \part One day you are playing a card game with one friend. Suppose there are 191 cards in total. You
        and your friend take turns to draw 1 to 5 cards once. The one who draws the last card will lose the
        game. Suppose you always draw the cards first, and your friend is wise enough. Design a winning
        strategy and proof that you can always win.
        \begin{solution}
            We may consider different cases at the end of the game. When 2 cards remain, and it
            is my turn to take cards, I can take 1, and the opponent is left with 1 card,
            therefore losing the game. Similarly, when 3 cards remain, I can take 2 and leave
            1. This winning position goes on until there are 7 remaining cards, where, no
            matter how many cards I take, he will be in the winning position as the case
            reduces to 1 to 6 remaining cards. Therefore, when the party who is left with 7
            cards to draw must lose the game. When there are 8 cards remaining, I can take 1
            card to reduce it to 7 and force my opponent on the losing position. Similarly,
            when there are 9 remaining, I take 2, ..., until there are 12 remaining for me and
            I take 5 to make it 7. Again, if I am left with 13 cards, no matter how many I take,
            it will reduce to 8 to 12 remaining cards, and my opponent can always make sure that
            I am left with 7 cards, thus losing.

            From the above discussion, we can observe that if we can ensure the opponent is
            left with $6n+1$ cards to take where $n\in\Z^+$, we can always win. Therefore, we
            can come up with this strategy:

            \textbf{Strategy:} I first take 4 cards, so that the opponent is left with 
            $191-4=187=6\times31+1$ cards. Suppose the opponent then take $p\in[1,5]$ cards
            in each round, I take $(6 - p)$ cards, and repeat until the game ends.

            Now we formally prove this strategy.

            \begin{psol}
                Denote $P(n):$ ``When there are $(6n+1)$ cards remaining, the player who make
                a step from this state by taking $p$ cards must lose if the opponent
                takes $(6-p)$ cards.'' We prove $\forall n\geq1:P(n)$.

                \textbf{Base case:} $P(1)$ (i.e. 7 remaining cards).
                In the table below, the columns $P$ record number of cards taken by the player
                making a move from the state of $(6n+1)$ cards, columns $O$ record the opponent.
                The numbers in the brackets denote the cards remaining.
                \begin{center}
                    \begin{tabular}{|c||c|c|c||c|}
                        \hline
                                                        & \textbf{P}    & \textbf{O}    & \textbf{P}    & \textbf{Result of P} \\
                        \hline
                        $\bm{P(1):}$ \textbf{Case 1}    & 1 (6)         & 5 (1)         & 1 (0)         & \textbf{Lose} \\
                        $\bm{P(1):}$ \textbf{Case 2}    & 2 (5)         & 4 (1)         & 1 (0)         & \textbf{Lose} \\
                        $\bm{P(1):}$ \textbf{Case 3}    & 3 (4)         & 3 (1)         & 1 (0)         & \textbf{Lose} \\
                        $\bm{P(1):}$ \textbf{Case 4}    & 4 (3)         & 2 (1)         & 1 (0)         & \textbf{Lose} \\
                        $\bm{P(1):}$ \textbf{Case 5}    & 5 (2)         & 1 (1)         & 1 (0)         & \textbf{Lose} \\
                        \hline
                    \end{tabular}
                \end{center}
                Therefore, regardless how many cards the player takes, he will always lose if the
                opponent is playing optimally.

                Therefore, $P(1)$ holds.

                \textbf{Inductive Step}: Assume that for some $k\geq1$, $P(k)$ holds. Consider
                $P(k+1)$, that is, we have $[6(k+1)+1]$ cards remaining. When compared with
                $P(k)$, we have $[6(k+1)+1]-(6k+1)=6k+7-6k-1=6$ more cards.

                Observe that in each round, the player takes $p$ cards, and the opponent takes
                $(6-p)$ cards, removing a total of $p+(6-p)=6$ cards. Then, after one round from
                $P(k+1)$, the number of remaining cards reduces by 6, reducing the case back to
                $P(k)$, which is assumed to hold.

                Therefore, we have $P(k)\Rightarrow P(k+1)$.

                By the principle of mathematical induction, $P(n)$ holds for all $n\geq1$.

                Now, integrate this $P(n)$ with our strategy. We first take away 4 cards, leaving
                187 remaining cards, which corresponds to the $P(31)$ case. Since we have shown
                $\forall n\geq 1 P(n)$, then $P(31)$ must hold. And notice that after we take the
                initial 4 cards, it is the opponent who starts to make a move from this state,
                therefore, the opponent must lose, i.e., I must win.
            \end{psol}
        \end{solution}

    \end{parts}
    
    \newpage 

    \question Basics on Sets

    \begin{parts}
        
        \part Determine the cardinality of the following sets:
        \begin{enumerate}[label=(\roman*)]
            \item $A = \set{\set{0, 1}, 2, 3, \mtset}$
            \item The power set $\pwset{B}$ of $B = \set{1, 2, \mtset, |x|=2}$
            \item The set $C=\set{x\in\N\mid x^2\leq 100}$ (Note that the set of natural numbers $\N$ includes zero).
            \item The set $D = \set{x\text{ is a letter}\mid x\text{ does not appear in the word ``artificial''}}$.
        \end{enumerate}
        \begin{solution}
            \begin{enumerate}[label=(\roman*)]
                \item $\card{A} = \boxed{4}$
                \item $\card{\pwset{B}} = 2^{\card{B}} = 2^4 = \boxed{16}$,
                provided that the set $B$ has two integers, one empty set, and one proposition.
                \item Note that $C = [0, 10]$. Therefore, $\card{C} = \boxed{11}$.
                \item Assuming that capital and small letters are the same, then
                $\card{D} = 26-7 =\boxed{19}$.
            \end{enumerate}
        \end{solution}

        \part For $A,B,C$ in part \ref{part@6@1}, determine the cardinalities of the following sets:
        \begin{enumerate}[label=(\roman*)]
            \item $A \cap B$
            \item $B \cup (C\cap\mtset) - A$
            \item $(A\cap B)\times(B\cap C)$
        \end{enumerate}
        \begin{solution}
            \begin{enumerate}[label=(\roman*)]
                \item $A \cap B = \set{2, \mtset} \Rightarrow \card{A\cap B} = \boxed{2}$.
                \item Consider:\begin{align*}
                    B \cup (C\cap\mtset) - A 
                    &= B\cup\mtset - A \\
                    &= B-A \\
                    &= \set{1, 2, \mtset, |x|=2} - \set{\set{0, 1}, 2, 3, \mtset} \\
                    &= \set{1, |x|=2} \\
                    &\Rightarrow \card{B \cup (C\cap\mtset) - A} = \boxed{2}
                \end{align*}
                \item Consider:\begin{align*}
                    (A\cap B)\times(B\cap C)
                    &= \set{2, \mtset} \times \set{1, 2} \\
                    &= \set{(2, 1), (2, 2), (\mtset, 1), (\mtset, 2)} \\
                    &\Rightarrow \card{(A\cap B)\times(B\cap C)} = \boxed{4}
                \end{align*}
            \end{enumerate}
        \end{solution}

    \end{parts}

    \newpage

    \question Venn Diagrams

    \begin{parts}
        
        \part Identify the set represented in the following figure:
        \begin{center}
            \begin{venndiagram3sets}
                \fillACapC
                \fillBCapC
            \end{venndiagram3sets}
        \end{center}
        \begin{solution}
            The set represented in the figure is $(A\cup B)\cap C$.
        \end{solution}

        \part Draw Venn diagram for $(A\cup B) - (\overline{A\cap C})$.
        \begin{solution}
            \begin{center}
                \begin{venndiagram3sets}[labelNotABC=$U$,shade=cyan!40]
                    \fillACapC
                \end{venndiagram3sets}
            \end{center}
        \end{solution}

    \end{parts}

    \newpage

    \question Set Theory and Logic

    \begin{parts}
        
        \part Determine whether $\forall\text{ sets }A,B,C,D:(A\cap B)\times(C\cap D)
        =(A\times C)\cap(B\times D)$.
        \begin{solution}
            To show that the two sets are equal, we show that whenever an element
            is in the L.H.S. set, it must also be in the R.H.S. set, and vice versa.
            
            Recall that $x\in (A\cap B) \equiv (x\in A) \land (x\in B)$
            and $(x, y) \in (A\times B) \equiv (x\in A) \land (y\in B)$.
            Therefore, for L.H.S., we have:
            \begin{align*}
                (x,y)\in(A\cap B)\times(C\cap D)
                &\equiv (x\in A\cap B)\land(y\in C\cap D) \\
                &\equiv (x\in A\land x\in B) \land (y \in C \land y \in D) \\
                &\equiv x\in A \land y \in C \land x \in B \land y \in D \qquad\text{(Associative Law)}\\
                &\equiv (x\in A \land y \in C) \land (x \in B \land y \in D) \\
                &\equiv (x,y)\in(A\times C) \land (x,y)\in(B\times D) \qquad\text{(Definition of Cartesian Product)}\\
                &\equiv (x,y)\in(A\times C)\cap(B\times D)
            \end{align*}
            Therefore, whenever we have $(x,y)\in(A\cap B)\times(C\cap D)$, we must also have
            $(x,y)\in(A\times C)\cap(B\times D)$, i.e., the two sets are equal, and the given
            proposition holds for all sets $A, B, C, D$.
        \end{solution}

        \part Prove $\forall\text{ arbitrary sets }A,B,C: (A\cup B)-C = (A-C)\cup(B-C)$.
        \begin{psol}
            To proof the equality of two sets, we show that whenever an element
            is in the L.H.S. set, it must also be in the R.H.S. set, and vice versa.

            Recall the definitions that
            $x\in(A\cup B) \equiv (x\in A) \lor (x\in B)$ and
            $x\in(A-B) \equiv (x\in A) \land \neg(x\in B)$. Then, for L.H.S., we have
            \begin{align*}
                x\in\left[(A\cup B)-C\right]
                &\equiv x\in(A\cup B)\land \neg(x\in C) \\
                &\equiv (x\in A \lor x\in B) \land (x\notin C)\\
                &\equiv (x\in A \land x\notin C) \lor (x\in B \land x\notin C)\\
                &\equiv [x\in(A-C)] \lor [x\in(B-C)]\\
                &\equiv x\in[(A-C)\cup (B-C)]
            \end{align*}
            Hence, we have shown that the sets on L.H.S. and R.H.S. are equal. Since $A,B,$
            and $C$ are chosen to be arbitrary, therefore the proposition also holds for any
            arbitrary sets.
        \end{psol}

    \end{parts}

    \newpage

    \question Relations

    \begin{parts}
        
        \part For the relation $R_1=\{(x, y)\in\Z\times\Z\mid x+2y\text{ is an even number}\}$,
        prove or disprove that $R_1$ is

        \begin{subparts}

            \subpart reflexive
            \begin{psol}
                To verify reflexivity, we need to check whether $xR_1x$, i.e., whether
                $x+2x=3x$ is an even number. Note that any arbitrary even number can be
                expressed in the form of $2n$ for some $n\in\Z$.

                \textbf{Case 1:} when $x$ is an even number, i.e., $x=2n$.
                Then, we have $3x=3(2n)=6n=2(3n)$, which is an even number. Therefore,
                $xR_1x$ holds when $x$ is an even number.

                \textbf{Case 2:} when $x$ is an odd number, i.e., $x=2n+1$.
                Then, we have $3x=3(2n+1)=6n+3=2(3n+1)+1$, which is an odd number.
                Therefore, $x{R_1}x$ does not hold when $x$ is an odd number.

                Having exhausted all the cases and shown that $\exists x:(x,x)\notin R_1$,
                we conclude that \fbox{$R_1$ is not reflexive}.
            \end{psol}

            \subpart symmetric
            \begin{psol}
                To verify symmetry, we need to check whether $yR_1x$ whenever $yR_1x$ for all
                $x,y\in\Z$.

                Let $x$ be an arbitrary even number, i.e., $x=2n$ for some $n\in\Z$,
                and $y$ be an arbitrary odd number, i.e., $y=2m+1$ for some $m\in\Z$.
                
                Then, for $xR_1y$, we have $x+2y=2n+2(2m+1)=2(n+2m+1)$, which is an even number,
                so $xR_1y$ holds.

                However, for $yR_1x$, we have $y+2x=(2m+1)+2(2n)=2(m+2n)+1$, which is an odd number,
                so $yR_1x$ does not hold.

                Having found a counterexample such that $(x,y)\in R_1$ but $(y, x)\in R_1$,
                we conclude that \fbox{$R_1$ is not symmetric}.
            \end{psol}

            \subpart transitive
            \begin{psol}
                To verify transitivity, we need to check whether $xR_1z$ whenever
                $xR_1y$ and $yR_1z$ for all $x,y,z\in\Z$.
                
                We can show this by contradiction. Assume that $xR_1y$ and $yR_1z$ hold,
                but $xR_1z$ does not hold.
                It implies that $x+2z$ is an odd number. Note that since $2z$ must be even,
                then $x$ must be an odd number. For $xR_1y$ to hold, $x+2y$ must be even,
                and when $x$ is odd, $2y$ must also be odd, which is impossible as $2y$ must
                be even for any integer $y$.

                Therefore, our assumption is wrong, and we have shown that whenever $xR_1y$
                and $yR_1z$ hold, $xR_1z$ must also hold. Hence, we conclude that
                \fbox{$R_1$ is transitive}.
            \end{psol}

        \end{subparts}

        \part Determine correctness of a statement.
        \begin{solution}
            The statement is not correct.

            The logical fallacy is at the statement ``Take an arbitrary $b$ such that $aRb$.''
            The statement assumes that such $b$ exists, which is not necessarily true.
            Symmetry only ensures that if $aRb$ holds, then $bRa$ must also hold, but it does not
            ensure that $aRb$ alone must hold.
        \end{solution}

    \end{parts}

\end{questions}

\end{document}