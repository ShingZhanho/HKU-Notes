\documentclass[answers]{exam}
\usepackage[english]{babel}
\usepackage[style=apa,backend=biber]{biblatex}
\usepackage[
    format=plain,
    labelsep=newline,
    font=small,
    labelfont=bf,
    textfont=it,
    justification=justified,
    singlelinecheck=off
    ]{caption}
\usepackage{csquotes}
\usepackage{float}
\usepackage[top=2.5cm,bottom=2.5cm,left=3cm,right=3cm]{geometry}
\usepackage{graphicx}
\usepackage[hidelinks,colorlinks=true,linkcolor=blue,filecolor=blue,urlcolor=blue,citecolor=blue]{hyperref}
    \usepackage[nameinlink]{cleveref} %nameinlink ensures that the entire element is clickable in the pdf, not just the number
\usepackage{indentfirst}
\usepackage{multicol}
\usepackage[skip=1em,indent]{parskip}
\usepackage{tabularx}
\usepackage{times}
\pagestyle{foot}
\cfoot{Page \thepage\ of \numpages}
\shadedsolutions

\renewcommand{\thesubpart}{(\arabic{subpart})}
\renewcommand{\subpartlabel}{\thesubpart}

\begin{document}

\begin{center}
    \textbf
    {\Large{COMP2120 Computer Organisation} \\
    \large{24/25 Semester 2} \\
    \large{Assignment 1}}
\end{center}

\begin{questions}

    \question Write down the logical expression $P=f(A, B, C)$ corresponding to the following truth table
    (simplification of logical expression is not required):

    \begin{table}[H]
        \centering
        \begin{tabular}{|c|c|c||c|}
            \hline
            $A$ & $B$ & $C$ & $P$ \\
            \hline
            0 & 0 & 0 & 1 \\
            0 & 0 & 1 & 0 \\
            0 & 1 & 0 & 0 \\
            0 & 1 & 1 & 1 \\
            1 & 0 & 0 & 0 \\
            1 & 0 & 1 & 1 \\
            1 & 1 & 0 & 1 \\
            1 & 1 & 1 & 1 \\
            \hline
        \end{tabular}
    \end{table}

    \begin{solution}
        Since there are less rows with $P=0$ than rows with $P=1$, it is easier to work by Product-of-Sums.
        First, identify the rows with $P=0$, which are $(0, 0, 1)$, $(0, 1, 0)$ and $(1, 0, 0)$.
        For each of the rows, construct a product of all the variables, such that the term equals 1,
        then apply NOT to the term to invert it to 0.
        This gives:
        \begin{align*}
            (0, 0, 1) &\implies \overline{\overline{A} \cdot \overline{B} \cdot C} \\
            (0, 1, 0) &\implies \overline{\overline{A} \cdot B \cdot \overline{C}} \\
            (1, 0, 0) &\implies \overline{A \cdot \overline{B} \cdot \overline{C}}
        \end{align*}

        Simplify each term by applying De Morgan's Theorem, then combine them with AND,
        we have:
        \[
            \boxed{P = \left(A+B+\overline{C}\right)\cdot\left(A+\overline{B}+C\right)\cdot\left(\overline{A}+B+C\right)}
        \]
    \end{solution}

    \question Consider a 16-bit 2's complement representation.

    \begin{parts}
        
        \part What is the largest (most positive) number and the smallest (most negative) value in this representation scheme?

        \begin{solution}
            Recall that in 2's complement representation, the most significant bit (MSB) is the sign bit.
            The largest number is when the MSB is 0 and all other bits are 1's, i.e., $0111\,1111\,1111\,1111_2 = \boxed{32\,767_{10}}$.

            Similarly, the smallest number is when the MSB is 1 and all other bits are 0's, i.e., $1000\,0000\,0000\,0000_2 = \boxed{-32\,768_{10}}$.
        \end{solution}

        \part Write down the bit pattern representing 18, -18, 25, and -25, respectively.

        \begin{solution}
            The bit patterns are:
            \begin{table}[H]
                \centering
                \begin{tabular}{|c|c|}
                    \hline
                    Decimal & Bit Pattern \\
                    \hline
                    18 & $\texttt{0000\,0000\,0001\,0010}_2$ \\
                    -18 & $\texttt{1111\,1111\,1110\,1110}_2$ \\
                    25 & $\texttt{0000\,0000\,0001\,1001}_2$ \\
                    -25 & $\texttt{1111\,1111\,1110\,0111}_2$ \\
                    \hline
                \end{tabular}
            \end{table}
        \end{solution}

        \part What are the values of the above bit patterns if they are treated as unsigned integers?

        \begin{solution}
            When treated as unsigned integers, the values are:
            \begin{table}[H]
                \centering
                \begin{tabular}{|c|c|}
                    \hline
                    Bit Pattern & \texttt{uint} Value \\
                    \hline
                    $\texttt{0000\,0000\,0001\,0010}_2$ & 18 \\
                    $\texttt{1111\,1111\,1110\,1110}_2$ & 65\,518 \\
                    $\texttt{0000\,0000\,0001\,1001}_2$ & 25 \\
                    $\texttt{1111\,1111\,1110\,0111}_2$ & 65\,511 \\
                    \hline
                \end{tabular}
            \end{table}
        \end{solution}

        \part Add the bit patterns together for the following:

        \begin{multicols}{4}
            \begin{subparts} 
                \subpart $18 + 25$
                \subpart $18 + (-25)$
                \subpart $(-18) + 25$
                \subpart $(-18) + (-25)$
            \end{subparts}
        \end{multicols}

        \begin{solution}
            The bit patterns required are:
            \begin{multicols}{2}\begin{enumerate}[label=(\arabic*)]
                \item $18+25=43$
                \begin{equation*}\begin{aligned}[t]\begin{arithmetic}[t]
                    \texttt{0000\,0000\,0001\,0010} & ($+18$) \\
                +   \texttt{0000\,0000\,0001\,1001} & ($+25$) \\
                    \texttt{0000\,0000\,0010\,1011} & ($+43$) \\
                \end{arithmetic}\end{aligned}\end{equation*}

                \item $18+(-25)=-7$
                \begin{equation*}\begin{aligned}[t]\begin{arithmetic}[t]
                    \texttt{0000\,0000\,0001\,0010} & ($+18$) \\
                +   \texttt{1111\,1111\,1110\,0111} & ($-25$) \\
                    \texttt{1111\,1111\,1111\,1001} & ($-7$) \\
                \end{arithmetic}\end{aligned}\end{equation*}

                \item $(-18)+25=7$
                \begin{equation*}\begin{aligned}[t]\begin{arithmetic}[t]
                    \texttt{1111\,1111\,1110\,1110} & ($-18$) \\
                +   \texttt{0000\,0000\,0001\,1001} & ($+25$) \\
                    \texttt{\cancel{1}\,0000\,0000\,0000\,0111} & ($+7$) \\
                \end{arithmetic}\end{aligned}\end{equation*}

                \item $(-18)+(-25)=-43$
                \begin{equation*}\begin{aligned}[t]\begin{arithmetic}[t]
                    \texttt{1111\,1111\,1110\,1110} & ($-18$) \\
                +   \texttt{1111\,1111\,1110\,0111} & ($-25$) \\
                    \texttt{\cancel{1}\,1111\,1111\,1101\,0101} & ($-43$) \\
                \end{arithmetic}\end{aligned}\end{equation*}
            \end{enumerate}\end{multicols}
        \end{solution}

    \end{parts}

    \question Prove that the multiplication of an $n$-bit binary number $A$ and an $m$-bit binary number $B$
    gives a product $A\times B$ of no more than $(n+m)$ bits.

    \begin{solution}
        Note that $A\in[0, 2^n-1]$ and $B\in[0, 2^m-1]$, $m, n\geq 1$, and $m+n \geq 2$.

        The maximum value of $A\times B$ is given by:
        \begin{align*}
            \max(A\times B) &= \max(A) \times \max(B) \\
            &= (2^n-1) \times (2^m-1) \\
            &= 2^n \times 2^m - 2^n - 2^m + 1 \\
            &= 2^{n+m} - 2^{n} - 2^{m} + 1
        \end{align*}
        Note that to represent the term $2^{n+m}$, we need at least $(n+m+1)$ bits, and
        the maximum value representable by $(n+m)$ bits is $2^{n+m}-1$.
        Therefore, we need to show that:
        \begin{equation} \label{eq:q3-proposition}
            \forall m,n\in[1,+\infty)\cap\mathbb{Z}, \quad2^{n+m} - 2^{n} - 2^{m} + 1 < 2^{n+m}
        \end{equation}

        Assume that $m$ is constant, when $n=1$, we have:
        \begin{align*}
            \text{L.H.S.} &= 2^{1+m} - 2^{1} - 2^{m} + 1 \\
            &= 2^{m+1} - 2 - 2^{m} + 1 \\
            &= 2^{m+1} - 2^{m} - 1 \\
            &= 2^{m} (2 - 1) - 1 \\
            &= 2^{m} - 1
        \end{align*}
        and $\text{R.H.S.} = 2^{m+1}$, so $\text{L.H.S.} < \text{R.H.S.}$ clearly holds.
        Therefore, the proposition holds for $n=1$.

        Now, assume that the proposition holds for $n=k$ where $k\in[1,+\infty)\cap\mathbb{Z}$, i.e.,
        \begin{equation} \label{eq:q3-induction-hypothesis}
            \forall k\in[1,+\infty)\cap\mathbb{Z}, \quad 2^{k+m} - 2^{k} - 2^{m} + 1 < 2^{k+m}
        \end{equation}
        We need to show that the proposition holds for $n=k+1$, i.e.,
        \begin{align*}
            \text{L.H.S.} &= 2^{k+1+m} - 2^{k+1} - 2^{m} + 1 \\
            &= 2^{k+m} \cdot 2 - 2^{k} \cdot 2 - 2^{m} + 1 \\
            &= 2^{k+m} - 2^{m} - 2^{k} + 1 + 2^{k+m} - 2^{k}
        \end{align*}
        and
        \begin{align*}
            \text{R.H.S.} &= 2^{k+1+m} \\
            &= 2^{k+m} \cdot 2 \\
            &= 2^{k+m} + 2^{k+m}
        \end{align*}
        Subtract $2^{k+m}$ from both sides, we have:
        \[
            \text{L.H.S.} = \underbrace{2^{k+m} - 2^{m} - 2^{k} + 1}_{\text{Induction Hypothesis (Eq. \ref{eq:q3-induction-hypothesis})}} - 2^{k} \qquad \text{and} \qquad
            \text{R.H.S.} = 2^{k+m}
        \]
        Observe that by the induction hypothesis, we have $2^{k+m} - 2^{m} - 2^{k} + 1 < 2^{k+m}$,
        and since $2^{k} > 0$, we have:
        \[
            \text{L.H.S.} < 2^{k+m} - 2^{k} < 2^{k+m}
        \]
        Therefore, the proposition holds for $n=k+1$.

        By the principle of mathematical induction, the proposition holds for all $n\in[1,+\infty)\cap\mathbb{Z}$.

        Now, proposition \ref{eq:q3-induction-hypothesis} is partially proven for all $n$ and constant $m$.
        Observe that the proposition is symmetric in $m$ and $n$, i.e., $m$ and $n$ are dummy variables.
        Therefore, it is trivial to show that the proposition holds for all $m\in[1,+\infty)\cap\mathbb{Z}$ and constant $n$.
        Hence, we have shown that the proposition holds for all $m,n\in[1,+\infty)\cap\mathbb{Z}$.

        Now, we can conclude that the $A\times B = 2^{n+m} - 2^{n} - 2^{m} + 1 < 2^{n+m}$, which means
        that the product $A\times B$ can be represented by no more than $(n+m)$ bits.
        
        \hfill \textbf{Q.E.D.}
    \end{solution}

    \question Verify the validity of the multiplication of integers (2's complement) procedure in the lecture notes.
    (Give the prove)

    \begin{solution}
        It may be helpful to recall that for any $n$-bit 2's complement number $\left(a_{n-1}a_{n-2}\dots a_2a_1a_0\right)_2$,
        where $a\in[0, 1]$, the value of the number is given by
        \begin{equation*}
            A=-2^{n-1}a_{n-1} + \sum_{i=0}^{n-2} 2^i a_i
        \end{equation*}

        Also, recall that the multiplication procedure in the lecture notes involves sign-extension and negation
        of the multiplicand and multiplier, it would be helpful to first prove that these two operations are valid.

        \textbf{Proof} of sign-extension:

        Suppose $A$ is sign-extended to $A'$ of $m$ bits, where $m>n$. Then, $A'$ can be expressed as:
        \[
            A' = \underbrace{\left(a_{n-1}a_{n-1}\dots a_{n-1}a_{n-2}\dots a_2a_1a_0\right)_2}_{m\text{ bits}}
        \]
        and its value is given by:
        \[
            A' = -2^{m-1}a_{n-1} + \sum_{i=0}^{m-2} 2^i a_j \text{, where }
            j = \begin{cases}
                i & \text{if } i \in [0, n-2] \\
                n-1 & \text{if } i \geq n-1
            \end{cases}
        \]
        By splitting the summation, we have:
        \begin{align*}
            A' &= -2^{m-1}a_{n-1} + \sum_{i=0}^{n-2} 2^i a_i + \sum_{i=n-1}^{m-2} 2^i a_{n-1} \\
            &= -2^{m-1}a_{n-1} + \sum_{i=0}^{n-2} 2^i a_i + a_{n-1} \cdot \sum_{i=n-1}^{m-2} 2^i \\
            &= -2^{m-1}a_{n-1} + \sum_{i=0}^{n-2} 2^i a_i + a_{n-1} \cdot \left(\frac{2^{n-1}\left(2^{m-2-(n-1)+1}-1\right)}{2-1}\right) \\
            % &= -2^{m-1}a_{n-1} + \sum_{i=0}^{n-2} 2^i a_i + a_{n-1} \cdot 2^{n-1} \cdot \left(2^{m-n}-1\right) \\
            &= -2^{m-1}a_{n-1} + \sum_{i=0}^{n-2} 2^i a_i + a_{n-1} \cdot \left(2^{m-1} - 2^{n-1}\right) \\
            &= \cancel{-2^{m-1}a_{n-1}} + \sum_{i=0}^{n-2} 2^i a_i + \cancel{2^{m-1} a_{n-1}} - 2^{n-1} a_{n-1} \\
            &= -2^{n-1} a_{n-1} + \sum_{i=0}^{n-2} 2^i a_i \\ &= A
        \end{align*}
        We can conclude that the value of $A'$ is the same as the value of $A$,
        therefore, sign-extension is valid.

        \hfill \textbf{Q.E.D.}

        \textbf{Proof} of negation:
        Recall that the negation of a number $A$ is given by taking its 2's complement,
        i.e. apply bitwise NOT to the number, then add 1 to the result.

        Consider a number $B$ of $n$ bits, which is given by
        $\left(\overline{a_{n-1}}\,\overline{a_{n-2}} \dots \overline{a_2}\,\overline{a_1}\,\overline{a_0}\right)_2 + 1$.
        Then, the value of $B$ is given by:
        \[
            B = -2^{n-1}\overline{a_{n-1}} + \sum_{i=0}^{n-2} 2^i \overline{a_i} + 1
        \]
        Also recall that for any bit $a_i$, we have $\overline{a_i} = 1 - a_i$.

        Now, consider $A+B$, we have:
        \begin{align*}
            A+B &= -2^{n-1}a_{n-1} + \sum_{i=0}^{n-2} 2^i a_i -2^{n-1}\overline{a_{n-1}} + \sum_{i=0}^{n-2} 2^i \overline{a_i} + 1 \\
            &= -2^{n-1}a_{n-1} + \sum_{i=0}^{n-2} 2^i a_i -2^{n-1}(1-a_{n-1}) + \sum_{i=0}^{n-2} 2^i (1-a_i) + 1 \\
            &= -2^{n-1}a_{n-1} + \sum_{i=0}^{n-2} 2^i a_i -2^{n-1} + 2^{n-1} a_{n-1} + \sum_{i=0}^{n-2} 2^i - \sum_{i=0}^{n-2} 2^i a_i + 1 \\
            &= \cancel{-2^{n-1}a_{n-1}} + \cancel{\sum_{i=0}^{n-2} 2^i a_i} - 2^{n-1} + \cancel{2^{n-1} a_{n-1}} + \sum_{i=0}^{n-2} 2^i - \cancel{\sum_{i=0}^{n-2} 2^i a_i} + 1 \\
            &= -2^{n-1} + \frac{2^0\cdot\left(2^{n-2-0+1}-1\right)}{2-1} + 1 \\
            &= -2^{n-1} + 2^{n-1} - 1 + 1 \\
            A+B &= 0 \\
            \text{Therefore, } B &= -A
        \end{align*}

        We can conclude that the negation of a number $A$ is indeed valid,
        since it gives the negative of the number, i.e., $-A$.

        \hfill \textbf{Q.E.D.}

        Now, we can verify the multiplication procedure in the lecture notes.

        \textbf{Proof} of multiplication procedure:

        First, sign-extension is performed on the multiplicand and multiplier to ensure that they are both of the same bit length,
        which is proven to be valid above.

        Now consider the multiplication of two $n$-bit 2's complement numbers $A$ and $B$,
        where $A$ is the multiplicand and $B$ is the multiplier. We have:
        \[
            A\times B = \left(-2^{n-1}a_{n-1} + \sum_{i=0}^{n-2} 2^i a_i\right) \times \left(-2^{n-1}b_{n-1} + \sum_{j=0}^{n-2} 2^j b_j\right)
        \]

        \textbf{Case 1:} $A, B \geq 0$ (i.e., $a_{n-1} = b_{n-1} = 0$), our desired expression of the product is:
        \begin{equation}
            A\times B = \left(\sum_{i=0}^{n-2} 2^i a_i\right) \times \left(\sum_{j=0}^{n-2} 2^j b_j\right)
        \end{equation}
        Consider the procedure given by the lecture notes, $\forall b_j \in B = 1$, $A$ is shifted to the left by $j$ bits,
        which is equivalent to $A \times 2^j$, and then added to the product. Therefore,
        \begin{align*}
            A\times B &= \left(\sum_{i=0}^{n-2}2^ia_i\right)\times 2^0 \times b_0 + \left(\sum_{i=0}^{n-2}2^ia_i\right)\times 2^1 \times b_1 + \cdots + \left(\sum_{i=0}^{n-2}2^ia_i\right)\times 2^{n-1} \times b_{n-1} \\
            &= \sum_{i=0}^{n-2} 2^i a_i \cdot \left(2^0 b_0 + 2^1 b_1 + \cdots + 2^{n-1} b_{n-1}\right) \\
            &= \sum_{i=0}^{n-2} 2^i a_i \cdot \left(\sum_{j=0}^{n-2} 2^j b_j + 2^{n-1}b_{n-1}\right) \\
            &= A \times B \qquad \text{(since } b_{n-1} = 0\text{)}
        \end{align*}
        Therefore, the multiplication procedure holds for two non-negative numbers.

        \textbf{Case 2:} $A \geq 0, B < 0$ (i.e., $a_{n-1} = 0, b_{n-1} = 1$), our desired expression of the product is:
        \begin{equation}
            A\times B = \left(\sum_{i=0}^{n-2} 2^i a_i\right) \times \left(-2^{n-1} + \sum_{j=0}^{n-2} 2^j b_j\right)
        \end{equation}
        The procedure states: \begin{enumerate}
            \item For each $j \in [0, n-2]$ where $b_j = 1$, shift $A$ to the left by $j$ bits, which is equivalent to $A \times 2^j$.
            \item Sum all terms given by step 1, which gives $\sum_{j=0}^{n-2} A \times 2^j \times b_j$.
            \item For the MSB of $B$, if $b_{n-1} = 1$, then negate $A$ and shift it to the left by $(n-1)$ bits, which is equivalent to $-A \times 2^{n-1} \times b_{n-1}$.
            \item Add the results of step 2 and step 3 together, which gives:
            \begin{align*}
                A\times B &= \sum_{j=0}^{n-2} A \times 2^j \times b_j - A \times 2^{n-1} \times b_{n-1} \\
                &= A \times \left(-2^{n-1}b_{n-1} + \sum_{j=0}^{n-2}2_jb_j\right) \\
                &= \left(\sum_{i=0}^{n-2} 2^i a_i\right) \times \left(-2^{n-1} + \sum_{j=0}^{n-2} 2^j b_j\right)
            \end{align*}
        \end{enumerate}
        This matches our desired expression of the product, therefore,
        the multiplication procedure holds for $A \geq 0, B < 0$.

        \textbf{Case 3:} $A < 0, B \geq 0$ (i.e., $a_{n-1} = 1, b_{n-1} = 0$). By considering the commutativity of multiplication,
        and the fact that $A$ and $B$ are dummy variables, it is trivial to show that the multiplication procedure holds for this case as well.

        \textbf{Case 4:} $A < 0, B < 0$ (i.e., $a_{n-1} = 1, b_{n-1} = 1$), our desired expression of the product is:
        \begin{equation}
            A\times B = \left(-2^{n-1} + \sum_{i=0}^{n-2} 2^i a_i\right) \times \left(-2^{n-1} + \sum_{j=0}^{n-2} 2^j b_j\right)
        \end{equation}

        The multiplication procedure for this case is identical to that of Case 2.
        Notice how $A$ was not expressed in its full form (i.e. $-2^{n-1}a_{n-1} + \sum_{i=0}^{n-2} 2^i a_i$)
        in the proof of Case 2. This implies that the calculation procedure is independent of the sign of $A$,
        and therefore, the multiplication procedure holds for this case as well.

        By now, we have shown that the multiplication procedure in the lecture notes holds for all cases of $A$ and $B$,
        therefore, the multiplication procedure is valid.

        \hfill \textbf{Q.E.D.}
    \end{solution}

    \question Any floating-point representation used in a computer can represent only certain real numbers exactly; all
    others must be approximated. If $A'$ is the stored value approximating the real value $A$, then the relative
    error, $r$, is expressed as
    \begin{equation*}
        r = \frac{A - A'}{A}
    \end{equation*}
    Represent the decimal quantity $+0.4$ in the following floating-point format: base $= 2$; exponent: biased, $4$
    bits; significand, $7$ bits. What is the relative error?

    \begin{solution}
        First, convert $0.4$ to binary and normalise it:
        \begin{align*}
            0.4_{10} &= 0.0110011001100\dots_2 \\
            &= 1.\underbrace{1001100}_{\text{significand}}1100\dots_2 \times 2^{-2}
        \end{align*}
        Then, find the biased exponent:
        \begin{equation*}
            \text{Biased exponent} = -2 + (2^3 - 1) = 5_10 = 0101_2
        \end{equation*}
        Therefore, the floating-point representation of $0.4$ is:
        \[
            A' = \underbrace{0}_{\text{sign}} \underbrace{0101}_\text{exponent} \underbrace{1001100}_\text{significand}
        \]
        Now, we find the stored value of $A'$:
        \begin{align*}
            A' &= (1 + 2^{-1} + 2^{-4} + 2^{-5}) \times 2^{-2} \\
            &= 2^{-2} + 2^{-3} + 2^{-6} + 2^{-7} \\
            &= 0.3984375_{10}
        \end{align*}
        Now, we can calculate the relative error:
        \begin{align*}
            r &= \frac{A - A'}{A} \\
            &= \frac{0.4-0.3984375}{0.4} \\
            &= \frac{0.0015625}{0.4} \\
            &= \boxed{0.00391} \text{(corr. to 3 sig. figs.)}
        \end{align*}
    \end{solution}

    \question Consider a $40$-bit floating point representation with a sign bit $S$, an exponent $E$ (biased, $11$ bits), and a
    significand $f$ ($28$ bits). The value is
    \[
        V = (-1)^S \cdot 1.f \cdot 2^{E - 1023}
    \]
    Here, $E=11\dots111_2$ and $f=00\dots000_2$ do not have special meanings.

    \begin{parts}
        \part Write down the largest positive number that can be represented.

        \begin{solution}
            \begin{align*}
                \text{Largest value} &= \left(1+\sum_{i=-28}^{-1} 2^i\right) \cdot 2^{2^{11}-1-1023} \\
                &= \left(1+2^{-28}\cdot(2^{28}-1)\right) \cdot 2^{1024} \\
                &= (1+1-2^{-28}) \cdot 2^{1024} \\
                &= \boxed{2^{1025} - 2^{996}}
            \end{align*}
        \end{solution}

        \part Write down the smallest positive number that can be represented.

        \begin{solution}
            \begin{align*}
                \text{Smallest value} &= 1 \cdot 2^{-1023} \\
                &= \boxed{2^{-1023}}
            \end{align*}
        \end{solution}

        \part Write down the bit pattern representing the value $15.3125$.

        \begin{solution}
            First, convert $15.3125$ to binary:
            \begin{align*}
                15.3125_{10} &= 1111.0101_2 \\
                &= 1.1110101_2 \times 2^3
            \end{align*}
            Now, find the biased exponent:
            \begin{align*}
                \text{Biased exponent} &= 3 + (2^{11-1} - 1) \\
                &= 3 + 1023 = 1026_{10} = 1000\,0000\,010_2
            \end{align*}
            Combining the sign bit, exponent, and significand, we have:
            \[
                \text{Bit pattern} = \underbrace{0}_{\text{sign}} \underbrace{1000\,0000\,010}_{\text{exponent}} \underbrace{1110\,1010\,0000\,0000\,0000\,0000\,0000}_{\text{significand}}
                = \boxed{\texttt{0x402EA00000}}
            \]
        \end{solution}

        \part Write down the value represented by the bit pattern \texttt{0xC06F800000}.

        \begin{solution}
            Convert the hexadecimal to binary:
            \begin{align*}
                \texttt{0xC06F800000} &= \texttt{1100\,0000\,0110\,1111\,1000\,0000\,0000\,0000\,0000\,0000} \\
                &= \underbrace{1}_{\text{sign}} \underbrace{1000\,000\,110}_{\text{exponent}} \underbrace{1111\,1000\,0000\,0000\,0000\,0000\,0000}_{\text{significand}}
            \end{align*}
            Now, calculate the value:
            \begin{align*}
                V &= (-1)^1 \cdot 1.1111000_2 \cdot 2^{1030-1023} \\
                &= -1.11111000_2 \cdot 2^7 \\
                &= -11111100_2 \\
                &= \boxed{-252_{10}}
            \end{align*}
        \end{solution}

        \part If we assign 16 bits and 23 bits for exponent $E$ and significand $f$ , respectively. What is the largest
        positive number that can be represented ? Discuss what is the relation between range and precision in
        floating point number representation?

        \begin{solution}
            \textit{Omitted.}
        \end{solution}
    \end{parts}
    
\end{questions}

\end{document}