\section{Se situer dans le temps}

Dans ce cours, on continue à développer notre connaissance du passé composé du dernier semestre,
et on introduit deux nouveaux temps : l'imparfait et le plus-que-parfait. On va aussi apprendre
à combiner ces temps pour raconter des histoires et décrire des situations dans le passé.

\subsection{Les temps du passé}

\subsubsection{Le passé composé}

\begin{frbox}{Conjugaison du passé composé}
    La formation du passé composé est la suivante :

    {
        \centering
        \conjugelem{avoir} ou \conjugelem{être} au présent + \conjugelem{participe passé}\par
    }

    Généralement, on utilise \conjugelem{avoir} comme auxiliaire, sauf pour les 17 verbes, ou les
    verbes pronominaux, qui se conjuguent avec \conjugelem{être}.

    Quand on utilise \conjugelem{être} comme auxiliaire, le participe passé s'accorde en genre
    et en nombre avec le sujet. Quand on utilise \conjugelem{avoir}, le participe passé ne s'accorde
    pas, sauf \textbf{si le complément d'objet direct (COD) précède le verbe}.
\end{frbox}

Par exemple :

\begin{itemize}
    \item{} [\textit{manger}, avec \conjugelem{avoir}, ne s'accorde pas] : J\hl{\textbf{'ai mangé}} une pomme.
    \item{} [\textit{aller}, avec \conjugelem{être}, s'accorde] : Elle \hl{\textbf{est allée}} au marché.
    \item{} [\textit{voir}, avec \conjugelem{avoir}, s'accorde car le COD précède] : La lettre \hl{\textbf{que}} j\hl{\textbf{'ai vue}} était importante.
    
    Dans cet exemple, le COD \textit{que} (qui remplace \textit{la lettre}) précède le verbe,
    donc le participe passé \textit{vue} s'accorde en genre (féminin) et en nombre (singulier) avec
    \textit{la lettre}.
\end{itemize}

Les 17 verbes sont
\begin{enumerate*}[label=(\arabic*)]
    \item aller \item arriver \item descendre* \item devenir \item entrer \item monter* \item mourir \item naître \item partir
    \item passer* \item rentrer \item rester \item retourner* \item revenir \item sortir* \item tomber \item venir
\end{enumerate*}

\begin{frwarn}
    Les verbes marqués d'un astérisque (*) peuvent aussi se conjuguer avec \conjugelem{avoir}
    lorsqu'ils sont utilisés de manière transitive, c'est-à-dire lorsqu'ils ont un complément
    d'objet direct (COD). Par exemple :
    \begin{itemize}
        \item Il \hl{\textbf{a retourné}} le livre à la bibliothèque.
    \end{itemize}
\end{frwarn}

\begin{frbox}[violet]{Utilisation du passé composé}
    On utilise le passé composé pour :
    \begin{itemize}
        \item Décrire des actions complètes dans le passé.
        
        \qquad\textit{Hier, je suis allé(e) à l'université.}
        \item Indiquer des événements spécifiques qui se sont produits à un moment précis.
        
        \qquad\textit{Elle a fini ses devoirs il y a deux heures.}
        \item Raconter des événements dans une séquence chronologique.
        
        \qquad\textit{D'abord, nous avons visité le musée, puis nous avons déjeuné au café.}
    \end{itemize}
\end{frbox}

\subsubsection{L'imparfait}

\begin{frbox}{Conjugaison de l'imparfait}
    La formation de l'imparfait est la suivante :

    {
        \centering
        \conjugelem{verbe} (1\iere{} personne pluriel (nous)) au présent
        $-$ \conjugelem{-ons} $+$ les terminaisons de l'imparfait\par
    } 

    Les terminaisons de l'imparfait sont :
    \begin{table}[H]
        \centering
        \begin{tabular}{rcccccc}
            \textbf{Personne} & je & tu & il/elle/on & nous & vous & ils/elles \\
            \hline
            \textbf{Terminaison} & -ais & -ais & -ait & -ions & -iez & -aient
        \end{tabular}
    \end{table}

    \textbf{Exception :} \conjugelem{être} utilise la racine \conjugelem{ét-} :
    j'étais, tu étais, il/elle/on était, nous étions, vous étiez, ils/elles étaient.

    \textbf{Attention :} Si la racine du verbe se termine par \conjugelem{-g} ou \conjugelem{-c},
    on ajoute un \conjugelem{e} ou remplace le \conjugelem{c} par \conjugelem{ç} avant les terminaisons
    qui commencent par \conjugelem{a} ou \conjugelem{o} pour conserver la prononciation douce.
\end{frbox}

Le verbe \conjugelem{être} est la seule exception à la règle de formation de l'imparfait.

Par exemple :
\begin{itemize}
    \item{} [\textit{mange}, la racine termine par \conjugelem{-g}] : Il \hl{\textbf{mang{\color{red}e}ait}} un croissant tous les matins.
    \item{} [\textit{finir}] : Tu \hl{\textbf{finissais}} tes devoirs avant le dîner.
    \item{} [\textit{vendre}] : Ils \hl{\textbf{vendaient}} des fruits au marché.
    \item{} [\textit{commencer}, la racine termine par \conjugelem{-c}] : Les gens \hl{\textbf{commen{\color{red}ç}aient}} à travailler à huit heures.
    \item{} [\textit{être}] : J\hl{\textbf{'étais}} très heureux(se) quand j'étais enfant.
\end{itemize}

\begin{frbox}[violet]{Utilisation de l'imparfait}
    On utilise l'imparfait pour :
    \begin{itemize}
        \item Décrire des actions habituelles ou répétées dans le passé.
        
        \qquad\textit{Quand j'étais enfant, je jouais au parc tous les samedis.}
        \item Décrire des états ou des conditions dans le passé.
        
        \qquad\textit{Il faisait beau ce jour-là.}
        \item Fournir des descriptions de personnes, de lieux ou de situations dans le passé.
        
        \qquad\textit{La maison était grande et avait un jardin magnifique.}
    \end{itemize}
\end{frbox}

\subsection{Combiner le passé composé et l'imparfait}

Les deux temps sont souvent utilisés ensemble pour raconter des histoires ou décrire des situations
dans le passé.