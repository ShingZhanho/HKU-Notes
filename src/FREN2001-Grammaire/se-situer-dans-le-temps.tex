\section{Se situer dans le temps}

Dans ce cours, on continue à développer notre connaissance du passé composé du dernier semestre,
et on introduit deux nouveaux temps : l'imparfait et le plus-que-parfait. On va aussi apprendre
à combiner ces temps pour raconter des histoires et décrire des situations dans le passé.

D'ailleurs, on va apprendre le gérondif et le présent du subjonctif dans ce cours.

\subsection{Les temps du passé}

\subsubsection{Le passé composé}

\begin{frbox}{Conjugaison du passé composé}
    {
        \centering
        \conjugelem{avoir} ou \conjugelem{être} au présent + \conjugelem{participe passé}\par
    }

    Généralement, on utilise \conjugelem{avoir} comme auxiliaire, sauf pour les 17 verbes, ou les
    verbes pronominaux, qui se conjuguent avec \conjugelem{être}.

    \textbf{Accord du participe passé} en nombre et en genre avec le sujet :
    \begin{itemize}
        \item avec \conjugelem{avoir} : \textbf{Jamais}, sauf si le complément d'objet direct (COD)
        précède le verbe.
        \item avec \conjugelem{être} : \textbf{Toujours},
        sauf si le complément d'objet indirect (COI) précède le verbe, ou
        si le complément d'objet direct (COD) suit le verbe.
    \end{itemize}
\end{frbox}

Par exemple :

\begin{itemize}
    \item{} [\textit{manger}, avec \conjugelem{avoir}, ne s'accorde pas] : J'\hl{\textbf{ai mangé}} une pomme.
    \item{} [\textit{aller}, avec \conjugelem{être}, s'accorde] : Elle \hl{\textbf{est allée}} au marché.
    \item{} [\textit{voir}, avec \conjugelem{avoir}, s'accorde car le COD « que » précède] : La lettre \hl{\textbf{que}} j'\hl{\textbf{ai vue}} était importante.
    \item{} [\textit{se laver}, verbe pronominal, ne s'accorde pas car le COD « les mains » suit] : Elle \hl{\textbf{s'est lavé}} les mains.
    \item{} [\textit{téléphoner}, avec \conjugelem{être}, ne s'accorde pas car le COI « se » précède] : Elles \hl{\textbf{se sont téléphoné}} hier.
\end{itemize}

Les 17 verbes sont
\begin{enumerate*}[label=(\arabic*)]
    \item aller \item arriver \item descendre* \item devenir \item entrer \item monter* \item mourir \item naître \item partir
    \item passer* \item rentrer \item rester \item retourner* \item revenir \item sortir* \item tomber \item venir
\end{enumerate*}

\begin{frwarn}
    Les verbes marqués d'un astérisque (*) peuvent aussi se conjuguer avec \conjugelem{avoir}
    lorsqu'ils sont utilisés de manière transitive, c'est-à-dire lorsqu'ils ont un complément
    d'objet direct (COD). Par exemple :
    \begin{itemize}
        \item Il \hl{\textbf{a retourné}} le livre à la bibliothèque.
    \end{itemize}
\end{frwarn}
\vspace*{-1.5em}
\begin{frbox}[violet]{Utilisation du passé composé}
    On utilise le passé composé pour :
    \begin{itemize}
        \item Décrire des actions complètes dans le passé.
        
        \qquad\textit{Hier, je suis allé(e) à l'université.}
        \item Indiquer des événements spécifiques qui se sont produits à un moment précis.
        
        \qquad\textit{Elle a fini ses devoirs il y a deux heures.}
        \item Raconter des événements dans une séquence chronologique.
        
        \qquad\textit{D'abord, nous avons visité le musée, puis nous avons déjeuné au café.}
    \end{itemize}
\end{frbox}

\subsubsection{L'imparfait}

\begin{frbox}{Conjugaison de l'imparfait}
    {
        \centering
        \conjugelem{verbe} (1re personne pluriel (nous)) au présent
        $-$ \conjugelem{-ons} $+$ les terminaisons de l'imparfait\par
    } 

    Les terminaisons de l'imparfait sont :
    \begin{table}[H]
        \centering
        \begin{tabular}{rcccccc}
            \textbf{Personne} & je & tu & il/elle/on & nous & vous & ils/elles \\
            \hline
            \textbf{Terminaison} & -ais & -ais & -ait & -ions & -iez & -aient
        \end{tabular}
    \end{table}

    \textbf{Exception :} \conjugelem{être} utilise la racine \conjugelem{ét-} :
    j'étais, tu étais, il/elle/on était, nous étions, vous étiez, ils/elles étaient.

    \textbf{Attention :} Si la racine du verbe se termine par \conjugelem{-g} ou \conjugelem{-c},
    on ajoute un \conjugelem{e} ou remplace le \conjugelem{c} par \conjugelem{ç} avant les terminaisons
    qui commencent par \conjugelem{a} ou \conjugelem{o} pour conserver la prononciation douce.
\end{frbox}

Le verbe \conjugelem{être} est la seule exception à la règle de formation de l'imparfait.

Par exemple :
\begin{itemize}
    \item{} [\textit{mange}, la racine termine par \conjugelem{-g}] : Il \hl{\textbf{mang{\color{red}e}ait}} un croissant tous les matins.
    \item{} [\textit{finir}] : Tu \hl{\textbf{finissais}} tes devoirs avant le dîner.
    \item{} [\textit{vendre}] : Ils \hl{\textbf{vendaient}} des fruits au marché.
    \item{} [\textit{commencer}, la racine termine par \conjugelem{-c}] : Les gens \hl{\textbf{commen{\color{red}ç}aient}} à travailler à huit heures.
    \item{} [\textit{être}] : J'\hl{\textbf{étais}} très heureux(se) quand j'étais enfant.
\end{itemize}

\begin{frbox}[violet]{Utilisation de l'imparfait}
    On utilise l'imparfait pour :
    \begin{itemize}
        \item Décrire des actions habituelles ou répétées dans le passé.
        
        \qquad\textit{Quand j'étais enfant, je jouais au parc tous les samedis.}
        \item Décrire des états ou des conditions dans le passé.
        
        \qquad\textit{Il faisait beau ce jour-là.}
        \item Fournir des descriptions de personnes, de lieux ou de situations dans le passé.
        
        \qquad\textit{La maison était grande et avait un jardin magnifique.}
    \end{itemize}
\end{frbox}

\subsubsection{Combiner le passé composé et l'imparfait}

Les deux temps sont souvent utilisés ensemble pour raconter des histoires ou décrire des situations
dans le passé.

\begin{frbox}[violet]{Utilisation combinée du passé composé et de l'imparfait}
    Pour raconter une action en cours dans le passé qui est interrompue par une acction spécifique,
    courte, ou soudain, on utilise l'imparfait pour l'action en cours et le passé composé pour l'action
    qui interrompt.
\end{frbox}

\subsubsection{Le plus-que-parfait}

\begin{frbox}{Conjugaison du plus-que-parfait}
    {
        \centering
        \conjugelem{avoir} ou \conjugelem{être} à l'imparfait + \conjugelem{participe passé}\par
    }

    Les règles d'accord du participe passé sont les mêmes que pour le passé composé.
\end{frbox}

Exemples :
\begin{itemize}
    \item{} [\textit{finir}, avec \conjugelem{avoir}] : J'\hl{\textbf{avais fini}} mes devoirs avant le dîner.
    \item{} [\textit{partir}, avec \conjugelem{être}] : Elle \hl{\textbf{était partie}} avant mon arrivée.
\end{itemize}

\begin{frbox}[violet]{Utilisation du plus-que-parfait}
    On utilise le plus-que-parfait pour :
    \begin{itemize}
        \item Indiquer une action qui s'est produite avant une autre action dans le passé.
        
        \qquad\textit{Il avait déjà mangé quand je suis arrivé(e).}
    \end{itemize}
\end{frbox}

\subsection{Vocabulaires temporels et expressions}

\subsubsection{Vocabulaires utiles pour se situer dans le temps}

Quand on se réfère aux événements du passé, du présent ou du futur depuis différents
moments dans le temps, on utilise différents mots. Voici quelques vocabulaires utiles
pour indiquer le cadre temporel des événements.

\begin{table}[H]
    \centering
    \begin{tabular}{ll}
        \multicolumn{1}{c}{\textbf{Moment de parole : présent}} & \multicolumn{1}{|c}{\textbf{Moment de parole : passé ou futur}} \\
        \hline
        \multicolumn{2}{c}{\cellcolor{gray!25}Événements/Actions du passé} \\
        \multilinecell{hier} & \multilinecell{la veille\\le jour d'avant (familier)\\le jour précédent (formel)} \\
        \hline
        \multilinecell{avant-hier} & \multilinecell{l'avant-veille\\deux jours avant (familier)\\deux jours plus tôt/auparavant (formel)} \\
        \hline
        \multilinecell{il y a deux jours} & \multilinecell{deux jours avant (familier)\\deux jours plus tôt/auparavant (formel)} \\
        \hline
        \multilinecell{lundi dernier} & \multilinecell{le lundi d'avant (familier)\\le lundi précédent (formel)} \\

        \multicolumn{2}{c}{\cellcolor{gray!25}Événements/Actions du présent} \\
        \multilinecell{aujourd'hui} & \multilinecell{ce jour-là} \\
        \hline
        \multilinecell{maintenant\\à présent\\en ce moment} & \multilinecell{à ce moment-là } \\
        \hline
        \multilinecell{cette semaine} & \multilinecell{cette semaine-là} \\

        \multicolumn{2}{c}{\cellcolor{gray!25}Événements/Actions du futur} \\
        \multilinecell{demain} & \multilinecell{le lendemain\\le jour d'après (familier)\\le jour suivant (formel)} \\
        \hline
        \multilinecell{après-demain} & \multilinecell{le surlendemain\\deux jours après (familier)\\deux jours plus tard (formel)} \\
        \hline
        \multilinecell{l'année prochaine} & \multilinecell{l'année suivante\\l'année d'après (familier)\\l'année suivante (formel)} \\
        \hline
    \end{tabular}
    \caption{Vocabulaires pour se situer dans le temps}
\end{table}

\subsubsection{\textit{depuis}, \textit{pendant}, \textit{pour}, etc.}

En anglais, touts les mots comme « depuis », « pendant », et « pour » peuvent être traduits par « for »,
mais en français, ils sont utilisés différemment. On va comparer comment les utiliser pour référer une période de temps.

\begin{multicols}{2}
    \begin{frbox}[violet]{\textit{depuis}}
        On utilise \textit{depuis} pour indiquer une action :
        \begin{itemize}
            \item qui a commencé dans le passé et qui est encore en cours au moment présent
                (souvent avec le présent).

                \quad\textit{J'étudie le français depuis deux ans.}
            \item qui a commencé dans le passé et qui était encore en cours lors d'une autre action dans le passé se produisant
                (souvent avec l'imparfait).

                \quad\textit{Il parlait depuis 30 minutes quand je suis arrivé.}
        \end{itemize}
    \end{frbox}

    \begin{frbox}[violet]{\textit{pendant}}
        On utilise \textit{pendant} pour indiquer une action qui se produit dans le passé ou le futur sans relation avec le présent.

        \quad\textit{J'ai appris jouer au piano pendant 4 ans.}

        \quad\textit{Il parlera pendant 30 minutes.}
    \end{frbox}

    \begin{frbox}[violet]{\textit{pour}}
        On utilise \textit{pour} pour indiquer seulement une action dans le futur.

        \quad\textit{Je vais rester ici pour deux semaines.}
    \end{frbox}

    \begin{frbox}[violet]{\textit{en}}
        On utilise \textit{en} pour indiquer ou stresser la durée nécessaire pour accomplir une action.

        \quad\textit{J'ai fini mes devoirs en deux heures.}

        \quad\textit{Je mange en 20 minutes.}
    \end{frbox}

    \begin{frbox}[violet]{\textit{de \dots à \dots} / \textit{à partir de \dots jusau'à \dots}}
        On utilise \textit{de \dots à \dots} ou \textit{à partir de \dots jusau'à \dots} pour donner le début et la fin définis d'une action.

        \quad\textit{J'ai travaillé de 9h à 17h.}

        \quad\textit{Je visiterai la France à partir de juin jusqu'à août.}
    \end{frbox}

    \begin{frbox}[orange]{Astuce}
        Quand on utilise \textit{depuis} pour parler d'une action qui a commencé dans le passé et qui est encore en cours,
        on peut aussi utiliser les expressions \textit{Il y a \dots que}, \textit{Voilà \dots que}, ou \textit{Ça/Cela fait \dots que}.

        Ces deux phrases sont équivalentes :

        \quad\textit{J'étudie le français depuis deux ans.}

        \quad\textit{Il y a deux ans que j'étudie le français.}

        \textbf{Attention} : Ces expressions doivent se situer au début de la phrase.
    \end{frbox}
\end{multicols}

\subsection{Le participe présent et le gérondif}

\subsubsection{Le participe présent}

\begin{frbox}{Conjugaison du participe présent}
    {
        \centering
        \conjugelem{verbe} (1re personne pluriel (nous)) au présent
        $-$ \conjugelem{-ons} $+$ \conjugelem{-ant}\par
    }
    
    Les formations irrégulières du participe présent sont :

    \hfill avoir $\to$ ayant \hfill être $\to$ étant \hfill savoir $\to$ sachant \hfill $\hphantom{.}$
\end{frbox}

Exemples : 
\begin{multicols}{3}
    \begin{itemize}
        \item{} [\textit{parler}] : parlant
        \item{} [\textit{finir}] : finissant
        \item{} [\textit{vendre}] : vendant
    \end{itemize}
\end{multicols}

L'utilisation du participe présent est rare en français et le participe présent sert principalement de formation du gérondif.

\subsubsection{Le gérondif}

\begin{frbox}{Conjugaison du gérondif}
    {
        \centering
        en + \conjugelem{participe présent}\par
    }
\end{frbox}

Exembles :
\begin{itemize}
    \item{} [\textit{parler}] : Il ne faut pas manger \hl{\textbf{en parlant}}.
    \item{} [\textit{finir}] : Elle a réussi son examen \hl{\textbf{en finissant}} tous les exercices.
    \item{} [\textit{jeter}] : \hl{\textbf{En jetant}} moins de plastique, on peut protéger l'environnement.
\end{itemize}

\begin{frbox}[violet]{Utilisation du gérondif}
    On utilise le gérondif pour :
    \begin{itemize}
        \item Indiquer la manière ou le moyen par lequel une action est accomplie.
        
        \qquad\textit{Il a réussi son examen en étudiant tous les jours.}
        \item Exprimer une action simultanée à une autre action.
        
        \qquad\textit{Elle écoute de la musique en travaillant.}
    \end{itemize}
\end{frbox}

\subsection{Le subjonctif}

Le subjonctif n'est pas un temps, mais un mode.
On utilise le subjonctif pour exprimer des actions ou des états subjectifs, incertains, ou hypothétiques.
Ce semestre, on s'apercevra brièvement du présent du subjonctif.

\subsubsection{Le présent du subjonctif}

\begin{frbox}{Conjugaison du présent du subjonctif}
    \vspace*{-1em}
    \begin{table}[H]
        \centering
        \begin{tabular}{rl}
            \textbf{Personne} & \textbf{Formation} \\
            \hline
            je & \conjugelem{verbe} (3e personne pluriel (ils/elles)) au présent de l'indicatif $-$ \conjugelem{-ent} $+$ \conjugelem{-e} \\
            tu & \conjugelem{verbe} (3e personne pluriel (ils/elles)) au présent de l'indicatif $-$ \conjugelem{-ent} $+$ \conjugelem{-es} \\
            il/elle/on & \conjugelem{verbe} (3e personne pluriel (ils/elles)) au présent de l'indicatif $-$ \conjugelem{-ent} $+$ \conjugelem{-e} \\
            nous & \conjugelem{verbe} (1re personne pluriel (nous)) au imparfait de l'indicatif \\
            vous & \conjugelem{verbe} (2e personne pluriel (vous)) au imparfait de l'indicatif \\
            ils/elles & \conjugelem{verbe} (3e personne pluriel (ils/elles)) au présent de l'indicatif
        \end{tabular}
    \end{table}
    \vspace*{-1em}

    Il y a des formations irrégulières pour certains verbes courants, par exemple :
    \begin{table}[H]
        \centering
        \begin{tabular}{|r|llllll|}
            \hline
            \diagbox{\textbf{Verbe}}{\textbf{Personne}} & que je & que tu & qu'il/elle/on & que nous & que vous & qu'ils/elles \\
            \hline
            \conjugelem{être} & sois & sois & soit & soyons & soyez & soient \\
            \conjugelem{avoir} & aie & aies & ait & ayons & ayez & aient \\
            \hline
            \conjugelem{aller} & aille & ailles & aille & allions & alliez & aillent \\
            \conjugelem{faire} & fasse & fasses & fasse & fassions & fassiez & fassent \\
            \conjugelem{pouvoir} & puisse & puisses & puisse & puissions & puissiez & puissent \\
            \conjugelem{vouloir} & veuille & veuilles & veuille & voulions & vouliez & veuillent \\
            \conjugelem{savoir} & sache & saches & sache & sachions & sachiez & sachent \\
            \hline
            \conjugelem{falloir} & - & - & faille & - & - & - \\
            \conjugelem{pleuvoir} & - & - & pleuve & - & - & - \\
            \hline
        \end{tabular}
    \end{table}
\end{frbox}

\subsubsection{Utilisation du présent du subjonctif}

Il est utile de mémoriser ces expressions dans lesquelles le subjonctif est généralement utilisé :

\begin{frwarn}
    Le subjonctif présent est seulement utilisé si les deux subjets des deux clauses sont différents.
    Sinon, utilisez l'infinitif.
    Par exemple :
    \begin{itemize}
        \item Correct : Je voudrais qu'il \hl{\textbf{vienne}}.
        \item Incorrect : Je voudrais que je \hl{\textbf{vienne}}.
        
        $\to$ Corrigé : Je voudrais \hl{\textbf{venir}}.
    \end{itemize}
\end{frwarn}

\begin{multicols}{2}
    \begin{frbox}[violet]{Exprimer l'obligation et la nécessité}
        \begin{itemize}
            \item Il faut que \dots
            \item Il est nécessaire que \dots
            \item Il est interdit que \dots
            \item Il est important que \dots
            \item Il est essentiel que \dots
            \item Il est indispensable que \dots
        \end{itemize}
    \end{frbox}

    \begin{frbox}[violet]{Exprimer l'incertitude}
        Dans certaines expressions, on utilise l'indicatif dans la forme affirmative.
        Mais dans la forme négative ou interrogative, on utilise le subjonctif parce que l'action devient incertaine.
        \begin{itemize}
            \item Je pense que ($+$ indicatif) \dots $\to$ Je ne pense pas que/Pensez-vous que ($+$ subjonctif) \dots
            \item ne pas croire que \dots
            \item ne pas dire que \dots
            \item ne pas imaginer que \dots
            \item ne pas supposer que \dots
            \item ne pas être sûr(e)/certain(e) que \dots
            \item ne pas avoir l'impression que \dots
            \item ne pas savoir que \dots
            \item ne pas affirmer que \dots
            \item ne pas déclarer que \dots
            \item Il est possible que \dots
            \item Il se peut que \dots
            \item Il semble que \dots
        \end{itemize}
    \end{frbox}

    \columnbreak

    \begin{frbox}[violet]{Exprimer le souhait et le désir}
        \begin{itemize}
            \item Je voudrais que \dots
            \item J'aimerais que \dots
            \item Je souhaite que \dots
            \item Je préfère que \dots
            \item Je suis désolé(e) que \dots
            \item Je suis content(e) que \dots
            \item Je suis triste que \dots
            \item J'ai peur que \dots
            \item Je suis fâché(e) que \dots
            \item J'exige que \dots
            \item Il vaut mieux que \dots
            \item C'est bien que \dots
            \item Il est urgent que \dots
            \item Je recommande que \dots
            \item Je regrette que \dots
            \item Je comprends que \dots
        \end{itemize}
    \end{frbox}

    \begin{frbox}[violet]{Après certaines conjonctions}
        \begin{itemize}
            \item afin que/pour que \dots
            \item avant que \dots
            \item jusqu'à ce que \dots
            \item en attendant que \dots
            \item de peur/crainte que \dots
            \item bien que \dots
            \item à condition que/pourvu que \dots
            \item sans que \dots
            \item de façon/manière que \dots
            \item à moins que \dots
        \end{itemize}
    \end{frbox}
\end{multicols}

\begin{frwarn}
    L'expression « avant que » nécessite l'utilisation du subjonctif, alors que « après que » est suivie de l'indicatif.
\end{frwarn}

Dans ce semestre, il n'est pas nécessaire que vous puissiez utiliser les autres temps du subjonctif (passé, imparfait, plus-que-parfait).