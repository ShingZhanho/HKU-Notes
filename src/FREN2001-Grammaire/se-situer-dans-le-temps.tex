\section{Se situer dans le temps}

Dans ce cours, on continue à développer notre connaissance du passé composé du dernier semestre,
et on introduit deux nouveaux temps : l'imparfait et le plus-que-parfait. On va aussi apprendre
à combiner ces temps pour raconter des histoires et décrire des situations dans le passé.

\subsection{Les temps du passé}

\subsubsection{Le passé composé}

\begin{frbox}{Conjugaison du passé composé}
    La formation du passé composé est la suivante :

    {
        \centering
        \conjugelem{avoir} ou \conjugelem{être} au présent + \conjugelem{participe passé}\par
    }

    Généralement, on utilise \conjugelem{avoir} comme auxiliaire, sauf pour les 17 verbes, ou les
    verbes pronominaux, qui se conjuguent avec \conjugelem{être}.

    \textbf{Accord du participe passé} en nombre et en genre avec le sujet :
    \begin{itemize}
        \item avec \conjugelem{avoir} : \textbf{Jamais}, sauf si le complément d'objet direct (COD)
        précède le verbe.
        \item avec \conjugelem{être} : \textbf{Toujours},
        sauf si le complément d'objet indirect (COI) précède le verbe, ou
        si le complément d'objet direct (COD) suit le verbe.
    \end{itemize}
\end{frbox}

Par exemple :

\begin{itemize}
    \item{} [\textit{manger}, avec \conjugelem{avoir}, ne s'accorde pas] : J'\hl{\textbf{ai mangé}} une pomme.
    \item{} [\textit{aller}, avec \conjugelem{être}, s'accorde] : Elle \hl{\textbf{est allée}} au marché.
    \item{} [\textit{voir}, avec \conjugelem{avoir}, s'accorde car le COD « que » précède] : La lettre \hl{\textbf{que}} j'\hl{\textbf{ai vue}} était importante.
    \item{} [\textit{se laver}, verbe pronominal, ne s'accorde pas car le COD « les mains » suit] : Elle \hl{\textbf{s'est lavé}} les mains.
    \item{} [\textit{téléphoner}, avec \conjugelem{être}, ne s'accorde pas car le COI « se » précède] : Elles \hl{\textbf{se sont téléphoné}} hier.
\end{itemize}

Les 17 verbes sont
\begin{enumerate*}[label=(\arabic*)]
    \item aller \item arriver \item descendre* \item devenir \item entrer \item monter* \item mourir \item naître \item partir
    \item passer* \item rentrer \item rester \item retourner* \item revenir \item sortir* \item tomber \item venir
\end{enumerate*}

\begin{frwarn}
    Les verbes marqués d'un astérisque (*) peuvent aussi se conjuguer avec \conjugelem{avoir}
    lorsqu'ils sont utilisés de manière transitive, c'est-à-dire lorsqu'ils ont un complément
    d'objet direct (COD). Par exemple :
    \begin{itemize}
        \item Il \hl{\textbf{a retourné}} le livre à la bibliothèque.
    \end{itemize}
\end{frwarn}
\vspace*{-1.5em}
\begin{frbox}[violet]{Utilisation du passé composé}
    On utilise le passé composé pour :
    \begin{itemize}
        \item Décrire des actions complètes dans le passé.
        
        \qquad\textit{Hier, je suis allé(e) à l'université.}
        \item Indiquer des événements spécifiques qui se sont produits à un moment précis.
        
        \qquad\textit{Elle a fini ses devoirs il y a deux heures.}
        \item Raconter des événements dans une séquence chronologique.
        
        \qquad\textit{D'abord, nous avons visité le musée, puis nous avons déjeuné au café.}
    \end{itemize}
\end{frbox}

\subsubsection{L'imparfait}

\begin{frbox}{Conjugaison de l'imparfait}
    La formation de l'imparfait est la suivante :

    {
        \centering
        \conjugelem{verbe} (1\iere{} personne pluriel (nous)) au présent
        $-$ \conjugelem{-ons} $+$ les terminaisons de l'imparfait\par
    } 

    Les terminaisons de l'imparfait sont :
    \begin{table}[H]
        \centering
        \begin{tabular}{rcccccc}
            \textbf{Personne} & je & tu & il/elle/on & nous & vous & ils/elles \\
            \hline
            \textbf{Terminaison} & -ais & -ais & -ait & -ions & -iez & -aient
        \end{tabular}
    \end{table}

    \textbf{Exception :} \conjugelem{être} utilise la racine \conjugelem{ét-} :
    j'étais, tu étais, il/elle/on était, nous étions, vous étiez, ils/elles étaient.

    \textbf{Attention :} Si la racine du verbe se termine par \conjugelem{-g} ou \conjugelem{-c},
    on ajoute un \conjugelem{e} ou remplace le \conjugelem{c} par \conjugelem{ç} avant les terminaisons
    qui commencent par \conjugelem{a} ou \conjugelem{o} pour conserver la prononciation douce.
\end{frbox}

Le verbe \conjugelem{être} est la seule exception à la règle de formation de l'imparfait.

Par exemple :
\begin{itemize}
    \item{} [\textit{mange}, la racine termine par \conjugelem{-g}] : Il \hl{\textbf{mang{\color{red}e}ait}} un croissant tous les matins.
    \item{} [\textit{finir}] : Tu \hl{\textbf{finissais}} tes devoirs avant le dîner.
    \item{} [\textit{vendre}] : Ils \hl{\textbf{vendaient}} des fruits au marché.
    \item{} [\textit{commencer}, la racine termine par \conjugelem{-c}] : Les gens \hl{\textbf{commen{\color{red}ç}aient}} à travailler à huit heures.
    \item{} [\textit{être}] : J'\hl{\textbf{étais}} très heureux(se) quand j'étais enfant.
\end{itemize}

\begin{frbox}[violet]{Utilisation de l'imparfait}
    On utilise l'imparfait pour :
    \begin{itemize}
        \item Décrire des actions habituelles ou répétées dans le passé.
        
        \qquad\textit{Quand j'étais enfant, je jouais au parc tous les samedis.}
        \item Décrire des états ou des conditions dans le passé.
        
        \qquad\textit{Il faisait beau ce jour-là.}
        \item Fournir des descriptions de personnes, de lieux ou de situations dans le passé.
        
        \qquad\textit{La maison était grande et avait un jardin magnifique.}
    \end{itemize}
\end{frbox}

\subsubsection{Combiner le passé composé et l'imparfait}

Les deux temps sont souvent utilisés ensemble pour raconter des histoires ou décrire des situations
dans le passé.

\begin{frbox}[violet]{Utilisation combinée du passé composé et de l'imparfait}
    Pour raconter une action en cours dans le passé qui est interrompue par une acction spécifique,
    courte, ou soudain, on utilise l'imparfait pour l'action en cours et le passé composé pour l'action
    qui interrompt.
\end{frbox}

\subsubsection{Le plus-que-parfait}

\begin{frbox}{Conjugaison du plus-que-parfait}
    La formation du plus-que-parfait est la suivante :

    {
        \centering
        \conjugelem{avoir} ou \conjugelem{être} à l'imparfait + \conjugelem{participe passé}\par
    }

    Les règles d'accord du participe passé sont les mêmes que pour le passé composé.
\end{frbox}

Exemples :
\begin{itemize}
    \item{} [\textit{finir}, avec \conjugelem{avoir}] : J'\hl{\textbf{avais fini}} mes devoirs avant le dîner.
    \item{} [\textit{partir}, avec \conjugelem{être}] : Elle \hl{\textbf{était partie}} avant mon arrivée.
\end{itemize}

\begin{frbox}[violet]{Utilisation du plus-que-parfait}
    On utilise le plus-que-parfait pour :
    \begin{itemize}
        \item Indiquer une action qui s'est produite avant une autre action dans le passé.
        
        \qquad\textit{Il avait déjà mangé quand je suis arrivé(e).}
    \end{itemize}
\end{frbox}

\subsubsection{Vocabulaires utiles pour se situer dans le temps}

Quand on se réfère aux événements du passé, du présent ou du futur depuis différents
moments dans le temps, on utilise différents mots. Voici quelques vocabulaires utiles
pour indiquer le cadre temporel des événements.

\begin{table}[H]
    \centering
    \begin{tabular}{ll}
        \multicolumn{1}{c}{\textbf{Moment de parole : présent}} & \multicolumn{1}{|c}{\textbf{Moment de parole : passé ou futur}} \\
        \hline
        \multicolumn{2}{c}{\cellcolor{gray!25}Événements/Actions du passé} \\
        \multilinecell{hier} & \multilinecell{la veille\\le jour d'avant (familier)\\le jour précédent (formel)} \\
        \hline
        \multilinecell{avant-hier} & \multilinecell{l'avant-veille\\deux jours avant (familier)\\deux jours plus tôt/auparavant (formel)} \\
        \hline
        \multilinecell{il y a deux jours} & \multilinecell{deux jours avant (familier)\\deux jours plus tôt/auparavant (formel)} \\
        \hline
        \multilinecell{lundi dernier} & \multilinecell{le lundi d'avant (familier)\\le lundi précédent (formel)} \\

        \multicolumn{2}{c}{\cellcolor{gray!25}Événements/Actions du présent} \\
        \multilinecell{aujourd'hui} & \multilinecell{ce jour-là} \\
        \hline
        \multilinecell{maintenant\\à présent\\en ce moment} & \multilinecell{à ce moment-là } \\
        \hline
        \multilinecell{cette semaine} & \multilinecell{cette semaine-là} \\

        \multicolumn{2}{c}{\cellcolor{gray!25}Événements/Actions du futur} \\
        \multilinecell{demain} & \multilinecell{le lendemain\\le jour d'après (familier)\\le jour suivant (formel)} \\
        \hline
        \multilinecell{après-demain} & \multilinecell{le surlendemain\\deux jours après (familier)\\deux jours plus tard (formel)} \\
        \hline
        \multilinecell{l'année prochaine} & \multilinecell{l'année suivante\\l'année d'après (familier)\\l'année suivante (formel)} \\
        \hline
    \end{tabular}
    \caption{Vocabulaires pour se situer dans le temps}
\end{table}

\subsubsection{\textit{depuis}, \textit{pendant}, \textit{pour}, etc.}