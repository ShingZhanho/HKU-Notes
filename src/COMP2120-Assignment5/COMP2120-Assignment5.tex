% !TEX program = latexmk
% !TEX options = --shell-escape -synctex=1 -interaction=nonstopmode -file-line-error -pdf -cd -outdir=. ./COMP2120-Assignment5.tex

\documentclass[answers]{exam}
\usepackage[english]{babel}
\usepackage[style=apa,backend=biber]{biblatex}
\usepackage[
    format=plain,
    labelsep=newline,
    font=small,
    labelfont=bf,
    textfont=it,
    justification=justified,
    singlelinecheck=off
    ]{caption}
\usepackage{csquotes}
\usepackage{float}
\usepackage[top=2.5cm,bottom=2.5cm,left=3cm,right=3cm]{geometry}
\usepackage{graphicx}
\usepackage[hidelinks,colorlinks=true,linkcolor=blue,filecolor=blue,urlcolor=blue,citecolor=blue]{hyperref}
    \usepackage[nameinlink]{cleveref} %nameinlink ensures that the entire element is clickable in the pdf, not just the number
\usepackage{indentfirst}
\usepackage{multicol}
\usepackage[skip=1em,indent]{parskip}
\usepackage{tabularx}
\usepackage{times}
\pagestyle{foot}
\cfoot{Page \thepage\ of \numpages}
\shadedsolutions

\renewcommand{\thesubpart}{(\arabic{subpart})}
\renewcommand{\subpartlabel}{\thesubpart}

\begin{document}

\begin{center}
    \textbf
    {\Large{COMP2120 Computer Organisation} \\
    \large{24/25 Semester 2} \\
    \large{Assignment 5}}
\end{center}

\begin{questions}

    \question Consider a Serial Interface (e.g. Modem), containing a Control \& Status Register and
        two Buffer Registers , Input and Output Buffer Register, residing in memory location
        \texttt{SCSR}, \texttt{SBRI} and \texttt{SBRO},
        The \texttt{SCSR} has the following format:

    \begin{verbatim}
        Bit 0       =1 if Device Error
        Bit 1       =1 if Device Ready
        Bit 2       =0 if next operation is Write, 1, Read
        Bit 3-5     =000 if speed = 4800 bps
                    =001 if speed = 9600 bps
                    =010 if speed = 19200 bps
                    =011 if speed = 57600 bps
                    =100 if spped = 115200 bps
        Bit 6       =0 if odd parity, 1 if even parity
    \end{verbatim}

    Write an assembly program, using any instructions set (you may invent your own
    instructions) to output an array of 10 characters by Program I/O, to the serial port,
    using a speed of 115200 bps and even parity.
    
    To simplify the problem, you may assume that the array of characters is stored in
    memory location \texttt{LINE}, with one character in one word.
    
    Only source program is needed.

    \begin{solution}
        In the following program, hexadecimal decimals are supported by prefixing with ``\texttt{0x}''.
\begin{minted}{asm}
                .data
    base:       .word   0x0
    arrsize:    .word   0xa
                .text
    main:                                   ; entry point of the program
                ld      base, R1            ; R1 = 0, used as ctr
                ld      arrsize, R2         ; R2 = 10
                ld      0x1, R3             ; const int r3 = 1;
    dev_rdy:    ld      SCSR, R3
                and     R3, 0x2, R3         ; check if device ready
                bz      dev_rdy             ; branch to dev_rdy if R3=0 (bit 1 =0, not rdy)
    loop:       ld      LINE(R1), R4        ; load char from array
                st      R4, SBRO            ; R4 = LINE[R1] (displacement addressing mode)
                add     R3, R1, R1          ; R1++
                st      0x60, SCSR          ; set speed and parity
                st      R4, SBRO            ; output char to serial port
                sub     R2, R1, R5          ; R5 = R2 - R1
                bnz     loop                ; branch to loop if R5 != 0
                rtn
\end{minted}
    \end{solution}

    \pagebreak
    
    \question Given the data path of a CPU as in Assignment 4 with the modification that the
        \texttt{MBR} provides data to both \texttt{S1-Bus} and \texttt{S2-Bus}. Consider another instruction set,
        which allows memory operands, and the addressing mode information is stored in the
        same byte as the register operand. Describe the data transfer/transformation for the
        following 2-word instruction:
        \vspace*{-0.2cm}\begin{verbatim}
            ADD OFF(R1), R2, R3
        \end{verbatim}\vspace*{-0.6cm}
        which will get the first operand from memory whose address is given by \texttt{OFF+R1}
        (displacement addressing mode), add it to \texttt{R2} and put the result in \texttt{R3}. \texttt{OFF} is stored in
        the word following the instruction:
        \begin{center}
            \begin{tabular}{|c|c|c|c|}
                \hline
                \texttt{ADD} & \texttt{R1} (Disp mode) & \texttt{R2} & \texttt{R3} \\
                \hline 
                \multicolumn{4}{c}{} \\
                \hline
                \multicolumn{4}{|c|}{\texttt{OFF}} \\
                \hline
            \end{tabular}
        \end{center}

        \begin{solution}
\begin{verbatim}
MAR <- PC                               ; (IF)
PC <- PC + 4                            ; PC points to OFF
MBR <- mem[MAR]
IR <- MBR
...                                     ; (DI)
Read Register File for R1 at RFOUT1     ; (OF)
ALU_A <- RFOUT1
MAR <- PC                               ; MAR has address of 2nd word
PC <- PC + 4                            ; PC points to next instruction
MBR <- mem[MAR]                         ; MBR has value of 2nd word
ALU_B <- MBR
ALU_C <-(via ALU)- ALU_A + ALU_B        ; calculate R1 + OFF
MAR <- ALU_C
MBR <- mem[MAR]                         ; MBR has value at address R1 + OFF
ALU_A <- MBR
Read Register File for R2 at RFOUT2
ALU_B <- RFOUT2
ALU_C <-(via ALU)- ALU_A + ALU_B        ; calculate mem[R1+OFF] + (R2)
RFIN <- ALU_C
Write Register File to R3 with value in RFIN
\end{verbatim}
        \end{solution}
    
\end{questions}

\end{document}