\question Given the matrix
$A=\begin{pmatrix}
    3 & 2 \\
    -1 & 0
\end{pmatrix}$.

\begin{parts}
    \part Find itw two eigenpairs with the eigenvectors normalized to unit vectors;
    \begin{solution}
        \begin{align*}
            \begin{vmatrix}
                3-\lambda & 2 \\
                -1 & -\lambda
            \end{vmatrix} &= 0 \\
            (3-\lambda)(-\lambda) - (-2) &= 0 \\
            \lambda^2 - 3\lambda + 2 &= 0 \\
            (\lambda - 1)(\lambda - 2) &= 0 \\
            \lambda_1 &= 1, \lambda_2 = 2
        \end{align*}
        \begin{multicols}{2}
            For $\lambda_1 = 1$,
            \begin{align*}
                \begin{augmatrix}{2}{\left[}{\right]}
                    2 & 2 & 0 \\
                    -1 & -1 & 0
                \end{augmatrix}
                &\xrightarrow[R_2+\dfrac{1}{2}R_1]{\dfrac{1}{2}R_1}
                \begin{augmatrix}{2}{\left[}{\right]}
                    1 & 1 & 0 \\
                    0 & 0 & 0
                \end{augmatrix}
            \end{align*}
            Then, $x_2$ is free, and the eigenvector is
            \[
                \bvec{v_1} = x_2\begin{pmatrix}
                    -1 \\ 1
                \end{pmatrix}, x_2 \neq 0.
            \]
            For convenience, we pick $x_2 = 1$, and normalize the eigenvector:
            \[
                \bvec{\widehat{v}_1} = \frac{1}{\sqrt{2}}\begin{pmatrix}
                    -1 \\ 1
                \end{pmatrix}.
            \]
            \columnbreak

            For $\lambda_2 = 2$,
            \begin{align*}
                \begin{augmatrix}{2}{\left[}{\right]}   
                    1 & 2 & 0 \\
                    -1 & -2 & 0
                \end{augmatrix}
                &\xrightarrow{R_2 + R_1}
                \begin{augmatrix}{2}{\left[}{\right]}
                    1 & 2 & 0 \\
                    0 & 0 & 0
                \end{augmatrix}
            \end{align*}
            Then, $x_2$ is free, and the eigenvector is
            \[
                \bvec{v_2} = x_2\begin{pmatrix}
                    -2 \\ 1
                \end{pmatrix}, x_2 \neq 0.
            \]
            For convenience, we pick $x_2 = 1$, and normalize the eigenvector:
            \[
                \bvec{\widehat{v}_2} = \frac{1}{\sqrt{5}}\begin{pmatrix}
                    -2 \\ 1
                \end{pmatrix}.
            \]
            \columnbreak
        \end{multicols}

        Therefore, the two eigenpairs are
        \[
            \boxed{(\lambda_1, \bvec{\widehat{v}_1}) = \left(1, \frac{1}{\sqrt{2}}\begin{pmatrix}
                -1 \\ 1
            \end{pmatrix}\right), \quad
            (\lambda_2, \bvec{\widehat{v}_2}) = \left(2, \frac{1}{\sqrt{5}}\begin{pmatrix}
                -2 \\ 1
            \end{pmatrix}\right).}
        \]
    \end{solution}

    \part Hence, or otherwise, compute $A^{10}$;
    \begin{solution}
        For $D = \begin{pmatrix}
            \bvec{v_1} & \bvec{v_2}
        \end{pmatrix} = \begin{pmatrix}
            -1 & -2 \\
            1 & 1
        \end{pmatrix}$, and $D = \begin{pmatrix}
            \lambda_1 & 0 \\
            0 & \lambda_2
        \end{pmatrix} = \begin{pmatrix}
            1 & 0 \\
            0 & 2
        \end{pmatrix}$,
        by diagonalisation, we have
        \[
            A = VDV^{-1} = \begin{pmatrix}
                -1 & -2 \\
                1 & 1
            \end{pmatrix}
            \begin{pmatrix}
                1 & 0 \\
                0 & 2
            \end{pmatrix}
            \begin{pmatrix}
                -1 & -2 \\
                1 & 1
            \end{pmatrix}^{-1}.
        \]
        Then,
        \[
            A^{10} = VD^{10}V^{-1} = \begin{pmatrix}
                -1 & -2 \\
                1 & 1
            \end{pmatrix}
            \begin{pmatrix}
                1^{10} & 0 \\
                0 & 2^{10}
            \end{pmatrix}
            \begin{pmatrix}
                1 & 2 \\
                -1 & -1
            \end{pmatrix}
        \]
        Calculating, we get
        \[
            A^{10} = \boxed{\begin{pmatrix}
                2047 & 2046 \\
                -1023 & -1022
            \end{pmatrix}}.
        \]
    \end{solution}

    \part Express the dominant direction of $A^{100}\bvec{v}$ in a unit vector (where $\bvec{v}$ is a random vector).
    \begin{solution}
        Since $\left|\lambda_2\right| > \left|\lambda_1\right|$, the dominant direction of $A^{100}\bvec{v}$ 
        is along the eigenvector $\bvec{v_2}$. Therefore, the normalised dominant direction is
        \[
            \boxed{\begin{pmatrix}
                -\frac{2}{\sqrt{5}} \\[6pt] \frac{1}{\sqrt{5}}
            \end{pmatrix}}
        \]
    \end{solution}
\end{parts}