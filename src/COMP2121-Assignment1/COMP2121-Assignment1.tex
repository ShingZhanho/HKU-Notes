\documentclass[answers]{exam}
\usepackage[x11names]{xcolor}
\usepackage[left=1.5cm,right=1.5cm,top=1.5cm,bottom=2cm]{geometry}
\usepackage{newtxtext,newtxmath}
\usepackage{amsmath}
\usepackage{bm}
\usepackage{float}
\usepackage{multicol,multirow}
\usepackage{tabularx}
\usepackage[inline,shortlabels]{enumitem}
\usepackage{cancel}
\usepackage{venndiagram}
% for long multiplications
\makeatletter
    \providecommand\text\mbox
    \newenvironment{arithmetic}[1][]{\begin{tabular}[#1]{Al}}{\end{tabular}}
    \newcolumntype{A}{>{\bgroup\def~{\phantom{0}}$\@testOptor}r<{\@gobble\\$\egroup}}
    \def\@testOptor\ignorespaces#1#2\\{%
    \ifx#1\times
        \@OperatorRow{#1}{#2}\@tempa%
    \else\ifx#1+
        \@OperatorRow+{#2}\@tempa%
    \else\ifx#1\discretionary% detects the soft hyphen, \-
        \@ShortSubtractRow{#2}\@tempa%
    \else\ifx#1-
        \@OperatorRow-{#2}\@tempa%
    \else
        \@NormalRow{#1#2}\@tempa%
    \fi\fi\fi\fi
    \@tempa}
    \def\@OperatorRow#1#2#3{%
    \@IfEndRow#2\@gobble\\{%
        \def#3{\underline{{}#1 #2}\\}%
    }{%
        \def#3{\underline{{}#1 #2{}}}%
    }}

\def\@NormalRow#1#2{%
    \@IfEndRow#1\@gobble\\{%
        \def#2{#1\\}%
    }{%
        \def#2{#1{}}%
    }}

\def\@IfEndRow#1\@gobble#2\\#3#4{%
    \ifx#2\@gobble
        #4%
    \else
        #3%
    \fi}

\makeatother

\pagestyle{foot}
\cfoot{Page \thepage\ of \numpages}
\bracketedpoints
\renewcommand{\solutiontitle}{\noindent\textbf{Answers:}\par\noindent}

% Logic symbols
\newcommand{\lxor}{\oplus}
\newcommand{\limp}{\rightarrow}
\newcommand{\liff}{\leftrightarrow}
\newcommand{\ltrue}{\bm{T}}
\newcommand{\lfalse}{\bm{F}}

% Set symbols
\newcommand{\set}[1]{\left\{#1\right\}} % set braces
\newcommand{\card}[1]{\left|#1\right|} % cardinality of a set
\newcommand{\pwset}[1]{\mathcal{P}(#1)} % power set
\newcommand{\mtset}{\varnothing} % empty set
\renewcommand{\emptyset}{\varnothing} % empty set

\renewcommand{\thesubpart}{(\arabic{subpart})}
\renewcommand{\subpartlabel}{\thesubpart}

% commands for question 4b
\newcommand{\tnot}[1]{\texttt{NOT}\left(#1\right)}
\newcommand{\topr}[3]{\texttt{#1}\left(#2, #3\right)}

% psol (proof-solution) environment
\newenvironment{psol}{
    \renewcommand{\solutiontitle}{\noindent\textbf{Proof:}\par\noindent}
    \begin{solution}
}{
    \par\hfill\textbf{Q.E.D.}
    \end{solution}
    \renewcommand{\solutiontitle}{\noindent\textbf{Answers:}\par\noindent}
}

\newcommand{\ds}{\displaystyle}

\allowdisplaybreaks

\newcommand{\A}{\mathbb{A}}
\newcommand{\B}{\mathbb{B}}
\newcommand{\C}{\mathbb{C}}
\newcommand{\D}{\mathbb{D}}
\newcommand{\E}{\mathbb{E}}
\newcommand{\F}{\mathbb{F}}
\newcommand{\K}{\mathbb{K}}
\newcommand{\N}{\mathbb{N}}
\newcommand{\Q}{\mathbb{Q}}
\newcommand{\R}{\mathbb{R}}
\newcommand{\T}{\mathbb{T}}
\newcommand{\X}{\mathbb{X}}
\newcommand{\Y}{\mathbb{Y}}
\newcommand{\Z}{\mathbb{Z}}

\begin{document}

\begin{center}
    \textbf
    {\Large{COMP2121 Discrete Mathematics} \\
    \large{25/26 Semester 1} \\
    \large{Assignment 1}}\\
    SHING, Zhan Ho Jacob \qquad 3036228892
\end{center}

\textit{
    To distinguish, tautologies are represented as a bold T ($\ltrue$),
    and contradictions a bold F ($\lfalse$).
}

\begin{questions}
    
    \question English Statements and Logic

    \begin{parts}

        \part $(C \land \neg M) \limp \neg S$
        \begin{solution}
            The statement translates to ``If Lady Furina has clues and does not have motive,
            then she does not solve the case.''
        \end{solution}

        \part Rewrite sentence:
        \begin{solution}
            The required logical expression is $T \limp (S\land C)$.
        \end{solution}

        \part Verify if an argument is valid:
        \begin{solution}
            If we rewrite the argument in logical expressions, we have:
            \begin{enumerate}
                \item $S \limp C$ \label{q1c:stm1}
                \item $\neg C \limp \neg T$ \label{q1c:stm2}
                \item Therefore, from propositions \ref{q1c:stm1} and \ref{q1c:stm2}, $S \limp T$ \label{q1c:stm3}
            \end{enumerate}
            For the argument to be valid, we require the conclusion (Proposition \ref{q1c:stm3})
            is never false while all the premises (Propositions \ref{q1c:stm1} and \ref{q1c:stm2}) are true.
            It can be verified by trying to find a counterexample, i.e., combinations of $S$, $C$, and $T$
            such that the conclusion is $\lfalse$ while all its premises are $\ltrue$.

            For the conclusion to be $\lfalse$, the only possible case is when $S$ is $\ltrue$ and
            $T$ is $\lfalse$.

            For Proposition \ref{q1c:stm1} to be $\ltrue$ when $S$ is $\ltrue$, $C$ can only be
            $\ltrue$.
            
            When $C$ is $\ltrue$ and $T$ is $\lfalse$, Proposition \ref{q1c:stm2} is $\ltrue$.

            Therefore, when $S$ is $\ltrue$, $C$ is $\ltrue$, and $T$ is $\lfalse$, all premises
            are $\ltrue$ but the conclusion is $\lfalse$, i.e., the argument is \textbf{invalid}.
        \end{solution}

    \end{parts}

    \question A Logic Puzzle

    \begin{solution}
        Denote $Q_1:$ ``If I asked you (A) if B is the spy, would you say Ja?'',
        and $Q^A_1$ as the actual question ``Is B the spy?''. Define similarly for
        $Q_2:$ ``If I asked you if you (C) were the knave, would you say Ja?'',
        $Q^A_2:$ ``Are you (C) the knave?'',
        $Q_3:$ ``If I asked you (C) if A is the spy, would you say Ja?'',
        and $Q^A_3:$ ``Is A the spy?''.

        We solve the puzzle by the following tree of assumptions:
        \begin{itemize}
            \item \textbf{Assume ``Ja'' means ``Yes'' and ``Da'' means ``No''.}
            \begin{itemize}
                \item \textbf{Assume A is a knight.}
                
                When we ask $Q_1$ to A, A answers ``Yes''.
                Since A is a knight, he tells the truth, so when we further ask $Q^A_1$ to A,
                A would also answer ``Yes'', which means B is indeed the spy.
                
                Then, C must be the knave.
                
                Since C always lies, when we ask $Q_2$ to C,
                C answers ``Yes'', which is a lie, so when we further ask $Q^A_2$ to C,
                C would actually answer ``No'', which is also a lie, i.e., C is indeed
                the knave.
                
                Now, we ask $Q_3$ to C, and C answers ``No'', which is a lie,
                so when we further ask $Q^A_3$ to C, C would actually answer ``Yes'', which
                is also a lie, i.e., A is not the spy.
                
                \fbox{\textbf{This is a valid solution.}}
                We will further examine other combinations.

                \item \textbf{Assume A is a knave.}
                
                When we ask $Q_1$ to A, A answers ``Yes''. Since A is a knave, he lies,
                so when we further ask $Q^A_1$ to A, A would actually answer ``No'',
                which is also a lie, i.e., B is indeed the spy.
                
                Then, C must be a knight.

                Since C always tells the truth, when we ask $Q_2$ to C, C answers ``Yes'',
                so when we further ask $Q^A_2$ to C, C would also answer ``Yes'', i.e., C
                is the knave. This contradicts our assumption that C is a knight.

                \item \textbf{Assume A is a spy.}
                
                Since A is a spy, when we ask $Q_1$ to A, A can either tell a truth or lie.

                \begin{itemize}
                    \item \textbf{Assume A answering ``Yes'' to $\bm{Q_1}$ is a truth.}
                    
                    When we further ask $Q^A_1$ to A, A indeed answers ``Yes''. Note that
                    this can also have two possibilities: either A answering ``Yes'' to
                    $Q^A_1$ is a truth, or A answering ``Yes'' to $Q^A_1$ is a lie.

                    \begin{itemize}
                        \item \textbf{Assume A answering ``Yes'' to $\bm{Q^A_1}$ is a truth.}
                        
                        Then, B is indeed the spy, which contradicts our assumption that
                        A is the spy.

                        \item \textbf{Assume A answering ``Yes'' to $\bm{Q^A_1}$ is a lie.}
                        
                        Then, B is not the spy.

                        \begin{enumerate}
                            \item \textbf{Assume C is a knight.}
                            
                            Since C always tells the truth, when we ask $Q_2$ to C, C answers
                            ``Yes'', so when we further ask $Q^A_2$ to C, C would also
                            answer ``Yes'', i.e., C is the knave. This contradicts our
                            assumption that C is a knight.

                            \item \textbf{Assume C is a knave.}
                            
                            Since C always lies, when we ask $Q_2$ to C, C answers ``Yes'',
                            which is a lie, so when we further ask $Q^A_2$ to C, C would
                            actually answer ``No'', which is also a lie, i.e., C is indeed
                            the knave.

                            Then, we ask $Q_3$ to C, and C answers ``No'', which is a lie,
                            so when we further ask $Q^A_3$ to C, C would actually answer
                            ``Yes'', which is also a lie, i.e., A is not the spy. This,
                            again, contradicts our assumption that A is the spy.
                        \end{enumerate}
                    \end{itemize}
                    
                    \item \textbf{Assume A answering ``Yes'' to $\bm{Q_1}$ is a lie.}
                    
                    When we further ask $Q^A_1$ to A, A would actually answer ``No''.
                    Again, this can have two possibilities.

                    \begin{itemize}
                        \item \textbf{Assume A answering ``No'' to $\bm{Q^A_1}$ is a truth.}
                        
                        Then, B is not the spy.

                        \begin{enumerate}
                            \item \textbf{Assume C is a knight.}
                            
                            When we ask $Q_2$ to C, C answers ``Yes'', so when we further
                            ask $Q^A_2$ to C, C would indeed answer ``Yes'', i.e., C is the
                            knave. This contradicts our assumption that C is a knight.

                            \item \textbf{Assume C is a knave.}
                            
                            When we ask $Q_2$ to C, C answers ``Yes'', which is a lie,
                            so when we further ask $Q^A_2$ to C, C would actually answer
                            ``No'', which is also a lie, i.e., C is indeed the knave.

                            Then, we ask $Q_3$ to C, and C answers ``No'', which is a lie,
                            so when we further ask $Q^A_3$ to C, C would actually answer
                            ``Yes'', which is also a lie, i.e., A is not the spy. This,
                            again, contradicts our assumption that A is the spy.
                        \end{enumerate}

                        \item \textbf{Assume A answering ``No'' to $\bm{Q^A_1}$ is a lie.}
                        
                        Then, B is indeed the spy. This cannot hold as A is the spy.
                    \end{itemize}
                \end{itemize}
            \end{itemize}

            \item \textbf{Assume ``Ja'' means ``No'' and ``Da'' means ``Yes''.}
            
            \begin{itemize}
                \item \textbf{Assume A is a knight.}
                
                When we ask $Q_1$ to A, A answers ``No'', so when we further ask $Q^A_1$ to
                A, A would say ``Yes'', i.e., B is the spy.
                
                Then, C must be the knave.

                When we ask $Q_2$ to C, C answers ``No'', which is a lie, i.e., C would 
                indeed answer ``Yes'' to $Q^A_2$, which is also a lie, i.e., C is not the
                knave. This contradicts our assumption that C is the knave.

                \item \textbf{Assume A is a knave.}
                
                When we ask $Q_1$ to A, A answers ``No'', which is a lie, so when we further
                ask $Q^A_1$ to A, A would actually answer ``Yes'', which is also a lie, i.e.,
                B is not the spy.

                Then, since B is not the spy, and A is already a knave, B must be the knight,
                and C must be the spy.

                When we ask $Q_2$ to C, C answers ``No'', which can be either a truth or a lie.

                \begin{itemize}
                    \item \textbf{Assume C answering ``No'' to $\bm{Q_2}$ is a truth.}
                    
                    Then, when we ask $Q^A_2$ to C, C would answer ``Yes'', which can also be
                    either a truth or a lie, i.e., C is the knave or C is not the knave.

                    When C answering ``Yes'' to $Q^A_2$ is a lie, then C is not the knave,
                    which checks as we deduced that C is the spy.

                    Next, we ask $Q_3$ to C, and C answers ``Yes'', which can be either a truth
                    or a lie.

                    When C answering ``Yes'' to $Q_3$ is a truth, then when we further ask $Q^A_3$
                    to C, C would answer ``No'', i.e., A can either be the spy (when C is lying)
                    or A is not the spy (when C is telling the truth).

                    When C was indeed telling the truth when answering ``No'' to $Q^A_3$, then A
                    is not the spy, which checks as A is a knave.

                    We have shown that it is possible that when ``Ja'' means ``No'' and ``Da''
                    means ``Yes'', A is a knave, B is a knight, and C is a spy.
                    \fbox{\textbf{This is another valid solution.}}
                \end{itemize}
            \end{itemize}
        \end{itemize}

        At this point, we have shown that there are at least two valid solutions:
        \begin{center}
            \begin{tabular}{cc|ccc}
                \textbf{``Ja'' means} & \textbf{``Da'' means} & \textbf{A} & \textbf{B} & \textbf{C} \\
                \hline
                Yes & No & Knight & Spy & Knave \\
                No & Yes & Knave & Knight & Spy
            \end{tabular}
        \end{center}
        It is not useful to continue discussing other cases, as, without extra information,
        it is impossible to determine with certainty the roles of A, B, and C.

        Final answer: \fbox{\textbf{There does not exist a unique solution that can determine the
        roles of A, B, and C with certainty.}}
    \end{solution}

    \question Logical Operators

    \begin{parts}
        
        \part Number of possible 2-to-1 logical operators for inputs $A, B\in\{0,1\}$
        and output $C\in \{0,1\}$.
        \begin{solution}
            Note that for two inputs $A$ and $B$, there are four possible combinations,
            i.e., $(A, B) \in \{(0,0), (0, 1), (1, 0), (1, 1)\}$. A logical operator
            is a function $f$ that maps the combinations of inputs to an output, i.e.,
            $f: \{0, 1\}^2 \to \{0, 1\}$. Consider the truth table of $f$:
            \begin{center}
                \begin{tabular}{ccc}
                    $\bm{A}$ & $\bm{B}$ & $\bm{C = f(A, B)}$ \\
                    \hline
                    0 & 0 & $f(0, 0)$ \\
                    0 & 1 & $f(0, 1)$ \\
                    1 & 0 & $f(1, 0)$ \\
                    1 & 1 & $f(1, 1)$ 
                \end{tabular}
            \end{center}
            Two logical operators $f$ and $g$ are said to be the same if and only if
            for all combinations of inputs, $f(A, B) = g(A, B)$. Therefore, the problem
            reduces to counting the number of different ways to fill the truth table,
            which is given by $2^4 = 16$.

            Therefore, there are \fbox{\textbf{16}} possible 2-to-1 logical operators.
        \end{solution}

        \part Implement \texttt{NOT}, \texttt{AND}, \texttt{OR}, \texttt{XOR}, and
        \texttt{IMPLIES} using only the logical operator \texttt{NAND}. 
        \begin{solution}
            \begin{itemize}
                \item \textbf{\texttt{NOT} operator}
                
                Note that $\texttt{NAND}(A, A) \equiv \neg (A \land A) \equiv \neg A \lor \neg A \equiv \neg A$.
                
                Therefore, the \texttt{NOT} operator is implemented as $\boxed{\texttt{NOT}(A) = \texttt{NAND}(A, A)}$.

                \item \textbf{\texttt{OR} operator}
                
                Note that $\neg(\neg A \land \neg B) \equiv A \lor B$,
                which means $\texttt{OR}(A, B)=\texttt{NAND}(\texttt{NOT}(A), \texttt{NOT}(B))$.
                
                Further expand the $\texttt{NOT}$ operators as implemented before, we have:
                $\boxed{\texttt{OR}(A, B)=\texttt{NAND}\left[\texttt{NAND}(A, A), \texttt{NAND}(B,B)\right]}$.

                \item \textbf{\texttt{AND} operator}
                
                Note that if we apply \texttt{NAND} on the result of $\texttt{NAND}(A, B)$, we have:
                $$\neg\left[\neg\left(A\land B\right)\land\neg\left(A\land B\right)\right]
                \equiv \left(A\land B\right)\lor\left(A \land B\right)
                \equiv A \land \left(B \lor B\right)
                \equiv A \land B$$

                Therefore, the \texttt{AND} operator is implemented as
                \fbox{$\topr{AND}{A}{B}=\topr{NAND}{\topr{NAND}{A}{B}}{\topr{NAND}{A}{B}}$}.

                \item \textbf{\texttt{XOR} operator}
                
                Recall that $A\lxor B
                \equiv (A\lor B)\land\neg(A\land B)$. From this expression, we continue to derive:
                \begin{align*}
                    (A\lor B)\land\neg(A\land B)
                    &\equiv (A\land\neg(A\land B))\lor(B\land\neg(A\land B)) \qquad\text{(Distributive Law)}\\
                    &\equiv \neg[\neg(A\land\neg(A\land B))\land\neg(B\land\neg(A\land B))] \qquad\text{(De Morgan's Law)}\\
                    &\equiv \topr{NAND}{
                        \topr{NAND}{A}{\topr{NAND}{A}{B}}
                    }{
                        \topr{NAND}{B}{\topr{NAND}{A}{B}}
                    }
                \end{align*}

                Therefore, \texttt{XOR} is implemented as \fbox{$\topr{XOR}{A}{B}
                =\topr{NAND}{
                        \topr{NAND}{A}{\topr{NAND}{A}{B}}
                    }{
                        \topr{NAND}{B}{\topr{NAND}{A}{B}}
                    }$}.

                \item \textbf{\texttt{IMPLIES} operator}
                
                By logical equivalence, we have $A \limp B \equiv \neg A \lor B$.

                Expand the expression, we have:
                \begin{align*}
                    \neg A\lor B
                    &\equiv \topr{OR}{\tnot{A}}{B} \\
                    &\equiv \topr{NAND}{A}{\tnot{B}} \\
                    &\equiv \topr{NAND}{A}{\topr{NAND}{B}{B}}
                \end{align*}

                Therefore, \texttt{IMPLIES} is as \fbox{$
                \topr{IMPLIES}{A}{B} = 
                \topr{NAND}{A}{\topr{NAND}{B}{B}}
                $}.
            \end{itemize}
        \end{solution}
    \end{parts}

    \question Quantifiers and Predicates

    \begin{parts}
        
        \part Rewrite expression
        $\neg(\exists y\neg(\forall x(P(x)\land Q(y)))\limp\exists z R(z))$.
        \begin{solution}
            \begin{align*}
                \neg[\exists y\neg\{\forall x[P(x)\land Q(y)]\}\limp\exists zR(z)]
                &\equiv \neg[\exists y{\exists x[\neg P(x)\lor\neg Q(y)]}\limp\exists zR(z)]\\
                &\equiv \neg[\neg\{\exists x\exists y[\neg P(x)\lor\neg Q(y)]\}\lor\exists zR(z)]\\
                &\equiv \neg[\forall x\forall y[P(x)\land Q(y)]\lor\exists zR(z)]\\
                &\equiv \neg[(\forall xP(x)\land\forall yQ(y))\lor\exists zR(z)]\\
                &\equiv \neg(\forall xP(x)\land\forall yQ(y))\land\forall z\neg R(z)\\
                &\equiv \boxed{[\exists x\neg P(x)\lor\exists y\neg Q(y)]\land\forall z\neg R(z)}
            \end{align*}
        \end{solution}

        \part Prove that $\forall xP(x)\land\exists xQ(x) \equiv \forall x \exists y (P(x)\land Q(y))$.
        \begin{psol}
            Rename the variable on the left-hand side, we have:
            $$\forall xP(x)\land\exists yQ(y) \equiv \forall x\exists y(P(x)\land Q(y))$$
            To prove equivalence, we prove both the sufficiency and necessity.
            \begin{enumerate}
                \item \textbf{Sufficiency:} $\forall xP(x)\land\exists yQ(y) \Rightarrow \forall x\exists y(P(x)\land Q(y))$
                
                Assume that the L.H.S. is true, then for all $x$, $P(x)$ must hold.
                Also, there must exist at least one $y$, say $y_0$, such that $Q(y_0)$ holds.
                Therefore, for all $x$, we can always find a $y$, which is $y_0$, such that
                $P(x)\land Q(y)$ holds. We have displayed that the R.H.S. must be true if
                the L.H.S. is true.

                \item \textbf{Necessity:} $\forall xP(x)\land\exists yQ(y) \Leftarrow \forall x\exists y(P(x)\land Q(y))$
                
                Assume that the R.H.S. (i.e., $\forall x\exists y(P(x)\land Q(y))$) is true,
                then for all $x$, we can always find a $y$, say $y_0$, such that $P(x)\land Q(y_0)$
                holds. Therefore, for all $x$, $P(x)$ must hold, and there must exist at least one $y$,
                which is $y_0$, such that $Q(y_0)$ holds. We have displayed that the L.H.S. must be true if
                the R.H.S. is true.
            \end{enumerate}
            Since both the sufficiency and necessity have been proved, we conclude that
            $\forall xP(x)\land\exists xQ(x) \liff \forall x \exists y (P(x)\land Q(y))$
            is a tautology, i.e., the two statements are logically equivalent.
        \end{psol}

        \part Justify whether $\exists x \forall y P(x, y) \limp \forall y \exists x P(x, y)$
        is always a tautology for any predicate $P(x, y)$.
        \begin{solution}
            We propose that the given proposition is not always a tautology, and we attempt
            to find an example, such that the proposition is a contradiction.

            $\exists x \forall y P(x,y) \limp \forall y \exists x P(x,y)$ is a contradiction
            iff $\exists x \forall y P(x,y)$ is $\ltrue$ and $\forall y \exists x P(x,y)$ is
            $\lfalse$.

            If the antecedent is true, then, for at least one $x$, say $x_0$, $P(x_0, y)$
            holds for all $y$.

            If the consequent is false, then, for at least one $y$, say $y_0$, there does not
            exist any $x$ such that $P(x, y_0)$ holds.

            However, if $P(x_0, y)$ holds for all $y$, then $P(x_0, y_0)$ must also hold.
            Therefore, such $y_0$ does not exist, and the consequent can never be false if
            the antecedent is true.

            Therefore, we cannot find a counterexample such that the antecedent is true
            while the consequent is false. Hence, the given proposition is always a tautology.
        \end{solution}

    \end{parts}

    \question Proofs

    \begin{parts}
        
        \part Determine the validity of a proof.
        \begin{solution}
            Step 1: Valid. This is implication law ($P \limp Q \equiv \neg P \lor Q$).

            Step 2: Valid. Renaming the variable $x$ to $y$ does not change the meaning of
            the statement. This is also the distribution law of $\forall$ over $\land$ --
            $\forall x P(x) \land \forall y Q(y) \equiv \forall x (P(x) \land Q(x))$.

            Step 3: Valid. This step combines application of several laws:
            \begin{align*}
                &\equiv \forall x[(P(x)\land\neg Q(x)) \lor (Q(x)\land\neg Q(x))] \qquad\text{(Distribution law)}\\
                &\equiv \forall x[(P(x)\land\neg Q(x)) \lor \lfalse] \qquad\text{(Contradiction law)}\\
                &\equiv \forall x[P(x)\land\neg Q(x)] \qquad\text{(Identity law)}
            \end{align*}

            Step 4: Valid. This is the simplification (i.e., $P\land Q \limp P$).

            Step 5: Valid. First, since $\forall x P(x)$ is true, then $P(c)$ must be true
            for any $c$ in the universe of dicourse. Then, since there exists at least one
            $x$, $c$ in this case, such that $P(x)$ is true, then $\exists x P(x)$ is true.
        \end{solution}

        \part Proof by Induction.
        \begin{psol}
            Denote $P(n) : \ds{\sum_{k=1}^n \cos(kx) = \frac{1}{2}\left(\frac{\sin\left[\left(n+\frac{1}{2}\right)x\right]}{\sin\left(\frac{1}{2}x\right)}-1\right)}$.

            \textbf{Base Case:} $n=1$.
            \begin{equation*}
                \text{L.H.S.} = \sum_{k=1}^1 \cos(kx) = \cos(x)
            \end{equation*}
            \begin{align*}
                \text{R.H.S.} &= \frac{1}{2}\left(\frac{\sin\frac{3}{2}x}{\sin\frac{1}{2}x}-1\right)\\
                &= \frac{1}{2} \left(\frac{\sin\frac{3}{2}x-\sin\frac{1}{2}x}{\sin\frac{1}{2}x}\right)\\
                &= \frac{1}{2} \left(\frac{2\cos x\sin\frac{1}{2}x}{\sin\frac{1}{2}x}\right)\\
                &= \cos x \\
                &= \text{L.H.S.}
            \end{align*}
            Therefore, $P(1)$ holds.

            \textbf{Inductive Step:} Assume that $P(n)$ holds for $n\geq 1$. We show that $P(n+1)$ also holds.
            \begin{align*}
                \text{L.H.S.} &= \sum_{k=1}^{n+1} \cos(kx) \\
                &= \sum_{k=1}^n \cos(kx) + \cos\left((n+1)x\right) \\
                &= P(n) + \cos\left((n+1)x\right) \\
                &= \frac{1}{2}\left(\frac{\sin\left[\left(n+\frac{1}{2}\right)x\right]}{\sin\frac{1}{2}x}-1\right) + \cos\left((n+1)x\right) \\
                &= \frac{1}{2}\left(\frac{\sin\left[\left(n+\frac{1}{2}\right)x\right]-\sin\frac{1}{2}x}{\sin\frac{1}{2}x}\right) + \cos\left((n+1)x\right) \\
                &= \frac{1}{2}\left(\frac{2\cos\left[\frac{n+1}{2}x\right]\sin\left[\frac{n}{2}x\right]}{\sin\frac{1}{2}x}\right) + \cos\left((n+1)x\right) \\
                &= \frac{\cos\left[\frac{n+1}{2}x\right]\sin\left[\frac{n}{2}x\right] + \cos\left((n+1)x\right)\sin\frac{1}{2}x}{\sin\frac{1}{2}x} \\
                &= \frac{\sin\left[\frac{2n+1}{2}x\right]-\sin\frac{1}{2}x+\sin\left[\frac{2n+3}{2}x\right]-\sin\left[\frac{2n+1}{2}x\right]}{2\sin\frac{1}{2}x} \\
                &= \frac{1}{2}\left(\frac{\sin\left[\frac{2n+3}{2}x\right]}{\sin\frac{1}{2}x}-1\right) \\
                &= \frac{1}{2}\left(\frac{\sin\left[\left(\left(n+1\right)+\frac{1}{2}\right)x\right]}{\sin\frac{1}{2}x}-1\right) \\
                &= \text{R.H.S.}
            \end{align*}
            Therefore, $P(n) \Rightarrow P(n+1)$.

            By the principle of mathematical induction, $P(n)$ holds for all $n\geq 1$.
        \end{psol}

    \end{parts}

    \question Basics on Sets

    \begin{parts}
        
        \part Cardinality.
        \begin{solution}
            \begin{enumerate}[label=(\roman*)]
                \item $\card{A} = \boxed{4}$
                \item $\card{\pwset{B}} = 2^{\card{B}} = 2^4 = \boxed{16}$,
                provided that the set $B$ has two integers, one empty set, and one proposition.
                \item Note that $C = [0, 10]$. Therefore, $\card{C} = \boxed{11}$.
                \item Assuming that capital and small letters are the same, then
                $\card{D} = 26-7 =\boxed{19}$.
            \end{enumerate}
        \end{solution}

        \part Set Operations.
        \begin{solution}
            \begin{enumerate}[label=(\roman*)]
                \item $A \cap B = \set{2, \mtset} \Rightarrow \card{A\cap B} = \boxed{2}$.
                \item Consider:\begin{align*}
                    B \cup (C\cap\mtset) - A 
                    &= B\cup\mtset - A \\
                    &= B-A \\
                    &= \set{1, 2, \mtset, |x|=2} - \set{\set{0, 1}, 2, 3, \mtset} \\
                    &= \set{1, |x|=2} \\
                    &\Rightarrow \card{B \cup (C\cap\mtset) - A} = \boxed{2}
                \end{align*}
                \item Consider:\begin{align*}
                    (A\cap B)\times(B\cap C)
                    &= \set{2, \mtset} \times \set{1, 2} \\
                    &= \set{(2, 1), (2, 2), (\mtset, 1), (\mtset, 2)} \\
                    &\Rightarrow \card{(A\cap B)\times(B\cap C)} = \boxed{4}
                \end{align*}
            \end{enumerate}
        \end{solution}

    \end{parts}

    \question Venn Diagrams

    \begin{parts}
        
        \part Set in Figure 2.
        \begin{solution}
            The set represented in Figure 2 is $(A\cup B)\cap C$.
        \end{solution}

        \part Draw Venn diagram for $(A\cup B) - (\overline{A\cap C})$.
        \begin{solution}
            \begin{center}
                \begin{venndiagram3sets}[labelNotABC=$U$,shade=cyan!40]
                    \fillACapC
                \end{venndiagram3sets}
            \end{center}
        \end{solution}

    \end{parts}

    \question Set Theory and Logic

    \begin{parts}
        
        \part Determine whether $\forall\text{ sets }A,B,C,D:(A\cap B)\times(C\cap D)
        =(A\times C)\cap(B\times D)$.
        \begin{solution}
            To show that the two sets are equal, we show that whenever an element
            is in the L.H.S. set, it must also be in the R.H.S. set, and vice versa.
            
            Recall that $x\in (A\cap B) \equiv (x\in A) \land (x\in B)$
            and $(x, y) \in (A\times B) \equiv (x\in A) \land (y\in B)$.
            Therefore, for L.H.S., we have:
            \begin{align*}
                (x,y)\in(A\cap B)\times(C\cap D)
                &\equiv (x\in A\cap B)\land(y\in C\cap D) \\
                &\equiv (x\in A\land x\in B) \land (y \in C \land y \in D) \\
                &\equiv x\in A \land y \in C \land x \in B \land y \in D \qquad\text{(Associative Law)}\\
                &\equiv (x\in A \land y \in C) \land (x \in B \land y \in D) \\
                &\equiv (x,y)\in(A\times C) \land (x,y)\in(B\times D) \qquad\text{(Definition of Cartesian Product)}\\
                &\equiv (x,y)\in(A\times C)\cap(B\times D)
            \end{align*}
            Therefore, whenever we have $(x,y)\in(A\cap B)\times(C\cap D)$, we must also have
            $(x,y)\in(A\times C)\cap(B\times D)$, i.e., the two sets are equal, and the given
            proposition holds for all sets $A, B, C, D$.
        \end{solution}

        \part Prove $\forall\text{ arbitrary sets }A,B,C: (A\cup B)-C = (A-C)\cup(B-C)$.
        \begin{psol}
            To proof the equality of two sets, we show that whenever an element
            is in the L.H.S. set, it must also be in the R.H.S. set, and vice versa.

            Recall the definitions that
            $x\in(A\cup B) \equiv (x\in A) \lor (x\in B)$ and
            $x\in(A-B) \equiv (x\in A) \land \neg(x\in B)$. Then, for L.H.S., we have
            \begin{align*}
                x\in\left[(A\cup B)-C\right]
                &\equiv x\in(A\cup B)\land \neg(x\in C) \\
                &\equiv (x\in A \lor x\in B) \land (x\notin C)\\
                &\equiv (x\in A \land x\notin C) \lor (x\in B \land x\notin C)\\
                &\equiv [x\in(A-C)] \lor [x\in(B-C)]\\
                &\equiv x\in[(A-C)\cup (B-C)]
            \end{align*}
            Hence, we have shown that the sets on L.H.S. and R.H.S. are equal. Since $A,B,$
            and $C$ are chosen to be arbitrary, therefore the proposition also holds for any
            arbitrary sets.
        \end{psol}

    \end{parts}

    \question Relations

    \begin{parts}
        
        \part For the relation $R_1=\{(x, y)\in\Z\times\Z\mid x+2y\text{ is an even number}\}$,
        prove or disprove that $R_1$ is

        \begin{subparts}

            \subpart reflexive
            \begin{psol}
                To verify reflexivity, we need to check whether $xR_1x$, i.e., whether
                $x+2x=3x$ is an even number. Note that any arbitrary even number can be
                expressed in the form of $2n$ for some $n\in\Z$.

                \textbf{Case 1:} when $x$ is an even number, i.e., $x=2n$.
                Then, we have $3x=3(2n)=6n=2(3n)$, which is an even number. Therefore,
                $xR_1x$ holds when $x$ is an even number.

                \textbf{Case 2:} when $x$ is an odd number, i.e., $x=2n+1$.
                Then, we have $3x=3(2n+1)=6n+3=2(3n+1)+1$, which is an odd number.
                Therefore, $x{R_1}x$ does not hold when $x$ is an odd number.

                Having exhausted all the cases and shown that $\exists x:(x,x)\notin R_1$,
                we conclude that \fbox{$R_1$ is not reflexive}.
            \end{psol}

            \subpart symmetric
            \begin{psol}
                To verify symmetry, we need to check whether $yR_1x$ whenever $yR_1x$ for all
                $x,y\in\Z$.

                Let $x$ be an arbitrary even number, i.e., $x=2n$ for some $n\in\Z$,
                and $y$ be an arbitrary odd number, i.e., $y=2m+1$ for some $m\in\Z$.
                
                Then, for $xR_1y$, we have $x+2y=2n+2(2m+1)=2(n+2m+1)$, which is an even number,
                so $xR_1y$ holds.

                However, for $yR_1x$, we have $y+2x=(2m+1)+2(2n)=2(m+2n)+1$, which is an odd number,
                so $yR_1x$ does not hold.

                Having found a counterexample such that $(x,y)\in R_1$ but $(y, x)\in R_1$,
                we conclude that \fbox{$R_1$ is not symmetric}.
            \end{psol}

            \subpart transitive
            \begin{psol}
                To verify transitivity, we need to check whether $xR_1z$ whenever
                $xR_1y$ and $yR_1z$ for all $x,y,z\in\Z$.
                
                We can show this by contradiction. Assume that $xR_1y$ and $yR_1z$ hold,
                but $xR_1z$ does not hold.
                It implies that $x+2z$ is an odd number. Note that since $2z$ must be even,
                then $x$ must be an odd number. For $xR_1y$ to hold, $x+2y$ must be even,
                and when $x$ is odd, $2y$ must also be odd, which is impossible as $2y$ must
                be even for any integer $y$.

                Therefore, our assumption is wrong, and we have shown that whenever $xR_1y$
                and $yR_1z$ hold, $xR_1z$ must also hold. Hence, we conclude that
                \fbox{$R_1$ is transitive}.
            \end{psol}

        \end{subparts}

        \part Determine correctness of a statement.
        \begin{solution}
            The statement is not correct.

            The logical fallacy is at the statement ``Take an arbitrary $b$ such that $aRb$.''
            The statement assumes that such $b$ exists, which is not necessarily true.
            Symmetry only ensures that if $aRb$ holds, then $bRa$ must also hold, but it does not
            ensure that $aRb$ alone must hold.
        \end{solution}

    \end{parts}

\end{questions}

\end{document}