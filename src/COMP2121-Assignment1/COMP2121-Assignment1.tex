\documentclass[answers]{exam}
\usepackage[x11names]{xcolor}
\usepackage[left=1.5cm,right=1.5cm,top=1.5cm,bottom=2cm]{geometry}
\usepackage{newtxtext,newtxmath}
\usepackage{amsmath}
\usepackage{bm}
\usepackage{float}
\usepackage{multicol,multirow}
\usepackage{tabularx}
\usepackage[inline,shortlabels]{enumitem}
\usepackage{cancel}
\usepackage{venndiagram}
% for long multiplications
\makeatletter
    \providecommand\text\mbox
    \newenvironment{arithmetic}[1][]{\begin{tabular}[#1]{Al}}{\end{tabular}}
    \newcolumntype{A}{>{\bgroup\def~{\phantom{0}}$\@testOptor}r<{\@gobble\\$\egroup}}
    \def\@testOptor\ignorespaces#1#2\\{%
    \ifx#1\times
        \@OperatorRow{#1}{#2}\@tempa%
    \else\ifx#1+
        \@OperatorRow+{#2}\@tempa%
    \else\ifx#1\discretionary% detects the soft hyphen, \-
        \@ShortSubtractRow{#2}\@tempa%
    \else\ifx#1-
        \@OperatorRow-{#2}\@tempa%
    \else
        \@NormalRow{#1#2}\@tempa%
    \fi\fi\fi\fi
    \@tempa}
    \def\@OperatorRow#1#2#3{%
    \@IfEndRow#2\@gobble\\{%
        \def#3{\underline{{}#1 #2}\\}%
    }{%
        \def#3{\underline{{}#1 #2{}}}%
    }}

\def\@NormalRow#1#2{%
    \@IfEndRow#1\@gobble\\{%
        \def#2{#1\\}%
    }{%
        \def#2{#1{}}%
    }}

\def\@IfEndRow#1\@gobble#2\\#3#4{%
    \ifx#2\@gobble
        #4%
    \else
        #3%
    \fi}

\makeatother

\pagestyle{foot}
\cfoot{Page \thepage\ of \numpages}
\bracketedpoints
\renewcommand{\solutiontitle}{\noindent\textbf{Answers:}\par\noindent}

% Logic symbols
\newcommand{\lxor}{\oplus}
\newcommand{\limp}{\rightarrow}
\newcommand{\liff}{\leftrightarrow}
\newcommand{\ltrue}{\bm{T}}
\newcommand{\lfalse}{\bm{F}}

% Set symbols
\newcommand{\set}[1]{\left\{#1\right\}} % set braces
\newcommand{\card}[1]{\left|#1\right|} % cardinality of a set
\newcommand{\pwset}[1]{\mathcal{P}(#1)} % power set
\newcommand{\mtset}{\varnothing} % empty set
\renewcommand{\emptyset}{\varnothing} % empty set

\renewcommand{\thesubpart}{(\arabic{subpart})}
\renewcommand{\subpartlabel}{\thesubpart}

% commands for question 4b
\newcommand{\tnot}[1]{\texttt{NOT}\left(#1\right)}
\newcommand{\topr}[3]{\texttt{#1}\left(#2, #3\right)}

% psol (proof-solution) environment
\newenvironment{psol}{
    \renewcommand{\solutiontitle}{\noindent\textbf{Proof:}\par\noindent}
    \begin{solution}
}{
    \par\hfill\textbf{Q.E.D.}
    \end{solution}
    \renewcommand{\solutiontitle}{\noindent\textbf{Answers:}\par\noindent}
}

\newcommand{\ds}{\displaystyle}

\allowdisplaybreaks

\newcommand{\A}{\mathbb{A}}
\newcommand{\B}{\mathbb{B}}
\newcommand{\C}{\mathbb{C}}
\newcommand{\D}{\mathbb{D}}
\newcommand{\E}{\mathbb{E}}
\newcommand{\F}{\mathbb{F}}
\newcommand{\K}{\mathbb{K}}
\newcommand{\N}{\mathbb{N}}
\newcommand{\Q}{\mathbb{Q}}
\newcommand{\R}{\mathbb{R}}
\newcommand{\T}{\mathbb{T}}
\newcommand{\X}{\mathbb{X}}
\newcommand{\Y}{\mathbb{Y}}
\newcommand{\Z}{\mathbb{Z}}

\begin{document}

\begin{center}
    \textbf
    {\Large{COMP2121 Discrete Mathematics} \\
    \large{25/26 Semester 1} \\
    \large{Assignment 1}}\\
    SHING, Zhan Ho Jacob \qquad 3036228892
\end{center}

\textit{
    To distinguish, tautologies are represented as a bold T ($\ltrue$),
    and contradictions a bold F ($\lfalse$).
}

\begin{questions}
    
    \question English Statements and Logic

    \begin{parts}

        \part $(C \land \neg M) \limp \neg S$
        \begin{solution}
            The statement translates to ``If Lady Furina has clues and does not have motive,
            then she does not solve the case.''
        \end{solution}

        \part Rewrite sentence:
        \begin{solution}
            The required logical expression is $(S \land C) \limp T$.
        \end{solution}

        \part Verify if an argument is valid:
        \begin{solution}
            If we rewrite the argument in logical expressions, we have:
            \begin{enumerate}
                \item $S \limp C$ \label{q1c:stm1}
                \item $\neg C \limp \neg T$ \label{q1c:stm2}
                \item Therefore, from propositions \ref{q1c:stm1} and \ref{q1c:stm2}, $S \limp T$ \label{q1c:stm3}
            \end{enumerate}
            For the argument to be valid, we require the conclusion (Proposition \ref{q1c:stm3})
            is never false while all the premises (Propositions \ref{q1c:stm1} and \ref{q1c:stm2}) are true.
            It can be verified by trying to find a counterexample, i.e., combinations of $S$, $C$, and $T$
            such that the conclusion is $\lfalse$ while all its premises are $\ltrue$.

            For the conclusion to be $\lfalse$, the only possible case is when $S$ is $\ltrue$ and
            $T$ is $\lfalse$.

            For Proposition \ref{q1c:stm1} to be $\ltrue$ when $S$ is $\ltrue$, $C$ can only be
            $\ltrue$.
            
            When $C$ is $\ltrue$ and $T$ is $\lfalse$, Proposition \ref{q1c:stm2} is $\ltrue$.

            Therefore, when $S$ is $\ltrue$, $C$ is $\ltrue$, and $T$ is $\lfalse$, all premises
            are $\ltrue$ but the conclusion is $\lfalse$, i.e., the argument is \textbf{invalid}.
        \end{solution}

    \end{parts}

    \question A Logic Puzzle

    \begin{solution}
        First, assume that ``Ja'' means ``Yes'' and ``Da'' means ``No''.

        Assume that A is a knight, then based on his answer, which is the truth, B is
        a spy. Then, C must be the knave since there is exactly one person of one type,
        i.e., the answers from C are lies. For the first question that C was asked,
        he said he would answer ``Ja'', but since he is a knave, that means C would
        actually answer ``Da'' upon being asked. If he answers ``Da (No)'' to the question
        whether A is a knave, he would be lying, and it would mean that A is indeed a
        knave, which is contraditory to our initial assumption. Therefore, this case
        is not true.

        Now, we assume A to be the knave, then when he says he would answer ``Ja'' to
        the question whether B is the spy, he is lying and he would actually answer ``Da''.
        And if so he does, ``B is the spy'' would in fact be the truth, so B is in fact
        the spy. Then C would have to be a knight. Since he always tells the truth, then
        when asked if C were a knave, he would answer ``Ja'', which is the truth. This is
        impossible as C could only be a knight. Therefore, this case is not true.

        Now, we assume A to be the spy, then he can either tells a truth or lie. Assume
        that A answering ``Ja'' to ``If I asked you if B is the spy, would you say Ja?''
        is a truth
    \end{solution}

    \question Logical Operators

    \begin{parts}
        
        \part Number of possible 2-to-1 logical operators for inputs $A, B\in\{0,1\}$
        and output $C\in \{0,1\}$.
        \begin{solution}
            Note that for two inputs $A$ and $B$, there are four possible combinations,
            i.e., $(A, B) \in \{(0,0), (0, 1), (1, 0), (1, 1)\}$. A logical operator
            is a function $f$ that maps the combinations of inputs to an output, i.e.,
            $f: \{0, 1\}^2 \to \{0, 1\}$. Consider the truth table of $f$:
            \begin{center}
                \begin{tabular}{ccc}
                    $\bm{A}$ & $\bm{B}$ & $\bm{C = f(A, B)}$ \\
                    \hline
                    0 & 0 & $f(0, 0)$ \\
                    0 & 1 & $f(0, 1)$ \\
                    1 & 0 & $f(1, 0)$ \\
                    1 & 1 & $f(1, 1)$ 
                \end{tabular}
            \end{center}
            Two logical operators $f$ and $g$ are said to be the same if and only if
            for all combinations of inputs, $f(A, B) = g(A, B)$. Therefore, the problem
            reduces to counting the number of different ways to fill the truth table,
            which is given by $2^4 = 16$.

            Therefore, there are \fbox{\textbf{16}} possible 2-to-1 logical operators.
        \end{solution}

        \part Implement \texttt{NOT}, \texttt{AND}, \texttt{OR}, \texttt{XOR}, and
        \texttt{IMPLIES} using only the logical operator \texttt{NAND}. 
        \begin{solution}
            \begin{itemize}
                \item \textbf{\texttt{NOT} operator}
                
                Note that $\texttt{NAND}(A, A) \equiv \neg (A \land A) \equiv \neg A \lor \neg A \equiv \neg A$.
                
                Therefore, the \texttt{NOT} operator is implemented as $\boxed{\texttt{NOT}(A) = \texttt{NAND}(A, A)}$.

                \item \textbf{\texttt{OR} operator}
                
                Note that $\neg(\neg A \land \neg B) \equiv A \lor B$,
                which means $\texttt{OR}(A, B)=\texttt{NAND}(\texttt{NOT}(A), \texttt{NOT}(B))$.
                
                Further expand the $\texttt{NOT}$ operators as implemented before, we have:
                $\boxed{\texttt{OR}(A, B)=\texttt{NAND}\left[\texttt{NAND}(A, A), \texttt{NAND}(B,B)\right]}$.

                \item \textbf{\texttt{AND} operator}
                
                Note that if we apply \texttt{NAND} on the result of $\texttt{NAND}(A, B)$, we have:
                $$\neg\left[\neg\left(A\land B\right)\land\neg\left(A\land B\right)\right]
                \equiv \left(A\land B\right)\lor\left(A \land B\right)
                \equiv A \land \left(B \lor B\right)
                \equiv A \land B$$

                Therefore, the \texttt{AND} operator is implemented as
                \fbox{$\topr{AND}{A}{B}=\topr{NAND}{\topr{NAND}{A}{B}}{\topr{NAND}{A}{B}}$}.

                \item \textbf{\texttt{XOR} operator}
                
                Recall that $A\lxor B
                \equiv (A\lor B)\land\neg(A\land B)$. From this expression, we continue to derive:
                \begin{align*}
                    (A\lor B)\land\neg(A\land B)
                    &\equiv (A\land\neg(A\land B))\lor(B\land\neg(A\land B)) \qquad\text{(Distributive Law)}\\
                    &\equiv \neg[\neg(A\land\neg(A\land B))\land\neg(B\land\neg(A\land B))] \qquad\text{(De Morgan's Law)}\\
                    &\equiv \topr{NAND}{
                        \topr{NAND}{A}{\topr{NAND}{A}{B}}
                    }{
                        \topr{NAND}{B}{\topr{NAND}{A}{B}}
                    }
                \end{align*}

                Therefore, \texttt{XOR} is implemented as \fbox{$\topr{XOR}{A}{B}
                =\topr{NAND}{
                        \topr{NAND}{A}{\topr{NAND}{A}{B}}
                    }{
                        \topr{NAND}{B}{\topr{NAND}{A}{B}}
                    }$}.

                \item \textbf{\texttt{IMPLIES} operator}
                
                By logical equivalence, we have $A \limp B \equiv \neg A \lor B$.

                Expand the expression, we have:
                \begin{align*}
                    \neg A\lor B
                    &\equiv \topr{OR}{\tnot{A}}{B} \\
                    &\equiv \topr{OR}{\topr{NAND}{A}{A}}{B} \\
                    &\equiv \topr{NAND}{
                        \topr{NAND}{\topr{NAND}{A}{A}}{\topr{NAND}{A}{A}}
                    }{
                        \topr{NAND}{B}{B}
                    }
                \end{align*}

                Therefore, \texttt{IMPLIES} is as \fbox{$
                \topr{IMPLIES}{A}{B} = 
                \topr{NAND}{
                        \topr{NAND}{\topr{NAND}{A}{A}}{\topr{NAND}{A}{A}}
                    }{
                        \topr{NAND}{B}{B}
                    }
                $}.
            \end{itemize}
        \end{solution}
    \end{parts}

    \question Quantifiers and Predicates

    \begin{parts}
        
        \part Rewrite expression
        $\neg(\exists y\neg(\forall x(P(x)\land Q(y))))\limp\exists z R(z)$.
        \begin{solution}
            \begin{align*}
                \neg(\exists y\neg(\forall x(P(x)\land Q(y))))\limp\exists z R(z)
                &\equiv \neg(\exists y\neg(\forall xP(x)\land\forall xQ(y)))\limp\exists zR(z) \\
                &\equiv \neg(\exists y(\neg\forall xP(x)\lor\neg Q(y)))\limp\exists zR(z) \\
                &\equiv \neg(\exists y(\exists x\neg P(x) \lor\neg Q(y)))\limp\exists zR(z) \\
                &\equiv \forall y\neg(\exists x\neg P(x)\lor\neg Q(y))\limp\exists zR(z) \\
                &\equiv \forall y(\neg\exists x\neg P(x)\land\neg(\neg Q(y)))\limp\exists zR(z) \\
                &\equiv \forall y(\forall x\neg(\neg P(x))\land Q(y))\limp\exists zR(z) \\
                &\equiv \forall y(\forall xP(x) \land Q(y)) \limp\exists zR(z) \\
                &\equiv \forall y\forall xP(x)\land\forall yQ(y)\limp\exists zR(z)\\
                &\equiv (\forall xP(x))\land(\forall yQ(y)) \limp\exists zR(z) \\
                &\equiv \neg(\forall xP(x))\land(\forall yQ(y)) \lor \exists zR(z)\\
                &\equiv \neg\forall xP(x) \lor \neg\forall yQ(y) \lor \exists zR(z)\\
                &\equiv \exists x\neg P(x) \lor \exists y\neg Q(y) \lor \exists zR(z)
            \end{align*}
        \end{solution}

        \part Prove that $\forall xP(x)\land\exists xQ(x) \equiv \forall x \exists y (P(x)\land Q(y))$.
        \begin{psol}
            Rename the variable on the left-hand side, we have:
            $$\forall xP(x)\land\exists yQ(y) \equiv \forall x\exists y(P(x)\land Q(y))$$
            To prove equivalence, we prove both the sufficiency and necessity.
            \begin{enumerate}
                \item \textbf{Sufficiency:} $\forall xP(x)\land\exists yQ(y) \Rightarrow \forall x\exists y(P(x)\land Q(y))$
                
                Assume that the L.H.S. is true, then for all $x$, $P(x)$ must hold.
                Also, there must exist at least one $y$, say $y_0$, such that $Q(y_0)$ holds.
                Therefore, for all $x$, we can always find a $y$, which is $y_0$, such that
                $P(x)\land Q(y)$ holds. We have displayed that the R.H.S. must be true if
                the L.H.S. is true.

                \item \textbf{Necessity:} $\forall xP(x)\land\exists yQ(y) \Leftarrow \forall x\exists y(P(x)\land Q(y))$
                
                Assume that the R.H.S. (i.e., $\forall x\exists y(P(x)\land Q(y))$) is true,
                then for all $x$, we can always find a $y$, say $y_0$, such that $P(x)\land Q(y_0)$
                holds. Therefore, for all $x$, $P(x)$ must hold, and there must exist at least one $y$,
                which is $y_0$, such that $Q(y_0)$ holds. We have displayed that the L.H.S. must be true if
                the R.H.S. is true.
            \end{enumerate}
            Since both the sufficiency and necessity have been proved, we conclude that
            $\forall xP(x)\land\exists xQ(x) \liff \forall x \exists y (P(x)\land Q(y))$
            is a tautology, i.e., the two statements are logically equivalent.
        \end{psol}

        \part Justify whether $\exists x \forall y P(x, y) \limp \forall y \exists x P(x, y)$
        is always a tautology for any predicate $P(x, y)$.
        \begin{solution}
            We propose that the given proposition is not always a tautology, and we attempt
            to find an example, such that the proposition is a contradiction.

            $\exists x \forall y P(x,y) \limp \forall y \exists x P(x,y)$ is a contradiction
            iff $\exists x \forall y P(x,y)$ is $\ltrue$ and $\forall y \exists x P(x,y)$ is
            $\lfalse$.

            If the antecedent is true, then, for at least one $x$, say $x_0$, $P(x_0, y)$
            holds for all $y$.

            If the consequent is false, then, for at least one $y$, say $y_0$, there does not
            exist any $x$ such that $P(x, y_0)$ holds.

            However, if $P(x_0, y)$ holds for all $y$, then $P(x_0, y_0)$ must also hold.
            Therefore, such $y_0$ does not exist, and the consequent can never be false if
            the antecedent is true.

            Therefore, we cannot find a counterexample such that the antecedent is true
            while the consequent is false. Hence, the given proposition is always a tautology.
        \end{solution}

    \end{parts}

    \question Proofs

    \begin{parts}
        
        \part Determine the validity of a proof.
        \begin{solution}
            Step 1: Valid. This is implication law ($P \limp Q \equiv \neg P \lor Q$).

            Step 2: Valid. Renaming the variable $x$ to $y$ does not change the meaning of
            the statement. This is also the distribution law of $\forall$ over $\land$ --
            $\forall x P(x) \land \forall y Q(y) \equiv \forall x (P(x) \land Q(x))$.

            Step 3: Valid. This step combines application of several laws:
            \begin{align*}
                &\equiv \forall x[(P(x)\land\neg Q(x)) \lor (Q(x)\land\neg Q(x))] \qquad\text{(Distribution law)}\\
                &\equiv \forall x[(P(x)\land\neg Q(x)) \lor \lfalse] \qquad\text{(Contradiction law)}\\
                &\equiv \forall x[P(x)\land\neg Q(x)] \qquad\text{(Identity law)}
            \end{align*}

            Step 4: Valid. This is the simplification (i.e., $P\land Q \limp P$).

            Step 5: Valid. First, since $\forall x P(x)$ is true, then $P(c)$ must be true
            for any $c$ in the universe of dicourse. Then, since there exists at least one
            $x$, $c$ in this case, such that $P(x)$ is true, then $\exists x P(x)$ is true.
        \end{solution}

        \part Proof by Induction.
        \begin{psol}
            Denote $P(n) : \ds{\sum_{k=1}^n \cos(kx) = \frac{1}{2}\left(\frac{\sin\left[\left(n+\frac{1}{2}\right)x\right]}{\sin\left(\frac{1}{2}x\right)}-1\right)}$.

            \textbf{Base Case:} $n=1$.
            \begin{equation*}
                \text{L.H.S.} = \sum_{k=1}^1 \cos(kx) = \cos(x)
            \end{equation*}
            \begin{align*}
                \text{R.H.S.} &= \frac{1}{2}\left(\frac{\sin\frac{3}{2}x}{\sin\frac{1}{2}x}-1\right)\\
                &= \frac{1}{2} \left(\frac{\sin\frac{3}{2}x-\sin\frac{1}{2}x}{\sin\frac{1}{2}x}\right)\\
                &= \frac{1}{2} \left(\frac{2\cos x\sin\frac{1}{2}x}{\sin\frac{1}{2}x}\right)\\
                &= \cos x \\
                &= \text{L.H.S.}
            \end{align*}
            Therefore, $P(1)$ holds.

            \textbf{Inductive Step:} Assume that $P(n)$ holds for $n\geq 1$. We show that $P(n+1)$ also holds.
            \begin{align*}
                \text{L.H.S.} &= \sum_{k=1}^{n+1} \cos(kx) \\
                &= \sum_{k=1}^n \cos(kx) + \cos\left((n+1)x\right) \\
                &= P(n) + \cos\left((n+1)x\right) \\
                &= \frac{1}{2}\left(\frac{\sin\left[\left(n+\frac{1}{2}\right)x\right]}{\sin\frac{1}{2}x}-1\right) + \cos\left((n+1)x\right) \\
                &= \frac{1}{2}\left(\frac{\sin\left[\left(n+\frac{1}{2}\right)x\right]-\sin\frac{1}{2}x}{\sin\frac{1}{2}x}\right) + \cos\left((n+1)x\right) \\
                &= \frac{1}{2}\left(\frac{2\cos\left[\frac{n+1}{2}x\right]\sin\left[\frac{n}{2}x\right]}{\sin\frac{1}{2}x}\right) + \cos\left((n+1)x\right) \\
                &= \frac{\cos\left[\frac{n+1}{2}x\right]\sin\left[\frac{n}{2}x\right] + \cos\left((n+1)x\right)\sin\frac{1}{2}x}{\sin\frac{1}{2}x} \\
                &= \frac{\sin\left[\frac{2n+1}{2}x\right]-\sin\frac{1}{2}x+\sin\left[\frac{2n+3}{2}x\right]-\sin\left[\frac{2n+1}{2}x\right]}{2\sin\frac{1}{2}x} \\
                &= \frac{1}{2}\left(\frac{\sin\left[\frac{2n+3}{2}x\right]}{\sin\frac{1}{2}x}-1\right) \\
                &= \frac{1}{2}\left(\frac{\sin\left[\left(\left(n+1\right)+\frac{1}{2}\right)x\right]}{\sin\frac{1}{2}x}-1\right) \\
                &= \text{R.H.S.}
            \end{align*}
            Therefore, $P(n) \Rightarrow P(n+1)$.

            By the principle of mathematical induction, $P(n)$ holds for all $n\geq 1$.
        \end{psol}

    \end{parts}

    \question Basics on Sets

    \begin{parts}
        
        \part Cardinality.
        \begin{solution}
            \begin{enumerate}[label=(\roman*)]
                \item $\card{A} = \boxed{4}$
                \item $\card{\pwset{B}} = 2^{\card{B}} = 2^4 = \boxed{16}$,
                provided that the set $B$ has two integers, one empty set, and one proposition.
                \item Note that $C = [0, 10]$. Therefore, $\card{C} = \boxed{11}$.
                \item Assuming that capital and small letters are the same, then
                $\card{D} = 26-7 =\boxed{19}$.
            \end{enumerate}
        \end{solution}

        \part Set Operations.
        \begin{solution}
            \begin{enumerate}[label=(\roman*)]
                \item $A \cap B = \set{2, \mtset} \Rightarrow \card{A\cap B} = \boxed{2}$.
                \item Consider:\begin{align*}
                    B \cup (C\cap\mtset) - A 
                    &= B\cup\mtset - A \\
                    &= B-A \\
                    &= \set{1, 2, \mtset, |x|=2} - \set{\set{0, 1}, 2, 3, \mtset} \\
                    &= \set{1, |x|=2} \\
                    &\Rightarrow \card{B \cup (C\cap\mtset) - A} = \boxed{2}
                \end{align*}
                \item Consider:\begin{align*}
                    (A\cap B)\times(B\cap C)
                    &= \set{2, \mtset} \times \set{1, 2} \\
                    &= \set{(2, 1), (2, 2), (\mtset, 1), (\mtset, 2)} \\
                    &\Rightarrow \card{(A\cap B)\times(B\cap C)} = \boxed{4}
                \end{align*}
            \end{enumerate}
        \end{solution}

    \end{parts}

    \question Venn Diagrams

    \begin{parts}
        
        \part Set in Figure 2.
        \begin{solution}
            The set represented in Figure 2 is $(A\cup B)\cap C$.
        \end{solution}

        \part Draw Venn diagram for $(A\cup B) - (\overline{A\cap C})$.
        \begin{solution}
            \begin{center}
                \begin{venndiagram3sets}[
                    labelNotABC=$U$,
                    shade=blue!30
                    ]
                    \fillOnlyA \fillOnlyB \fillACapBNotC \fillBCapCNotA
                \end{venndiagram3sets}
            \end{center}
        \end{solution}

    \end{parts}

\end{questions}

\end{document}