\documentclass[answers]{exam}
\usepackage[x11names]{xcolor}
\usepackage[left=1.5cm,right=1.5cm,top=1.5cm,bottom=2cm]{geometry}
\usepackage{newtxtext,newtxmath}
\usepackage{amsmath}
\usepackage{bm}
\usepackage{float}
\usepackage{multicol,multirow}
\usepackage{tabularx}
\usepackage[inline,shortlabels]{enumitem}
\usepackage{cancel}
\usepackage{venndiagram}
% for long multiplications
\makeatletter
    \providecommand\text\mbox
    \newenvironment{arithmetic}[1][]{\begin{tabular}[#1]{Al}}{\end{tabular}}
    \newcolumntype{A}{>{\bgroup\def~{\phantom{0}}$\@testOptor}r<{\@gobble\\$\egroup}}
    \def\@testOptor\ignorespaces#1#2\\{%
    \ifx#1\times
        \@OperatorRow{#1}{#2}\@tempa%
    \else\ifx#1+
        \@OperatorRow+{#2}\@tempa%
    \else\ifx#1\discretionary% detects the soft hyphen, \-
        \@ShortSubtractRow{#2}\@tempa%
    \else\ifx#1-
        \@OperatorRow-{#2}\@tempa%
    \else
        \@NormalRow{#1#2}\@tempa%
    \fi\fi\fi\fi
    \@tempa}
    \def\@OperatorRow#1#2#3{%
    \@IfEndRow#2\@gobble\\{%
        \def#3{\underline{{}#1 #2}\\}%
    }{%
        \def#3{\underline{{}#1 #2{}}}%
    }}

\def\@NormalRow#1#2{%
    \@IfEndRow#1\@gobble\\{%
        \def#2{#1\\}%
    }{%
        \def#2{#1{}}%
    }}

\def\@IfEndRow#1\@gobble#2\\#3#4{%
    \ifx#2\@gobble
        #4%
    \else
        #3%
    \fi}

\makeatother

\pagestyle{foot}
\cfoot{Page \thepage\ of \numpages}
\bracketedpoints
\renewcommand{\solutiontitle}{\noindent\textbf{Answers:}\par\noindent}

% Logic symbols
\newcommand{\lxor}{\oplus}
\newcommand{\limp}{\rightarrow}
\newcommand{\liff}{\leftrightarrow}
\newcommand{\ltrue}{\bm{T}}
\newcommand{\lfalse}{\bm{F}}

% Set symbols
\newcommand{\set}[1]{\left\{#1\right\}} % set braces
\newcommand{\card}[1]{\left|#1\right|} % cardinality of a set
\newcommand{\pwset}[1]{\mathcal{P}(#1)} % power set
\newcommand{\mtset}{\varnothing} % empty set
\renewcommand{\emptyset}{\varnothing} % empty set

\renewcommand{\thesubpart}{(\arabic{subpart})}
\renewcommand{\subpartlabel}{\thesubpart}

% commands for question 4b
\newcommand{\tnot}[1]{\texttt{NOT}\left(#1\right)}
\newcommand{\topr}[3]{\texttt{#1}\left(#2, #3\right)}

% psol (proof-solution) environment
\newenvironment{psol}{
    \renewcommand{\solutiontitle}{\noindent\textbf{Proof:}\par\noindent}
    \begin{solution}
}{
    \par\hfill\textbf{Q.E.D.}
    \end{solution}
    \renewcommand{\solutiontitle}{\noindent\textbf{Answers:}\par\noindent}
}

\newcommand{\ds}{\displaystyle}

\allowdisplaybreaks

\newcommand{\A}{\mathbb{A}}
\newcommand{\B}{\mathbb{B}}
\newcommand{\C}{\mathbb{C}}
\newcommand{\D}{\mathbb{D}}
\newcommand{\E}{\mathbb{E}}
\newcommand{\F}{\mathbb{F}}
\newcommand{\K}{\mathbb{K}}
\newcommand{\N}{\mathbb{N}}
\newcommand{\Q}{\mathbb{Q}}
\newcommand{\R}{\mathbb{R}}
\newcommand{\T}{\mathbb{T}}
\newcommand{\X}{\mathbb{X}}
\newcommand{\Y}{\mathbb{Y}}
\newcommand{\Z}{\mathbb{Z}}

\begin{document}

\begin{center}
    \textbf
    {\Large{COMP2121 Discrete Mathematics} \\
    \large{25/26 Semester 1} \\
    \large{Assignment 1}}\\
    SHING, Zhan Ho Jacob \qquad 3036228892
\end{center}

\textit{
    To distinguish, tautologies are represented as a bold T ($\ltrue$),
    and contradictions a bold F ($\lfalse$).
}

\begin{questions}
    
    \question English Statements and Logic

    \begin{parts}

        \part $(C \land \neg M) \limp \neg S$
        \begin{solution}
            The statement translates to ``If Lady Furina has clues and does not have motive,
            then she does not solve the case.''
        \end{solution}

        \part Rewrite sentence:
        \begin{solution}
            The required logical expression is $(S \land C) \limp T$.
        \end{solution}

        \part Verify if an argument is valid:
        \begin{solution}
            If we rewrite the argument in logical expressions, we have:
            \begin{enumerate}
                \item $S \limp C$ \label{q1c:stm1}
                \item $\neg C \limp \neg T$ \label{q1c:stm2}
                \item Therefore, from propositions \ref{q1c:stm1} and \ref{q1c:stm2}, $S \limp T$ \label{q1c:stm3}
            \end{enumerate}
            For the argument to be valid, we require the conclusion (Proposition \ref{q1c:stm3})
            is never false while all the premises (Propositions \ref{q1c:stm1} and \ref{q1c:stm2}) are true.
            It can be verified by trying to find a counterexample, i.e., combinations of $S$, $C$, and $T$
            such that the conclusion is $\lfalse$ while all its premises are $\ltrue$.

            For the conclusion to be $\lfalse$, the only possible case is when $S$ is $\ltrue$ and
            $T$ is $\lfalse$.

            For Proposition \ref{q1c:stm1} to be $\ltrue$ when $S$ is $\ltrue$, $C$ can only be
            $\ltrue$.
            
            When $C$ is $\ltrue$ and $T$ is $\lfalse$, Proposition \ref{q1c:stm2} is $\ltrue$.

            Therefore, when $S$ is $\ltrue$, $C$ is $\ltrue$, and $T$ is $\lfalse$, all premises
            are $\ltrue$ but the conclusion is $\lfalse$, i.e., the argument is \textbf{invalid}.
        \end{solution}

    \end{parts}

\end{questions}

\end{document}