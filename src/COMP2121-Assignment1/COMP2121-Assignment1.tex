\documentclass[answers]{exam}
\usepackage[english]{babel}
\usepackage[style=apa,backend=biber]{biblatex}
\usepackage[
    format=plain,
    labelsep=newline,
    font=small,
    labelfont=bf,
    textfont=it,
    justification=justified,
    singlelinecheck=off
    ]{caption}
\usepackage{csquotes}
\usepackage{float}
\usepackage[top=2.5cm,bottom=2.5cm,left=3cm,right=3cm]{geometry}
\usepackage{graphicx}
\usepackage[hidelinks,colorlinks=true,linkcolor=blue,filecolor=blue,urlcolor=blue,citecolor=blue]{hyperref}
    \usepackage[nameinlink]{cleveref} %nameinlink ensures that the entire element is clickable in the pdf, not just the number
\usepackage{indentfirst}
\usepackage{multicol}
\usepackage[skip=1em,indent]{parskip}
\usepackage{tabularx}
\usepackage{times}
\pagestyle{foot}
\cfoot{Page \thepage\ of \numpages}
\shadedsolutions

\renewcommand{\thesubpart}{(\arabic{subpart})}
\renewcommand{\subpartlabel}{\thesubpart}

\begin{document}

\begin{center}
    \textbf
    {\Large{COMP2121 Discrete Mathematics} \\
    \large{25/26 Semester 1} \\
    \large{Assignment 1}}\\
    SHING, Zhan Ho Jacob \qquad 3036228892
\end{center}

\textit{
    To distinguish, tautologies are represented as a bold T ($\ltrue$),
    and contradictions a bold F ($\lfalse$).
}

\begin{questions}
    
    \question English Statements and Logic

    \begin{parts}

        \part $(C \land \neg M) \limp \neg S$
        \begin{solution}
            The statement translates to ``If Lady Furina has clues and does not have motive,
            then she does not solve the case.''
        \end{solution}

        \part Rewrite sentence:
        \begin{solution}
            The required logical expression is $(S \land C) \limp T$.
        \end{solution}

        \part Verify if an argument is valid:
        \begin{solution}
            If we rewrite the argument in logical expressions, we have:
            \begin{enumerate}
                \item $S \limp C$ \label{q1c:stm1}
                \item $\neg C \limp \neg T$ \label{q1c:stm2}
                \item Therefore, from propositions \ref{q1c:stm1} and \ref{q1c:stm2}, $S \limp T$ \label{q1c:stm3}
            \end{enumerate}
            For the argument to be valid, we require the conclusion (Proposition \ref{q1c:stm3})
            is never false while all the premises (Propositions \ref{q1c:stm1} and \ref{q1c:stm2}) are true.
            It can be verified by trying to find a counterexample, i.e., combinations of $S$, $C$, and $T$
            such that the conclusion is $\lfalse$ while all its premises are $\ltrue$.

            For the conclusion to be $\lfalse$, the only possible case is when $S$ is $\ltrue$ and
            $T$ is $\lfalse$.

            For Proposition \ref{q1c:stm1} to be $\ltrue$ when $S$ is $\ltrue$, $C$ can only be
            $\ltrue$.
            
            When $C$ is $\ltrue$ and $T$ is $\lfalse$, Proposition \ref{q1c:stm2} is $\ltrue$.

            Therefore, when $S$ is $\ltrue$, $C$ is $\ltrue$, and $T$ is $\lfalse$, all premises
            are $\ltrue$ but the conclusion is $\lfalse$, i.e., the argument is \textbf{invalid}.
        \end{solution}

    \end{parts}

\end{questions}

\end{document}