% for long multiplications
\makeatletter
    \providecommand\text\mbox
    \newenvironment{arithmetic}[1][]{\begin{tabular}[#1]{Al}}{\end{tabular}}
    \newcolumntype{A}{>{\bgroup\def~{\phantom{0}}$\@testOptor}r<{\@gobble\\$\egroup}}
    \def\@testOptor\ignorespaces#1#2\\{%
    \ifx#1\times
        \@OperatorRow{#1}{#2}\@tempa%
    \else\ifx#1+
        \@OperatorRow+{#2}\@tempa%
    \else\ifx#1\discretionary% detects the soft hyphen, \-
        \@ShortSubtractRow{#2}\@tempa%
    \else\ifx#1-
        \@OperatorRow-{#2}\@tempa%
    \else
        \@NormalRow{#1#2}\@tempa%
    \fi\fi\fi\fi
    \@tempa}
    \def\@OperatorRow#1#2#3{%
    \@IfEndRow#2\@gobble\\{%
        \def#3{\underline{{}#1 #2}\\}%
    }{%
        \def#3{\underline{{}#1 #2{}}}%
    }}

\def\@NormalRow#1#2{%
    \@IfEndRow#1\@gobble\\{%
        \def#2{#1\\}%
    }{%
        \def#2{#1{}}%
    }}

\def\@IfEndRow#1\@gobble#2\\#3#4{%
    \ifx#2\@gobble
        #4%
    \else
        #3%
    \fi}

\makeatother

\pagestyle{foot}
\cfoot{Page \thepage\ of \numpages}
\bracketedpoints
\renewcommand{\solutiontitle}{\noindent\textbf{Answers:}\par\noindent}

% Logic symbols
\newcommand{\lxor}{\oplus}
\newcommand{\limp}{\rightarrow}
\newcommand{\liff}{\leftrightarrow}
\newcommand{\ltrue}{\bm{T}}
\newcommand{\lfalse}{\bm{F}}

% Set symbols
\newcommand{\set}[1]{\left\{#1\right\}} % set braces
\newcommand{\card}[1]{\left|#1\right|} % cardinality of a set
\newcommand{\pwset}[1]{\mathcal{P}(#1)} % power set
\newcommand{\mtset}{\varnothing} % empty set
\renewcommand{\emptyset}{\varnothing} % empty set

\renewcommand{\thesubpart}{(\arabic{subpart})}
\renewcommand{\subpartlabel}{\thesubpart}

% commands for question 4b
\newcommand{\tnot}[1]{\texttt{NOT}\left(#1\right)}
\newcommand{\topr}[3]{\texttt{#1}\left(#2, #3\right)}

% psol (proof-solution) environment
\newenvironment{psol}{
    \renewcommand{\solutiontitle}{\noindent\textbf{Proof:}\par\noindent}
    \begin{solution}
}{
    \par\hfill\textbf{Q.E.D.}
    \end{solution}
    \renewcommand{\solutiontitle}{\noindent\textbf{Answers:}\par\noindent}
}

\newcommand{\ds}{\displaystyle}

\allowdisplaybreaks