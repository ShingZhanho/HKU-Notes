\documentclass[answers]{exam}
\usepackage[english]{babel}
\usepackage[style=apa,backend=biber]{biblatex}
\usepackage[
    format=plain,
    labelsep=newline,
    font=small,
    labelfont=bf,
    textfont=it,
    justification=justified,
    singlelinecheck=off
    ]{caption}
\usepackage{csquotes}
\usepackage{float}
\usepackage[top=2.5cm,bottom=2.5cm,left=3cm,right=3cm]{geometry}
\usepackage{graphicx}
\usepackage[hidelinks,colorlinks=true,linkcolor=blue,filecolor=blue,urlcolor=blue,citecolor=blue]{hyperref}
    \usepackage[nameinlink]{cleveref} %nameinlink ensures that the entire element is clickable in the pdf, not just the number
\usepackage{indentfirst}
\usepackage{multicol}
\usepackage[skip=1em,indent]{parskip}
\usepackage{tabularx}
\usepackage{times}
\pagestyle{foot}
\cfoot{Page \thepage\ of \numpages}
\shadedsolutions

\renewcommand{\thesubpart}{(\arabic{subpart})}
\renewcommand{\subpartlabel}{\thesubpart}

\begin{document}

\begin{center}
    \textbf
    {\Large{COMP2120 Computer Organisation} \\
    \large{24/25 Semester 2} \\
    \large{Assignment 2}}
\end{center}

\begin{questions}

    \question Consider a simple 32-bit processor with the data path as shown in fig. 1. The
        processor has 32 general purpose registers. There are 3 buses, S1-bus, S2-bus and D-bus
        connecting the registers for data movement. The register files has 2 read ports and 1
        write port (i.e. it can perform 2 read and 1 write at the same time).

    \begin{figure}[H]
        \centering
        \includegraphics[width=0.7\textwidth]{figure.pdf}
        \caption{A simplified CPU}
    \end{figure}

    The processor has instructions which specifies 3 operands explicitly (namely, 2 source
    and 1 destination operands). The leftmost byte of the instruction represents the oper-
    ation to be performed, such as \texttt{ADD}, \texttt{SUB} etc. For arithmetic and logic operations, the
    operands must be in registers. Hence the 3 bytes will give the addresses of operands in
    the register file. There will be a direct path connecting these 3 bytes in the \texttt{IR}
    (Instruction Register) to the address of the register file, so that when you perform read/write
    on register file, the register specified in these bytes will be accessed.
    
    If the instruction is \texttt{LOAD} or \texttt{STORE} to load a word from memory to register, and
    vice versa, the source operand (\texttt{LOAD}) or destination operand (\texttt{STORE}) refer to a memory
    address. How to find this address is specified by Addressing Mode. In this machine, for
    simiplicity, the memory operand byte (source/destination) will always be \texttt{1111 1111} (or
    in hex \texttt{0xff}), which means that the actual 32-bit memory address will be given in the
    word following the instruction (see example program below).
    
    The \texttt{ALU} has the following operations: \texttt{ADD}, \texttt{SUB}, bitwise \texttt{AND}, \texttt{OR}, and \texttt{NOT}. For
    operations with only one operand (e.g. \texttt{NOT}), source operand 1 is used, and source
    operand 2 is empty.
    
    Finally, there is a branch instruction, which performs conditional or unconditional
    branch as specified in the cc field of the instruction. The branch address is specified in
    the word following the instruction, the same as in \texttt{LOAD}/\texttt{STORE} instruction.

    \section*{Instruction Format}

    \subsection*{Arithmetic/Logic Instruction}

    The instruction format of the machine (except \texttt{LOAD}/\texttt{STORE}/\texttt{BRANCH}):
    \begin{table}[H]
        \centering
        \begin{tabular}{|c|c|c|c|}
            \hline
            Opcode & Source Operand 1 & Source Operand 2 & Destination Operand \\
            \hline
        \end{tabular}
    \end{table}
    The instructions can be categorized into the following types:
    \begin{itemize}
        \item \textbf{Arithmetic Operations}
        \begin{verbatim}
ADD     R1, R2, R3  ; R3 <- R1 + R2
SUB     R1, R2, R3  ; R3 <- R1 - R2
        \end{verbatim}
        \item \textbf{Logic Operations}
        \begin{verbatim}
AND     R1, R2, R3  ; R3 <- R1 AND R2
OR      R1, R2, R3  ; R3 <- R1 OR R2
NOT     R1, R3      ; R3 <- NOT R1
        \end{verbatim}
        \item \textbf{Data Movement Instruction}
        \begin{verbatim}
MOV     R1, R3      ; R3 <- R1
        \end{verbatim}
    \end{itemize}
    Note that in the \texttt{NOT} and \texttt{MOV} operations, source operand 2 field is not used and will be set as \texttt{0000 0000}.

    \subsection*{Load/Store Instruction}

    \begin{solution}
        
    \end{solution}
    
\end{questions}

\end{document}