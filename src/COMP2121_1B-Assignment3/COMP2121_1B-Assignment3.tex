\documentclass[answers]{exam}
\usepackage[english]{babel}
\usepackage[style=apa,backend=biber]{biblatex}
\usepackage[
    format=plain,
    labelsep=newline,
    font=small,
    labelfont=bf,
    textfont=it,
    justification=justified,
    singlelinecheck=off
    ]{caption}
\usepackage{csquotes}
\usepackage{float}
\usepackage[top=2.5cm,bottom=2.5cm,left=3cm,right=3cm]{geometry}
\usepackage{graphicx}
\usepackage[hidelinks,colorlinks=true,linkcolor=blue,filecolor=blue,urlcolor=blue,citecolor=blue]{hyperref}
    \usepackage[nameinlink]{cleveref} %nameinlink ensures that the entire element is clickable in the pdf, not just the number
\usepackage{indentfirst}
\usepackage{multicol}
\usepackage[skip=1em,indent]{parskip}
\usepackage{tabularx}
\usepackage{times}
% =============================================================
% Title formats
% =============================================================
\titleformat{\section}
	{\Large\sffamily\bfseries}
	{}
	{0em}
	{{\parbox{\dimexpr\linewidth-2\fboxsep\relax}{\thesection.\space#1}}}
	[]

\titleformat{name=\section,numberless}
	{\Large\sffamily\bfseries}
	{}
	{0em}
	{{\parbox{\dimexpr\linewidth-2\fboxsep\relax}{#1}}}
	[]

\titleformat{\subsection}
	{\sffamily\bfseries}
	{\thesubsection}
	{0em}
	{{\parbox{\dimexpr\linewidth-2\fboxsep\relax}{#1}}}
	[]

\titleformat{name=\subsection,numberless}
    {\sffamily\bfseries}
    {}
    {0em}
    {{\parbox{\dimexpr\linewidth-2\fboxsep\relax}{#1}}}
    []

\titleformat{\subsubsection}
	{\sffamily\bfseries}
	{\thesubsubsection}
	{0.5em}
	{#1}
	[]

% =============================================================
% Exam package
% =============================================================
\bracketedpoints
\pointname{\%}
\pointsinmargin

\newcommand{\mcqAns}[1]{\fbox{\bfseries\sffamily #1}}

\begin{document}

\begin{center}
    \textbf
    {\Large{COMP2121 Discrete Mathematics} \\
    \large{25/26 Semester 1} \\
    \large{Assignment 3}}\\
    SHING, Zhan Ho Jacob \qquad 3036228892
\end{center}

\begin{questions}
    
    \question Basics Probabilities and Random Variables

    \begin{parts}

        \part[4] There are 5 boxes, 3 of which contain 10 HKD each, and 2 of which are empty. Bob will randomly
            select 3 boxes, and he will win the total amount of money in the selected boxes. What is the
            probability that he will win at least 20 HKD?
        \begin{solution}
            Let $X$ be the total amount of money Bob wins.
            $P(X\geq20)=1-P(X=0)-P(X=10)=1-0-\frac{\binom{3}{2}}{\binom{5}{3}}=1-\frac{3}{10}=\boxed{\frac{7}{10}}$.
        \end{solution}

        \part[4] Suppose you toss a fair coin for multiple times. Define $L$ to be the number of tosses you have to
            make until either of the following is satisfied:
        \begin{enumerate}[label=(\arabic*)]
            \item The number of ``head'' has reached 3.
            \item The total number of tosses reaches 5.
        \end{enumerate}
        Determine $\E(L)$.
        \begin{solution}
            \allbold{If $L=3$:} $P(L=3) = (0.5)^3 = 0.125$.
            
            \allbold{If $L=4$:} $P(L=4) = \binom{3}{2}(0.5)^3 (0.5) = 0.1875$.

            \allbold{If $L=5$:} The 5th toss is irrelevant.
                $P(L=5)=(0.5)^4+\binom{4}{1}(0.5)^3(0.5)+\binom{4}{2}(0.5)^2(0.5)^2=0.6875$
            
            \[
                \E(L)=3\cdot P(L=3)+4\cdot P(L=4)+5\cdot P(L=5)=\boxed{4.5625}
            \]
        \end{solution}

        \part[4] The joint distribution $P(X,Y)$ of two random variables $X$ and $Y$ is shown in the following table:
        \begin{center}
            \begin{tabular}{|c|c|c|c|}
                \hline
                \diagbox{$X$}{$Y$} & 0 & 1 & 2 \\
                \hline
                -1 & 0 & 1/6 & 1/6 \\ \hline
                0 & 1/6 & 1/6 & 0 \\ \hline
                1 & 0 & 1/6 & 1/6 \\ \hline
            \end{tabular}
        \end{center}
        Show that $X$ and $Y$ are uncorrelated but not independent.
        \begin{solution}
            Marginal probabilities:
            \[
                P_X(x)=\sum_{\text{all }y} P(X=x, Y=y)=\begin{cases}
                    \frac{2}{6}, & x=-1\\
                    \frac{2}{6}, & x=0 \\
                    \frac{2}{6}, & x=1
                \end{cases}=\frac{1}{3}
            \]
            and
            \[
                P_Y(y)=\sum_{\text{all }x} P(X=x, Y=y)=\begin{cases}
                    \frac{1}{6}, & y=0\\
                    \frac{1}{2}, & y=1\\
                    \frac{1}{3}, & y=2
                \end{cases}.
            \]
            Then,
            \[
                \E(X) = \sum_{\text{all }x} x\cdot P_X(x) = 0 \qquad
                \E(Y) = \sum_{\text{all }y} y\cdot P_Y(y) = \frac{1}{2} + 2\cdot\frac{1}{3}=\frac{7}{6} \qquad
                \E(XY) = \sum_{\text{all }x} \sum_{\text{all }y} xy\cdot P(x, y)=0.
            \]
            Also,
            \[
                \cov(XY)=\E(XY)-\E(X)\E(Y)=0
            \]
            Hence, \fbox{$X$ and $Y$ are not correlated}.

            However, $P(-1, 0)=0\neq\frac{1}{18}=P_X(-1)P_Y(0)$. Therefore, \fbox{$X$ and $Y$ are not independent}.
        \end{solution}

        \part[4] Alice and Bob are playing a game. At the start, Alice randomly chooses a real number $x \in[0, 5]$.
            The game consists of $n$ rounds. In each round, Alice secretly tosses a fair coin. If the outcome is
            ``head'', she tells Bob $X_i = x + 1$, otherwise, she tells him $X_i = x - 1$. At the end, Bob computes
            the estimate
            \[
                \hat{x}=\frac{1}{n}\sum_{i=1}^{n} X_i.
            \]
            Determine how large $n$ should be to ensure that $|\hat{x}-x|<\frac{1}{2}$ with a probability of at least 0.9.
        \begin{solution}
            Let $Y_i = X_i - x$. Since $X_i = x \pm 1$ with probability $0.5$, $Y_i$ takes values $\pm 1$ with probability $0.5$.
            \[
                \E(Y_i) = 1(0.5) + (-1)(0.5) = 0, \quad \var(Y_i) = \E(Y_i^2) - (\E(Y_i))^2 = 1 - 0 = 1.
            \]
            Let $\overline{Y} = \hat{x} - x = \frac{1}{n}\sum_{i=1}^n (X_i - x) = \frac{1}{n}\sum_{i=1}^n Y_i$. Then
            \[
                \E(\overline{Y}) = 0, \quad \var(\overline{Y}) = \frac{1}{n^2}\sum_{i=1}^n \var(Y_i) = \frac{n}{n^2} = \frac{1}{n}.
            \]
            We want to find $n$ such that
            \begin{align*}
                P\left(|\hat{x}-x|<\frac{1}{2}\right) \geq 0.9 &\iff 1 - P\left(|\hat{x}-x| \geq \frac{1}{2}\right) \geq 0.9 \\
                &\iff P\left(|\overline{Y}| \geq \frac{1}{2}\right) \leq 0.1.
            \end{align*}
            Using Chebyshev's inequality,
            \[
                P\left(|\overline{Y}| \geq \frac{1}{2}\right) \leq \frac{\var(\overline{Y})}{(1/2)^2} = \frac{1/n}{1/4} = \frac{4}{n}.
            \]
            To ensure the condition holds, we require
            \[
                \frac{4}{n} \leq 0.1 \implies n \geq 40.
            \]
            Thus, $n$ should be at least \fbox{40}.
        \end{solution}
    \end{parts}

    \question Basics on Graphs

    \begin{parts}
        \part[6] Let $V = \{a, b, c, d\}$ and $E = \{(a, b), (a, c), (a, d), (d, c)\}$. Count the number of subgraphs of the
            graph $G = (V, E)$.
        \begin{solution}
            By cases:
            \begin{multicols}{2}
                \begin{enumerate}[label=\textbf{(\arabic*)}]
                    \item Null graph: 1 subgraph.
                    \item $V'=\{a,b,c,d\}\subseteq V$: $2^{\left|E\right|}=2^4=16$ subgraphs.
                    \item $V'=\{a,b,c\}\subset V$: $2^2=4$ subgraphs.
                    \item $V'=\{a,b,d\}\subset V$: $2^2=4$ subgraphs.
                    \item $V'=\{a,c,d\}\subset V$: $2^3=8$ subgraphs.
                    \item $V'=\{b,c,d\}\subset V$: $2^1=2$ subgraphs.
                    \item $V'=\{a,b\},\{a,c\},\{a,d\}\subset V$: $2\cdot3=6$ subgraphs.
                    \item $V'=\{b,c\},\{b,d\}\subset V$: $1\cdot2=2$ subgraphs.
                    \item $V'=\{c,d\}\subset V$: $2^1=2$ subgraphs.
                    \item $V'=\{a\},\{b\},\{c\},\{d\}\subset V$: $1\cdot4=4$ subgraphs.
                \end{enumerate}
            \end{multicols}
            Therefore, there are \fbox{49} subgraphs.
        \end{solution}

        \part[4] Draw a simple, connected, undirected graph consisting of 1 vertex of degree 1, 2 vertices of degree
            2, 3 vertices of degree 3, and 4 vertices of degree 4. Is the graph you draw planar or not? Justify
            your answer.
        \begin{solution}
            \begin{multicols}{2}
                \begin{figure}[H]
                    \centering
                    \includegraphics[width=.5\textwidth]{figs/q2b.pdf}
                    \caption{Graph for Question (\ref{question@2})(\ref{part@2@2})}
                \end{figure}

                \columnbreak

                The required graph is shown on the left.
                As visually presented, the graph satisfies all the requirements and has no crossing edges.

                Also, we find that $v=|V|=10$, $e=|E|=15$, and $r=7$. By Euler's formula, $v+r-e=2$ holds.
                Therefore, the graph is \fbox{planar}.
            \end{multicols}
        \end{solution}

        \part[6] How many simple, undirected graphs that do not contain any cycle can you construct with a fixed
            vertex set $\{A, B, C, D, E\}$? Explain your answer.
        \begin{solution}
            
        \end{solution}

        \part[4] The diameter of a graph is defined to be the length of the shortest path between the most distanced
            vertices. What is the diameter of the graph you drew in Question (\ref{question@2})(\ref{part@2@2})?
        \begin{solution}
            We start from vertex $a$ (degree 1) and perform Breadth-First Search (BFS) to find the furthest vertex.
            Then, we have:
            \begin{center}
                \begin{tabular}{|r|c|c|c|c|}
                    \hline
                    \bfseries Distance & 1 & 2 & 3 & 4 \\ \hline
                    \bfseries Vertices & $b$ & $c$ & $d,g,h,j$ & $e,f,g,i$ \\ \hline
                \end{tabular}
            \end{center}
            The furthest vertices from $a$ are $e,f,g,i$ with distance 4.
            Also, to get to $a$ from any vertex other than $b$, at least 2 edges are needed.
            Therefore, the diameter of the graph is \fbox{4}.
        \end{solution}
    \end{parts}

    \question Proofs with Graphs

    \begin{parts}
        \part[6] Let $G$ be a simple, undirected graph with at least two vertices. If $G$ has exactly one vertex of degree
            1, show that there exists a cycle in $G$.
        \begin{psol}
            Assume, for contradiction, that $G$ is acyclic. Then, $G$ is a tree.
            In a tree, there are at least two vertices of degree 1 (the leaves).
            This contradicts the assumption that $G$ has exactly one vertex of degree 1.
            Therefore, our assumption is false, and $G$ must contain at least one cycle.
        \end{psol}

        \part[8] A subset set of edges $E'$ is called an edge cut of a graph $G = (V, E)$ if the subgraph $(V, E - E')$ of
            $G$ is disconnected. The edge connectivity of a graph is the minimum number of edges in an edge
            cut. Suppose $G$ is an undirected graph with connectivity 2 and at least 3 vertices. Prove that there
            exists a cycle in $G$.
        \begin{psol}
            
        \end{psol}
    \end{parts}

    \question Generalized Square Graphs

    Consider the family of simple, undirected graphs $G_{m,n} = (V_{m,n}, E_{m,n})$, where $m, n$ are positive integers.
    The set of vertices is defined by $V_{m,n} = \{(i, j) \in \mathbb{Z}^2 : 1 \leq i \leq m, 1 \leq j \leq n\}$, and the set of edges is defined
    by the following property: for any two distinct vertices $v_1 = (i_1, j_1), v_2 = (i_2, j_2) \in V_{m,n}$, $\{v_1, v_2\} \in E_{m,n}$
    if and only if $i_1 = i_2$ or $j_1 = j_2$.

    \begin{parts}
        \part[4] Determine the number of edges in $G_{m,n}$ for generic $m, n$.
        \begin{solution}
            On the $i$-th row, there are vertices $(i,1), (i,2), \ldots, (i,n)$, totalling $n$ vertices.
            On the $j$-th column, there are vertices $(1,j), (2,j), \ldots, (m,j)$, totalling $m$ vertices.
            In any row, each vertex connects to $n-1$ other vertices; in each column, each vertex connects to $m-1$ other vertices.
            Therefore, $\deg(v)=m+n-2$ for any vertex $v\in V_{m,n}$.

            By Handshaking, $\displaystyle\sum \deg(v) = 2|E_{m,n}| \Rightarrow |E_{m,n}| = \frac{1}{2} \displaystyle\sum \deg(v) = \boxed{\frac{1}{2} mn(m+n-2)}$.
        \end{solution}

        \part[6] Is $G_{3,3}$ planar? Prove your statement.
        \begin{psol}
            Consider the following arrangement of $G_{3,3}$:
            \begin{figure}[H]
                \centering
                \includegraphics[width=.3\textwidth]{figs/q4b.pdf}
                \caption{An arrangement of $G_{3,3}$}
            \end{figure}

            Choose two sets of vertices: $\{A,E,I\}$ and $\{B,D,F\}$.
            We can directly make connections: $AB, AD, EB, ED, EF, IF$.
            Other connections can be made via intermediate vertices: $AI=AC\rightarrow CI, IB=IH\rightarrow HB, ID=IG\rightarrow GD$.
            Then, we have found a subdivision of $K_{3,3}$ in $G_{3,3}$.
            By Kuratowski's Theorem, \fbox{$G_{3,3}$ is non-planar}.
        \end{psol}

        \part[6] Find all $(m,n)$ such that $G_{m,n}$ is planar.
        \begin{solution}
            Note that $G_{m,n}$ and $G_{n,m}$ are isomorphic.
            Without loss of generality, assume $m\leq n$.
            Also, $G_{1,n}$ is isomorphic to $K_n$, which is planar for $n\leq 4$.

            For $G_{2,n}$, $G_{2,1}$ is planar because $G_{1,2}$ is planar.
            $G_{2,2}$ is planar as it is a square.
            A planar arrangement of $G_{2,3}$ is shown below:
            \begin{figure}[H]
                \centering
                \includegraphics[width=.3\textwidth]{figs/q4c.pdf}
                \caption{A planar arrangement of $G_{2,3}$}
            \end{figure}
            For $G_{2,4}$, we can verify that it is planar using Euler's formula.
            Namely, $v=8$, $e=16$, and $r=10$ satisfy $v+r-e=2$.
            However, for $G_{2,5}$, we find that every row is a $K_5$ in its own, by Kuratowski's Theorem, it is non-planar.
            For $n\geq 5$, $G_{2,n}$ contains $G_{2,5}$ as a subgraph, so it is non-planar.
            Therefore, when $m=2$, $G_{2,n}$ is planar for $n\leq 4$.

            For $m=3$, $G_{3,n}$ is planar for $n=1,2$ as they are isomorphic to $G_{1,3}$ and $G_{2,3}$ respectively.
            However, $G_{3,3}$ is non-planar as shown in part (\ref{part@4@2}). For $n\geq 3$, $G_{3,n}$ contains $G_{3,3}$ as a subgraph, so it is non-planar.
            Therefore, when $m=3$, $G_{3,n}$ is planar for $n\leq 2$.

            For $m=4$, $G_{4,n}$ is planar for $n=1,2$ as they are isomorphic to $G_{1,4}$ and $G_{2,4}$ respectively.
            However, for $n\geq 3$, $G_{4,n}$ contains $G_{3,3}$ as a subgraph, so it is non-planar.
            Therefore, when $m=4$, $G_{4,n}$ is planar for $n\leq 2$.

            For $m\geq 5$, $G_{m,n}$ must contain $K_5$ as a subgraph, so it is non-planar.

            In conclusion, the pairs $(m,n)$ such that $G_{m,n}$ is planar are:
            \[
                \boxed{(1,1), (1,2), (1,3), (1,4), (2,1), (2,2), (2,3), (2,4), (3,1), (3,2), (4,1), (4,2)}.
            \]
        \end{solution}

        \part[4] Find the necessary and sufficient condition for $G_{m,n}$ to have an Eulerian circuit.
        \begin{solution}
            For $G_{m,n}$ to have an Eulerian circuit, all vertices must have even degree, i.e., $(m+n-2)$ is even and is an integer.
            Note that subtracting 2 does not change the parity of a number.
            Therefore, the necessary and sufficient condition is that \fbox{$m+n$ is even, i.e., $m$ and $n$ have the same parity}.
        \end{solution}
    \end{parts}

    \question Graph Coloring

    \begin{parts}
        \part[10] Recall that a proper region coloring is an assignment of a color to each region in a planar embedding
            of a graph such that adjacent regions receive different colors. Count the number of distinct ways of
            region coloring the graph $D_4$ (see below) with $k$ colors for any positive integer $k$.

        \begin{solution}
            Represent each region by a vertex, and connect the vertices if the corresponding regions are adjacent, we obtain the following graph $G$:
            \begin{figure}[H]
                \centering
                \includegraphics[width=.3\textwidth]{figs/q5a.pdf}
                \caption{Graph $G$ of $D_4$ for region coloring}
            \end{figure}
            
            By deletion-contraction, the chromatic polynomial $P(G,k)$ is given by:
            \begin{align*}
                P(G,k) &= P(G-\set{B,E}, k) - P(G/\set{B,E}, k) \\
                &= P(C_6, k) - k(k-1)^2(k-2)^2 \\
                &= (k-1)^6 + (k-1) - k(k-1)^2(k-2)^2
            \end{align*}

            Also, note that $\gamma(G)=4$, then $P(G,k)$ is valid for $k\geq4$.

            Therefore, the number of distinct ways of region coloring the graph $D_4$ with $k$ colors is
            \[\boxed{(k-1)^6 + (k-1) - k(k-1)^2(k-2)^2\quad\forall k\geq4}.\]
        \end{solution}

        \part[8] Recall that a proper vertex coloring is an assignment of a color to each vertex in a graph such that
            adjacent vertices receive different colors. Count the number of distinct ways to vertex coloring the
            graph below with $k$ colors for any positive integer $k$.
        \begin{solution}
            
        \end{solution}
    \end{parts}

    \question Applications of Graph Theory

    \begin{parts}
        \part[4] Seven people that assist to a conference want to have lunch together at a roundtable during the
            three days that the conference lasts. In order to get to know each other better, they decide to sit
            in such a way that two people are next to each other at most once. Can they achieve their goal?
            And what happens if the congress lasts 5 days? 
        \begin{solution}
            Represent each person by a vertex, and connect two vertices if the corresponding people sit next to each other.
            Then, each lunch arrangement corresponds to a $C_7$ in the graph.
            Also, each edge can appear in at most one $C_7$.

            Note that $K_7$ has $\binom{7}{2}=21$ edges, and each $C_7$ contains 7 edges.
            Therefore, at most $\lfloor\frac{21}{7}\rfloor=3$ different lunch arrangements can be made such that
            two people are next to each other at most once.

            If the congress lasts 3 days, \fbox{they can achieve their goal}.
            
            However, if the congress lasts 5 days, \fbox{they cannot achieve their goal}.
        \end{solution}

        \part[8] At a dancing party, 10 participants form pairs to dance at each round. Suppose after many rounds,
            it is found that everyone has danced with exactly 2 other participants. How many different records
            could there be, if one is to record all pairs of participants who have danced? (Note that the records
            are not required to contain information on in which order have the participants danced.) 
        \begin{solution}
            Represent each participant by a vertex, connect two vertices if the corresponding participants have danced together.
            Then, we have a graph with 10 vertices, each of degree 2.
            We count the number of distinct graphs possible.

            
        \end{solution}
    \end{parts}

\end{questions}

\end{document}